\chapter{Yongning Na tones in dynamic"=synchronic perspective}
\label{chap:yongningnatonesinadynamicsynchronicperspective}

\epigraph{The past century of phonetic research has illuminated our understanding of the production of sound, the properties of the acoustic signals, and to a~certain extent, the perception of speech sounds. But the search for the originating causes of sound change itself remains one of the most recalcitrant problems of phonetic science.}{\citep[1]{labov1979}}

%Command \noindent added to avoid having a first indent in cases where a paragraph starts after an epigraph without an intervening title.
{\noindent}The synchronic description proposed in the present volume provides a~basis for studying the
historical dynamics of tone in Na: as mentioned in \sectref{sec:theoreticalbackdrop}, a~dynamic approach to synchrony brings out patterns of synchronic
\isi{variation} which, in turn, offer some glimpses into diachronic evolution. The study of
variability is especially crucial to the study of tone. Variability in tone
patterns tends to be high in level"=tone systems with rich morphotonology, and {diachronic} change
tends to be more rapid than in other areas of the linguistic system (such as syntax). 

The argument that tonological models
should be designed in such a~way as to accommodate
patterns of \isi{variation} is found e.g.~in a~discussion of Bambara, a~\ili{Mande} language:

\begin{quotation}
Clearly, any hypothesis about the system that underlies the tonal
productions of Bambara speakers should be able to account, with minimal adjustments, for observed
patterns of \isi{variation}.~\citep[8]{creissels1992}\footnote{\textit{Original text}: il est clair que toute hypothèse sur l’organisation du système
sous"=jacent à un corpus de productions tonales de bambarophones doit pouvoir au prix d’un minimum
d’aménagements rendre compte de possibilités éventuelles de {variation}.}
\end{quotation}

As more data becomes available about the dialectal diversity of Na, it will be possible to investigate patterns of contact and \isi{variation} with increasing precision. Four topics are discussed here: {structural} gap"=filling, disyllabification, {analogy}, and the influence of {bilingualism} with {Mandarin}. 


\section[Gap"=filling in tonal paradigms]{Gap"=filling in tonal paradigms: The example of subject"=plus"=verb phrases}
\label{sec:howthesuffixacquiresitslmorhtoneafteramtoneverb}
\is{gap-filling|textbf}

Structural \is{gap-filling}gap"=filling causes a~change in the phonological system when an~allophone has
drifted far enough away from its original pronunciation for a~new combination to nest itself in the abandoned slot. For
instance, in Yongning Na it is likely that the sound [\ipa{ʁ}] was originally an~empty"=onset filler (see Appendix A, \sectref{sec:theinitialvoiceduvularfricativeasaphonemicizedemptyonsetfiller}). The syllable /\ipa{ʁo}/ in
/\ipa{ɑ˩ʁo˧}/ ‘house’ is \is{comparative method (historical linguistics)}reconstructed as a~simple *\ipa{o} at the proto"=\ili{Naish} stage \citep{jacquesetal2011}; it remains
phonemically onsetless to this day in \ili{Laze} ([\ipa{ɑ˥wu˥}], phonemically /\ipa{ɑ˥u˥}/) and in \ili{Naxi}
([\ipa{mi˧wu˩}], phonemically /\ipa{mi˧u˩}/). After *\ipa{o} syllables came to be realized as [\ipa{ʁo}] (at the surface phonological level) in Yongning Na, there
remained no [\ipa{o}] or [\ipa{wo}] syllables. But this phonetic slot was filled by syllables
with other origins: the syllable /\ipa{wo}/ is now firmly attested, in examples such as /\ipa{wo˥}/ ‘hard’, /\ipa{wo˩\textsubscript{b}}/ ‘{classifier} for teams of oxen’, /\ipa{wo˩kɤ\#˥}/ ‘swing’, and /\ipa{wo˩˥}/ ‘turnip leaves’. The introduction of [\ipa{wo}] syllables precipitated the phonemicization of what was originally an~empty"=onset"=filler:
the syllable /\ipa{ʁo}/ in /\ipa{ɑ.ʁo}/ ‘house’ must now be analyzed as composed of two phonemes, an initial
/\ipa{ʁ}/ and the vowel /\ipa{o}/.

\begin{sidewaystable}[p]
	\caption{\label{tab:thetonepatternsofsubjectplusverbBIS}The tone patterns of subject"=plus"=verb combinations, in
		surface phonological transcription.}
	\begin{tabularx}{\textheight}{ l@{\hspace{6mm}} Q l@{\hspace{6mm}} l@{\hspace{6mm}} l@{\hspace{6mm}} l@{\hspace{6mm}} Q }
		\lsptoprule
		& tone of verb & & & & &\\ \cmidrule{2-7}	
		tone of noun & H & M\textsubscript{a} & M\textsubscript{b} & L\textsubscript{a} & L\textsubscript{b} & MH\\ \midrule
		LM, LH & L.H & L.M+M & L.M+M & L.H & L.H & L.MH\\
		M & M.M+L & M.M+M & M.M+M & M.L & M.L & M.MH\\
		L & M.M+L & L.L  & M.M+M & L.L & L.L~/ M.L & L.L\\
		H & M.M+L & M.M+L & M.M+L & M.MH & M.MH & M.L\\
		MH & M.H & M.H & M.H & M.MH & M.MH & M.H\\ \addlinespace \hdashline \addlinespace
		M & M.M.M+L & M.M.M+M & M.M.M+M & M.M.L & M.M.L & M.M.MH\\
		\#H & M.M.M+L & M.M.M+L & M.M.M+L & M.M.MH & M.M.MH & M.M.L\\
		MH\# & M.M.MH & M.M.MH & M.M.MH & M.M.MH & M.M.MH & M.M.H\\
		H\$ & M.M.M+L & M.M.M+L & M.M.M+L~/ M.M.M+H & M.M.MH & M.M.MH & M.H.L\\
		L & L.L.L & L.L.L & L.L.L & L.L.L & L.L.L & L.L.H\\
		L\# & M.L.L & M.L.L & M.L.L & M.L.L & M.L.L & M.L.L\\
		LM+MH\# & L.M.M+L & L.M.M+L & L.M.M+L & L.M.MH & L.M.MH & L.M.H\\
		LM+\#H & L.M.M+L & L.M.M+M & L.M.M+M & L.M.L & L.M.MH & L.M.MH\\
		LM & L.M.M+L & L.M.M+M & L.M.M+M & L.M.L & L.M.L & L.M.MH\\
		LH & L.H.L & L.H.L & L.H.L & L.H.L & L.H.L & L.H.L\\
		H\# & M.H.L & M.H.L & M.H.L & M.H.L & M.H.L & M.H.L\\
		\lspbottomrule
	\end{tabularx}
\end{sidewaystable}

\is{form!surface|(}
\is{gap-filling}Gap"=filling can also take place in tonal paradigms. This section is devoted to a~plausible example from subject"=plus"=verb constructions. 

\tabref{tab:thetonepatternsofsubjectplusverbcombinationsinsurfacephonologicaltranscription}, repeated here as \tabref{tab:thetonepatternsofsubjectplusverbBIS}, presents the tonal behaviour of combinations of a~\is{monosyllables}monosyllabic or disyllabic subject noun with a~verb. Two contexts were used to arrive at underlying tone categories: S+V, and
S+V+\textsc{perfective}. For instance, ‘the guests arrive’ is /\ipa{hĩ˧-bæ˧
	tsʰɯ˧˥}/, and addition of the {perfective} yields /\ipa{hĩ˧-bæ˧
	tsʰɯ˧-ze˥}/ ‘the guests have arrived’. The tone pattern for this combination of subject and predicate is described as /M.M.MH/, and further analyzed as \mbox{//MH\#//}: a~MH \is{tonal contour}contour associating to the last
syllable.

A~challenge raised by the data set in \tabref{tab:thetonepatternsofsubjectplusverbBIS} concerns the analysis of the surface phonological tone sequences ending in M+L, M+M and M+H. Remember that, in these notations, the tone which follows the ‘+’ sign is that carried by the {perfective} \mbox{/\ipa{-ze˧}/} when placed after the subject"=plus"=verb combination. The {question} is how the {perfective} acquires its /L/, /M/ or /H/ tone in these combinations. The full list of the expressions at issue is: M.M+L, M.M.M+L, M.M+M, M.M.M+M, M.M.M+H, L.M+M, L.M.M+M, and L.M.M+L. Among these, those
ending in /M+L/, as in (\ref{ex:tigerwalked}), and those ending in /M+M/, as in (\ref{ex:tigerdied}), are commonplace. On the other hand, there only exists one sequence ending in /M+H/: it results from the
combination of a~\mbox{//H\$//}-tone subject and a~\mbox{//M\textsubscript{b}//}-tone verb, as in (\ref{ex:shecatjumped}). 

\begin{exe}
	\ex
	\label{ex:tigerwalked}
	\ipaex{lɑ˧ se˧-ze˩}\\ 
	\gll lɑ˧		se˥		-ze˧\textsubscript{b}\\
	tiger	to\_walk	\textsc{pfv}\\
	\glt ‘the tiger walked’ (input tones: M on noun and H on verb)
\end{exe}

\begin{exe}
	\ex
	\label{ex:tigerdied}
	\ipaex{lɑ˧ ʂɯ˧-ze˧}\\ 
		\gll lɑ˧		ʂɯ˧\textsubscript{a}		-ze˧\textsubscript{b}\\
		tiger	to\_die		\textsc{pfv}\\
		\glt ‘the tiger died’ (input tones: M on noun and M on verb)
\end{exe}
	
\begin{exe}
	\ex
	\label{ex:shecatjumped}
	\ipaex{hwɤ˧mi˧ tsʰo˧-ze˥}\\ 
		\gll hwɤ˧mi˥\$		tsʰo˧\textsubscript{b}	-ze˧\textsubscript{b}\\
		she\_cat	to\_jump	\textsc{pfv}\\
		\glt ‘the she-cat jumped’ (input tones: H\$ on noun and M\textsubscript{b} on verb)
\end{exe}
	
The hypothesis proposed here is that the pattern ending in /M+H/
is an~\is{innovative (phonological form)}{innovation}.

The {perfective} can receive one of three tones in subject"=plus"=verb plus {perfective} constructions: \mbox{/M/}, \mbox{/H/}, or \mbox{/L/}. Cases in which
the {perfective} carries \mbox{/M/} tone are the simplest: the morpheme does not receive any tone assignment from what
precedes, and surfaces with its lexical M tone. The surface strings \mbox{/M.M}+\mbox{M/} (for \is{monosyllables}monosyllabic nouns) and \mbox{/M.M.M}+\mbox{M/} (for
disyllabic nouns) can therefore be analyzed as \mbox{//M//}. As for \mbox{/L.M}+\mbox{M/} (for \is{monosyllables}monosyllabic nouns) and \mbox{/L.M.M}+\mbox{M/} (for
disyllabic nouns), they can be analyzed as \mbox{//LM//}. 

Cases where the {perfective} receives /H/ tone look like typical instances of the \is{floating tone}floating H tone, \mbox{//\#H//}. This
tone, which does not surface \is{form!in isolation}in isolation but can be manifested on a~following syllable (\sectref{sec:afloatinghtonewithcomparativeevidencepointingtoitsorigin}), is
frequently observed in Yongning Na. It is the lexical tone of a~class of nouns, exemplified by ‘little brother’, realized \is{form!in isolation}in isolation as /\ipa{gi˧zɯ˧}/ ‘little brother’, and yielding /\ipa{gi˧zɯ˧ ɲi˥}/ when followed by the \isi{copula}, as explained in~\sectref{sec:afloatinghtonewithcomparativeevidencepointingtoitsorigin}.

It was noted in \sectref{sec:wordfinalandmorphologicalnucleusfinalHtones} that \mbox{//H\$//} tone shows signs of variability: it is the lexical tone for
which there is the greatest number of morphotonological variants, in various
morphosyntactic contexts. In subject"=plus"=verb combinations, its association with a~\mbox{//M\textsubscript{b}//}-tone verb allows
for two possibilities: M.M.M+L and M.M.M+H. The latter tonal string, M.M.M+H, is not attested in any
of the other subject"=plus"=verb combinations. A~hypothesis suggested by this distribution is that this
\is{variants}variant is an~\is{innovative (phonological form)}{innovation} which filled a~slot that was empty in the surface phonological forms.

Under the hypothesis that the tone pattern M.M.M+H on subject"=plus"=verb plus {perfective} combinations represents an~\is{innovative (phonological form)}{innovation}, prior to this \is{innovative (phonological form)}{innovation} the M.M.M+L surface pattern
could have been analyzed as \mbox{//\#H//}. The \is{floating tone}floating H tone was not manifested directly but triggered
a~lowering of following tones~-- in this instance, a~lowering of the tone of the {perfective}
morpheme.\footnote{The floating H tone of Yongning Na, transcribed as \mbox{//\#H//}, exists not only as a~lexical category on nouns (as discussed in \sectref{sec:afloatinghtonewithcomparativeevidencepointingtoitsorigin}), but also as the output of some syntactically restricted tone rules (morphotonological rules), such as those that apply in {compound} nouns. Cases where a~H tone does not surface but lowers the following tones (to L) are found in various areas of Yongning Na morphotonology. For instance, the phrase /\ipa{mv̩˩tɕo˧ se˧}/ ‘to walk downward’ depresses a~following {perfective} (//\ipa{-ze˧\textsubscript{b}}//, which has lexical M tone) to L: /\ipa{mv̩˩tɕo˧ se˧-ze˩}/ ‘(s)he walked downward’; this is interpreted as evidence of the presence of a~floating H tone in the expression ‘to walk downward’ (see \sectref{sec:themarkingofspatialorientationonverbs}).}

This state of affairs is reflected in the analysis in \tabref{tab:subjectverbcombinations}, which leaves out the
problematic \is{variants}variant M.M.M+H for the combination of a~subject carrying H\$ tone and a~verb carrying M\textsubscript{b} tone.

\begin{sidewaystable}[p]
	\caption{\label{tab:subjectverbcombinations}A phonological analysis of the tones of subject"=plus"=verb combinations, 
		% positing that tone \#H is reflected in the lowering of the tone of the postverbal morpheme, and 
		leaving aside the M.M.M+H variant of the combination of a~H\$-tone subject and a~M\textsubscript{b}-tone verb.}
	\begin{tabularx}{\textheight}{ l Q Q Q Q Q Q }
		\lsptoprule
		& tone of verb & & & & &\\ \cmidrule{2-7}
		tone of noun & H & M\textsubscript{a} & M\textsubscript{b} & L\textsubscript{a} & L\textsubscript{b} & MH\\ \midrule
		LM, LH & LM & LM & LH & LH & LH & LM+MH\#\\
		M & M & M & \#H & M.L & M.L & M.MH\\
		L & L & M & \#H & L & L & L\\
		H & \#H & \#H & \#H & MH\# & MH\# & L\#\\
		MH & H\# & H\# & H\# & MH\# & MH\# & H\#\\ \addlinespace \hdashline \addlinespace
		M & M & M & \#H & \#H & \#H & MH\#\\
		\#H & \#H & \#H & \#H & MH\# & MH\# & \#H\\
		MH\# & MH\# & MH\# & MH\# & MH\# & MH\# & \#H\\
		H\$ & \#H & \#H & \#H & MH\# & MH\# & H\#\\
		L & L & L & L & L & L & L+H\#\\
		L\# & L\#-- & L\#-- & L\#-- & L\#-- & L\#-- & L\#--\\
		LM+MH\# & LM--+\#H  & LM--+\#H  & LM--+\#H  & LM+MH\# & LM+MH\# & LM+H\$\\
		LM+\#H & LM-- & LM-- & LM--+\#H  & LH-- & LM+MH\# & LM+MH\#\\
		LM & LM-- & LM-- & LM--+\#H  & LH-- & LH-- & LM+MH\#\\
		LH & LH-- & LH-- & LH-- & LH-- & LM+MH\# & LH--\\
		H\# & H\#-- & H\#-- & H\#-- & H\#-- & H\#-- & H\#--\\
		\lspbottomrule
	\end{tabularx}
\end{sidewaystable}


\tabref{tab:subjectverbcombinations} is a~\is{comparative method (historical linguistics)}reconstruction of the set of \is{tone rules}tone rules that applied in
subject"=plus"=verb constructions prior to the appearance of the M.M.M+H \is{variants}variant. If it represents a~historical reality, this \is{comparative method (historical linguistics)}reconstructed stage is likely to have shallow time depth: the amount of observed idiolectal and dialectal diversity suggests that such a~change can take place within a~couple of generations. At the \is{comparative method (historical linguistics)}reconstructed stage represented in \tabref{tab:subjectverbcombinations}, a~tone rule must be specified, to the effect that \mbox{//\#H//} tone in subject"=plus"=verb combinations does not
surface, but depresses following tones to L. In view of the general architecture of the Na tone
system, this rule is not an~\textit{ad hoc} device to explain away an~unaccountable observation:
a~rule to the same effect operates in other morphosyntactic contexts.

In the present state of the language, on the other hand, the M.M.M+H \is{variants}variant has settled in, and its simplest phonological interpretation is as the
manifestation of a~\is{floating tone}floating H tone~-- an interpretation that conflicts with the earlier system. Interpretation of the M.M.M+H pattern as \is{form!underlying}underlying \mbox{//\#H//} causes an~in"=depth modification in the
system: as the \mbox{//\#H//} slot in the system comes to be occupied by the new, innovative form,
the M.M.M+L surface pattern, which could previously be analyzed as reflecting an~underlying \mbox{//\#H//},
requires a~new interpretation, as do the other surface patterns ending in /M+L/ in \tabref{tab:thetonepatternsofsubjectplusverbBIS}. 

A~possible phonological reanalysis in view of the entire system would be as a~\is{floating tone}floating L tone, //\#L//, which would thus enter the language’s tone system. The surface phonological
patterns in subject"=plus"=verb plus {perfective} combinations would then be interpreted as in \tabref{tab:analysisofthetonesofsubjectverbcombinationspositingfloatingLtones}. The //\#L//
category is highlighted, bringing out its relatively pervasive presence in the table.

\begin{sidewaystable}[p]
	\caption{\label{tab:analysisofthetonesofsubjectverbcombinationspositingfloatingLtones}A phonological analysis of the tones of subject"=plus"=verb combinations positing floating L tones.}
	\begin{tabularx}{\textheight}{ l Q Q Q Q Q Q }
		\lsptoprule
		& tone of verb & & & & &\\ \cmidrule{2-7}
		tone of noun & H & M\textsubscript{a} & M\textsubscript{b} & L\textsubscript{a} & L\textsubscript{b} & MH\\ \midrule
		LM, LH & LH & LM & LM & LH & LH & LM+MH\#\\
		M & \shadedcell \#L & M & M & M.L & M.L & M.MH\\
		L & \shadedcell \#L & L & M & L & L & L\\
		H & \shadedcell \#L & \shadedcell \#L & \shadedcell \#L & MH\# & MH\# & L\#\\
		MH & H\# & H\# & H\# & MH\# & MH\# & H\#\\ \addlinespace \hdashline \addlinespace
		M & \shadedcell \#L & M & M & \shadedcell \#L & \shadedcell \#L & MH\#\\
		\#H & \shadedcell \#L & \shadedcell \#L & \shadedcell \#L & MH\# & MH\# & \shadedcell \#L\\
		MH\# & MH\# & MH\# & MH\# & MH\# & MH\# & \shadedcell \#L\\
		H\$ & \shadedcell \#L & \shadedcell \#L & \lshadedcell \#H / \#L & MH\# & MH\# & H\#\\
		L & L & L & L & L & L & L+H\#\\
		L\# & L\#-- & L\#-- & L\#-- & L\#-- & L\#-- & L\#--\\
		LM+MH\# & LM--+\#H  & LM--+\#H  & LM--+\#H  & LM+MH\# & LM+MH\# & LM+H\$\\
		LM+\#H & LM--+\#H  & LM-- & LM-- & LH-- & LM+MH\# & LM+MH\#\\
		LM & LM--+\#H  & LM-- & LM-- & LH-- & LH-- & LM+MH\#\\
		LH & LH-- & LH-- & LH-- & LH-- & LM+MH\# & LH--\\
		H\# & H\#-- & H\#-- & H\#-- & H\#-- & H\#-- & H\#--\\
		\lspbottomrule
	\end{tabularx}
\end{sidewaystable}

Devoting the whole of the present section to the discussion of one isolated tonal \is{variants}variant may seem
dreadfully disproportionate. This \is{variants}variant deserves special attention, however, because it illustrates the
constant tension between \is{form!underlying}underlying forms and surface phonological forms, shedding light on types of
evolution taking place in level"=tone systems. From the point of view of surface phonological forms,
the innovative expression discussed here can be viewed in the light of a~\isi{simplification}: at the
(hypothetical) \is{conservative (phonological form)}conservative stage presented in \tabref{tab:subjectverbcombinations}, for the combination of \mbox{//H\$//} and \mbox{//M\textsubscript{b}//} there is
a~H tone in the input, and none in the output; by contrast, in the innovative form there is
an~output H tone echoing the input H, creating a~better fit between input and output. From the
point of view of underlying forms, on the other hand, the new combination creates an~analytical puzzle for linguists~-- and probably for language learners too. Cases like this one allow for
several analytical options and hence hold potential for {diachronic} change.
\is{form!surface|)}

\section{Disyllabification}
\label{sec:disyllabification}

\is{disyllabification|textbf}
\is{disyllabification|(}

As mentioned at the outset of Chapter~\ref{chap:compoundnouns}, many roots that used to be phonologically
distinct have become \is{homophony}homophonous in Na, as in other \il{Sino-Tibetan}Sino"=Tibetan languages that have undergone
considerable \isi{phonological erosion} (such as \ili{Tujia}, Bai, \ili{Namuyi}, or \ili{Shixing}~/ Xumi). As a~consequence, there
exists a~strong tendency towards disyllabification. If each tonal combination of two
monosyllables created a~different tonal category for the resulting \is{disyllables}disyllable, this could
multiply the number of tones by squaring: six tones on monosyllables could yield 6×6=36 tones on
disyllables. The observed number is much smaller: eleven tone categories for disyllabic nouns. The study of the relationship between the tones of monosyllables and those of disyllables
holds promise for an~understanding of the dynamics of the tone system.


\subsection{A dynamic analysis of compound nouns}
\label{sec:adynamicanalysisofcompoundnouns}

The analysis of \is{compounds}compound nouns in Chapter~\ref{chap:compoundnouns} aimed to bring out the relationship between input nouns
and the resulting \is{compounds}compound. The notations chosen for the tones of compounds emphasized their internal
makeup. For instance, the combination of a~\#H-tone determiner and a~LM"=tone head yields a~M.H
surface phonological tone pattern, as in (\ref{ex:horseskin}).

\begin{exe}
	\ex
	\label{ex:horseskin}
	\ipaex{ʐwæ˧-ɣɯ˥}\\ 
		\gll ʐwæ˥		ɣɯ˩˧\\
		horse		skin\\
		\glt ‘horse’s skin’ (DetermCompounds6.24, 7.67-68, 12.44)
\end{exe}

The processes leading from the input tones to the tone of the \is{compounds}compound can be interpreted as follows: the lexical tone of the determiner, being a~\is{floating tone}floating H tone (never expressed on the lexical item itself, only on a~following syllable), associates to the
second syllable of the \is{compounds}compound. The assignment of surface
tones then takes place according to the general rules governing the association of tone \#H in
Yongning Na. Since there is a~following syllable within the \isi{tone group} to host it (namely, the
second syllable of the \is{compounds}compound), the H tone attaches there, and the first syllable of the \is{compounds}compound
receives M by default (through Rule~2). The notation used for this combination in Chapter~\ref{chap:compoundnouns} is \#H--, using the symbol
‘--' to stand for the last syllable of the first part of the \is{compounds}compound. This notation, while it is
fairly complex, appears adequate insofar as it reflects a~hypothesis about the way in which the
tone pattern of the \is{compounds}compound obtains. Such notations are referred to below as \textit{source"=oriented}.

In terms of end result, on the other hand, the \is{compounds}compound in (\ref{ex:horseskin}) belongs in tone category H\#: it
carries a~final H tone, which does not move. Disyllabic compounds made up of the combination of a~\#H-tone
determiner and a~LM"=tone head therefore feed into the lexical category of H\# disyllables. Notation
as H\# is referred below as \textit{result"=oriented}.

Likewise, the source"=oriented notation --L corresponds to the
result"=oriented notation L\#: assigning a~L tone after the \is{juncture (inside a tone group)}juncture between the two parts of the disyllabic \is{compounds}compound yields the same result as assigning a~final L tone to the entire
expression. \tabref{tab:sourceorientedandresultorientednotationsofthetonesofcompoundsthreeexamples}
provides a~summary.

\begin{table}%[t]
\caption{Source"=oriented and result"=oriented notations of the tones of compounds: three examples.}
\begin{tabularx}{\textwidth}{ l Q l l }
\lsptoprule
	 &  & \multicolumn{2}{l}{phonological analysis}\\ \cmidrule{3-4}
	input tones & surface phonological tone & source"=oriented & result"=oriented\\\midrule
	\#H and LM & M.H & \#H-- & H\#\\
	M and LM & M.L & --L & L\#\\
	M and L & M.L & --L & L\#\\
\lspbottomrule
\end{tabularx}
\label{tab:sourceorientedandresultorientednotationsofthetonesofcompoundsthreeexamples}
\end{table}

The table presenting the tone patterns of disyllabic compounds (\tabref{tab:surfacemonosyllabicmonosyllables} of Chapter~\ref{chap:compoundnouns}) is rewritten below as \tabref{tab:thetonesofdisyllabiccompounds}, adopting a~result"=oriented notation, eliminating all references to the \is{juncture (inside a tone group)}juncture between
the two parts of the \is{compounds}compound (transcribed by means of the symbol ‘--' in \tabref{tab:sourceorientedandresultorientednotationsofthetonesofcompoundsthreeexamples}). Each
row corresponds to a~tonal category of determiners, and each column to a~tonal category of
heads.\footnote{The same treatment cannot be extended to compounds of more than two syllables: it is not
	possible to describe the tone patterns of these compounds without referring to the {juncture} between
	the two input nouns, except by changing the entire notation, for instance specifying the tone of
	each syllable.}

\begin{table}%[t]
\caption{The tones of disyllabic compounds, adopting a~result"=oriented notation. The four combinations transcribed differently from \tabref{tab:sourceorientedandresultorientednotationsofthetonesofcompoundsthreeexamples} are set in italics.}
{\renewcommand{\arraystretch}{1.25}
\begin{tabularx}{\textwidth}{ l@{\hspace{8mm}} Q Q Q Q Q }
\lsptoprule
	tone & LH; LM & M & L & \#H & MH\#\\ \midrule
	LM & LM & LM & LM & \tikzmark{6a} LM+\#H & \tikzmark{5a} LM+MH\#\\
	LH & LH & L & LH & \hspace*{\fill}\tikzmark{6e} & \hspace*{\fill}\tikzmark{5e}\\
	M & \textit{L\#} & \#H & \textit{L\#} & \#H & MH\#\\
	L & \tikzmark{1a} L & & & & \hspace*{\fill}\tikzmark{1e}\\
	\#H & \textit{H\#} & \tikzmark{2a}\#H & & \hspace*{\fill}\tikzmark{2e} & \textit{L\#}\\
	MH & \tikzmark{3a} H\# & & \hspace*{\fill}\tikzmark{3e} & \tikzmark{4a} H\$ & \hspace*{\fill}\tikzmark{4e}\\
\lspbottomrule
\end{tabularx}}
\DrawBox[dashed]{1a}{1e}
\DrawBox[dashed]{2a}{2e}
\DrawBox[dashed]{3a}{3e}
\DrawBox[dashed]{4a}{4e}
\DrawBox[dashed]{5a}{5e}
\DrawBox[dashed]{6a}{6e}
\label{tab:thetonesofdisyllabiccompounds}
\end{table}

All of the tone categories observed on \is{disyllables}disyllabic nouns in Yongning Na are found in \tabref{tab:thetonesofdisyllabiccompounds}, except
M. This reveals that the synchronically productive \is{tone rules}tone rules that apply in compounds feed into all
of the tone categories of disyllables, apart from M.

\subsection{Possible origins for disyllables in view of their tone}
\label{sec:possibleoriginsfordisyllablesonthebasisoftheirtoneabirdseyeview}


\tabref{tab:possibleoriginsofdisyllabicitemsinviewofcurrentlyproductivetonerules} flips around the morphotonological rules set out in Chapters~\ref{chap:compoundnouns} and~\ref{chap:combinationsofnounswithgrammaticalwords} to provide
an~overview of possible origins of \is{disyllables}disyllabic items, in view of currently productive tone rules. The
indication ‘–’ means that no example was found. The mention \textit{dubious} is given for H\# tone as a~product of \is{suffixes}suffixation because there is no firmly"=attested pattern of {correspondence} between monosyllables and suffixed forms carrying \mbox{//H\#//} tone, only isolated tokens whose analysis is problematic. For instance, /\ipa{tse˧mi˥}/ ‘cigarette lighter’ has \mbox{//H\#//} tone and looks like the product of {suffixation}, but there is no corresponding {monosyllable} and hence no possibility (from this dialect alone) to establish a~tone {correspondence} between root and suffixed form.

The bird’s-eye view in \tabref{tab:possibleoriginsofdisyllabicitemsinviewofcurrentlyproductivetonerules} can provide a~hint for the analysis of disyllabic words whose \isi{etymology} is
unclear. 

\begin{table}%[t]
\caption{Possible origins of disyllabic items, in view of currently productive tone rules.}
\begin{tabularx}{\textwidth}{ Q Q Q Q }
\lsptoprule
	tone & compounding & \is{suffixes}suffixation & \is{prefixes}prefixation\\\midrule
	M & -- & yes & yes\\
	\#H & yes & yes & --\\
	MH\# & yes & -- & yes\\
	H\$ & yes & yes & yes\\
	L & yes & yes & yes\\
	L\# & yes & -- & yes\\
	LM+MH\# & yes & -- & --\\
	LM+\#H & yes & yes & --\\
	LM & yes & yes & --\\
	LH & yes & yes & --\\
	H\# & yes & \textit{dubious} & --\\
\lspbottomrule
\end{tabularx}
\label{tab:possibleoriginsofdisyllabicitemsinviewofcurrentlyproductivetonerules}
\end{table}

\subsection{Recovering the tones of nouns on the basis of compounds}
\label{sec:recoveringthetonesofnounsonthebasisofcompounds}

It is tempting to try to work backwards from the tones of compounds to those of their constituting elements. For
instance, ‘elder sibling (brother or sister)’ is commonly realized as /\ipa{ə˧mv̩˩}/ (tone: L\#), but
it has a~\is{variants}variant with MH\# tone: /\ipa{ə˧mv̩˧˥}/. The tone of the coordinative compound
/\ipa{ə˧mv̩˧-gi˥zɯ˩}/ ‘brothers’ (made up of ‘elder sibling’ + ‘younger brother’) is the one
expected for an~input MH\# tone, not an~input L\# tone. This could suggest
that it is the MH\# \is{variants}variant of ‘elder sibling’, /\ipa{ə˧mv̩˧˥}/, that went into the creation of the \is{compounds}compound. Seen in this light,
the rarity of the MH\# \is{variants}variant in present"=day speech, where /\ipa{ə˧mv̩˩}/ is far more common, suggests
that the MH\# \is{variants}variant /\ipa{ə˧mv̩˧˥}/ is not a~recent \is{innovative (phonological form)}{innovation} but a~form which is currently losing ground to
/\ipa{ə˧mv̩˩}/.

The greatest care must be exercised when attempting to recover tones in this way, however, since different tone rules may have applied at different stages of the language’s history. As pointed out by Nathan Hill (p.c.\ 2016), there is at present no way to exclude the possibility that the MH\# \is{variants}variant /\ipa{ə˧mv̩˧˥}/ for ‘elder sibling’ was inferred from the \is{compounds}compound (whatever the origin of the \is{compounds}compound's tone may be) and constitutes an \is{innovative (phonological form)}{innovation}. 

\is{disyllabification|)}

\section{Analogy}
\label{sec:analogy}

\subsection{General principles}
\label{sec:generalprinciples}

\is{analogy|textbf}Analogy is the process whereby word forms perceived as irregular are reshaped so as to conform with more
common forms: at some point, a~speaker of \ili{English} who was in doubt about the past tense for \textit{dive} reasoned that \textit{dove} is to \textit{dive} as \textit{drove} is to \textit{drive}, and introduced an innovative form, \textit{dove}, which has now become standard in North America, replacing the earlier form \textit{dived}. From a~morphological point of view, \isi{analogy} can be viewed as a~process of
regularization. From the point of view of phonetic change, on the other hand, the piecemeal changes
introduced by \isi{analogy} tend to obfuscate regular correspondences.

Case studies of analogical reanalysis reveal the complexity of individual situations. For
instance, in the \ili{Bantu} language Eton, the stem of the \isi{possessive} ‘my’ ends in
/\ipa{ɔ}/ in association with nouns of classes 1 and 3: /\ipa{-amɔ}/, and in /\ipa{a}/ elsewhere: /\ipa{-ama}/. This
irregularity is due to a~mechanism of analogical morphophonological reanalysis that changed the
original /\ipa{a}/ of the class 1/3 forms to /\ipa{ɔ}/. In Eton, there is a~|\ipa{ɔ}| morphoneme whose
morphologically"=conditioned realizations include /\ipa{wa}/; commonly occurring
sequences of /\ipa{w}/+/\ipa{a}/, although separated by a~morpheme {boundary}, were reinterpreted as
realizations of this morphoneme \citep{vandevelde2008}. In this example, morphophonological
\isi{analogy} disregards morphological boundaries. %This is not the least of the paradoxes of \isi{analogy},
%which has the potential to create morphophonological alternations \citep{blevinsetal2009} as well
%as to inhibit phonetic change in some contexts \citep{blevinsetal2009b}.

Analogy is by definition irregular and unpredictable. One may nonetheless believe that evidence from case studies gradually adds up.

\begin{quotation}
	[I]t is possible to some extent
	to constrain the space of hypotheses involving \isi{analogy}, and research
	on the general principles of \isi{analogy} is of utmost importance for historical
	linguistics. \citep[239]{jacques2016}
\end{quotation}

To date, studies about the principles of \isi{analogy} \citep[e.g.][]{kurylowicz1944, lahiri2000, hill2007, blevinsetal2009, juge2013, hill2014} contain little about tone, and studies about tone (\cites[e.g.][]{pike1948}{fromkinTONE1978}{pulleyblank1986}[229-231]{gussenhoven2004}) contain little about \isi{analogy}, even though it seems intuitively clear that morphotonology can be subject to analogical levelling just like other aspects of morphophonology. 


\subsection{Analogy in Yongning Na morphotonology}
\label{sec:applicationtoyongningna}

Traces of \isi{analogy} are also found among the tones of compounds and of affixed forms, as was pointed out in several places (in \sectref{sec:anindependentsetoffactscompoundgivennamesandtermsofaddress}, \ref{sec:lexicalizedcompoundsofnadjstructure}, \ref{sec:lhtoneroots}, \ref{sec:concludinggeneralobservations} \& \ref{sec:!nominalization}). 
It appears highly plausible that the tantalizingly similar, but not identical tonal paradigms of
\is{classifiers}classifiers~-- H\textsubscript{a} and H\textsubscript{b}, M\textsubscript{a} and M\textsubscript{b}, MH\textsubscript{a} and MH\textsubscript{b}, L\textsubscript{a}, L\textsubscript{b} and L\textsubscript{c} (see Chapter~\ref{chap:classifiers})~-- have undergone
a~degree of analogical levelling, without becoming fully identical. The existence of variants for
some combinations, and the consultants' occasional hesitations and confusions (errors) during elicitation
sessions, all point to the presence of contradictory pressures: on the one hand the tendency towards analogical
\isi{simplification}, and on the other hand the tendency to maintain the distinct identity of the
different classes. This is a~field where the description of a~single language variety reaches its limits, and
a~variationist approach would be called for. This study could be based on the closest language
varieties: studying phenomena of accommodation between speakers within the hamlet under study, then extending the investigation to dialects spoken in the plain of Yongning and its close vicinity. 

%A~\is{comparative method (historical linguistics)}{diachronic}"=comparative analysis confirms the plausibility of \isi{analogy} as a~major factor in the history of this aspect of Yongning Na morphotonology. But in order to understand this argument, a~hypothesis concerning the origin of the system needs to be set out first.

%\begin{quotation}
%	In Burmo-Qiangic languages other than \ili{Rgyalrongic} (except the Burmish branch), final stops are invariably lost. In the case of \ili{Naish} loss of final obstruents had already happened at the proto-\ili{Naish} stage.
%	
%	There is some evidence that the final stops in pre-proto-\ili{Naish}\footnote{Proto-\ili{Naish} only includes materials which can be shown to have been present in the common ancestor of the three languages that constitute the present"=day \ili{Naish} subgroup (Yongning Na, Naxi, and \ili{Laze}); pre-proto-\ili{Naish}, on the other hand, is a~construct in which the correspondences found between the three languages are projected as far back as the comparison with archaic languages allows. Within the \il{Sino-Tibetan}Sino"=Tibetan family, archaic languages include \ili{Rgyalrongic} languages (Khroskyabs, Horpa, Situ, \ili{Japhug}, Tshobdun and Zbu), Written \ili{Tibetan}, and Old \ili{Burmese}. For further information on the \is{comparative method (historical linguistics)}reconstruction of proto-\ili{Naish}, the reader is referred to \citet{jacquesetal2011}.} left a~trace in the patterning of tonal alternation in the \is{numerals}numeral"=plus"=classifier paradigms. ({\dots}) [T]he comparison of the three \ili{Naish} languages Na, \ili{Laze} and Naxi reveals that numerals under 10 can be classified into groups based on their tonal alternations. The numerals 3, 7, 9 and 10 have specific alternations, but \{1, 2\}, \{4, 5\} and \{6, 8\} respectively always have the same tonal patterns. The group \{6, 8\} is particularly significant, as it is the only group of non-contiguous numerals, and both 6 and 8 have final obstruents in \isi{conservative} languages (\ili{Tibetan} \textit{drug} and \textit{brgʲad}, for instance).
	
%	Thus, it can be hypothesized that (i)~although final stops were lost, they were partially transphonologized as tonal contrasts, and (ii)~the development of the classifier system in \ili{Naish} predates the loss of final stops. \citep[143]{jacques.morphology2017}
%\end{quotation}

%“Somewhat paradoxically, in Rgyalrong languages, otherwise known for their polysynthetic and irregular verbal morphology \citep{jackson14morpho, jacques12incorp}, numerals and classifiers present relatively simple and predictable alternations” \citep[135]{jacques.morphology2017}. Jacques's argument is that these alternations are cognate with the \is{numerals}numeral"=plus"=classifier paradigms in Lolo-\ili{Burmese} and \ili{Naish} (cases of \isi{suppletion} found across Burmo"=Qiangic constitute evidence of shared innovations, not parallel innovations) but have been thoroughly simplified by analogical levelling.

%\begin{quotation}
%The  fact that some numerals have two competing prefixal forms (for instance \ipa{kɯmŋu-} vs.\ \ipa{kɯmŋɤ-} for \ipa{kɯmŋu} `five') shows that \isi{analogy} is still synchronically at work in the system, and therefore that a~massive generalization of one particular allomorph is likely to have occurred several times in the history of \ili{Japhug} and other \ili{Rgyalrongic} languages, on the basis of phonological alternations otherwise attested in the language. \citep[147]{jacques.morphology2017}
%\end{quotation}


\section{The influence of bilingualism with Mandarin}
\label{sec:theinfluenceofbilingualismwithchinese}

\is{bilingualism|textbf}
\il{Mandarin|textbf}

Language contact is known to be a~key factor in linguistic
change. An~exemplary illustration of how the study of present"=day contact dynamics can shed light on prosodic systems is the analysis of Kagoshima \ili{Japanese} by \citet{kubozono2007}. The Kagoshima dialect has two prosodic patterns for words: one (Tone A) with a~high tone on the penultimate syllable (i.e.\ a~fall from the penultimate to the last syllable), and the other (Tone B) with a~high tone on the final syllable (i.e.\ no fall in pitch in the course of the word). At the time of study, this dialect was undergoing tonal change through influence from Tokyo \ili{Japanese}, the national standard: words that involve an~abrupt pitch fall in Tokyo tended to be reinterpreted as carrying Tone B, and vice versa. This sheds light on the issue of the tonal or accentual nature of this prosodic system: “the tonal changes in {question} can best be understood if an~accentual analysis of Kagoshima \ili{Japanese} \isi{prosody} is adopted in preference to the traditional tonal analysis” (\citealt[348]{kubozono2007}; supporting evidence from a~follow"=up study of twenty speakers is reported by \citealt{otaetal2016}).

Since the present volume is synchronic in orientation, past
contact between Na, \ili{Tibetan}, Chinese, \ili{Pumi}, \ili{Lisu}, \ili{Naxi} and other
languages will not be investigated (apart from brief remarks in \sectref{sec:thetonegroupasbuildingblockofutterancesanditsroleinconveyinginformationstructure}). Instead, this section focuses
on the current landscape of \isi{language contact}, in which \ili{Mandarin} has, by
far, the leading role. To take the example of the main consultant,
{Mandarin} is the only language other than her mother tongue of which she
has any knowledge.\footnote{Since moving to Lijiang (2010), she has had relatively frequent
	contacts with {Naxi} speakers, however, and this has led to at least one amendment to her Na vocabulary. The {Naxi} are
	referred to by the Na as //\ipa{nɑ˩hĩ\#˥}//, by a~calque of the word structure of the {Naxi} word /\ipa{nɑ˩çi˧}/,
	made up of an~{endonym} which is segmentally identical in both languages (/\ipa{nɑ}/), and of the word for
	‘person, human being’: {Naxi} /\ipa{çi˧}/, Na //\ipa{hĩ˥}//. Initially, the main consultant used this Na pronunciation when
	discussing with {Naxi} people in Lijiang. But to a~{Naxi} listener, the realization /\ipa{nɑ˩hĩ˥}/ does not sound right:
	the {Naxi} do not have nasalization in the syllable for ‘person, human being’. Whether at {Naxi}
	speakers’ suggestion, or through a~spontaneous process of adjustment, she began to refer to the
	{Naxi} as /\ipa{nɑ˩ɕi˥}/, denasalizing the second syllable. This amounts to borrowing the word from
	{Naxi}, instead of calquing it with Na morphemes.}
The guiding principle in focusing on the influence of \ili{Mandarin} is that “extracting as much historical information from clear contact phenomena as possible before attempting greater time depths may be the order of investigation most likely to be fruitful” \citep[485]{souag2010}.

\ili{Mandarin} is a~latecomer to this area. The feudal chieftain of Yongning
spoke Na, and the Na language had a~dominant situation in the plain of
Yongning up until the mid-20\textsuperscript{th} century. There were few (Han) Chinese
migrants to Yongning, and they learnt Na, which was the locally
dominant language, used in the Yongning marketplace by speakers
of other languages, such as \ili{Pumi}, \ili{Lisu}, and \ili{Nosu} ({Nuosu}~/ \ili{Yi}). While
there can be no doubt that the Na language received various influences
in the course of its development, \isi{bilingualism} was not widespread:
speakers of other languages were bilingual in Yongning Na, rather than the
other way round. Numerous Na speakers had very little command of other
languages, or none at all. This situation is somewhat uncommon in
this area, at the border between Sichuan and Yunnan. For instance, the
small community of Na speakers in the neighbouring county of Muli \zh{木里} (Shuiluo \zh{水落} township) are bilingual in \ili{Shixing} (Xumi) and have some command of
\ili{Tibetan} and \ili{Pumi}; and the variety of Na spoken in Guabie \zh{瓜別} has long been
influenced by other languages, in particular \ili{Pumi} and \ili{Nosu}.

While Yongning still preserves the role of a~meeting place and market
place in the eyes of inhabitants of neighbouring mountain villages,
for instance for the \ili{Nosu} and \ili{Pumi} people from small villages
around Yongning \citep[85]{wellens2006}, language shift from Na to
\ili{Mandarin} is now under way in Yongning. All of the Na here have some
command of \ili{Mandarin}. While some members of the community regret the
fact that their language is falling into gradual disuse, proficiency in
\ili{Mandarin}~-- one of the keys to success in society~-- tends to be ranked
far above proficiency in Na. The blending of Na with \ili{Mandarin}, rather than being stigmatized, is accepted with tolerance. This attitude facilitates language
change. While a~pool of \isi{variation} is present at every moment and for
any language, linguistic change in the strict sense requires the
acceptance of innovative, deviant forms by the community of speakers. 

All the level"=tone systems spoken within China are currently subject to the same pressure towards reinterpretation of their tones by {analogy} with those of {Mandarin}. The Yongning Na sociolinguistic scene currently offers interesting opportunities for
studying the effects of \isi{bilingualism}~-- specifically, \textsc{non"=egalitarian bilingualism}, to take
up a~notion from \citet{haudricourt1961b} – in languages with widely different tone systems. The account of Na
tone presented in Chapters~\ref{chap:thelexicaltonesofnouns} to \ref{chap:toneassignmentrulesandthedivisionoftheutteranceintotonegroups} of this volume
makes it clear how much this system differs from that of the \il{Sinitic}Chinese dialects
to which speakers of Na are currently exposed (\il{Mandarin!Southwestern}Southwestern Mandarin
and \il{Mandarin!Standard}Standard Mandarin). As discussed in Chapter~\ref{chap:arealandtypologicaldiscussion}, tones
in Yongning Na are phonetically simple, consisting of three levels and combinations thereof, whereas tones in \ili{Mandarin} are phonetically \is{complex tones}complex. The
tone system of Yongning Na includes some oppositions that are
neutralized when a~word is said \is{form!in isolation}in isolation. When they learn
\ili{Mandarin}, Na speakers come to terms with a~differently
structured tone system: one in which (leaving aside marginal phenomena
of toneless syllables and \isi{tone sandhi}) each syllable has its own tone,
which surfaces as such \is{form!in isolation}in isolation. The discrepancy between the
underlying forms and the \is{form!surface}surface forms of Yongning Na tones makes them
difficult to handle for bilingual
speakers who have more exposure to \ili{Mandarin} than to Na.\footnote{On the effects of bilingualism with Mandarin on the tone systems of other minority languages in rural southwest China, see the two case studies presented by \citet{stanfordandevans2012}. These concern (i)~Sui (Tai"=Kadai family), which has a system of six phonetically complex tones in unchecked syllables and two in checked syllables, and (ii)~Southern Qiang (Sino"=Tibetan family), whose tone system is based on a~binary opposition between H and L levels.}


\subsection{Loss of tone categories that do not surface in isolation}
\label{sec:thelossoftonecategoriesnotreflectedinsurfaceformsinisolation}

Among younger speakers, especially those who went to boarding school, and who predominantly use Southwestern \ili{Mandarin}, there is a~tendency to overlook
the differences that are neutralized \is{form!in isolation}in isolation, such as that between the \mbox{//H\#//} and \mbox{//H\$//} tones. The surface\is{form!surface} tone
pattern of a~word is reinterpreted as its underlying pattern, causing an upheaval in the architecture of the tonal system. For instance, the family name of the main language consultant is realized \is{form!in isolation}in isolation as
/\ipa{lɑ˧tʰɑ˧mi˥}/, and the {associative} form (‘the
Latami clan, the Latami family’) is /\ipa{lɑ˧tʰɑ˧mi˧}=\ipa{ɻ̍˥}/. This reveals that the H tone on the last syllable of /\ipa{lɑ˧tʰɑ˧mi˥}/ is the realization of a~\textit{gliding} H tone (tone category H\$), and the underlying form is //\ipa{lɑ˧tʰɑ˧mi˥\$}//. But if one ignores these alternations, and takes the M.M.H surface tone pattern in /\ipa{lɑ˧tʰɑ˧mi˥}/ at face value, one will interpret the word as carrying a~H tone on its last syllable, i.e.\ as belonging in phonological category \mbox{//H\#//} (//\ipa{lɑ˧tʰɑ˧mi˥\#}//). As a~consequence, when building phrases, e.g.~creating an {associative} form (‘the
Latami clan’) by addition of the \is{suffixes}suffix
/\ipa{=ɻ̍˩}/, the H tone is left sitting on the last syllable of the lexical word, hence $\dagger$\ipa{lɑ˧tʰɑ˧mi˥=ɻ̍˩}. This is what happens in the speech of consultant F5, who is proficient in both Na and \ili{Mandarin}. (She was aged 35 at the time of fieldwork.)

An especially difficult opposition to learn is that between the LM and
LH patterns over disyllables, because it only surfaces when the word
is followed by a~\is{clitics}clitic. For instance, //\ipa{bo˩mi˧}// ‘sow,
female pig’ and //\ipa{bo˩ɬɑ˥}// ‘boar, male pig’ carry the same tones
not only \is{form!in isolation}in isolation (/\ipa{bo˩mi˥}/ and /\ipa{bo˩ɬɑ˥}/) but also in most other contexts. The few
contexts in which their tone patterns are disambiguated are exemplified by
/\ipa{bo˩mi˧=bv̩˧}/ ‘of (a) sow’ vs.\ /\ipa{bo˩ɬɑ˥=bv̩˩}/
‘of (a) boar’, where the H part of the lexical LH pattern
results in the lowering of the following \isi{possessive} particle, to L. Under such circumstances, it does not come as a~surprise that the lexical opposition between LM and LH
should be lost by some speakers, such as F5. In her
speech, the tones of ‘sow’ and ‘boar’ are strictly identical: LM and LH have merged.



\subsection{Simplification of morphosyntactic tone rules}
\label{sec:thesimplificationofmorphosyntactictonerules}

\is{tone rules}

Examining the tables in Chapters \ref{chap:compoundnouns} to \ref{chap:verbsandtheircombinatoryproperties}, the complexity of Na tonal morphosyntax may look
mind"=boggling. But the rules are productive, and the syntactic structures at
issue (subject"=plus"=verb, object"=plus"=verb, \is{compounds}compound nouns{\dots}) are so frequent that they are not
particularly challenging to learn as part of the process of first language acquisition, for children steeped in a~Na linguistic environment. On the other
hand, for a~speaker with limited practice, these combinations become
problematic: they become difficult to remember and apply if one does not practise the language regularly. This holds true of the visiting linguist, as well as of Na speakers below age sixty, many of whom use more Mandarin than Na in daily life. 

Evidence about ongoing language change can be gathered from deviant patterns, which to the linguist
are harbingers of language change. They include a~range of phenomena, from occasional slips of the
tongue to ingrained habits.

Instances of hesitation and of pattern \isi{simplification} can be found in the speech of M23, a~bilingual
language consultant. In subject"=plus"=verb and object"=plus"=verb combinations as well as in compounds, //L// tone, which surfaces as /M/ \is{form!in isolation}in isolation, tends to be neutralized with \mbox{//M//} tone: consultant M23
realizes ‘the sheep came’ as /\ipa{jo˧ {\kern2pt}|{\kern2pt} tsʰɯ˩-ze˩}/ (example (\ref{ex:sheeparrived}) of Chapter~\ref{chap:verbsandtheircombinatoryproperties}), instead of (\ref{ex:sheeparrivedCLASSICAL}). In F4's speech, the M.L.L pattern for ‘the sheep came’ is a~\is{variants}variant condemned as a~slip of the tongue; in M23's speech, it has become the
usual form. 

\begin{exe}
	\ex
	\label{ex:sheeparrivedCLASSICAL}
	\ipaex{jo˩ tsʰɯ˩-ze˩˥}\\
	\gll jo˩	tsʰɯ˩\textsubscript{a}	-ze˧\textsubscript{b}\\
	sheep		to\_come.\textsc{pst}		\textsc{pfv}\\
	\glt ‘the sheep have come’
\end{exe}

Likewise, in M23's speech the combination ‘sheep’s muzzle’ is //\ipa{jo˧-ɲi˧gɤ˧}//, i.e.\ a~simple concatenation of the surface forms of the two nouns,
unlike the \is{conservative (phonological form)}{conservative} pattern found in F4’s speech, shown in (\ref{ex:sheepnose}).

\begin{exe}
	\ex
	\label{ex:sheepnose}
	\ipaex{jo˩-ɲi˩gɤ˩˥}\\
	\gll jo˩	ɲi˧gɤ˧\\
	sheep		nose/muzzle\\
	\glt ‘sheep's muzzle’
\end{exe}

In the \is{conservative (phonological form)}{conservative} form, (\ref{ex:sheepnose}), the tone of the \is{compounds}compound is phonologically identical with the lexical L tone of the determiner, //\ipa{jo˩}// ‘sheep’. This yields //\ipa{jo˩-ɲi˩gɤ˩}// ‘sheep’s muzzle’, surfacing
as /\ipa{jo˩-ɲi˩gɤ˩˥}/ following the post"=lexical addition of a~final H tone, due to Rule~7: all-L
tone groups are not allowed in Yongning Na; if a~\isi{tone group} only contains L tones, a~post"=lexical H tone is added to its last
syllable. A speaker needs a~good command of the grammar of Na to implement the \is{conservative (phonological form)}{conservative} tone pattern in
this \is{compounds}compound. The L tone of ‘sheep’, which does not
even surface \is{form!in isolation}in isolation, has the effect of imposing itself onto no fewer than three syllables in
succession in the \is{compounds}compound, overriding the lexical tone of the head noun. The realization of ‘nose, muzzle’ \is{form!in isolation}in isolation is /{\kern2pt}\ipa{ɲi˧gɤ˧}/; in (\ref{ex:sheepnose}), it is changed to /{\kern2pt}\ipa{ɲi˩gɤ˩}/. Such tonal processes are alien to \ili{Mandarin}. In light of this discrepancy, it is understandable that some less proficient
speakers of Yongning Na who are exposed to \ili{Mandarin} on a~day"=to"=day basis should come to have hesitations, and should (occasionally or regularly) go for a~simple succession of the tones as they surface in
isolation, as is the case in their second language (\ili{Mandarin}), instead of applying the rules of Yongning Na tonal grammar.

These examples illustrate the complexity of phenomena of language \isi{variation} and change. As mentioned above, a~change can
constitute a~\isi{simplification} from one point of view and a~complexification from other points of
view. Saying /\ipa{jo˧ {\kern2pt}|{\kern2pt} tsʰɯ˩-ze˩}/ for ‘the sheep arrived’, with a M.L.L tone pattern instead of the conservative L.L.L pattern shown in (\ref{ex:sheeparrivedCLASSICAL}), can be seen as a~\isi{simplification}
insofar as the subject bears the same tone as \is{form!in isolation}in isolation. But it is a~complexification insofar as it
increases the frequency of occurrence of contexts in which the lexical L tone of ‘sheep’ does not
\is{form!surface}surface, making it more difficult for language learners to arrive at the identity of this word’s
lexical tone.


\subsection{Straightening out irregular tone patterns}
\label{sec:thestraighteningoutofirregulartonepatterns}

In addition to losing some tone categories, less proficient speakers tend to regularize irregular
patterns, for want of having memorized the exceptions. For instance, the word for ‘powder, flour’ is
/\ipa{tsɑ˧bɤ˧}/, with M tone. According to the synchronic rules that govern the tone patterns of \is{compounds}compound nouns,
the combination of this word with /\ipa{lv̩˧mi˧}/, ‘stone’, should yield a~simple M-tone output,
$\dagger$\ipa{lv̩˧mi˧-tsɑ˧bɤ˧}. (The \is{compounds}compound means ‘fine sand’.) However, in the speech of the older generation of
speakers, compounds involving /\ipa{tsɑ˧bɤ˧}/,
‘powder, flour’, are irregular: they all carry L tone on their second half (the head noun ‘powder, flour’). They belong in a~set of expressions referred to as ‘\ili{Tibetan} compounds' in \sectref{sec:thenounflourpowder} and \sectref{sec:anindependentsetoffactscompoundgivennamesandtermsofaddress}: compounds involving \ili{Tibetan} loanwords, and whose second part systematically receives L tone. Examples were shown in \tabref{tab:powder} and \tabref{tab:Names}. This irregularity is lost in the speech of less proficient speakers, who realize these compounds with a~M.M.M.M tonal string
(phonological analysis: /M/), as $\ddagger${\kern2pt}\ipa{lv̩˧mi˧"=tsɑ˧bɤ˧} ‘fine sand’,
$\ddagger${\kern2pt}\ipa{qʰɑ˧dze˧"=tsɑ˧bɤ˧} ‘sweetcorn flour’, $\ddagger${\kern2pt}\ipa{dze˧ɭɯ˧"=tsɑ˧bɤ˧} ‘wheat flour’, and so on.


\subsection{Cases where MH tone fails to unfold: Towards a~syllable"=tone system?}
\label{sec:caseswheremhtonefailstosplitintotwolevelshintofapotentialforevolutiontowardsasyllabletonesystem}

\ili{Naxi}, a~close relative of Yongning Na, has few phenomena of tone change. In the A-sher dialect, the
reduction of a~morpheme carrying H tone results in reassociation of this tone to the syllable to its
left \citep{michaud2006d,michaudetal2007d}. Informal observations and exchanges with
\ili{Naxi} speakers living in the city of Lijiang suggest that even these simple instances of tone change
are disappearing from Lijiang \ili{Naxi}. For instance, the conditional is /\ipa{se˥}/ in A-sher, but this syllable is preceded by a~\is{floating tone}floating H tone (its full form can be transcribed as //\ipa{˥}~\ipa{se˥}//). This is presumably the historical
product of the reduction of disyllabic *\ipa{ɭɯ˥~se˥}. The phenomenon is still reported in
a~dictionary compiled from 1995 to 2012: “This word is fairly unique in that it triggers the mid
or low tone of the preceding syllable to become a~rising tone”
\citep[337]{pinsonetal2012}. Impressionistic observations made in Lijiang around 2010 suggest that this
morpheme becomes simplified to /\ipa{se˥}/ in the speech of younger speakers. It is not at all
unlikely that increasing familiarity with \ili{Mandarin} is exerting an~influence on \ili{Naxi}, leading to the
reinterpretation of its tones as units attached to the syllable, rather than levels that can combine
among themselves.

Of the four tones of \ili{Naxi}, High, Mid, Low and Rising, the last appears especially revealing in this
respect. It is clearly an \is{innovative (phonological form)}{innovation} which emerged in a~system that contained three
levels: High, Mid and Low. The following historical scenario can be proposed: rising contours appeared in the \ili{Naxi} language
through processes of syllable reduction, and became lexicalized on some words. Lexicalized instances of rising tones paved the way
for the assignment of a~/LH/ tone sequence to \ili{Mandarin} words with rising tone. At that point, borrowings from \il{Mandarin!Southwestern}Southwestern Mandarin consolidated this marginal lexical tone category by giving it considerable lexical development. In recent decades, bilingualism with \ili{Mandarin} gradually tilted the perception of \ili{Naxi} tones
towards the syllable"=tone type, to the point that it is now an~issue whether the rising tone of \ili{Naxi}
is to be analyzed as a~combination of levels (L+H, or M+H) or as a~phonological unit (like contours in \ili{Mandarin}). It is unclear to what extent \ili{Naxi} speakers, most of whom are
highly proficient in Southwestern \ili{Mandarin}, keep the tone systems of \ili{Naxi} and \ili{Mandarin} cognitively
distinct. To venture a~hypothesis, the lexical
rising tone of \ili{Naxi} currently seems to behave essentially like an~indecomposable {contour}.

In Yongning Na, there is plentiful synchronic evidence that contours are to be analyzed as sequences of
levels. There are nonetheless some weak hints of a~tendency for MH contours to be treated as units
associated to one syllable. When saying MH"=tone \is{monosyllables}monosyllabic nouns in the frame ‘This is~{\dots}’,
consultant F4 occasionally produced variants with a~rising {contour} on the target noun. Examples are shown in (\ref{ex:isasheep}); the standard realization is provided in (\ref{ex:isasheepcorrect}).

\begin{exe}
	\ex
	\label{ex:isasheep}
	\ipaex{$\ddagger${\kern2pt}ʈʂʰɯ˧ {\kern2pt}|{\kern2pt} tsʰɯ˧˥ ɲi˩.{\kern3pt}{≈}{\kern3pt}$\ddagger${\kern2pt}ʈʂʰɯ˧ {\kern2pt}|{\kern2pt} tsʰɯ˧˥ ɲi˥.}\\ 
		\gll ʈʂʰɯ˥		tsʰɯ˧˥	ɲi˩\\
		\textsc{dem.prox}	goat	\textsc{cop}\\
		\glt ‘This is a sheep.’ 
\end{exe}

\Hack{\newpage}

\begin{exe}
	\ex
	\label{ex:isasheepcorrect}
	\ipaex{ʈʂʰɯ˧ {\kern2pt}|{\kern2pt} tsʰɯ˧ ɲi˥.}\\ 
	\gll ʈʂʰɯ˥		tsʰɯ˧˥	ɲi˩\\
	\textsc{dem.prox}	goat	\textsc{cop}\\
	\glt ‘This is a sheep.’
\end{exe}

When her attention was
drawn to these discrepancies, the consultant said: /\ipa{ɖɯ˧-bæ˧ lɑ˧ ɲi˥}/, “It’s just the
same”. Intonationally, realization of the MH \is{tonal contour}contour on the syllable to which it is lexically
attached tends to happen when the word is emphasized.

This tendency surfaces here and there in the recordings. An example is the phonetic realization of /\ipa{ʈʂʰæ˧-pɤ˥to˩}/ ‘even a~deer’ as
[\ipa{ʈʂʰæ˧˥-pɤ˥to˩}] (in the recording NounsEven.7), with (i)~a~H tone on [\ipa{pɤ˥}], due to reassociation of the H part of the
MH \is{tonal contour}contour of /\ipa{ʈʂʰæ˧˥}/ ‘deer’, as expected, and (ii)~a~MH {contour} on [\ipa{ʈʂʰæ˧˥}], due to
incomplete dissociation of the H part of its phonological MH \is{tonal contour}contour.

This is only a~weak tendency, however. It by no means warrants the conclusion that Yongning Na is on its
way towards adopting a~syllable"=tone system. A~detailed cross"=linguistic phonetic study would be
necessary to determine to what extent such tendencies are present among the world’s level"=tone
systems. Such a~study might reveal that Yongning Na is not at all exceptional in this respect. The
computation of tone sequences is not a~mechanical process, and slips of the tongue whereby
a~H tone does not fully dissociate from the syllable to which it is lexically attached may not come
as a~surprise to linguists with an~experience of level"=tone systems. This synchronic tendency
appeared well worth mentioning, however, in relation to the influence currently exerted by {Mandarin}.


\subsection{A topic for future research: Influence of language contact on intonation}
\label{sec:presentdaysociolinguisticsituationofyongningnacontactwithacomplextonesystemmandarinchinese}
\label{sec:presentdaysociolinguisticsituationeffectsontone}
\label{sec:presentdaysociolinguisticsituationeffectsonintonation}

Intonation is especially subject to carry"=over from one language to another in the speech of bilinguals. A~striking example involving \ili{Vietnamese} speakers in a~\ili{French}"=speaking environment is reported by \citet[401]{dungetal1998}. In the case of Yongning Na, a~full"=fledged study of \isi{intonation} should include a~description and analysis of the \isi{intonation} of Na speakers when using {Mandarin}, comparing it with their \isi{intonation} when using Na, and examining patterns of interaction. This is not an~easy topic. The main consultant only uses {Mandarin} reluctantly and hesitantly: she lacks confidence and feels awkward using this language. Her own evaluation is that she will never be able to speak “proper Chinese'', and will make do with “pig Chinese'' until her last breath. A~complicating factor is that the variety of Chinese to which she was occasionally exposed until 2000 was Southwestern \il{Mandarin!Southwestern}Mandarin, but since the year 2000 she has had regular exposure to Standard \il{Mandarin!Standard}Mandarin from listening to television. The tone systems of these two dialects of {Mandarin} are almost identical phonologically, but the four tones' phonetic templates differ enough between the two dialects to complicate accommodation to the \isi{intonation} patterns. Consultant F4 does her part in dialogues with {Mandarin}"=speaking relatives and acquaintances, but recording these exchanges and establishing a~transcription together with the participants would have run counter to our collaboration's implicit focus on her mother tongue, Yongning Na. The consultant's comfort zone was respected, and no pieces in {Mandarin} were recorded. 

As a~(lame) consolation for not being able to offer data on this topic, here is an example from another language showing that \isi{language contact} can exert a~strong influence on \isi{intonation}. Wolof, one of the nontonal languages of the Niger"=Congo family, has been described as having typological peculiarities such as the absence of any intonational marking of focus and the optional nature of the division of sentences into \isi{intonation} groups \citep{riallandetal2001}. Information structure is conveyed by verbal morphology: there are three “nonfocusing conjugations'' and three “focusing conjugations''; the latter “vary according to the syntactic status of the focused constituent: subject, verb, or complement (in the wide sense of any constituent that is neither subject nor main verb)'' \citep[895]{riallandetal2001}. As Wolof ascends “from its origins in the heartland of Senegal to the status of urban vernacular and national \textit{lingua franca}'' \citep[142]{mclaughlin2008}, it is acquired as a~second language by speakers of many other languages. Informal discussion with a~speaker of Wolof who also speaks some Bambara suggests that, in her speech, focus is clearly marked intonationally: pitch is raised on the focused constituent `Peer' (a proper name) in (\ref{ex:woloffoc}), as compared with (\ref{ex:wolofnonfoc}), where it is not under focus. 

{\largerpage}

\begin{exe}
	\ex
	\label{ex:woloffoc} % :)
	\gll {\dots} Peer moo ko lekk\\
	{} Peer \textsc{subjemph.3sg} \textsc{opr} to\_eat\\
	\glt ‘It was Peer who ate it.’ (\textsc{subjemph} stands for \textit{subject-emphatic}, and \textsc{opr} for \textit{object pronoun}.)
\end{exe}

\begin{exe}
	\ex
	\label{ex:wolofnonfoc} % :)
	\gll {\dots} Peer lekk na\\
	{} Peer to\_eat \textsc{pft.3sg}\\
	\glt ‘Peer has eaten.’ (No focused constituent.)
\end{exe}

This testifies to the fact that carry"=over of \isi{intonation} patterns from Bambara to Wolof can take place for bilingual speakers. Such situations can have far-reaching consequences for the evolution of the \isi{intonation} system.