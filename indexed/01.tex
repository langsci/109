\chapter{Introduction}
\label{chap:introduction}
\label{chap:1}

%\epigraph{The most important task in linguistics today~-- indeed, the only really important task~-- is to get out into the field and describe languages, while this can still be done.}{\citep[144]{dixon1997}}

The aim of this book is to provide a~description and analysis of the tone system of
Yongning Na (also known as Mosuo), a~Sino"=Tibetan language spoken in Southwest China.

The richness of this system is immediately apparent when one examines a~sentence. Example (\ref{ex:ihavetogoandtakemyluggagenow}) is the
first one that I transcribed: I had just arrived at my future teacher’s house; my luggage had
been left at someone’s house along the main road, some fifty meters from the house. I asked my
teacher’s son, who can speak fluent Mandarin, to translate an~explanation for me: “I have brought
a~lot of stuff; I have to go back [to the main road] and pick it up now”. This yielded (\ref{ex:ihavetogoandtakemyluggagenow}). Later I
elicited (\ref{ex:ihavetogoimafraidihavetoleave}) as a~simpler form.

\begin{exe}
  \ex \label{1}
  \begin{xlist}
    \ex
    \label{ex:ihavetogoandtakemyluggagenow}\label{1a}
    \gll njɤ˧	ʑi˩	bi˩	-zo˩	-ho˥.\\
    \textsc{1sg}	to\_take	to\_go	\textsc{obligative}	\textsc{desiderative}\\
    \glt ‘I have to go and take [my luggage] now.' (Field notes, 2006)

    \ex
    \label{ex:ihavetogoimafraidihavetoleave}\label{1b}
    \gll	njɤ˧	bi˧	-zo˧	-ho˩.\\
    \textsc{1sg}	to\_go	\textsc{obligative}	\textsc{desiderative}\\
    \glt ‘I have to go. / I’m afraid I have to leave.' (Field notes, 2006)
  \end{xlist}
\end{exe}

The difference in the lexical tone on the main verb (in \ref{ex:ihavetogoandtakemyluggagenow}: /\ipa{ʑi˩}/
‘to take’; in \ref{ex:ihavetogoimafraidihavetoleave}: /\ipa{bi˧}/ ‘to go’) is reflected in the tones of the following syllables, all the way to the end of the
sentence.

This book proposes an~analysis of the underlying system: lexical tone categories, phonological rules that apply inside tone groups, and combination rules that apply in various types of phrases (i.e.\ \textit{morphotonological} rules). As a~preview of the results concerning lexical tones,
(\ref{ex:ihavetogoandtakemyluggagenow}--\ref{ex:ihavetogoimafraidihavetoleave}) are provided below
(as \ref{ex:ihavetogoandtakemyluggagenow2}--\ref{ex:ihavetogoimafraidihavetoleave2}) with
morpheme"=level transcriptions indicating lexical tone by means of tone symbols supplemented by
subscript letters \textsubscript{a} \textsubscript{b} \textsubscript{c} to distinguish {subcategories} of lexical tones. The following chapters provide full details on this system, and describe how the tones of entire sentences obtain from the lexical tones. 

% \Hack{\newpage}

\begin{exe}
\ex
\begin{xlist}
\ex
\label{ex:ihavetogoandtakemyluggagenow2}
\ipaex{njɤ˧ ʑi˩ bi˩-zo˩-ho˥.}\\
\gll njɤ˩ 	ʑi˩\textsubscript{a}		bi˧\textsubscript{c}	-zo˧\textsubscript{a}		-ho˩\\
\textsc{1sg}	to\_take		to\_go	\textsc{obligative}	\textsc{desiderative}\\
\glt ‘I have to go and take [my luggage] now.' 

\ex
\label{ex:ihavetogoimafraidihavetoleave2}
\ipaex{njɤ˧ bi˧-zo˧-ho˩.}\\
\gll njɤ˩ 	bi˧\textsubscript{c}	-zo˧\textsubscript{a}		-ho˩\\
\textsc{1sg}	to\_go	\textsc{obligative}	\textsc{desiderative}\\
\glt ‘I have to go. / I’m afraid I have to leave.'
\end{xlist}
\end{exe}

To set the stage for these analyses,
\sectref{sec:presentationofthenalanguageandnasocietyandreviewofearlierstudies} presents the Na language and provides a~review of Na language studies. (Historical and ethnological perspectives are presented in Appendix B.) \sectref{sec:chronologyofthestudyelicitationproceduresandonlinematerials} sets out the research programme behind the present study, and presents the
language consultants, the data elicitation methods, and the online documentation available on this
language. A~quick grammatical sketch is provided in \sectref{sec:sketch}, presenting general properties of the language such as basic word order and the structure of {noun} and {verb} phrases, which serve as the backdrop to the discussion of morphotonology in the following chapters. 


\section{The Na language}
\label{sec:presentationofthenalanguageandnasocietyandreviewofearlierstudies}


\subsection{Endonym and exonyms}
\label{sec:endoexo}

Yongning Na is a~Sino"={Tibetan} language spoken in Southwest China, astride the border between the provinces of Yunnan and Sichuan, at a~latitude of
27{\textdegree}50’~N and a~longitude of 100{\textdegree}41’~E. Speakers of the language refer to it as /\ipa{nɑ˩ʐwɤ˥}/: ‘Na language’. The structure of this {compound}, made up of a~{noun} and a~verb, is shown in (\ref{ex:nalang}).

\begin{exe}
	\ex
	\label{ex:nalang}
	\ipaex{nɑ˩ʐwɤ˥}\\
	\gll nɑ˩˧		ʐwɤ˩\textsubscript{b}\\
	\isi{endonym}:~Na			to\_speak\\
	\glt ‘the language of the Na’, i.e.\ ‘the Na language’
\end{exe}


The name ‘Yongning Na’ was coined by Liberty \citet{lidz2006} by associating the people's \isi{endonym} with the name of the place where the language is spoken: 
% Map~\ref{map:1-1} locates this language on a~map of Asia showing the current geographical distribution of \il{Sino-Tibetan}Sino"=Tibetan languages.\footnote{Issues of language classification are addressed in \sectref{sec:thepositionofnaandnaxiwithinsinotibetan}.} 
the plain of Yongning \zh{永宁}, a~basin located close to Lake Lugu \zh{泸沽湖}, a~lake of about fifty square kilometres (see Map~\ref{map:1-1}). The lake creates a~microclimate that is suitable for farming despite the high altitude (about 2,800 meters above sea level).

\begin{mapfigure}[p!]
	\caption{A sketch map of the Yongning area. \textit{Designed by Jérôme Picard. Sources: Geofabrik, ASTER GDEM (a product of METI and NASA) and OpenStreetMap.}}
%	\includegraphics[width=.8\textwidth]{figures/map/27mai.jpg}
%	\includegraphics{figures/map/PNG_CMYK.png} % for test only: rasterized image
% \includegraphics{figures/map/PNG_Grayscale.png} % for test only: rasterized image
\includegraphics{figures/map/PDF_CMYK_1point4.pdf} % vector image for hardcopy
%\includegraphics{figures/map/PDF_Grayscale_1point4.pdf} % vector image for softcopy
%\includegraphics{figures/map/PDF_RGB_1point4.pdf} % vector image for online PDF

%	\includegraphics{figures/map/YongningMap.jpg}

	%	\includegraphics[width=.95\textwidth]{figures/ch1mapSN.pdf}
	\label{map:1-1}
\end{mapfigure}

%\begin{mapfigure}[t]
%	\caption{The distribution of Sino"=Tibetan languages. \textit{Source: Glottolog 2.7.}}
%	\includegraphics[width=\textwidth]{figures/map/Glottolog.jpg}
%	%	\includegraphics[width=.95\textwidth]{figures/ch1mapSN.pdf}
%	\label{map:1-1}
%\end{mapfigure}

%\hyperref[fig:map]{see map}

\begin{photofigure}[t]
	\caption{The plain of Yongning seen from the West, with Gemu Mountain (in~Na: \ipa{kɤ˧mv̩˧˥}) in the background. The Lake is behind the pass on the right"=hand side. Autumn 2006.}
	\includegraphics[width=\textwidth]{figures/YongningPlain08375.jpg}
\end{photofigure}

Yongning Na appears in the Glottolog database (glottolog.org) under the code \textit{yong1270}. The Ethnologue language code is NRU, an acronym for ‘Narua’, a~romanization of the expression shown in (\ref{ex:nalang}): /\ipa{nɑ˩ʐwɤ˥}/ ‘Na language’. The number of speakers is estimated at 47,000 in the Ethnologue database, based on the Summer Institute of Linguistics' own sources \citep{lewisetal2016}.\footnote{The figure of 47,000 speakers includes people who do not use the name /\ipa{nɑ˩ʐwɤ˥}/ to refer to their native language. This is a~drawback of the name ‘Narua’, proposed in a~report to the Summer Institute of Linguistics asking for a~language code distinct from {Naxi}. By contrast, the Glottolog inventory of languages adopts the name ‘Yongning Na’, based on principles such as that (i)~“language names (like city names) are loanwords, not code-switches” \citep{haspelmath2017}, a~principle which leads to favour ‘Na language’ over its equivalent in Na: /\ipa{nɑ˩ʐwɤ˥}/ ‘Na language’, romanized as ‘Narua’, and (ii)~“language names may have a modifier"=head structure”, so that the name ‘Yongning Na’ is interpreted with the intended meaning of ‘Yongning variety of the Na language’.} Ethnonymy reflects the high degree of ethnic, cultural and linguistic intricacy of the Sino"=Tibetan borderlands \citep{gros2014b}. \tabref{tab:thenamesofthenaendonymsandexonyms} presents (i)~two endonyms, (ii)~the name by which the \ili{Naxi} \zh{纳西} (a~closely related ethnic group) refer to the Na, and (iii)~a~Chinese exonym found in various sources, under various avatars, for close to two thousand years, and which currently enjoys renewed favour for reasons discussed in Appendix B (\sectref{sec:ethnicclass}). 

%%test
%\begin{figure}
%	% [t] to place at top; here: full page
%%	\includegraphics[width=\textwidth]{figures/map/ExportCarteMichaudjpg_14avril_ROGNEE_vuAM.jpg}
%	\includegraphics[width=\textwidth]{figures/map/mapPLACEHOLDER.jpg}
%	\caption{A sketch map of the Yongning area. \textit{Designed by Jérôme Picard. Sources: Geofabrik, ASTER GDEM (a product of METI and NASA) and OpenStreetMap.}}
%	\label{fig:map}
%\end{figure}


\begin{sidewaystable}[p]
	\caption{The names of the Na: endonyms and exonyms.}
	{\renewcommand{\arraystretch}{1.35}
		\begin{tabularx}{\textheight}{ Q l@{\hspace{5mm}} P{42mm} P{42mm} P{37mm} }
			\lsptoprule
			transcription & language	& romanized equivalents &	Chinese equivalents & meaning\\ \midrule
			\ipa{nɑ˩˧}	& Na	& Na \citep{cai1997,lidz2010} &  \textit{Nà} \zh{纳} \citep{yang2006} & ‘black’\\
			\ipa{ɬi˧-hĩ˧} & 	Na	& Hli"=khin \citep{rock1963}, Hli"=hing \citep{shih1993}	& \textit{Lǐxīn} \zh{里新}
			\citep[15]{shih2008} & ‘People of the Centre’\\
			\ipa{ly˧-çi˧}	& Naxi & 	Lü"=khi \citep{rock1963}	& \textit{Lǚxī} \zh{吕西} \citep[8]{guoetal1994} & as above: ‘People of the Centre’\\
			\textit{origin not established yet}	& Chinese &  Moso
			\citep{cordier1908,nishida1985,shih1993,mckhann1998,luo2008}, Mo"=So, Mosuo \citep{knodel1995} & \textit{Móshā} \zh{摩沙}, \textit{Móxiē} \zh{磨些}, \textit{Móxiē} \zh{麽些}, \textit{Móxiē} \zh{摩些}, \textit{Mósuō} \zh{摩娑}, \textit{Mòxiē} \zh{末些}, \textit{Móhuò} \zh{磨获}, \textit{Mòsuān} \zh{莫狻}, \textit{Mósuō} \zh{摩梭}	&
			\textit{not established yet}\\ \lspbottomrule
		\end{tabularx}}
		\label{tab:thenamesofthenaendonymsandexonyms}
	\end{sidewaystable}
	
The most likely interpretation of the \isi{endonym} /\ipa{nɑ˩˧}/ is that it means ‘black’. Use of ethnonyms meaning ‘black’ or ‘white’ is widespread in the area; in Yongning, the Na coexist with the \ili{Pumi} \zh{普米}, who call themselves /\ipa{ʈʰóŋmə}/ ‘white people’. 

%\begin{quotation}
%	In Southwest China, there is no noticeable difference between the characteristic skin colors or facial features of different ethnic groups; \isi{variation} within groups is at least as broad as \isi{variation} between groups.~({\dots}) Occasionally someone will comment that people of one ethnic group or another might be shorter or darker or have curlier hair than another, but these are not the important distinctions; the important distinctions are linguistic and cultural. So terms like “White Miao” or “Black \ili{Yi}” or “Red Lahu” do not refer to the phenotypic characteristics of their bodies, but rather to the things they wear. 
%\end{quotation}

\begin{quotation}
The designation \ipa{ʈʰóŋ} ‘white’ sets the {Pumi} apart from some surrounding ethnic groups whom they designate as \ipa{nʲæ̌} ‘black’: the \ipa{ɡoŋnʲæ̌} ‘{Nuòsū} (Yí) \zh{彝}’ (‘black skin’) and the \ipa{nʲæmə̌} ‘Na (Mósuō) \zh{摩梭}’ (‘black person’). \citep[2]{daudey2014}
\end{quotation}

%Command \noindent added to avoid having an indent. Proofreader suggestion: since this sentence continues the argument, it is better not to indent. 
{\noindent}Among the \ili{Yi} \zh{彝} (formerly known as ‘{Lolo}’), there is a~distinction between ‘black’ and ‘white’ castes. “The {Nasoid} groups are also known as Black {Lolo}, and the assimilated groups connected with them~-- either {Nasoid} groups who have become Sinicized, or others who have become assimilated to the {Nasoid} groups, often by capture or conquest~-- are called ‘white’ to denote the fact that they do not ‘fit’ in the {Nasoid} clan structure” \citep[53]{bradley1979}.

\begin{quotation}
In northeastern Yunnan and western Guizhou, the designation \textit{Nasu} (the black ones) refers to a~group of 
\ili{Yi} who were the overlords of a~series of feudal kingdoms between the 9\textsuperscript{th} and the 20\textsuperscript{th} centuries;
they were often contrasted to other, subordinate groups who referred to themselves as white.  In the Liangshan region of southwestern Sichuan, on the other hand, Black bones (\textit{Nuoho}, called Black \ili{Yi} in 
Chinese), and White bones (\textit{Quho}, called White \ili{Yi} in Chinese), refer to the aristocratic and commoner castes into which the society is divided~-- the term \textit{nuo}, or ‘Black' also means ‘heavy', ‘important', or ‘serious'. At the same time, the aristocratic caste is also associated with darker colored clothing ({\dots}).
%aristocratic women often dress entirely in black. 
In this case, it appears that historically the color of the clothing is derived from
the color name given to the people, rather than the other way around. What is most important here is to realize that the association of people and colors in this region has little or nothing to do with the imagined color of the people themselves, but rather is part of a~complex symbolic system that both reflects and is reflected in the styles and colors of people's clothing. \citep[102]{harrell2009}
\end{quotation}

The name of Yongning in Na is /\ipa{ɬi˧di˩}/, interpreted by \citet[23]{shih2010} as ‘the peaceful land’, relating it to the verb /\ipa{ɬi˥}/ ‘to rest, to relax'. This \isi{folk etymology} fits nicely with the author’s celebration of Na society's ideals of harmony (the title of the volume is \textit{Quest} \textit{for} \textit{harmony}:
\textit{The} \textit{Moso} \textit{traditions} \textit{of} \textit{sexual} \textit{union}
\textit{and} \textit{family} \textit{life}). But phonetic correspondences with \ili{Naxi} do not support this analysis, and demonstrate instead that the Na name of Yongning, /\ipa{ɬi˧di˩}/, means ‘the central land, the heartland’. The linguistic argument is as follows.

Yongning is called /\ipa{ly˧dy˩}/ in \ili{Naxi} \citep[201]{heetal2011}. This cannot be a~recent borrowing from Na, because Na does not have a~rounded close front vowel /\ipa{y}/. If the present form of the Na word were borrowed into \ili{Naxi}, /\ipa{ɬi˧di˩}/ would be interpreted as /\ipa{li˧di˩}/ by Naxi ears, with a~straightforward correspondence for /\ipa{d}/ and /\ipa{i}/ (which are present in both languages) and a~reinterpretation of Na /\ipa{ɬ}/ as Naxi /\ipa{l}/ in the absence of an unvoiced lateral in Naxi. The presence of the vowel /\ipa{y}/ in \ili{Naxi} /\ipa{ly˧dy˩}/ strongly suggests that the word is cognate with Na. It could be a~calque (a~root"=for"=root translation from Na to \ili{Naxi}, by a~bilingual speaker who was able to interpret the Na word), but it is not a~phonetic \is{loanwords}loanword.

The noun's second syllable is easily analyzed: it means ‘earth, place, land' (Na: /\ipa{di˩˥}/, \ili{Naxi}: /\ipa{dy˩}/), a~root found in many place names in both languages: it is used as a~locative nominalizer \citep[559]{lidz2010}. As for the first syllable, in view of Na data alone it could have a~number of different interpretations. It could indeed be related to the verb ‘to rest', /\ipa{ɬi˥}/, as in the \isi{folk etymology} of Yongning as ‘the land of rest, the peaceful land’ adopted by Shih Chuan"=kang.\footnote{The main language consultant reports this {folk etymology} in the document FolkEtymology, available online.
%	 This recording had not yet been transcribed and translated at the time of publication of the present volume.
} 
	 But it could equally be related to ‘moon', /\ipa{ɬi˧}/ (as in the disyllable /\ipa{ɬi˧mi˧}/ ‘moon'); to ‘ear', /\ipa{ɬi˧}/ (as in /\ipa{ɬi˧pi˩}/ ‘ear'); to ‘middle', /\ipa{ɬi˧}/ (as in /\ipa{ɬi˧gv̩˧}/\footnote{For the sake of simplicity, this noun is provided here in surface phonological transcription. Its underlying form is //\ipa{ɬi˧gv̩\#˥}//, with a~{floating} High tone. This tonal category is analyzed in \sectref{sec:afloatinghtonewithcomparativeevidencepointingtoitsorigin}.} ‘middle part'); or to ‘Bai' (an ethnic group), through truncation of the disyllable /\ipa{ɬi˧bv̩˧}/ ‘Bai'. Any of these roots combined with  /\ipa{di˩˥}/ ‘earth, place' would yield the form /\ipa{ɬi˧di˩}/ by application of regular tone rules. 

Moreover, the search needs to be extended further in view of the existence of some words that are irregular in terms of their tone patterns: the tones of some disyllabic words do not correspond to the tones of their two monosyllabic roots as expected in view of synchronically {productive} rules (as explained e.g.\ in \sectref{sec:exceptionalitems} and \sectref{sec:htoneroots}). This suggests that one needs to relax tonal constraints when searching for the origin of the first syllable of the name /\ipa{ɬi˧di˩}/. Widening the search to /\ipa{ɬi}/ roots of all tone categories yields the following additions to the list of possible origins for the first syllable in the name /\ipa{ɬi˧di˩}/ ‘Yongning': the nouns ‘roebuck' (/\ipa{ɬi˩}/), ‘turnip' (in the disyllable /\ipa{ɬi˩bi˩}/), ‘trousers' (in /\ipa{ɬi˩qʰwɤ˩}/), and ‘wrath, anger' (in /\ipa{ɬi˩ʁɑ˩}/), as well as the verbs /\ipa{ɬi˩}/ ‘to measure' and /\ipa{ɬi˧˥}/ ‘to dry in the sun'. Language-internal evidence thus allows for a~broad range of hypotheses: is /\ipa{ɬi˧di˩}/ ‘the peaceful land, the land of rest', ‘the land of the moon', ‘the land of ears', ‘the land of the middle', ‘the land of the Bai people', ‘the land of the roebuck', ‘the land of turnips', ‘the land of trousers', ‘the land of wrath', ‘the land of measurements' or ‘the land of sun-drying'? It would be unwise to exclude some of these possibilities on grounds of semantic implausibility: a~study of place names in various languages of China \citep{yangliquan2011} confirms the great extent of toponymic creativity.

The decisive evidence comes from comparison with \ili{Naxi}. Of all the above possibilities, only one is supported by the existence of a~cognate in \ili{Naxi}. The root /\ipa{ly˥}/ means ‘centre’ in \ili{Naxi}, as does /\ipa{ɬi˧}/ in Na. This leads to an interpretation of the name /\ipa{ɬi˧di˩}/ (and of \ili{Naxi} /\ipa{ly˧dy˩}/) as ‘the central land, the heartland’. This interpretation can then be passed on to historians and anthropologists; it makes excellent sense in view of Yongning's geographic position, and of the role of the Yongning area in the history of {Naish} peoples (about which see Appendix B), but the crucial evidence is linguistic, relying on the \is{comparative method (historical linguistics)}historical"=comparative method. In Na, /\ipa{ɬi˧di˩}/ can have more than ten different interpretations; likewise, in \ili{Naxi}, /\ipa{ly˧dy˩}/ could be given various etymologies, such as ‘land of grain’, ‘land of Asian crabapple, \textit{Malus asiatica}’, ‘central land', ‘land of watchfulness', or ‘quaking land, trembling land'. It is through looking for matches between the Na and \ili{Naxi} words (technically known as \textit{cognate words}), and examining their phonetic correspondences, that the final result can be arrived at.

‘The Centre, the Central land’, /\ipa{ɬi˧di˩}/, is an~apt designation from the point of view of linguistic richness, as most of the diversity of the \ili{Naish} language group (the lower"=level subgrouping to which Na belongs: see \sectref{sec:dialectclassificationyongningnaandnaxi}) is found in and around the plain of Yongning, within a~radius of less than a~hundred kilometres. The name ‘People of the Centre’, /\ipa{ɬi˧-hĩ˧}/,\footnote{For the sake of simplicity, this noun is provided here (and also in \tabref{tab:thenamesofthenaendonymsandexonyms}) in surface phonological transcription. Its underlying form is //\ipa{ɬi˧-hĩ\#˥}//, with a~{floating} High tone. This tonal category is analyzed in \sectref{sec:afloatinghtonewithcomparativeevidencepointingtoitsorigin}.} refers to the
inhabitants of the plain of Yongning. Ironically, this name is not in common use in the dialect under study here, which is located squarely inside the Yongning plain, whereas it is still used by a~community of speakers who moved from Yongning to the peripheral region of Shuiluo \zh{水落} (in the neighbouring county of Muli \zh{木里}) several centuries ago.


\subsection{Dialect classification and issues of phylogeny}
\label{sec:dialectclassificationyongningnaandnaxi}

\subsubsection{Dialect classification: The heritage of mid-20\textsuperscript{th} century surveys}
\label{sec:dialectclassification}

The language spoken in Yongning was investigated in 1979 by the linguists He 
Jiren \zh{和即仁} and Jiang Zhuyi \zh{姜竹仪}, who classified it among Eastern \ili{Naxi} dialects \citep[4, 104--116]{heetal1985}. The division of \ili{Naxi} into Eastern and Western dialects was initially advanced cautiously, as a~working hypothesis based on relatively short stays in the field in 1956 and 1957 as part of the national survey of the languages spoken within the borders of the People’s Republic of China.

\begin{quotation}
	From our analysis and comparison of the available linguistic and cultural materials, we propose a~preliminary division between two dialects, Western and Eastern. But due to the very short amount
	of time [that could be devoted to this research] and the shortcomings of our experience, it is
	difficult to tell whether this division tallies with the actual language
	situation{\dots}~\citep[120]{heetal1988}\footnote{\textit{Original text}: \zh{我们从现有的语言和人文材料来加以分析和比较,将纳西语初步分划为西部和东部两个方言。不过时间短促,经验不足,这样分划不知是否符合客观现实情况{\dots}{\dots}}}
\end{quotation}

%Command \noindent added to avoid having an indent. Proofreader suggestion: since this sentence continues the argument, it is better not to indent. 
{\noindent}The Western dialect area thus proposed by and large corresponds to the area ruled by the Naxi chieftains of Lijiang from the 14\textsuperscript{th} to the 18\textsuperscript{th} century (see Appendix B, \sectref{sec:feudal}). The Eastern dialect area is located to its east and north"=east, across the Yangtze river, in the present"=day counties of Ninglang \zh{宁蒗}, Yanyuan \zh{盐源}, Muli \zh{木里}, and Yanbian \zh{盐边}. Within the Eastern dialect area, three sub"=dialects were distinguished by He Jiren \& He Zhiwu: Yongning \zh{永宁}, Guabie \zh{瓜别} and Beiquba \zh{北蕖坝}. 

This classification came to be used as the standard in Chinese scholarship. It was also taken up in the inventory of languages maintained by the Summer Institute of Linguistics: \textit{Ethnologue}: \textit{Languages} \textit{of} \textit{the} \textit{World} \citep{gordon2005}. \il{Naxi|textbf}Naxi used to appear in Ethnologue under the language code NBF, which covered all the dialects, i.e.\ giving the name “\ili{Naxi}” the same extension as in Chinese scholarship. As from 2010, “Eastern” dialects of \ili{Naxi} were granted an~entry of their own in this inventory, under the romanized name “Narua” (code: NRU). The former language code NBF is now split into (i)~\ili{Naxi} proper (new code: NXQ), corresponding to “Western \ili{Naxi}” in Chinese terminology, and (ii)~“Narua” (code: NRU), corresponding to “Eastern \ili{Naxi}” in {Chinese} terminology. In detail, however, the division into dialects proposed for “Narua” is identical with that proposed by He \& Jiang for “Eastern \ili{Naxi}”. The total number of speakers was estimated at about 40,000 on the basis of early surveys \citep[107]{heetal1985}; the same figure is taken up by \citet{yang2009}. As mentioned at the outset of this chapter, the Ethnologue entry indicates a~figure of 47,000 as of 2012.

No large"=scale dialectal comparison was conducted in the half"=century that followed the first
dialect survey. The list of “subfamilies” 
(\textit{zhīxì} \zh{支系}) of the “Naxi nationality” (\textit{Nàxīzú} \zh{纳西族})
provided by \citet[5–9]{guoetal1994} could serve as a~useful reference for such a~survey, keeping in mind that this list was essentially based on ethnological criteria, rather
than on linguistic data. Reliable descriptions of the language varieties that these authors grouped under the label
“Naxi” are required for fine"=grained dialectological and comparative research. The present volume
aims to contribute to this long"=term endeavour by offering a~synchronic description of the tone system of one
specific language variety.

\subsubsection{Issues of phylogeny: The position of Na and Naxi within Sino"=Tibetan}
\label{sec:thepositionofnaandnaxiwithinsinotibetan}

The position of \ili{Naxi} and Na within \il{Sino-Tibetan}Sino"=Tibetan is a~topical issue in \il{Sino-Tibetan}Sino"=Tibetan historical
linguistics. \ili{Naxi} was classified as a~member of the “Lolo branch” (\ili{Yi}) by \citet{Shafer1955};
however, Shafer clarified that this language, to which he referred as “Mosso”, was among those for
which there was “[t]oo little data or too irregularly recorded” (p. 103, note 37). His classificatory
proposal for \ili{Naxi}, as for the other languages placed in the “Unclassified” set within the “Lolo
branch”, was thus tentative. \citet{Bradley1975} took up the issue on the basis of advances in the
comparative study of Lolo (\ili{Yi}) languages. He noted that “[w]hile a~large proportion of Nahsi
vocabulary is plausibly cognate to Proto"={Burmese}"=Lolo (*BL) and Proto"=Loloish (*L) forms
reconstructed in \citealt{Bradley1975b}, there is only limited systematic regularity of
\is{comparative method (historical linguistics)}correspondence. Moreover, the tonal and other developments postulated for *BL and *L by Matisoff are
not reflected in Nahsi”. The lack of regular \is{comparative method (historical linguistics)}correspondences, and the absence in \ili{Naxi} of shared
innovations deemed characteristic of Loloish and Burmese"=Lolo, led Bradley to conclude that \ili{Naxi} is
“certainly not a~Loloish language, and probably not a~Burmish language either” (p. 6).

Some scholars, especially in mainland China, nonetheless maintain the classification of \ili{Naxi} as
a~member of the \ili{Yi}/{\allowbreak}Lolo group. \citet{gaietal1990} base this renewed claim on the high
percentage of phonetically similar words between \ili{Naxi} and \ili{Yi}/{\allowbreak}Lolo, though without
verifying the regularity of sound correspondences. \citet{Lama2012} includes \ili{Naxi} among the set of
thirty"=seven Lolo"={Burmese} languages among which he proposes subgroupings by two methods, (i)~searching for
candidates for the status of shared innovations, and (ii)~conducting automated computation. The latter approach consists of performing Bayesian inference of phylogeny using \textsc{MrBayes}, and computing phylogenetic networks by means of  \textsc{SplitsTree}. These two software are applied to a~300-word list from the thirty"=seven languages. No judgments of cognacy
are passed on the 300 word sets fed as input to the computational procedure, which apparently
assumes cognacy in all cases as a~default hypothesis. By definition, these methods lead to proposals
for subgrouping without questioning the premise that the languages at issue all belong to the same
branch.

In light of the conclusion reached by \citet{Bradley1975}, if one looks outside Lolo"={Burmese} for
languages most closely related to \ili{Naxi} and Na, suggestive evidence comes from comparison with the
neighbouring languages \ili{Shixing} \zh{史兴语} (also known as Xumi; see \citealt{huangetal1991} and \citealt{chirkovaetal22013}) and \ili{Namuyi} \zh{纳木依语} (\citealt{lama1994}; see also \citealt{sun2001}, \citealt{yang2006}, and \citealt{lakhietal2010}), but full"=fledged comparison
has not been carried out yet, and the state of \isi{phonological erosion} of these languages is a~major
impediment to \is{comparative method (historical linguistics)}comparative studies.

Sun Hongkai, who included \ili{Shixing} (Xumi) and \ili{Namuyi} within a~“Qiangic” language group which he defined on
the basis of typological similarities, proposed that \ili{Naxi} (understood as encompassing \ili{Naxi} and Na)
is an “intermediate language” (“\textit{línjiè yǔyán} \zh{临界语言}”) between Loloish and Qiangic
\citep{sun1984}.\footnote{The original
	statement is the following: 
	\zh{“纳西语经常被认为是彝语支的语言,大家知道,纳西语在彝语支中是不太合套的一种语言} 
	[David Bradley, Proto"=Loloish, 1978: 14], 
	\zh{纳西语动词的互动范畴和羌语支完全一致。此外纳西语方言中还有一些语音、词汇和语法现象,与羌语支语言相一致。纳西语是羌语支和彝语支之间的“临界”语言。即在语言谱系分类上兼有两种语言集团的不同的特征。也就是说,纳西语同时兼有彝语支和羌语支所特有的某些特征,而动词互动范畴,则是纳西语兼有羌语支的一种重要的语法特征。”}~\citep[14]{sun1984}} This compromise view amounts to projecting the presence of \ili{Yi}"=like and Qiangic"=like
typological features into the indefinite past of {Naxi}. By a~similar reasoning, the presence of words
of \ili{Sinitic} (\il{Sinitic}Chinese), \il{Tai-Kadai}Tai"=Kadai and \il{Mon-Khmer}Mon"=Khmer origin in \ili{Vietnamese} could lead to its classification as an~intermediate language straddling the divide between these three language families.  Historical
linguists, however, favour an~approach in which borrowings and other changes in the language
are gradually identified, one layer after another, eventually resulting in a~detailed account of the
language’s evolution that includes the influences to which the language was subjected through the
ages. Thus Maspero, in his study on \ili{Vietnamese}, identified \il{Sinitic}Chinese elements as belonging to a~later layer than \ili{Tai-Kadai} and \il{Mon-Khmer}Mon"=Khmer elements:

\begin{quotation}
	Pre"=Annamite was born out of the fusion of a~\il{Mon-Khmer}Mon"=Khmer dialect with a~{Tai} dialect; the fusion may
	even have involved a~third language, which remains unidentified. At a~later period, the Annamite
	language borrowed a~huge number of Chinese words.~\citep[118]{maspero1912}\footnote{\textit{Original text}: Le préannamite est né de la fusion d'un dialecte mon"=khmer, d'un dialecte thai et peut-être même d'une troisième langue encore inconnue, et postérieurement, l'annamite a~emprunté une masse énorme de mots chinois.}
\end{quotation}

Four decades later, Haudricourt further attempted to tease apart the \il{Tai-Kadai}Tai"=Kadai and \il{Mon-Khmer}Mon"=Khmer components, emphasizing that the notion of “language fusion” can be misleading.

\begin{quotation}
	If we admit that there is no such thing as “fusion” between languages, and that genealogical
	relatedness must be assessed on the basis of core vocabulary and grammatical structure, we are led
	to consider that the modern form of a~language is not determined by its genealogical origin, but
	by the influences to which it is subjected in the course of its history.~(\citealt[121--122]{haudricourt1953a})\footnote{\textit{Original text}: Si l'on admet qu'il n'y a~pas de «~fusion~» de langues, et que l'apparentement généalogique doit être fondé sur le vocabulaire de base et la structure grammaticale, on sera conduit à penser que ce qui donne sa forme moderne à une langue n'est pas son origine généalogique, mais les influences qui s'exercent sur elle au cours de son histoire.}
\end{quotation}

Haudricourt identified a~greater proportion of \il{Mon-Khmer}Mon"=Khmer words in basic vocabulary as opposed to words of \il{Tai-Kadai}Tai stock, and came to the conclusion that \ili{Vietnamese} is a~\il{Mon-Khmer}Mon"=Khmer language. Importantly, this proposed phylogenetic affiliation by no means constitutes a~denial of the considerable influence of \isi{language contact} in the course of history~-- a~point emphasized e.g.\ by \citet[268-271]{dimmendaal2011}, citing \citet{manessy1990}. 

\begin{quotation}
	It is important to realise that there is no principled way in which one can argue that language x has become “mixed”, i.e.\ embedded with foreign language material, to an~extent where it should be classified as non"=genetic, or multi"=genetic. There are scales or degrees of borrowing, and it is precisely for this reason that it is \textit{not} a~useful taxonomic principle to talk about non"=genetic or multi"=genetic developments. \citep[271]{dimmendaal2011}
\end{quotation}

There is no hard"=and"=fast dividing line between cases of contact that are considered to result in language replacement~-- where the vocabulary inherited from an~earlier language is considered as a~substratum: piecemeal vestiges of a~language that has been replaced by another, e.g.~Basque or Celtic elements in \ili{Romance} languages~-- and cases where the earlier component still appears substantial enough to motivate classification of the modern language as belonging to that earlier component’s language family, e.g.~the \il{Mon-Khmer}Mon"=Khmer component in \ili{Vietnamese}.

Returning to Yongning Na, the traditional tools of \is{comparative method (historical linguistics)}comparative"=historical phonology appear as the most reliable to unravel this language’s history and clarify its relationship to other languages within the \il{Sino-Tibetan}Sino"=Tibetan family. There is widespread agreement about the method: “it is only by searching for lexical and morphological parallels on all sides and by establishing the phonetic equations for such parallels that we can finally decide the genetic relationship of a~doubtful group” \citep[98]{Shafer1955}. To what extent this endeavour is successful depends, as Shafer was keenly aware, on the empirical basis: the abundance and reliability of available data.

A tentative family tree was proposed in a~preliminary \is{comparative method (historical linguistics)}comparative study \citep{jacquesetal2011} based on Yongning Na, Lijiang \ili{Naxi}, and \ili{Laze}~-- a~language spoken in Muli County. The use of a~tree representation does not amount to downplaying the importance of areal diffusion. In the \il{Sino-Tibetan}Sino"=Tibetan family, waves of mutual influence are so strong that concerns about the applicability of the tree model have been voiced for decades \citep{benedict1972,matisoff1978}. Still, the tree model is useful, as one of the tools in the historical linguist’s toolbox: it serves to set out one’s working hypotheses about degrees of phylogenetic closeness between languages~-- hypotheses which serve as the basis for attempting comparisons and proposing reconstructions at various historical depths. In a~context of continuing debates about models and methods (see e.g.~\citealt{francois2014, jacques.listSAVE}), it appears necessary to emphasize that the researcher’s aim when proposing a~tree model is not to {float} new proposals about classification for classification’s sake, but to clarify assumptions made in historical comparison. The real aim is to document the evolution from the hypothesized common ancestor of a~language group to the attested language varieties.

{\largerpage}
The proposal that Yongning Na, Lijiang \ili{Naxi}, and \ili{Laze} join into a~\il{Naish|textbf}{Naish} lower"=level
subgroup is supported by shared innovations, in particular structural similarities in the tones of {numeral}"=plus"=classifier phrases that cannot have been acquired through contact (the argument will not be repeated here: readers are referred to \citealt{michaud2011c}). It is further proposed that
the {Naish} subgroup joins with \ili{Shixing} (Xumi) and \ili{Namuyi} into a~\il{Naic|textbf}Naic subgroup. At a~third, more speculative
level, {Naic} joins with {Ersuish} (called {Ersuic} in the historical study of \citealt{yu2012}; see also the reference grammar of \ili{Ersu} by \citealt{zhang2013ersu}) and Qiangic, to form
a~Na"=Qiangic node. Na"=Qiangic further joins with Lolo"={Burmese}, to form a~Burmo"={Qiangic} higher"=level
grouping, provisionally placed on a~par with Bodic, {Sinitic}, and other primary branches. This is
represented visually as~\figref{fig:atentativefamilytreeshowingthepositionofyongningnawithinaburmoqiangicbranchofsinotibetan}. This working hypothesis encourages the search for cognates between {Naic} languages, {Ersuish}, and
Qiangic. Needless to say, in this research programme, comparison with Lolo"={Burmese} languages is also
essential to progress in the historical study of {Naish} languages: all the languages listed in
\figref{fig:atentativefamilytreeshowingthepositionofyongningnawithinaburmoqiangicbranchofsinotibetan}
are uncontroversially related, as members of the \il{Sino-Tibetan}Sino"=Tibetan family, so that \is{comparative method (historical linguistics)}comparative analysis
of data from all these languages makes sense. 


\begin{figure}[p] %put on page to avoid pagebreak in fn 12
% 	\centering
	\resizebox{\textwidth}{!}{
	\begin{forest}
		where n children=0{tier=lect}{},
		for tree = {reversed,grow'=east,parent anchor=east,child anchor=west,anchor=west,l sep=1em,s sep = 0}%
		[Sino-\\Tibetan,align=left
		  [Burmo-\\Qiangic,align=left,tier=prim,l sep=2em
			[Lolo-\\Burmese,align=left,tier=sec [Loloish,tier=ish [Lahu] [Lisu] [\textit{etc.}]] [Burmish,tier=ish [Burmese] [\textit{etc.}]] ]
			[Na-\\Qiangic,align=left,tier=sec
			  [Qiangic,tier=ter
			[Rgyalrongic [Rgyalrongish,tier=ish [Situ] [Japhug] [Tshobdun] [Zbu]] [Lavrung,tier=ish [Thugsrje] [Njorogs]] [Hopa [Rtau] [Stodsde]] ]
			[Qiang,tier=ish [Northern Qiang] [Southern Qiang] ]
			[Muya,tier=ish [Northern Muya] [Southern Muya] ]
			[Queyu]
			[Zhaba?]
			[{Pumi (Prinmi)},tier=ish,calign with current [Northern Pumi] [Southern Pumi] ]
			[Tangut]
			  ]
			  [Ersuish,tier=ish [Ersu] [Lizu] [Tosu]]
			  [Naic,tier=ter [Namuyi] [Shixing (Xumi)] [Naish,tier=ish [Naxi] [Na (Mosuo)] [Laze]] ]
			]
		  ]
		  [Bodic\\(including\\Tibetan),align=left,tier=prim]
		  [Sinitic\\(Chinese),align=left,tier=prim]
		  [other\\primary\\branches,align=left,tier=prim]
%		  [{{\textit{other}}{\\}{\textit{\primary}}{\\}{\textit{\branches}}},align=left,tier=prim]
		]
	\end{forest}}
	\caption{A tentative family tree showing the position of Yongning Na within Sino-Tibetan.}\il{Sino-Tibetan|textbf}
	\label{fig:atentativefamilytreeshowingthepositionofyongningnawithinaburmoqiangicbranchofsinotibetan}
\end{figure}


%\begin{figure}%[t]
%	\includegraphics[width=\textwidth]{figures/TentativeFamilyTree.pdf}
%	\caption{A tentative family tree showing the position of Yongning Na within Sino"=Tibetan.}
%	\label{fig:atentativefamilytreeshowingthepositionofyongningnawithinaburmoqiangicbranchofsinotibetan}
%\end{figure}


Systematic comparison between {Naic} and Lolo"={Burmese} (conducted
e.g.~by~\citealt{li2015}) holds potential for clarifying to what extent their typological similarities are due to (i)~inheritance, (ii)~parallel changes (unrelated developments,
from a~typologically similar starting"=point: e.g.~the development of retroflex consonants from
initial consonant clusters), and (iii)~\isi{language contact}. There remains considerable room for
progress in the \is{comparative method (historical linguistics)}reconstruction of the various lower"=level subgroups within \il{Sino-Tibetan}Sino"=Tibetan, including
{Naish}; in turn, progress in documentation and lower"=level \is{comparative method (historical linguistics)}reconstruction holds potential for a~refined understanding of the broader picture of \il{Sino-Tibetan}Sino"=Tibetan historical linguistics.


\subsection{A review of Na language studies}
\label{sec:previousstudiesofthenalanguage}

Historical sources in Chinese offer fascinating glimpses into the language spoken in Yongning centuries ago. The \textit{Yuan {Yi} 
Tongzhi} {\kern-3pt}\zh{《元一统志》}{\kern-4pt}, a~book dated 1286, provides {Chinese} phonetic equivalents for present-day Lijiang and Yongning as \zh{样渠头} and \zh{楼头} (present"=day {Mandarin}: \textit{yàngqútóu} and \textit{lóutóu}), respectively. In the variety of {Chinese} recorded in the 14\textsuperscript{th}-century rhyme table \textit{Zhongyuan 
Yinyun} {\kern-3pt}\zh{《中原音韵》}{\kern-4pt}, the initial of \zh{头} is unvoiced; however, using the \is{comparative method (historical linguistics)}reconstruction of {\apostrophe}Phags-pa by \citet{coblin2007}, it is interpreted as *\ipa{dəw}, i.e.\ with the same voicing features as present-day Na and \ili{Naxi}. The 
name \zh{楼头} reconstructs as *\ipa{ləw dəw} \citep[487]{jacquesetal2011}, which is clearly cognate with the present-day name of Yongning in Na (/\ipa{ɬi˧di˩}/) and \ili{Naxi} (/\ipa{ly˧dy˩}/), discussed in Appendix B (\sectref{sec:shih19932010andweng1993}). This word by itself is sufficient to establish that this place name dates back at least eight centuries; it also provides evidence on a~disputed point of Chinese historical phonology, suggesting that the standard dialect of Yuan dynasty Chinese (Northern Mandarin) retained voiced obstruents \citep[487]{jacquesetal2011}. On the topic of tone, on the other hand, we have not been able to extract evidence from these early notations: examples are few, and too little is currently known about tone in both Yuan dynasty Chinese and Yuan dynasty {Naish} languages~-- not to mention the possibility that tone was simply ignored in the process of selecting a~Chinese equivalent for local terms. The notes of explorers of the turn of the 20\textsuperscript{th} century likewise provide little information about the language and less about tone, and will therefore not be discussed here; readers are referred to \citet{michaudetal2010} and references therein. The present review of Na language studies focuses on contemporary linguistic research.  
%\largerpage

\subsubsection{Information about Na in the \textit{Brief description of the Naxi language}}
\label{sec:heandjiang1985}

He Jiren \& Jiang Zhuyi’s \citeyear{heetal1985} \textit{Brief description of the Naxi language} mainly focuses on the dialects spoken in the Lijiang plain, but
the volume includes a~word list of Yongning Na, as well as some observations on phonology,
syntax, and dialectal diversity (pp. 107--116; see also \citealt{jiang1993}).

The transcription is not phonemicized, and may not be fully consistent. Only four tones are
transcribed over monosyllables: LM (\ipa{˩˧}), M (\ipa{˧}), ML (\ipa{˧˩}), and H (\ipa{˥}), whereas
the analysis presented in this volume brings out six categories (LM, LH, M, L, H and MH: see
Chapter~\ref{chap:thelexicaltonesofnouns}). He \& Jiang based their linguistic research on
an~{analogy} with \ili{Naxi}, a~language which both of them could speak: He Jiren as a~native speaker, Jiang
Zhuyi as a~second"=language learner. \ili{Naxi} has a~four"=way tonal opposition over monosyllables:
High, Mid, Low (realized phonetically as low"=falling),
and Rising. When listening to Yongning Na, He \& Jiang failed to distinguish the low"=rising and mid"=rising tones. Moreover, they report a~difference between M and H tones in word"=initial position, which in fact does not
exist. Of course, it may also be that the dialect they investigated differs considerably from
that spoken by the consultants who collaborated with me, but this is considered less likely in view of the great
phonetic proximity between their word list and the data reported here. It is not
uncommon for phonemic and tonal analyses based on data collected during short field trips to differ
greatly across authors, due to incomplete phonemicization of the data (as pointed out by
\citealt[329]{matisoff2004}). 
%For example, for \ili{Shixing} (a.k.a. Xumi), \citet{sun1983} reports 58 initials, whereas
%\citet{huangetal1991} report 43, for two subvarieties which are mutually intelligible
%\citep{chirkova2009}.

Returning to tones M and H in Yongning Na, in the language variety described here (Alawa dialect) these two tones
are neutralized to M \is{form!in isolation}in isolation (see \sectref{sec:thesynchronicfacts}). Phonetic realizations of this tone (which, under a~\is{Praguian phonology}Praguian
approach, can be referred to as an~\isi{architoneme}) vary freely in the upper half of the speaker’s tonal
space, so that an~investigator who starts out from the hypothesis that there are H-tone
monosyllables and M-tone monosyllables would be able to hear differences in pitch that seem to support
the hypothesis. The same process of pre"=existing assumptions affecting linguistic observation
happened to me in the early stages of fieldwork: in my initial transcriptions, there were H-tone
words and M-tone words. I later discovered that there was no such opposition \is{form!in isolation}in isolation. In Alawa, the
phonological M and H tones can only be brought out by placing the words in context,
e.g.~adding the {copula} (as explained in Chapter~\ref{chap:thelexicaltonesofnouns}). The distinction between H and M tones made by He
and Jiang on the basis of citation forms (words said \is{form!in isolation}in isolation) is thus spurious (unless, as mentioned above, there is
a~considerable gap between the tone systems of the dialects at issue). ‘Field’, glossed as ‘earth’
(\textit{dì} \zh{地}), is transcribed with a~High tone: /\ipa{lv̩˥}/, when in fact it carries Mid
tone: //\ipa{lv̩˧}// (the double slashes are used in this volume to distinguish lexical forms
from surface phonological ones). ‘Man’ is transcribed as /\ipa{xĩ˧}/, with M tone; this is indeed
the tone that the word carries \is{form!in isolation}in isolation, but its behaviour in context reveals that its lexical
tone is H (my transcription: //\ipa{hĩ˥}//).

The distribution of Low(-falling) and Low"=rising tones in He \& Jiang’s data is a~puzzle to me,
since no monosyllables are pronounced with Low tone \is{form!in isolation}in isolation in the language variety that I
investigated. Examples include ‘plain, flatlands’, ‘water’ and ‘goose’, transcribed as /\ipa{dy˧˩}/,
/\ipa{dʑi˧˩}/ and /\ipa{o˧˩}/; these three items have different tones in my data: LH tone for
‘plain’, L tone for ‘water’, and LM for ‘goose’. My best guess is that, in the speech of
He \& Jiang’s consultants, L was a~(relatively infrequent) free {variant} of LH in {citation form}. It
is also possible that their word list combines data from several speakers, and is not dialectally
homogeneous. The Naxi data in the same volume is a~case in point: the authors present the data as
coming from the dialect of the township of Qinglong \zh{青龙} (present"=day Changshui \zh{长水}), the
home of Jiang Zhuyi’s teacher He Zhiwu \zh{和志武}, but some data was contributed by He Jiren on
the basis of his native dialect, Yangxi \zh{漾西}. In the absence of indications about the origin of each piece of data, it is extremely difficult to determine which data comes from which
dialect.

  
There are also some issues with He \& Jiang’s transcription of vowels and consonants, as is to be
expected of initial field notes. Nasality is transcribed only in two syllables, /\ipa{xĩ}/ (as in
‘man’, transcribed /\ipa{xĩ˧}/; my data: //\ipa{hĩ˥}//) and /\ipa{ɣə̃r}/ (the only example is ‘bone’,
transcribed /\ipa{ʂa˧ɣə̃r˧}/; my transcription: //\ipa{ʂæ˩ɻ̍̃˩}//), whereas the investigation reported
in the present volume brings out eight nasal rhymes. Another point of difference is that He \&
Jiang do not transcribe the uvular consonants reported in Chapter~\ref{chap:vowelsandconsonants} of this volume. Such
discrepancies may be due to the fact that the variety described by He \& Jiang had fewer phonemes
than that described here; but it is not implausible that they failed to distinguish some sounds that
were in fact contrastive.

Conversely, some vowel differences transcribed by He \& Jiang may be spurious. The word list
contains examples of /\ipa{li}/ (as in /\ipa{li˧}/ ‘to look’) and /\ipa{lie}/ (as in /\ipa{lie˩˧}/
‘tea’). In my data ‘tea’ and ‘to look’ have the same initial and rhyme. The vowel /\ipa{i}/ is
slightly diphthongized towards [\ipa{e}], and thus close to [\ipa{lie}], which explains why it could be
sometimes heard as [\ipa{i}] and sometimes as [\ipa{ie}] before the investigator’s ear attunes to the
vowel system of Yongning Na. Once again, it is also theoretically possible that these two words did
not have the same phonemes in the dialect investigated by He \& Jiang.


\subsubsection{A~study of kinship terms, with phonetic observations: \citet{fu1980}}
\label{sec:fu1980astudyofkinshipterms}

The linguist Fu Maoji \zh{傅懋勣} visited Yongning in May and June 1979 with He Jiren and Jiang Zhuyi. He collected data in the
village of /\ipa{dʑɤ˩bv̩˧-ʁwɤ˩}/ (Jiabowa \zh{甲波瓦}) for a~study about kinship terms, presented at the 12\textsuperscript{th} International Conference on {Sino"=Tibetan} Languages and Linguistics (Paris,
1979), then published in Chinese and in {French} translation (\citealt{fu1980,fu1983}). The paper is a~testimony to the appeal of Na family structure beyond the circle of professional anthropologists (see Appendix B, \sectref{sec:anthropologicalresearchthefascinationofnafamilystructure}). The article has an~appendix containing notes about phonetic transcription. It is interesting to examine these notes in light of a~full"=fledged phonemic analysis, picking up in retrospect some groundbreaking observations, such as the recognition of uvular /\ipa{q}/, /\ipa{qʰ}/ and /\ipa{ʁ}/. Fu Maoji also recorded the approximant /\ipa{ɹ}/, noting that it can appear in front of a~vowel (i.e.\ as an~initial consonant) or constitute a~syllable on its own; this is no different from the analysis proposed in this volume (see Chapter~\ref{chap:vowelsandconsonants}), where the notation chosen is as a~retroflex, /\ipa{ɻ}{\kern2pt}/. 

However, Fu Maoji's felicitous insights come together with more puzzling proposals. For instance, his set of uvular consonants includes the fricative /\ipa{χ}/, which closer analysis of the target dialect would probably show never to contrast with velar realizations, as in all the Naxi and Na dialects recorded to date. Supposing that, during their joint field trip, He Jiren and Jiang Zhuyi examined Fu Maoji's notations, they could rightly be skeptical of inclusion of /\ipa{χ}/ in the inventory, and their doubts could then extend to Fu Maoji's (correct) proposal of uvular /\ipa{q}/ and /\ipa{qʰ}/ as distinct phonemes. There are further possible reasons for disbelief on their part: the uvular sounds [\ipa{q}] and [\ipa{qʰ}] are also found in \ili{Naxi}, where the scope of allophonic {variation} of velar stops extends well into the uvular region. This can lead speakers of {Naxi} to the assumption that uvulars in other {Naish} varieties are also allophones of velars. Moreover, the phonological environments in which /\ipa{k}/ and /\ipa{kʰ}/ contrast with /\ipa{q}/ and /\ipa{qʰ}/ in Yongning Na are restricted (for details, see Appendix A, \sectref{sec:velaranduvularstops}).

 Other problematic aspects of Fu Maoji's notation include (i)~a distinction between plain and laryngeally constricted /\ipa{u}/, /\ipa{v̩}/ and /\ipa{z̩}/ rhymes, (ii)~a proposed set of three rhotic vowels in addition to the approximant rhyme [\ipa{ɹ}], and (iii)~an analysis whereby /\ipa{i}/ contrasts with /\ipa{e}/ but /\ipa{i}/ is realized as [\ipa{e}] after apical and apical"=dental consonants. There is still some way to go from these notes to a~working phonemic notation. As for tone, Fu Maoji's classification into three categories, mid"=rising, high"=flat, and mid"=falling, demonstrates that he commendably chose to start fresh and establish the language's tonal categories on their own terms instead of interpreting them in terms of the \ili{Naxi} tone system; but again, he did not quite reach the stage at which the relevant categories would have emerged. The piecemeal nature of the report goes some way towards explaining why He Jiren and Jiang Zhuyi did not avail themselves of Fu Maoji's data in their 1985 book. It should however be remembered that conducting fieldwork in Yongning in 1979 was an~achievement in itself. 


\subsubsection{An outline of Yongning Na by Yang Zhenhong (\citeyear{yang2009})}
\label{sec:yang2009}

An outline of Yongning Na was published by Yang Zhenhong \zh{杨振洪}, a~speaker of this language
from /\ipa{ə˧bv̩˧-ʁwɤ˧}/ village (Abuwa \zh{阿布瓦村}), close to the current location of
the Yongning high school (original publication in Chinese: \citealt{yang2006d}; {English} translation
by Liberty Lidz, with improvements made after consulting with the author, published as \citealt{yang2009}). This outline by and large follows the structure of
He \& Jiang’s description of Naxi. Some parts of the discussion of phonetics and phonology may
require further analysis: among consonants, uvular and retroflex stops are not granted phonemic
status; concerning tone, the analysis is based on the four tones attested in Naxi, which entails
some limitations. Informal exchanges with the author (in 2011) suggest that there are in fact more
tone categories in the dialect at issue. Like various other researchers, Yang Zhenhong uses descriptive
tools developed for syllable"=tone systems such as those of \ili{Sinitic} languages, which do not
constitute a~fully adequate means to describe tone in Na. (A similar problem was encountered in studies
of \ili{Pumi}~-- also known as Prinmi~-- where advances were finally realized by researchers with a~knowledge of
other types of tone systems, such as those of \ili{Japanese} dialects: see
\citealt{ding2001,ding2006,jacques2011a}.)


\subsubsection{Lexical materials}
\label{sec:dictionary2013}

\textit{An anthology of everyday words and expressions in the Mosuo language} \citep{zhibaetal2013} presents vocabulary and expressions arranged by semantic field. The authors are a~native speaker from the Lake Lugu area and a~doctor in linguistics from Yunnan University. Their fieldwork is described as covering the Yongning plain and the Lake area, but with the Yongning plain as the main research area (p. 2). 

Approximations in phonetic notation are so numerous that they make the volume unreliable as a~work of reference. Voicing contrasts were challenging for the linguist in the team, whose training was mainly focused on the theory and practice of teaching Chinese as a~foreign language. Thus, the name of the mountain /\ipa{kɤ˧mv̩˧˥}/ is transcribed as /\ipa{gə⁵⁵mu⁵⁵}/, with a~voiced initial (p. 17 and elsewhere). The mountain's name in Chinese, \textit{Gemu} \zh{格姆山}, may have exerted an influence here. Conversely, the adjective /\ipa{dʑɤ˩\textsubscript{b}}/ ‘good’ is transcribed as /\ipa{tɕɑ¹³}/, with an unvoiced initial. Some phonemes, such as uvulars, are absent from the notations. 


\subsubsection{Collections of oral literature: Na ritual texts}
\label{sec:collectionsoforalliteraturenaritualtexts}

In the Yongning plain, {Tibetan} Buddhism (of the Gelugpa school) co"=exists with a~local tradition of
ritual practitioners, called /\ipa{dɑ˧pɤ˧}/. Unlike those of the
Naxi, the rituals are not written~-- although some written characters are used by the Na for computing days in divination: see
\citet{yang1985} and \citet[163-189]{lidazhu2015}. The absence of a~written form explains in part why these rituals have attracted
less attention than those of the Naxi. The much smaller size and socio"=political weight of Yongning
as compared with Lijiang, and the difficulty of access to the area until the late 20\textsuperscript{th} century, also
go a~long way towards explaining why there were few efforts for documenting this aspect of the
Na {oral tradition}. 

A bilingual collection published by a~native speaker of the language
\citep{azeming2013} contains (i)~a phonetic approximation of each syllable by means of a~Chinese
character, (ii)~a transcription of each syllable in the International Phonetic Alphabet, (iii)~a Chinese gloss for each syllable,
and (iv) a~translation into \il{Sinitic}Chinese for each line. (Most lines contain five, seven, or eight
syllables.) This volume is best reserved for readers already familiar with the language, who will
be able to identify part of the glosses by making allowance for transcription habits influenced by
the \textit{Pinyin} system for the romanization of Chinese. For instance, /\ipa{p}/, /\ipa{t}/ and
/\ipa{k}/ are apparently used to transcribe aspirated /\ipa{pʰ}/, /\ipa{tʰ}/ and /\ipa{kʰ}/,
respectively, like \textit{p}, \textit{t}, \textit{k} in \textit{Pinyin}: ‘white’, transcribed as
/\ipa{puə}/ in this book (p. 4), has an~aspirated initial in Na. Tone is not indicated.

During the 2010s, Latami Dashi started a~documentation programme aiming
at the publication of an~extensive collection of translated and annotated rituals with accompanying
video. This work was in progress at the time of publication of the present volume.

In"=depth study of Na rituals, and comparison with Naxi rituals, holds great promise for
an~improved understanding of Na cultural dynamics (see \citealt{mathieu2015} and references therein). This study would require a~good command of
\ili{Tibetan} philology, an in-depth knowledge of \ili{Tibetan} Buddhism, and other skills far beyond my field of expertise. A~piecemeal observation can
be offered to suggest the type of issues of cultural contact related to Na religion:
twentieth"=century reports suggest that monks and /\ipa{dɑ˧pɤ˧}/ coexisted peacefully, with
an~established division of labour. One would call the /\ipa{dɑ˧pɤ˧}/ to perform a~ritual when
killing pigs; the monks on prescribed days of the calendar; and both monks and /\ipa{dɑ˧pɤ˧}/ for
the most important events, such as funerals. This peaceful coexistence, and the occasions for
contact offered by the rites where both participated, apparently resulted in a~measure of
convergence. The Na /\ipa{dɑ˧pɤ˧}/ show signs of influence from the highly ritualized behaviour of
Buddhist monks. They prepare their rituals with an~attention to detail that approaches that of
Buddhist monks, asking for all the necessary paraphernalia in advance, such as butter, candles,
water, and different types of flour. By contrast, the \ili{Yi} ritual specialists (called ‘Bimo’ \zh{毕摩}
in Chinese) have a~habit of requesting objects and accessories suddenly at any point during a~ritual,
as if acting on their inspiration (Latami Dashi, p.c. 2008). The gestures of the /\ipa{dɑ˧pɤ˧}/
have also come to resemble those of monks. Conversely, some monks are reported to study the Na
horoscope~-- one of the fields of competence of the /\ipa{dɑ˧pɤ˧}/.

\subsubsection{Liberty \citet{lidz2010}, \textit{A descriptive grammar of Yongning Na (Mosuo)}}
\label{sec:lidz2010}

By far the most thorough description and analysis of Yongning Na to date is Liberty Lidz’s
Ph.D.\ dissertation \citep{lidz2010}, \textit{A descriptive grammar of Yongning Na}
(\textit{Mosuo}). It concerns the variety of Yongning Na spoken in the village of Luoshui \zh{落水},
on the shore of Lake Lugu. The dissertation, based on in"=depth fieldwork, provides a~description of the morphosyntax
of the language, and contains 150 pages of
transcribed and annotated narratives.

Concerning tone, it has been noted that “[t]he tonal system of Luoshui Narua calls for further
analysis. Surface phonological tones are transcribed employing three tonal levels, but a~reanalysis
in terms of two levels would seem possible in many cases. Mention is made of prolific {tone sandhi}
processes, but tantalisingly, these processes are not elaborated on” \citep{dobbsetal2016}. It is not unusual for reference grammars to leave open some issues of {prosody}, such as the status of tone, or of stress \citep[26]{zeitoun2007}; in the case of Yongning Na, a~division of labour has been tacitly established between Liberty Lidz and myself, whereby I would take up the task of conducting detailed investigations into tone. Such is the aim of the present volume.


\section{Project and method}
\label{sec:chronologyofthestudyelicitationproceduresandonlinematerials}

\subsection[The aim: Detailed description of a~level"=tone system]{The aim: Detailed description of a~level"=tone system of East Asia}
\label{sec:theaimofthepresentvolumetoprovideanindepthdescriptionofaneastasianleveltonesystem}

Tonal changes permeate numerous aspects of the morphosyntax of Yongning Na. Importantly, they are
not the product of a~small set of phonological rules, but of a~host of rules that are restricted to
specific morphosyntactic contexts. Guillaume Jacques (p.c.\ 2009) notes that irregular morphology in Yongning Na, as also in
other \ili{Naish} languages (and in \ili{Pumi}), mostly consists of irregular morphotonology. The richness of this aspect of the language calls for a~book"=length description, applying “the old philological virtue of exactitude”
\citep[152]{Scherer1885} to this system, in order to arrive at
a~reasonably comprehensive account.

A search for full"=fledged, book"=length descriptions of similar systems in other languages suggests that such
reference works remain relatively scarce, even concerning the extensive
{Bantu} branch of the Niger"=Congo family of languages, famous for the richness of its morphotonology.

\begin{quotation}
  Theoretical linguistics is primarily concerned with advancing the theoretical enterprise, and
  tends to produce short pieces~-- chapters, articles, squibs. It does not have the writing of
  grammars as a~priority, and few of the theoretical grammars of African languages written during
  the heyday of transformational theory during the 1960s and 1970s have stood the test of time. ({\dots})
  Are there enough grammars of sub"=Saharan African, especially {Bantu}, languages? The answer is
  no. ({\dots}) The overwhelming impression is that of the small number of real grammars, and the number
  is not increasing.~\citep[xxiii--xxiv]{nurse2011}
\end{quotation}

In addition to quantitative scarcity, there is also an~issue of breadth and depth of coverage for
those languages that have been the object of book"=length descriptions. Linguistic fieldwork consists in “going into a~community
where a~language is spoken, collecting data from fluent native speakers, analysing the data, and
providing a~comprehensive description, consisting of grammar, texts and dictionary”
\citep[12]{Dixon2007}. This all"=out endeavour entails decisive advantages for understanding the language as a~whole, as it functions in its social setting. But breadth of scope can occasionally conflict with depth of investigation of individual topics, such as tone. “No variety of Bambara has heretofore been the object of a~systematic tonological
description aiming at full coverage”\footnote{\textit{Original text}: aucun parler bambara ({\dots}) n’a jusqu’ici fait
	l’objet de ce qui mériterait d’être considéré comme une description tonologique systématique visant
	à l’exhaustivité.} (\citealt[199]{creissels1992}; see also \citealt{clements2000};
\citealt{hyman2005a}). “Even the ‘well described’ languages often suffer from a~lack of examples, by which to test the descriptive or theoretical
claims” \citep[xxiii]{nurse2011}. Similar observations recur in literature concerning tone systems from various areas of the world. In the field of Mesoamerican languages, the following plea for better documentation of tonal morphology emphasizes the urgency of the work.

\begin{quotation}
	We need full paradigms in grammars of tonal languages, not just rules, abstract representations or examples of how a~given form is used in a~natural context. This is a~cordial invitation to descriptive linguists
	to enrich the field with new data on inflection. It matters. It matters in a~time
	when most languages with complex morphology are dying. By doing so, we will
	be paying tribute both to the languages and to the field of linguistics, because in
	a~hundred years from now, when all of us are gone, it will only be our data that
	shall remain for future linguists to continue increasing our understanding of our
	human languages. \citep[134]{palancar2016}
\end{quotation}

In the field of \il{Sino-Tibetan}Sino"=Tibetan studies, occasional misrepresentation of the (often complex) tone systems is not unheard of, as noted e.g.~by \citet{sun2003b} for \ili{Tibetan} and \citet[188]{post2015} for Tani and languages of Northeast India in general. Fortunately, languages
with level"=tone systems and rich morphotonological systems are the object of active research. Two grammars of
\ili{Pumi}, a~language that uses two tonal levels, are now available \citep{daudey2014,ding2014}. The present account
of Yongning Na tone is intended as a~contribution to this development in Sino"=Tibetan studies. 

At the present stage, the aim is to arrive at a~precise description, which constitutes the necessary basis for further work. Consequently, this volume contains many tables setting out the paradigms in full. There is room for some progress in terms of economy of description: for instance, identifying a~set of morphotonological combination patterns as default for a~certain category of morphemes, and rewriting the data for other categories as \textit{identical to the standard pattern, except for{\dots}}, instead of setting out all the data in tabular form. Modelling the morphotonology of Yongning Na, with computer implementation (finite"=state modelling), is among the author's long"=term projects, mentioned in the conclusion (Chapter~\ref{chap:conclusion}).


\subsection{Theoretical backdrop}
\label{sec:theoreticalbackdrop}

This study is theory"=informed, not theory"=driven: the aim is not to bring selected data to bear on topical issues in phonological or morphological debates, but to attempt an~in"=depth description of a~language as it functions. This goal is common
to all linguists, and matters more than theoretical differences. An overarching guiding principle is to exercise the greatest vigilance to steer clear of Procrustean models. The theoretical backdrop to the present research is intended to be as unobtrusive as possible, as befits language description and analysis. Linguistic models will be mentioned as the necessity arises. 

In a~nutshell, the method used here essentially rests on the basic principles of classical structural"=functional phonology, as
set out in handbooks of phonological description (for instance \citealt[15,
  34--47]{martinet1956}). It is difficult to select a~quote that would neatly summarize
the main tenets of this approach. Martinet devoted an~entire volume to setting out \textit{A
  functional view of language} \citep{martinet1962}. The excerpt below is from another
{English}"=language volume, entitled \textit{The internal conditioning of phonological systems}.

\begin{quotation}
  A~dynamic conception of language presupposes that we do not deal with it as we would with a~dead
  body in the morgue, but try to look at it as a~means of satisfying some of the human needs, and
  essentially that of communication. In other terms, it derives from a~functional view of language~({\dots}). 
  [E]xperience has shown that even if language is often used for the satisfaction of
  other needs as, for instance, that of communion, it is, in the last analysis, mutual understanding
  that determines the choices of the speakers. ({\dots}) At every point in time, with every speaker,
  what is said and how it is said will show a~balance between the desire to communicate, and
  inertia, be it individual, i.e.\ reduction of energy, or social, i.e.\ preservation of traditional
  forms at the expense of personal comfort and communicative efficiency.~\citep[2–3]{martinet1996}
\end{quotation}

A~major source of change is the constant competition between the tendency
towards phonological integration on the one hand and the tendency towards phonetic economy on the
other. Phonological integration tends to fill structural gaps in phonological systems, while
phonetic economy tends to create phonological gaps. A~simple example can be drawn from tones: having five level tones (Top, High, Mid, Low, Bottom) could be seen as phonologically economical, since in such a~system tone alone allows
for numerous lexical distinctions, and the combinations of the five levels open up immense possibilities for tonal morphology and morphotonology. But having five level tones is phonetically uneconomical, because the distinction
between a~large number of level tones is perceptually difficult, e.g.~distinguishing sequences such as
Top+High, Top+Mid, and High+Mid.

In synchronic description, attention to the conflicting factors that are constantly at play in speech communication leads to adopt the method advocated by
Martinet under the name of “\is{dynamic synchrony|textbf}dynamic synchrony” \citep{martinet1990}. The focus is on synchronic
description, but flatly synchronic description is enriched by observations about current tensions within
the system, assessing which of the competing variants are innovative and which are {conservative}. Thus synchrony and {diachrony} gradually combine. Not much is known at present about the \is{comparative method (historical linguistics)}{diachrony} of level"=tone systems in Sino"={Tibetan}. Case studies in \isi{dynamic synchrony} can yield convergent evidence on {diachronic} tonal changes and their conditioning. Ultimately, what is needed is an~approach that attempts to formulate generalizations about sound
change that are independent of any particular language or language group. The aim is to build
an~inventory of types of sound change and arrive at an~improved understanding of the
conditions under which they occur. Haudricourt labels such an~approach \is{panchronic phonology|textbf}\textit{panchronic}
(\citealt{haudricourt1940,haudricourt1973b}; see also \citealt{hagegeetal1978}). \is{panchronic phonology}Panchronic laws are
obtained by induction from a~typological survey of precise {diachronic} events whose analysis brings
out their common conditions of appearance. In turn, \is{panchronic phonology}{panchronic} laws can be used to shed light on
individual historical situations. The idea is that, out of the pool of potential changes, the direction of evolution observed in a~given language depends in part on the state of its phonological system: which phonemic oppositions are found in the language, which phonotactic constraints they are subject to, which role they play in the morphophonology, and so on. (For an example, see \citealt{jacques2011a}.) 

A challenging mid"= to long"=term goal for historical research will consist in modelling the origin and evolution of level"=tone systems with the same degree of precision attained in studies of classical tonogenetic processes in \ili{Sinitic}, Austroasiatic and \il{Tai-Kadai}Tai"=Kadai, which constitute a~success story of \is{panchronic phonology}{panchronic} phonology. Diachronic comparison shows that a~voicing opposition can turn into a~phonation"=type opposition, a~tonal opposition, or a~vowel quality opposition \citep{haudricourt1965a,ferlus1979}. In the Mon language, for instance, the voicing opposition on initial consonants transphonologized to two contrastive phonation types; 
in \ili{Vietnamese}, it resulted in an increase in the number of lexical tones; and in Khmer, it resulted in an increase of the number of vowels. Phonation"=type oppositions are now known to be the first stage in the transphonologization. This stage is characterized by the relaxation of the larynx for syllables with formerly voiced initials. At that stage, the phonetic cues to this opposition include, in addition to \is{phonation types}phonation type proper, some differences in pitch, as well as differences in vowel articulation. This was already noted for a~{conservative} variety of Khmer by \citet{henderson1952}, and is confirmed by experimental studies of Mon (\citealt{abramsonetal2015} and references therein). At a~later stage, one or the other of the cues becomes dominant: this is where the evolution branches into the Khmer type (where vowel quality stabilizes as the new distinctive property) and the \ili{Vietnamese} type (where the distinctions become tonal). A~major structural parameter in this branching is whether the language already has tones at the time when the transphonologization takes place. If the language already has tones, the transphonologization of voicing
oppositions creates a~split in the tone system; otherwise it becomes a~vowel quality opposition, creating a~two"=way split in the vowel system. Examination of numerous East and Southeast Asian languages confirms this model, simultaneously offering opportunities for further refinements to the model through the study of tonogenetic processes in progress, e.g.~\citet{brunelle2012} on Cham; \citet{kirby2014} on Khmer; \citet{yangetal2015} on Lalo; and \citet{pittayaporn.kirby.laryngeal2017} on Cao Bằng \il{Tai-Kadai}Tai. 

Panchronic phonology is close (at least in my view) to \textit{\is{evolutionary phonology}evolutionary phonology}, a~theory of phonology that aims to combine insights from historical phonology and experimental phonetics, to provide “a~general link between neogrammarian discoveries, advances in modern phonetics, and phonological theory” \citep[xiii]{blevins2004}. The emphasis on phonetic bases of change, building on \citet{ohala1989}, encourages a~continuous and mutually profitable dialogue between experimental phonetics and historical phonology. Evolutionary phonology, like \is{panchronic phonology}{panchronic} phonology, thus constitutes a long"=term research programme that holds promise of an~increasing degree of
precision and explicitness in modelling historical change. While it is possible to pinpoint some differences in the stated principles and methods (see \citealt[601]{labov1994}, \citealt{andersen2006}, \citealt{iverson2006}, \citealt{mazaudonetal2007} and \citealt{smithetal2008}), I believe that the common aim~-- to explain
synchronic states in terms of the processes that lead up to them, and to arrive at general laws of
sound change~-- is more important than theoretical
differences, and that practitioners of \is{panchronic phonology}{panchronic} phonology, \is{evolutionary phonology}evolutionary phonology or other
approaches to the typology of sound systems and sound change share the same essential goals. 

This may seem far too much {diachronic} background for a~synchronic monograph, but I believe that it is useful for synchronic descriptions such as the present volume to have a~long"=term {diachronic} agenda. In historical linguistics, as in phonetics/phonology, “the devil is in the detail” \citep{nolan2003}, and the patient sifting of fine points of tonal description helps develop a~feel for the {diachronic} evolution of tone systems. (Chapter~\ref{chap:yongningnatonesinadynamicsynchronicperspective} is devoted to issues of synchronic \isi{variation} and {diachronic} change.) 


\subsection{Field trips and collaboration with consultants}
\label{sec:collaborationwithconsultants}

The present results are based on data collected since 2006. Four field trips to the village of Alawa were conducted from 2006 to 2009. Excluding the time spent on travel
and on organizational tasks, the {duration} of these stays was 50 days in 2006, 35 days in 2007, 58
days in 2008 and 40 days in 2009. From July 2011 to October 2012, I had the wonderful opportunity of
staying in China for long"=term fieldwork. As my main consultant had by that time moved to the town
of Lijiang to take care of a~granddaughter, I was based in this town too, working with her for
an~average of two hours a~day. In 2013, 2014, 2015 and 2016, I made short field trips to Lijiang and Yongning, still working mainly with the same consultant.

Currently standard procedures for data collection, as reflected in the ‘Method’ section of papers in
phonetics journals, tend to avoid any mention of personal contacts between the investigator and the
subjects. Such mentions would be worse than irrelevant, they would be suspicious, since exchanges
with consultants beyond providing instructions are viewed as a~contaminagen: a~threat to
the objectivity of the experiment. “The
subjects were unaware of the purpose of the experiment” is considered a~commendable state of
affairs. However, to linguists who have experience of working in
collaboration with consultants, whether in a~language lab or in the field, it is clear how deeply
the relationship established with the consultant influences research. A~close look at data
collection in language laboratories suggests that important dimensions in the selection of language consultants
and the formulation of instructions tend to be overlooked. Worries are seldom voiced about the bias
introduced by the use of professional linguists, or multilingual students, as subjects, despite the existence of well"=documented differences across speakers: clearly, different people have different abilities (see e.g.~\citealt{audibertetal2008}). 

Of course, different research purposes call for different data collection methods, and it
would be thoroughly unreasonable to expect all investigators to develop personal familiarity
with subjects who participate in their research. Nevertheless, it is clear that, in the process of
describing a~language, mutual understanding between the investigator and consultants is of the
essence. In this light, the personal details presented here are not simply fieldwork anecdotes: in
my view they represent relevant information on data collection.
%\footnote{Further thoughts about data collection are set out in \citet{niebuhretal2015}.}


\subsubsection{First steps in the search for consultants}
\label{sec:firststepsinthesearchforconsultants}

On the first field trip, Mr.\ Latami Dashi, a~native speaker of Na and a~researcher in ethnology
based in the Ninglang county seat to whom I had been introduced by Picus Ding, accompanied me to his
family’s house, where I was invited to reside during all my stays in Yongning. He volunteered to
work as a~consultant. (His code in the database of \ili{Naish} speakers is M18.) Mr.\ Latami has near"=native
command of both \il{Mandarin!Southwestern}Southwestern {Mandarin} and \il{Mandarin!Standard}Standard (Beijing) Mandarin. It was immediately obvious to both of
us, when we began an~elicitation session, that long years of daily practice of {Mandarin} had taken
their toll on his proficiency in Yongning Na. He offered to help find a~speaker who had a~relatively homogeneous
linguistic experience, having lived continuously in Yongning since childhood. For my part, I wanted
to work with a~male speaker, for a~technical reason: spectrogram reading and electroglottographic
analysis, two techniques that I planned to use, are easier to perform on data from male
speakers. Also, we agreed to look for a~speaker whose age ranged between 35 and 65. Younger speakers
have limited command of the language; as for the oldest speakers, they are often the most proficient in
Na, but at a~certain age speech becomes less audible and communication with strangers more
difficult.

Mr.\ Latami therefore set out to look for a~suitable consultant in the neighbourhood. The procedure
as he narrated it to me was the following. He invited a~candidate over to his place after dinner,
treated him to liquor and sunflower seeds, and launched a~conversation about how the language was
being lost by the younger generation. Then he explained that there was currently a~foreigner staying
in the house, who wanted to study and record everyday language; and he asked if the person would
agree to work as language consultant.

Several acquaintances were thus invited, said they would consider the proposal, and eventually
declined. There may have been a~number of reasons for this. Mr.\ Latami’s point of view is that, for
want of knowing the ins and outs of linguistic fieldwork, they were suspicious of potential misuse
of the information that they would provide, and wary of the blows that their reputation would suffer
if their name became associated with debatable materials about Na language and Na culture. There are
enough examples of ludicrously simplified depictions of Na culture produced to cater for the tourist
industry (as reviewed in Appendix B, \sectref{sec:presentdaysociologicalstudiestheimpactoftourismsincethe1990s}) to justify their
cautious stand.

On the other hand, these people have known Mr.\ Latami since he was a~child, and they would have
reason to trust that he would not collaborate in a~research project that may harm the image of the
community. In his own research on Na culture, Mr.\ Latami takes care to gather
viewpoints from a~number of relatives and acquaintances. After he has
finalized a~draft of one of his books, he circulates copies to Na people who are literate in Chinese and asks for their comments. Only after he
has received their criticisms, corrections and comments, and worked them into the final version,
does the book go to press. This offers no absolute guarantee against resentment on the part of 
community members concerning the contents of the book in its final form, but it could go some way
towards allaying suspicions.

The negative response of these Na speakers on being invited to participate in data collection can be considered as providing an~insight into traditional Na society as a~highly {conservative} agricultural society where deviant behaviour meets with sharp
reproach. This is not without consequences for language: “the strong networks typical of rural life” \citep[379]{milroyetal1985} favour not only archaism, but
also the development of innovations that tend to complexify morphology~-- a~process opposite to what
happens in cases of creolization. For instance, reflecting on the case of the development of person
agreement marking on complementizers in Bavarian German (reported by \citealt{bayer1984}), Peter
Trudgill speculates that this only happens in tightly"=knit rural communities. The example given is (\ref{ex:minga}), where \textit{ob} ‘whether’ receives second"=person agreement \textit{-st} \citep[82,
112--113]{trudgill2011}. 

\begin{exe}
  \ex
	 \label{ex:minga}
	 \gll {\dots} obst du noch Minga kummst\\
     {} whether you-\textsc{sg} to Munich come\\
     \glt ‘whether you are coming to Munich’
\end{exe}

To return to Yongning Na, the development of the rich morphotonology described in the present volume may have been favoured by the same social factors that
initially led potential consultants to decline sharing their knowledge of Na with me.\footnote{I~am aware that this link is hypothetical and looks a~lot like a~“just"=so story”: an \textit{ad hoc} and unverifiable speculation. Some observations on the dynamics of Yongning Na morphotonology are set out in Chapter~\ref{chap:yongningnatonesinadynamicsynchronicperspective}.}

Mr.\ Latami Dashi’s mother seconded her son’s efforts to convince potential consultants that they
should not be intimidated by the tasks proposed to them, explaining that the purpose was not to
collect folklore, but to study the everyday language, and that the initial stage of the work was as
simple as saying the words for ‘head’, ‘hand’, and so on. While she did not succeed in convincing
others, she eventually convinced herself, and volunteered as a~consultant.


\subsubsection{Main language consultant}
\label{sec:mainlanguageconsultantmrs}
\largerpage

My Na language teacher is Mrs. Latami Dashilame /\ipa{lɑ˧tʰɑ˧mi˥ ʈæ˧ʂɯ˧-lɑ˩mv̩˩}/. (Her code in the database of speakers of \ili{Naish} languages is F4.) She was born in
1950 into a~family of commoners~-- the majority group among the Na, distinct on the one hand from the chieftain's family, which constituted the nobility, and on the other hand from the serfs. Her birthplace is the hamlet called /\ipa{ə˧lɑ˧-ʁwɤ˧}/,\footnote{For the sake of simplicity, this noun is provided here in surface phonological transcription. Its underlying form is //\ipa{ə˧lɑ˧-ʁwɤ\#˥}//, with a~{floating} High tone. This tonal category is analyzed in \sectref{sec:afloatinghtonewithcomparativeevidencepointingtoitsorigin}.} close to the monastery of Yongning, called
\textit{dgra med dgon pa} in {Tibetan}, a~name rendered in Chinese as \textit{Zhāměisì} \zh{扎美寺}. The
full address is: Yúnnán province, Lìjiāng municipality, Nínglàng Yí
autonomous county, Yǒngníng district, Ālāwǎ 
village (\zh{云南省丽江市宁蒗彝族自治县永宁乡阿拉瓦村}). This place is referred to in this volume as ‘Alawa'.
The founding of this hamlet is recounted in the narrative Elders3 (about the narratives and other online resources, see \sectref{sec:transcribedandtranslatednarrativesandphonologicalmaterials}). My teacher later established a~home of her own in a~neighbouring hamlet, slightly closer to
the road leading to the Yongning marketplace.

My teacher is attached to traditions, closely associated with the teachings of her grandmother, whom
she remembers as an~outstanding character who tactfully managed a~large household. In narratives,
she refers to her grandmother as /\ipa{ə˧si˧}/, ‘great"=grandmother, ancestor of the third
generation’, the {term of address} used by her own children~-- a~way to point her out to the next
generation as a~model. My teacher is considered locally as a~connoisseur of Na customs: in the last two decades or so, villagers experiencing doubts about how a~certain ceremony should be
performed would come to ask her for instructions.

On the other hand, she is well aware of how deeply Na society has been transformed since her childhood, and she does not cling to a~bygone past. She is an~open"=minded character,
gaily deriding in retrospect the prejudices that used to prevail in Na villages. For instance, in an account of the introduction of vegetables
such as courgettes and eggplants in Yongning, she reports that distrustful and indignant villagers would warn, “Don’t eat those: they are grown in shit!” but that these new crops were eventually adopted,
along with traditional Chinese methods for fertilizing soil (narrative: Housebuilding2). In her childhood, my teacher was one of the
actors in a~film about the Na and their unusual family structure “without fathers or husbands” (discussed in Appendix B, \sectref{sec:anthropologicalresearchthefascinationofnafamilystructure}): \textit{‘A-zhu’
  marriage among the Naxi of Yongning}.\footnote{I was not able to access this film. Chinese title: {\kern-3pt}\zh{《永宁纳西族的阿注婚姻》}{\kern-4pt}. Black and white. Duration: about 60~minutes (6~film portions of 10~minutes each). Production date: about 1966. Advisor: Qiū Pǔ
  \zh{秋浦}. Scenario: Zhān Chéngxù \zh{詹承绪} and Yáng Guānghǎi \zh{杨光海}. Director: Yáng
  Guānghǎi \zh{杨光海}. Photography: Yuán Yáozhù \zh{袁尧柱}. Sound recording: Zhào Déwàng \zh{赵德旺}. 
  Animations: Zhèng Chéngyáng \zh{郑成杨}. Narration: Zhōu Qìngyú \zh{周庆瑜}. Summary: “Before Liberation, the Naxi of the people's commune of Yongning, in Ninglang Yi Autonomous County, lived in a feudal society, but they preserved distinctive characteristics of primeval matriarchal societies. They had matriarchal households in which the maternal side constituted the core of the family. They retained a style of marriage (‘A-zhu’ marriage) in which men did not take wives into their family, and women did not marry into another family. This documentary film records this form of matrimony according to the facts.” \textit{Original text:} \zh{“在云南省宁蒗彝族自治县永宁公社的纳西族,解放前处于封建领主社会,但长期以来还保存着原始母系社会特征,保存着以母系为核心的母系家庭,保存着男不娶,女不嫁的“阿注婚姻”。男阿注到女阿注家过夜,晚上来白天走的“半同居”婚姻生活。本片对这种婚姻形式,特点和母系家庭作了如实的记录。”} The film is part of a~series about “ethnic minorities” initiated in 1957: \zh{少数民族社会历史科学纪录片}.} Later, one of her sons became
an~anthropologist, specializing in Na society, and she met a~number of his
colleagues. She witnessed how the Na of Yongning became an~object of curiosity and fantasy, and
how Na culture became folklorized for the promotion of the tourist industry.\footnote{One example among many is a~report done for the {French} tabloid
  \textit{Paris Match}. The magazine's special issue “China is changing” (“La Chine change”, May 2001) contained no fewer
  than ten pages dedicated to the “Mosuo” (pp. 52--61), including an~interview with Latami Dashi and
  his mother Latami Dashilame (p. 60).} Her experiences and reflections shook some of the
beliefs that had been passed on to her by her grandmother, such as Buddhist
faith. While conscientiously going through the prescribed rituals on a~day"=to"=day basis, her belief in Buddhist teachings such as reincarnation was faltering, although without affecting her
commitment to the ideals of benevolence and respect of others. The narratives recorded show her
awareness of the cultural relativity of the waning customs and traditions of the Na, to which she
nonetheless remains attached.

\begin{photofigure}[t]
	\caption{The main language consultant, Mrs. Latami Dashilame (\ipa{lɑ˧tʰɑ˧mi˥ ʈæ˧ʂɯ˧-lɑ˩mv̩˩}), shopping at the Yongning marketplace. Spring 2008.}
	\includegraphics[width=0.7\textwidth]{figures/Ama08355.jpg}
\end{photofigure}

%{\newpage}
Unlike more traditional parents in whose view the monastery was the most prestigious prospect for
boys, she encouraged her children~-- girls as well as boys~-- to study in the Chinese school system,
which she felt was a~better gateway to an~existence free from daily toil in the fields. Her four
children have all found employment outside Yongning, one of them in the county town of Ninglang, two
in Lijiang, and one in faraway Shenzhen (Guangdong). Although she lived continuously in the village
from the time of her birth, and hardly ever left the plain of Yongning (and, indeed, seldom left her village)
until she came over to the city of Lijiang to look after a~newborn granddaughter in 2010, she was
always well aware of the wider world.

Her experience that, beneath the differences in local customs, the human heart is the same everywhere, surfaces in places in the narratives that she agreed to record during the course of our
collaboration. She likes to point out similarities between the situations described in her
narratives and present"=day situations. For instance, mentioning apprentice monks’ hopes of finding
a~good master, she brought out the {analogy} with my study of Na, for which I likewise needed
attentive teachers, as did her grandson at a~university in Kunming. Following her instruction, I
have always called her ‘mother’ (/\ipa{ə˧mɑ˧}/), and have been the grateful object of her affectionate care
throughout my stays. She is a~model of tact, masterfully fine"=tuning relations within the family and
beyond. She has been a~patient and encouraging teacher, despite her declining health and her heavy
workload as a~mother and a~grandmother. While conceivably proud of raising four children under harsh
circumstances, she has a~strong sense of humour and was never inclined to pose as guardian angel,
muse, or Madonna (to cite Baudelaire’s impassioned wording: “l’Ange gardien, la Muse et la Madone”).

\largerpage
She was a~stutterer during her adolescence, but later overcame this difficulty. I may not have
noticed had I not been informed, but I now interpret the rare cases of stuttering that occur in
recorded audio documents as remnants of this earlier difficulty. Also, she is known in the community
as a~person who talks fast. Finally, although she has never suffered from any major
otorhinolaryngological ailments, she has noticed changes in her voice over the years, and is no
longer able to sing the high"=pitched songs of the Na. The reason is partly social: according to
local habits, singing is an~activity for young people. The voices of singers over the age of forty (fifty at
a~push) are considered unattractive, and it is unusual for a~women past fifty to sing songs. On the
occasion when my teacher agreed to sing a~song (understanding that this may be a~useful part of
documenting the language, and that most members of the younger generation do not get to learn
traditional songs anymore), it appeared that lack of practice over the years had made her unable to
perform songs.

She never travelled to Na villages
outside the Yongning plain, such as Labai or the Na villages in Muli county. She can speak a~little
\il{Mandarin!Southwestern}Southwestern Mandarin {Chinese}, and borrows common {Chinese} words when speaking Na, but
her proficiency was very limited until she moved to Lijiang in 2010 and found herself in
a~predominantly Chinese"=speaking environment. Apart from initial vocabulary elicitation, which
was done in Chinese, fieldwork was conducted in Yongning Na, the consultant providing explanations in her own
language, without translation into Chinese. This certainly made the elicitation process slower than
it would have been with a~bilingual consultant, but monolingual fieldwork also has its advantages,
allowing the investigator to develop a~better command of the language.

In addition to phonological materials, the set of texts by consultant F4 grew from one traditional story to a~set of over one hundred monologues (addressed to the investigator) about topics that she selected. The consultant does not view herself as a~skilled performer of oral literature: she was not tutored to become the depositor of a~heritage of oral traditions, a~role reserved to men in the local tradition of ritual practitioners (called \ipa{dɑ˧pɤ˧}, and related to the Naxi \ipa{to˧mbɑ˩}: see \citealt{fangetal1995,lidazhu2015}). She gave in to the wishes of the investigator, a~linguist who was greedy for narratives and audio recordings in general. Knowing that I have a~side interest in history, ethnicity and sociology, she talked about life in Yongning in “the old times”, a~conveniently vague epoch which includes her childhood and youth. The result is a~set of monologues that belong to ordinary, casual speech as opposed to codified performance of oral texts (to use a~distinction set out in \citealt{dournes1990}). On the other hand, she is aware of how much is at stake when narrating {oral history}, and how narratives about the past can be used. (About the didactic and mobilizational power of personal stories in “an {oral history} regime”, see \citealt[100-105]{bulag2010}.) She therefore exerts caution to steer clear of topics that she feels to be sensitive. This results in a~degree of evenness and uniformity of style that entails strong limitations in terms of speaking styles (the recordings are monologues, and are predictably less lively than dialogues) as well as in terms of contents. No claims are made of having achieved a~balanced corpus in any sense. From the point of view of morphotonology, the main concern was to have sufficiently abundant materials to be able to confirm some (if not all) of the elicited tonal combinations by matching them with data which, in the phonetician's admittedly crude categories, belong to “spontaneous speech” as opposed to phonological elicitation sessions.

\subsubsection{Other language consultants}
\label{sec:otherlanguageconsultants}

During the first field trip (2006), as work with consultant F4 began, I reflected on possibilities to extend the work to other speakers. I had in mind the textbook arguments for working with several speakers. 

\begin{quotation}
  Data from varied sources can guard against distortions resulting from dressage, the observer’s
  paradox, faulty questioning, or prescriptive influences of one individual’s idiolect. Working
  with several speakers will provide the researcher with points of comparison so that he or she can
  learn to distinguish between reliable and unreliable data.~\citep[180–181]{Chelliahetal2011}
\end{quotation}

In the family where I stayed, two members were living in Yongning all the year round: F4, and
a~daughter"=in"=law: the wife of her second son. Her daughter"=in"=law (speaker code: F5), born in 1973,
accepted to act as language consultant for basic elicitation, while declining to record any
continuous texts such as narratives or dialogues, explaining that she did not feel up to the task. Unlike F4,
who after a~couple of days eliciting vocabulary agreed to record a~narrative, F5 maintained her
initial decision not to record anything other than short responses to my questions, except for a~few
short songs recorded in 2007.

Both F4 and F5 had relatively little spare time, so I would work with one or the other depending on who was available at the time. One day in December 2007 when both were busy, F5 asked a~niece to “replace”
her for a~work session. This offered an~opportunity to get insights into the speech of the younger
generation. This speaker, F6, was born in 1987. As a~high school student in the Ninglang county
seat, she only came back home for holidays. Vocabulary elicitation showed that she knew few words,
and that her Na phonological system was highly simplified. These observations are reported in
a~joint book chapter with Latami Dashi: “A description of endangered phonemic oppositions in Mosuo
(Yongning Na)” (\citealt{michaudetal2011}; Chinese translation: \citealt{mikemike-Alexis-michaudetal2010}).
{\largerpage}

In 2008, two more speakers were recorded. The first was Mr.\ /\ipa{ho˧dʑɤ˧tsʰe˥}/ (Chinese: He Jiaze \zh{何甲泽}), hereafter M21, born in 1942. He was a~retired cadre (\textit{gànbù} \zh{干部}), and had lived two
years in Kunming and three in Yongsheng. But elicitation tasks proved
challenging for him, due to some hearing difficulties: he reported full deafness in
one ear and highly reduced sensitivity in the other. His experience of various dialect areas also
made him more changeable in linguistic behaviour than the investigator would have
wished. Specifically, it did not prove feasible to elicit consistent tone patterns for {compound}
nouns from M21, as some patterns that had been elicited in one session were dismissed in another, only to be
reasserted with full confidence at a~later point. Later, M21’s youngest son, /\ipa{ɖɯ˩ɖʐɯ˧}/
(Chinese: He Duzhi \zh{何独知}), born in 1974, kindly agreed to participate in the linguistic
investigation. His command of Na appears comparable to that of F5. Like F5, M21 and M23 declined to
record anything except vocabulary, phrases and sentences. %This is interpreted as an~indirect indication that they received some exposure to the oral traditions, and still take them seriously enough to consider that
%they are not to be treated casually. The speakers who accept most readily to tell stories are not
%necessarily those who are most familiar with them. On the contrary: some speakers who have a~strong
%footing in another language and culture (in the case of \ili{Naish}: Chinese) can tell simplified versions
%of traditional stories all the more easily as they mean less to them.

\subsection{Elicitation methods}
\label{sec:elicitationmethods}
\largerpage[-1]
\subsubsection{Examination of transcribed texts and direct elicitation}
\label{sec:examinationoftranscribedtextsanddirectelicitation}
\largerpage

“Texts are the lifeblood of linguistic fieldwork. The only way to understand the grammatical
structure of a~language is to analyse recorded texts in that language” \citep[22]{Dixon2007}. Following classical methods in linguistic fieldwork, observations from continuous speech (in the
case of the present study, mostly narratives) are verified and further investigated through
elicitation. A~fair amount of transcribed narrative has been collected: more than four hours in total. But
a~considerably greater amount would be necessary to be able to study the language’s tonal grammar on
the basis of these texts alone. Not all possible tonal combinations in {compound} nouns are found in
these texts, and no amount of continuous speech would be enough to obtain all the combinations of
numerals from 1 to 100 with the various tonal categories of classifiers, as required for the study
of {numeral}"=plus"=classifier phrases (Chapter~\ref{chap:classifiers}). Systematic elicitation was therefore used to
investigate one area after another of the tonal grammar of Yongning Na.

Larry Hyman, reflecting on his study of the tones of another {Sino"=Tibetan} language (Thlangtlang
Lai), makes the following observation:
%LaTeX symbol for maths: × is composed as $\times$
\begin{quotation}
  Clearly the speaker had never heard or conceptualized noun phrases such as “pig’s friend’s
  grave’s price”, “chief’s beetle’s kidney basket”~({\dots}). It would not impress any psychologist, and
  it would definitely horrify an~anthropologist. ({\dots}) However, when I need to get 3×3×3×3=81 tonal combinations to test my rules, the available data may be limited, or the
  language may make it difficult to find certain tone combinations. I am personally thankful that
  speakers of Kuki"=Chin languages are willing to entertain such imaginary notions. It is most
  significant that the novel utterances are produced with the appropriate application of tone
  rules.~\citep[34]{hyman2007a}
\end{quotation}

%Command \noindent added to avoid having an indent. Proofreader suggestion: since this sentence continues the argument, it is better not to indent. 
{\noindent}Using this method, L.\ Hyman completed elicitation of data on Thlangtlang Lai tone within six hours
\citep[9]{hyman2007a}. Such swift progress is possible when the linguist is fortunate to work with
consultants possessing high metalinguistic abilities, who enter into the linguistic reasoning, and
collaborate with the linguist as colleagues. Famous cases include François Mandeville, a~speaker
of Chipewyan (Athabaskan family) who “possessed the extraordinary ability to dictate texts and to explain forms with
lucidity and patience” \citep[132]{li1964}. 

In the case of Yongning Na, the relationship with the consultants was somewhat different. In the
course of our collaboration, they developed an~understanding of a~linguist’s interests and aims. In
particular, dozens of hours working together made the main consultant familiar with the investigator’s body
language, so she would understand a~repetition"=beseeching glance upward from the laptop screen, or
volunteer an~explanation when a~lengthy pause suggested that the investigator was experiencing
a~doubt. On the other hand, the main consultant did not develop metalinguistic intuitions beyond the
notion of full \isi{homophony} between two words. She did not reach the stage where she would indicate that
the tone of a~word is like the tone of another word, much less name a category by letter or number. The main consultant did not become a~collaborator in the sense of studying tools for transcribing her language and discussing the inventory and analysis of phonemes and tonemes, i.e.\ gaining training in linguistics. So she did not provide feedback on analytical choices or on the transcriptions. This clarification appears useful, in light of the recommendation that “linguists should make it a~practice to explicitly
indicate whether the tonal categories have been recognised by native speaker consultants
and whether words cited have been confirmed in those categories by native speakers, or
whether those categories are the result of the linguist’s own analysis” \citep[638]{morey2014}. I~would be delighted to work with native speakers as fellow linguists, and hope that there will be opportunities for this in future.\footnote{I was fortunate to co"=supervise the M.A. thesis of a~native speaker who conducted a~comparative study \citep{a2016} on the tone systems of her own dialect~-- that of the village of Shèkuǎ \zh{舍垮}~-- and of Ālāwǎ \zh{阿拉瓦}, studied in the present volume. This was of the highest interest for me, of course. But A Hui's work did not include verification of the analyses that I propose for Ālāwǎ: A Hui worked out the tone categories of her own dialect, and she did a~comparison with transcribed recordings that I provided, without questioning my notations. The two systems are fairly different: that of Shèkuǎ is based on two levels only (High and Low), as against High, Mid and Low in Ālāwǎ.}


\largerpage
\subsubsection{The issue of cross"=speaker differences}
\label{sec:theissueofcrossspeakerdifferences}

One of the findings from the systematic study of {compound} nouns (reported in Chapter~\ref{chap:compoundnouns}) was the high
degree of cross"=speaker difference.

It is a~general observation that tone is highly susceptible to \is{variation!dialectal}dialectal variation, and in view of
existing reports it appears that the greater Yongning area is the part of the {Naish}"=speaking
area that has the richest tone systems and the greatest dialectal diversity. In this light,
differences observed within the same village, and even within the same family, did not come as
a~huge surprise.

Another factor of diversity is the gap between age groups created by ongoing language shift to
Chinese. F4’s four children are more proficient in Chinese than in Yongning Na. All four left Yongning to
work. Two married Han Chinese spouses without any command of Na; one married a~Na spouse from
a~different dialect area (\ipa{lo˧gv̩˩}; in Chinese: Běiqúbà \zh{北渠坝}) and difficulties of mutual comprehension led the wedded pair to communicate
in Chinese instead; one married a~Na spouse from Yongning (F5) but works in Shenzhen (Guangdong) and seldom returns
home. The generation of her four grandsons and granddaughters has even less command of Na, even
though it was F4 who took charge of them in their early years. But there are also notable
differences in tonal terms between the speech of F4 and that of her
daughter"=in"=law F5, despite their living under the same roof. Documenting the tone system of several
speakers is not simply a~matter of verifying data: the description is to be conducted
separately for each speaker, only comparing the results as a~second step. Another difficulty was
that transcribed texts are necessary in order to confirm at least part of the patterns obtained
through systematic elicitation, but speakers F5, M21 and M23 declined to record narratives. Had they agreed, arriving at a~sizeable collection of texts for each of the four speakers would have represented
a~formidable amount of work~-- not to mention the difficulty for the investigator of
keeping the four systems distinct.
% avoiding unwarranted carry"=over of transcription habits developed
% when working on data from one speaker.


\subsubsection{A dilemma: tonological depth vs.\ sociolinguistic breadth}
\label{sec:adilemmabreadthofcoverageofthetonesystemvsbreadthofsociolinguisticcoverage}

Clearly, it did not appear feasible to explore all areas of the tone system with the same degree of
detail for four speakers. This led to the following dilemma: either limiting the extent of
investigation (for instance, focusing on compound nouns, or on numeral"=plus"=classifier phrases)
but eliciting data from a~broad sample of speakers, to arrive at adequate sociolinguistic coverage;
or extending the investigation to more parts of the linguistic system, working towards a~complete
picture of the tonal grammar of one speaker, with occasional extension of
the investigation to other speakers.

The second option was preferred: attempting as complete a~description as possible of the entire
linguistic system. Work with consultant F4 appeared more promising because of her much stronger
proficiency in Na than the two younger consultants (F5 and M23) and of her relatively homogeneous
linguistic lifecourse, as opposed to M21’s long years of practice of various dialects of Na and of
Chinese. Basing the work primarily on data carefully verified with F4 appeared to be a~reliable
starting"=point for later comparative work, including cross"=speaker and cross"=dialect studies. 

The ultimate research goal, viewed as a~collective endeavour, is to document in great
detail the synchronic tone systems of a~number of research locations in the Na"=speaking area, and, on
this solid empirical basis, to conduct comparisons and gradually reconstruct the history of the
evolution of tone systems. Ideally, this would lead to a~complete account of the origin and
development of tone systems in {Naish}, shedding light on all the stages that led from a~non"=tonal
stage to each of the present"=day varieties. This is clearly a~long"=term endeavour; the first task to
be addressed is to attempt a~description of the tonal system of one dialect in its full synchronic
complexity.

The analyses of Yongning Na tone reported in the present volume are therefore based on data from
consultant F4, unless otherwise mentioned.


\subsection{Online materials}
\label{sec:onlinematerials}


\subsubsection{Transcribed and translated narratives and phonological materials}
\label{sec:transcribedandtranslatednarrativesandphonologicalmaterials}

A guiding principle in the present research is that a~close association between documentation and
research is highly profitable to both. If it is true, as \citet{Whalen2004} puts it,
that “the study of endangered languages has the potential to revolutionize linguistics”, and that
“the vanguard of the revolution will be those who study endangered languages”, then it is all the
more unfortunate that “enormous amounts of data~-- often the only information we have on
disappearing languages~-- remain inaccessible both to the language community itself, and to ongoing
linguistic research” (\citealt{thiebergeretal2006}; see also
\citealt{woodbury2003,woodbury2011}). “[L]anguage documentation as a~paradigm in linguistic
research” has been described as having benefits such as the following: 
\begin{quotation}
\begin{enumerate}[label=(\roman*)]
	\item  making analyses accountable to the primary material on which they are
based; 

	\item  providing future researchers with a~body of linguistic material to analyse
in ways not foreseen by the original collector of the data; and, equally importantly,

	\item  acknowledging the responsibility of the linguist to create records that can be
accessed by the speakers of the language and by their descendants. \citep[1]{thiebergeretal2016}
\end{enumerate}
\end{quotation}

The necessity of making primary data available was emphasized with specific reference to the analysis of tone at a~symposium on “Cross-linguistic studies of tonal phenomena”:
 
\begin{quotation}
The point to be made is extremely simple: declare the status of the primary data for what it is, and allow an evaluation of the data and its subsequent interpretation to take place in view of the declared status of the data. ({\dots}) I~in no way imply that the wealth of impressionistic data which we have on tonal phenomena are inherently wrong or that they misrepresent the situation in any particular language. ({\dots}) What I do, however, argue for is an unashamed scientific approach to the handling of tonal data, abiding with generally accepted criteria of scientific practice. This will not only raise the credibility of analyses, but may even lead to objective empirical testing of particular theories. \citep[366]{roux2001}
\end{quotation}

Accordingly, the recordings conducted in Yongning have been made freely available
online, document after document, since 2011. The recordings are mostly
narratives and lexical or phonological elicitation sessions. They are accompanied by metadata
(information about the recordings), and, to the greatest extent possible, by full transcriptions
and translations. A~list of documents is provided in the Abbreviations section.

The data is hosted by the Pangloss Collection\footnote{Address of the interface as of 2017: \url{http://lacito.vjf.cnrs.fr/pangloss/index\_en.html}}
\citep{michailovskyetal2014}, a~language archive developed at the
research centre \textit{Oral} \textit{Tradition}: \textit{Languages} \textit{and}
\textit{Civilizations} (LACITO) of the {French} National Centre for Scientific Research (CNRS) since
1994. The goal of this archive is to preserve and disseminate oral
literature and other linguistic materials in (mainly) endangered or poorly documented languages,
giving simultaneous access to sound recordings and text annotation. 
%“Paired with advances in digital media, accessible corpora of annotated language data not only allow for verification of current analyses; they will, in time, provide answers to as yet unknown research questions, as well as providing a~cultural record of value to the broader community” \citep{thiebergeretal2016}.

For some of the Yongning Na recordings, an~electroglottographic signal was collected simultaneously
with the audio. The electroglottographic signal allows for
high"=precision measurement of the voice’s fundamental frequency, as well as of other glottal
parameters.\footnote{About electroglottography, see the initial report of the invention: \citet{fabre1957};
a~synthesis: \citet{baken1992}; some caveats: \citet{orlikoff1998}; discussions about parameters
that can be measured: \citet{henrichetal2004b}, \citet{michaud2004b}; and applications to the study of specific linguistic issues,
e.g.~\citet{brunelleetal2010} and \citet{kuangetal2014}.} Documents that comprise
an~electroglottographic signal are accompanied by a~special icon in the list of resources (\figref{fig:autolist}). 

\begin{figure}%[t]
	\includegraphics[width=\textwidth]{figures/AutoList.png}
	\caption{Screen shot of the beginning of the list of Yongning Na resources in the Pangloss Collection.}
	\label{fig:autolist}
\end{figure}

The list of resources is generated automatically from the archive's catalogue. As of 2017, it was arranged by date of deposit. This does not make it easy to locate a~document in this list, or to get a~feel for the state of the corpus. A~description of the corpus is therefore provided as a~static HTML document. It is available by clicking on the language name (‘Na’) at top of page. This description is arranged by dialect and by type of contents. Explanations about data collection are also provided. A~passage from this presentation (which is maintained and gradually improved over the years) is shown in \figref{fig:static}.
%
\begin{figure}  %[t]
\includegraphics[width=\textwidth]{figures/ListOfResources.png}
\caption{Static HTML page presenting the Na resources.}
\label{fig:static}
\end{figure}
%
 For instance, the recordings of {numeral}"=plus"=classifier phrases which appear as a~list towards the bottom of \figref{fig:autolist} are arranged in table form on the presentation shown in \figref{fig:static}, making it much easier to see at a~glance which recordings are available. \figref{fig:apassagefromoneofthedocumentsasdisplayedonthewebinterface} shows
a~passage from one of the documents as displayed on the web interface: transcription, translation, and time"=aligned audio. The original files can also be downloaded (no login is required). 



\begin{figure}[t] 
\includegraphics[width=\textwidth]{figures/SampleDisplay.png}
\caption{A passage from one of the documents as displayed on the web interface: transcription, translation, and time"=aligned audio.}
\label{fig:apassagefromoneofthedocumentsasdisplayedonthewebinterface}
\end{figure}



The Na documents are archived with provisions for long"=term conservation, and will continue to be accessible despite future changes
that may take place in the Uniform Resource Locator of the Pangloss Collection’s web interface. Stable internet links to directly access a~specific location within a
text are currently under development at the archive; the aim is “to offer readers the means to interact instantly with digital versions of the primary data, indexed by transcripts'' \citep{thieberger2009}. Seamless navigation between grammars and data is clearly the way to go, providing one"=click links to the texts where cited examples are taken from. It is hoped that, by the time the next edition of this volume is released, tools for resolving a~multimedia document's identifier will be all set up and working, allowing for links from the digital book that will direct the user straight to the relevant passage in the online data. In the meantime, it did not
appear advisable to provide hyperlinks to the archive's current interface, because URLs are likely to
change in future.

Embedding audio excerpts of the cited examples in the PDF file of this volume would have been a~technical possibility, but this would not have reached the goal of connecting the analyses to the data: allowing the reader to navigate between the book and the online documentation where the examples can be examined in context. At present, it remains necessary for users to go
to the Pangloss Collection's online interface, and to locate the document at issue either on the internet page presenting the Na
documents or
in the automatically generated list of documents. 

The availability of these audio and electroglottographic data with synchronized transcriptions makes
it possible for interested persons to get a~feel for the data; it also allows for
further research into a~broad range of phonetic topics. Given the amount of time that is necessary
to produce a~state"=of"=the"=art experimental \is{experimental phonetics}phonetic investigation, it is simply impossible for
linguists who are working at the description of an~entire language (or of several languages) to
launch into a~phonetic study to substantiate and refine each of their observations; on the other
hand, it appears feasible to collect a~sufficient amount of data for interested colleagues to
conduct such a~study. To take the example of recorded data about {numeral}"=plus"=classifier phrases
(analyzed in Chapter~\ref{chap:classifiers}), here are two examples of phonetic phenomena that could be studied in the future
on the basis of the recorded Yongning Na data.





%\begin{enumerate}[label=(\roman*)]
%\item The implementation of tone. The electroglottographic signal has not been exploited so
%  far, except for its occasional use in auditory verification (the pitch can be clearer when
%  listening to the electroglotto"=graphic signal than when listening to the audio). This signal could
%  serve in future for a~phonetic study of the implementation of tone in Yongning Na. There is
%  a~large gap between phonological representations in terms of sequences of level tones, on the one
%  hand, and observed fundamental frequency curves, on the other: “both F\textsubscript{0} height and F\textsubscript{0} velocity are
%  relevant parameters ({\dots}) even for the simplest \is{level tones}level tone languages” \citep[1]{Yu2010}. A
%  study of the implementation of tone sequences in Na would be a~useful addition to the existing
%  literature on contextual tonal \isi{variation} and segmental effects on tone, as studied e.g.~by as
%  studied e.g.~by \citet{abramson1979a}, \citet{gsell1985} and \citet{gandouretal1992} for
%  \ili{Thai}, and \citet{xu1997,xu1998} for \ili{Mandarin}.
%\item The weakened (hypo"=articulated) realization of repeated words. When a~consultant
%  pronounces a~sequence of \is{numerals}numeral"=plus"=classifier phrases, such as /\ipa{ɖɯ˧-kʰwɤ˥ {\kern2pt}|{\kern2pt} ɲi˧-kʰwɤ˥ {\kern2pt}|{\kern2pt}
%    so˩-kʰwɤ˩˥ {\kern2pt}|{\kern2pt} ʐv̩˧-kʰwɤ˧}/ (‘one piece, two pieces, three pieces, four pieces{\dots}’), the tone
%  of the classifier changes (High, High, Low"=to"=High, Mid{\dots}) but its consonants and vowels do
%  not. As a~result, the speaker’s attention focuses on the realization of the new information
%  (essentially: the correct tone sequence for the phrase); in terms of the continuum from
%  hyper"=articulation to hypo"=articulation \citep{lindblom1990}, the classifier is
%  hypo"=articulated. Specifically, the unvoiced lateral /\ipa{ɬ}/ in classifiers such as /\ipa{ɬi˧}/
%  ‘month’ and /\ipa{ɬi˩}/ ‘armspan’ is occasionally realized as voiced, despite the existence of
%  a~voicing opposition between /\ipa{ɬ}/ and /\ipa{l}/ in Na. Pursuing such observations would shed
%  light on the field of allophonic dispersion of Na phonemes. Due to the nature of the corpus,
%  numerous tokens of each \is{numerals}numeral and classifier are available, offering a~good basis for
%  statistical treatments.
%\end{enumerate}

\subsubsection*{(i) The implementation of tone}
The electroglottographic signal has not been exploited so
	far, except for its occasional use in auditory verification (the pitch can be clearer when
	listening to the electroglottographic signal than when listening to the audio). This signal could
	serve in future for a~phonetic study of the implementation of tone in Yongning Na. There is
	a~large gap between phonological representations in terms of sequences of level tones, on the one
	hand, and observed fundamental frequency curves, on the other: from a~phonetic point of view, “both F\textsubscript{0} height and F\textsubscript{0} velocity are
	relevant parameters ({\dots}) even for the simplest level tone languages” \citep[1]{Yu2010}. A
	study of the implementation of tone sequences in Yongning Na would be a~useful addition to the existing
	literature on contextual tonal \isi{variation} and segmental effects on tone, as studied by \citet{abramson1979a}, \citet{gsell1985} and \citet{gandouretal1992} for
	\ili{Thai}, and \citet{xu1997,xu1998} for \ili{Mandarin}.

\subsubsection*{(ii) The weakened (hypo"=articulated) realization of repeated words}
\largerpage[2]
When a~consultant
	pronounces a~sequence of {numeral}"=plus"=classifier phrases, such as /\ipa{ɖɯ˧-kʰwɤ˥ {\kern2pt}|{\kern2pt} ɲi˧-kʰwɤ˥ {\kern2pt}|{\kern2pt}
		so˩-kʰwɤ˩˥ {\kern2pt}|{\kern2pt} ʐv̩˧-kʰwɤ˧}/ (‘one piece, two pieces, three pieces, four pieces{\dots}’), the tone
	of the classifier changes (e.g.~High in /\ipa{ɖɯ˧-kʰwɤ˥}/ and /\ipa{ɲi˧-kʰwɤ˥}/, Low"=to"=High in /\ipa{so˩-kʰwɤ˩˥}/, and Mid in /\ipa{ʐv̩˧-kʰwɤ˧}/) but its consonants and vowels do
	not. As a~result, the speaker’s attention focuses on the realization of the new information
	(essentially: the correct tone sequence for the phrase); in terms of the continuum from
	hyper"=articulation to hypo"=articulation \citep{lindblom1990}, the classifier is
	hypo"=articulated. Specifically, the unvoiced lateral /\ipa{ɬ}/ in classifiers such as /\ipa{ɬi˧}/
	‘month’ and /\ipa{ɬi˩}/ ‘armspan’ is occasionally realized as voiced, despite the existence of
	a~voicing opposition between /\ipa{ɬ}/ and /\ipa{l}/ in Na. Pursuing such observations would shed
	light on the field of allophonic dispersion of Na phonemes. Due to the nature of the corpus,
	numerous tokens of each {numeral} and classifier are available, offering a~good basis for
	statistical treatments.

Importantly, the documents are also open to entirely different uses, including aesthetic enjoyment of a~voice captured through the wonder of high"=fidelity recording, and preserved unchanged through the magic of digital technology. I for one am not insensitive to the luxury of listening to the Yongning Na recordings at leisure. After a~recording session is over, the qualms and concerns of fieldwork recede, leaving room for ``the fecund miracle of communication within solitude'', to cite Proust's eulogy of reading.\footnote{\textit{Original text:}~la lecture, dans son essence originale, dans ce miracle fécond d'une communication au sein de la solitude (\dots) (\textit{Journées de lecture}, in \textit{Pastiches et mélanges}, Paris: Gallimard, 1919, p. 257).}


\subsubsection{Dictionary}
\label{sec:dictionary}

A dictionary of Yongning Na \citep{michauddict2015} is available online (i)~as an~online dictionary in HTML format, (ii)~as a~PDF document, and (iii)~in database format. Entries and
examples have translations into {English}, Chinese and {French}. There are two supplementary files in
the version deposited in the HAL archive.
\begin{enumerate}[label=(\roman*)]
\item The file NaDictionary.xml contains the entire database in XML format following the ISO
  standard LMF \citep{francopoulo2013}.
\item The file NaDictionary.txt contains the entire database in SIL's Toolbox format (MDF).
  This is the source file on which the editing is done.
\end{enumerate}
This dictionary is conceived of as a~work in progress: successive versions will be released, probably
every two years or so. Hosting in the HAL preprint archive guarantees long"=term preservation.


\subsubsection{Bibliography}
\label{sec:bibliography}

An online bibliography of {Naish} studies (studies about {Naxi}, Na, {Laze} and related languages) was
started in 2015, as a~Zotero group \citep{duong2010zotero} called ‘{Naish} languages’. 
% References will gradually be
% added and enriched to provide multilingual information: for Chinese"=language references, the author
% name will be provided in Chinese characters, as well as in romanized Chinese (\textit{Pinyin}
% transcription); and titles and other relevant information will be translated into {English}. 
The
emphasis is on linguistics, but the team of contributors would also like to include
ethnological/{\allowbreak}anthropological, historical, and sociological work. Any visitor can
consult this bibliography; people interested in contributing to its enrichment and maintenance are
welcome to get in touch.


\section{A grammatical sketch of Yongning Na}
\label{sec:sketch}

This last introductory section is a~quick grammatical sketch of Yongning Na, presenting general properties of the language such as basic {word order} and the structure of noun and verb phrases, which serve as the backdrop to the discussion of morphotonology in the following chapters. A~more substantial sketch is found in \citet{lidz2016}.

%\subsection{Morphology}
%\label{sec:morphology}

There is no inflectional morphology in Na. Suppletion is only observed for ‘to go’: {nonpast} /\ipa{bi˧\textsubscript{c}}/, as in (\ref{ex:alone}), {past} /\ipa{hɯ˧}/, as in (\ref{ex:gone}), {past perfective} /\ipa{hɤ˩\textsubscript{a}}/, as in (\ref{ex:Caravans35went}), and {imperative} /\ipa{hõ˧}/, as in (\ref{ex:imperat}). Note that the {nonpast} form in (\ref{ex:alone}) is translated as a preterite, a tense commonly used for narratives in written {English}.
 
\begin{exe}
	\ex
	\label{ex:alone}
	\ipaex{ɖɯ˧-v̩˧ lɑ˧ \textbf{bi˥} {\kern2pt}|{\kern2pt} pi˧-dʑo˩ {\kern2pt}|{\kern2pt} ʐv̩˩-kv̩˩ tɕɯ˥-kv̩˩ mæ˩!}\\
	\gll ɖɯ˧-v̩˧						lɑ˧		\textbf{bi˧\textsubscript{c}}			pi˥		-dʑo˥				ʐv̩˩-kv̩˩					tɕɯ˧˥		-kv̩˧˥		mæ˧\\
	one-\textsc{clf}.individual		only	\textbf{to\_go.\textsc{nonpast}}		to\_say		\textsc{top}		four-\textsc{clf}	to\_lead	\textsc{abilitive}	\textsc{obviousness}\\
	\glt ‘If only one [man] went, he could lead four [horses]! / If a man went alone, he would lead up to four horses!’ (Caravans.119) 
\end{exe}

%\Hack{\newpage}

\begin{exe}
  	\ex
  	\label{ex:gone}
  	\ipaex{ɻ̍˩ʈʂʰe˧-ɖɯ˩mɑ˩ {\kern2pt}|{\kern2pt} tsʰi˧ɲi˧ {\kern2pt}|{\kern2pt} ə˧tse˧ \textbf{hɯ˧}-ɻ̍˩?}\\
  	\gll ɻ̍˩ʈʂʰe˧-ɖɯ˩mɑ˩	tsʰi˧ɲi\#˥	ə˧tse\$˥							\textbf{hɯ˧\textsubscript{c}}						-ɻ̍˩\\
  	given\_name				today		\textsc{interrog}:why		\textbf{to\_go.\textsc{pst}}		\textsc{inceptive}\\
  	\glt ‘Why has Erchei Ddeema gone away today? / How come Erchei Ddeema has left the house?’ (BuriedAlive2.14)
\end{exe}
  
\begin{exe}
 	\ex
 	\label{ex:Caravans35went}
 	\ipaex{“ɳæ˧=ɻ̍˩-se˩ {\kern2pt}|{\kern2pt} ə˧mv̩˩ le˩-ʝi˩{$\sim$}ʝi˩-ze˩! {\kern2pt}|{\kern2pt} se˧kʰɯ˩-ʁo˩ni˩ le˩-po˩ ʝi˩{$\sim$}ʝi˩-ze˩!” {\kern2pt}|{\kern2pt} pi˧-kv̩˩ mæ˩, {\kern2pt}|{\kern2pt} ho˧di˧ \textbf{hɤ˧}-dʑo˥!}\\
 	\gll ɳæ˧=ɻ̍˩		-se˧		ə˧mv̩˩				le˧-				ʝi˧\textsubscript{c} {$\sim$}	-ze˧				se˧kʰɯ˩-ʁo˩ni˩			le˧-				po˧˥			ʝi˧\textsubscript{c} {$\sim$}	-ze˧			pi˥							-kv̩˧˥					mæ˧							ho˧di˧		\textbf{hɤ˩\textsubscript{a}}		-dʑo˧\\
 	\textsc{2pl}	\textsc{top}	elder\_brother	\textsc{accomp}		to\_come	\textsc{red}	\textsc{pfv}	satin\_headdress		\textsc{accomp}	to\_bring		to\_come			\textsc{red}	\textsc{pfv}	to\_say		\textsc{abilitive}	\textsc{obviousness} 	Sichuan		\textbf{to\_go.\textsc{pst.pfv}}		\textsc{prog}\\
 	\glt ‘When [one's brothers] had gone away [on a caravan] to Sichuan, [people in the village] would say: “You, your brother is going to come back! [He] will bring back a~satin headdress [for you]!”~’ (Caravans.35)
\end{exe}
 
\begin{exe}
	\ex
	\label{ex:imperat}
	\ipaex{no˧ {\kern2pt}|{\kern2pt} wɤ˩˥ {\kern2pt}|{\kern2pt} tsʰo˧qʰwɤ˩ \textbf{hõ˩}!}\\
	\gll no˩			wɤ˩˥			tsʰo˧qʰwɤ˩		\textbf{hõ˧}\\
	\textsc{2sg}	again			tonight			\textbf{to\_go.\textsc{imperative}}\\
	\glt ‘Go again tonight!’ (Reward.63)
\end{exe}
 
\is{reduplication}Reduplication of verbs and adjectives is widely used to convey reflexivity and intensification (see \sectref{sec:reduplication}). 

In the lexicon, a major driving force towards the creation of disyllables is the pressure of \isi{homophony}, caused by the dramatic \isi{phonological erosion} of the {Naish} languages as compared with proto"={Sino"=Tibetan}. For instance, [{\kern1.3pt}\ipa{ʝi˧}] can mean ‘jar’, ‘ox’, ‘to do’, ‘to draw’ and ‘to inform’. ‘Jar’ is generally disyllabic: [{\kern1.3pt}\ipa{ʝi˧mi˧}], adding the syllable for ‘mother’, which has grammaticalized uses as a~female suffix and an augmentative suffix (see \sectref{sec:thegendersuffixes}). In that instance, the disyllabic form simply replaces the \is{monosyllables}monosyllable. 

%\subsection{Syntax}
%\label{sec:syntax}

\subsection{Word order}
\label{sec:wordorder}
\is{word order|textbf}

Word order is S+O+V, or, to be more accurate: “[i]n unmarked, non-idiomatic,
pragmatically neutral constructions, subject-object-verb {word order} is
most common” \citep[48]{lidz2007}. Adverbials can appear at various places. Agent marking is not obligatory: “agents are unmarked more frequently than
they are marked, both in conversation and narrative” \citep[51]{lidz2011}. Like the {agent} adposition /\ipa{ɳɯ˧}/, the {dative} \is{suffixes}suffix /\ipa{-ki˧}/ is optional. These morphemes allow for a~deviation from standard {word order}; they also serve purposes of disambiguation, as in (\ref{ex:teachyou}), where the presence of a noun phrase marked as {dative} clarifies that the verb /\ipa{so˩\textsubscript{a}}/ is to be understood in its meaning of ‘to teach’, and not in its meaning of ‘to study; to imitate’, since the latter's object cannot take a~{dative} particle. The context of (\ref{ex:teachyou}) is the following: the speaker saw me sitting ready for an elicitation session, silently hoping that someone could give me some of their time. There was a real uncertainty as to who that would be, as I was working with two consultants within the household at the time. No subject is indicated in the first part of the sentence (‘After feeding the pigs’), as it is obvious from the situation that the feeding is done by the person who speaks. On the other hand, the explicit mention of both semantic roles in the second part of the sentence thoroughly answers the concerns of its addressee.

\begin{exe}
	\ex
	\label{ex:teachyou}
	\ipaex{bo˩-hɑ˧ {\kern2pt}|{\kern2pt} le˧-ki˧ le˧-se˩-ze˩, {\kern2pt}|{\kern2pt} njɤ˧ ɳɯ˧ {\kern2pt}|{\kern2pt} no˧-ki˧ {\kern2pt}|{\kern2pt} nɑ˩ʐwɤ˧ so˩-bi˩!}\\
	\gll bo˩˧	hɑ˥		le˧-	ki˧\textsubscript{a}			le˧-		se˩\textsubscript{a}		-ze˧		njɤ˩				ɳɯ˧		no˩					-ki˧			nɑ˩ʐwɤ˥				so˩\textsubscript{a}						-bi˧\\
	pig		food		\textsc{accomp}		to\_give	\textsc{accomp}	to\_finish			\textsc{pfv}	\textsc{1sg}	\textsc{a}	\textsc{2sg}	\textsc{dat}	Na\_language		to\_teach/to\_study		\textsc{imm.fut}\\
	\glt ‘After feeding the pigs, I will teach you Na (myself) / I shall come over for a language lesson!’ (Sentence transcribed on the fly in 2008)
\end{exe}

As is well-attested across the Himalayas, nominals, including pronouns, need not appear if they can be understood from the discourse context. This applies to any definite argument of the verb, the head of a relative clause, or the head of a complex noun phrase. Another areal characteristic is that topic markers are the morphemes with the highest frequency of occurrence.

\subsection{Tense, aspect and modality}
\label{sec:tenseaspectmodality}
Tense, aspect and modality are expressed through a~host of post"=verbal particles and some pre"=verbal particles, such as {durative} /\ipa{tʰi˧}-/, {accomplished} /\ipa{le˧}-/, {perfective} /-\ipa{ze˧}/ and {experiential} /-\ipa{dʑɯ˧}/. ‘To go’, /\ipa{bi˧\textsubscript{c}}/, is grammaticalized to express immediate future, and /\ipa{se˩\textsubscript{a}}/ ‘to complete’ as a completion marker (both are placed post-verbally). Na stands at a~great typological distance from a~language such as Wolof, where one predicative marker associated to the verb encodes the greater part of the grammatical content \citep{creisselsetal1998, guerin2015}. In Na, as in Loloish (\ili{Yi}) languages such as Lahu \citep{matisoff1973a} or Lalo \citep{bjorverud1998}, particles have rich combinatory potential. Combinations among particles create a~wealth of highly dialect"=specific nuances; these nuances are best appreciated by examining examples in context. 


\subsection{Question formation}
\label{sec:qformation}
In yes/no questions, the verb is preceded by an interrogative particle /\ipa{-ə˩}/, as illustrated in (\ref{ex:areyoumydaughter}). In \textit{wh}-questions, exemplified by (\ref{ex:whogoeswithmother}), the interrogative \is{pronouns}pronoun occupies the same slot as the noun phrase would in a statement. 

\begin{exe}
	\ex
	\label{ex:areyoumydaughter}
	\ipaex{njɤ˧ mv̩˩ {\kern2pt}|{\kern2pt} ə˩-ɲi˩˥?}\\
	\gll njɤ˩				mv̩˩˥		ə˩-						ɲi˩\\
	\textsc{1sg}		daughter		\textsc{interrog}		\textsc{cop}\\
	\glt ‘Is this my daughter? / Are you my daughter?’ (BuriedAlive3.107)
\end{exe}

\begin{exe}
	\ex
	\label{ex:whogoeswithmother}
	\ipaex{ɲi˩ ɳɯ˥ {\kern2pt}|{\kern2pt} ə˧mɑ˧-qɑ˧ tɕʰo˧ bi˧˥?}\\
	\gll ɲi˩		ɳɯ˧						ə˧mɑ˧		-qɑ˧˥						tɕʰo˩	bi˧\textsubscript{c}\\
	\textsc{interrog}.who		\textsc{a}		mother		\textsc{comitative}		together				to\_go\\
	\glt ‘Who is going to go together with mama? / Who is going to accompany mama?’ (FoodShortage2.17)
\end{exe}




\subsection{Existential sentences}
\label{sec:existential}
There exist several \is{existentials}existential verbs in Na, as in many other {Sino"=Tibetan} languages. One is preferred for animates, as illustrated in (\ref{ex:isathome}). One is “used with things that stand, protrude, or are perpendicular to a
plane” \citep[358]{lidz2010}, as in (\ref{ex:barley}), and another is “used with objects within a container” \citep[361]{lidz2010}~-- including abstract containers such as narratives: see (\ref{ex:instory}). In contrast to the above three, there is an \is{existentials}existential for entities that are diffuse, as it were: those that can neither be counted nor held within a container (which would allow for making them quantifiable): an example is shown in (\ref{ex:athand}). Finally, there is a~generic \is{existentials}existential, illustrated in (\ref{ex:nothingtoeat}).\footnote{\citet[359]{lidz2010} reports an additional {existential} verb in the Luoshui dialect, “used for the passing of time”: “/\ipa{ku33}/ \textsc{exist.t} [Existential: Used with past existence of time] seems to have something of a~connotation of ‘pass,’ and may be a~fairly recent {grammaticalization} from a lexical verb”. In the Alawa dialect (studied in the present volume), this verb, /\ipa{gv̩˧\textsubscript{c}}/, retains the lexical meaning ‘to flow, to go by, to elapse (time); to take place, to occur (event)’.}

\begin{exe}
	\ex
	\label{ex:isathome}
	\ipaex{hĩ˧ ɑ˥ʁo˩ dʑo˩}\\
	\gll hĩ˥		ɑ˩ʁo˧		dʑo˩\textsubscript{b}\\
	person		home		\textsc{exist.animate}\\
	\glt ‘there are people at home’ (Reward.11)
\end{exe}

\begin{exe}
	\ex
	\label{ex:barley}
	\ipaex{tɕʰi˧ tʰi˧-di˥-ɲi˩ mæ˩!}\\
	\gll tɕʰi˥		tʰi˧-			di˩\textsubscript{a}		-ɲi˩		mæ˧\\
	thorn		\textsc{dur}	\textsc{exist}		\textsc{certitude}	\textsc{obviousness}\\
	\glt ‘There are some thorns [on barley], not? / [barley] has beard, hasn't it!’ (FoodShortage.43)
\end{exe}

\begin{exe}
	\ex
	\label{ex:instory}
	\ipaex{dɑ˧pɤ˧ qʰwæ˧-qo˩ {\kern2pt}|{\kern2pt} qv̩˧ɻ̍˧ {\kern2pt}|{\kern2pt} tʰi˧-ʑi˥-kv̩˩ mæ˩!}\\
	\gll dɑ˧pɤ˧						qʰwæ˧		-qo˧		 qv̩˧ɻ̍\#˥						tʰi˧-				ʑi˥						-kv̩˧˥										mæ˧\\
	priest\_of\_local\_religion		message		inside		name\_of\_a\_mountain	\textsc{dur}	\textsc{exist}		\textsc{abilitive}		\textsc{obviousness}\\
	\glt ‘Mount \ipa{qv̩˧ɻ̍\#˥} is mentioned in the tales of the Daba priests!’ \textit{Literally:} ‘Mount \ipa{qv̩˧ɻ̍\#˥} exists in the tales of the Daba priests’ (Mountains.120)
\end{exe}

\begin{exe}
	\ex
	\label{ex:athand}
	\ipaex{ə˧tso˧-mɤ˧-ɲi˩ {\kern2pt}|{\kern2pt} le˧-ʂe˧, {\kern2pt}|{\kern2pt} le˧-ʝi˥!}\\
	\gll ə˧tso˧-mɤ˧-ɲi˩		le˧-		ʂe˧\textsubscript{a}		le˧-	ʝi˥\\
	all\_sorts\_of\_things	\textsc{accomp}		to\_get		\textsc{accomp}		\textsc{exist}\\
	\glt ‘We get all sorts of things (all the necessary paraphernalia for a ritual, a feast{\dots}) [so that] we have it (at hand for when we need it) / We get all sorts of things ready (for the ritual / the feast)!’ (Source: field notes.)
\end{exe}

\begin{exe}
	\ex
	\label{ex:nothingtoeat}
	\ipaex{dzɯ˧-di˧ mɤ˧-dʑo˧˥}\\
	\gll dzɯ˥		-di˩				mɤ˧-			dʑo˧\textsubscript{b}\\
	to\_eat			\textsc{nmlz}		\textsc{neg}		\textsc{exist}\\
	\glt ‘there was nothing to eat / there was no food left’ (FoodShortage2.4)
\end{exe}


\subsection{Sentence-final particles}
\label{sec:sentfinparticles}

Among sentence-final particles, several are epistemics or evidentials: /\ipa{tsɯ˧˥}/ for hearsay, /\ipa{mv̩˧}/ for affirmation, /\ipa{le˩}/ for exclamation, and /\ipa{mæ˧}/ to convey obviousness. Combinations between sentence-final particles allow for a~considerable range of nuances. Their meaning in context obtains through an interplay with intonational cues: the three"=level tone system does not considerably constrain a sentence’s \isi{intonation}; it leaves ample room for the expression of nuances of doubt, surprise and other attitudes and emotions by intonational means, such as overall raising of the pitch register, pitch \isi{range expansion}, \isi{lengthening}, and changes in \is{phonation types}phonation type. (This topic is taken up in Chapter~\ref{chap:fromsurfacephonologicalformstophoneticrealizationintonationandtonalimplementation}.)


\subsection{Noun and noun phrase}
\label{sec:nounNP}

The classifier, preceded by a~{numeral} or a~{demonstrative}, follows the head noun, as in the phrase ‘a~dumb person’ in (\ref{ex:lake33}). A~\is{numerals}numeral and a~classifier also constitute a~well"=formed phrase, which can receive a~{suffix}, as in the phrase ‘a~family’ in the same example (\ref{ex:lake33}). The counting system is decimal. 

\begin{exe}
	\ex
	\label{ex:lake33}
	\ipaex{ɖɯ˧-ʑi˩=ɻ̍˩-dʑo˩ {\kern2pt}|{\kern2pt} zo˧bæ˩ {\kern2pt}|{\kern2pt} ɖɯ˧-v̩˧ dʑo˩ tsɯ˩ {\kern2pt}|{\kern2pt} mv̩˩!}\\
	\gll ɖɯ˧	ʑi˩\textsubscript{b}					=ɻ̍˩					-dʑo˥			zo˧bæ˩			ɖɯ˧		v̩˧							dʑo˩\textsubscript{b}		 tsɯ˧˥			mv̩˧\\
	one			\textsc{clf}.households		\textsc{associative}	\textsc{top}	dumb\_man	one		\textsc{clf}.individual		\textsc{exist}			\textsc{rep}		\textsc{affirm}\\
	\glt ‘It is said that, in a~household, there was a~dumb man.’ (Lake3.3)
\end{exe}

Classifiers are numerous; many still correspond transparently to a noun. Some ‘self-classifiers’ are used without a head noun, e.g.~/\ipa{ɖɯ˧"=kʰv̩˧˥}/ ‘one year’ (not {$\ddagger$}{\kern2pt}\ipa{kʰv̩˧˥ ɖɯ˧-kʰv̩˧˥}  ‘year one"=year’). Classifiers are discussed in Chapter~\ref{chap:classifiers}.

The order in compounds is \textit{determiner+determined}, but a~few lexicalized items made of \is{adjectives}adjective plus noun have the reverse order, e.g.~‘dry field’ /\ipa{pv̩˧lv̩˧}/ (‘dry’{\allowbreak}+‘field’). Compounds are discussed in Chapter~\ref{chap:compoundnouns}.

There is no gender or agreement. 

There are proximal and distal demonstratives. The proximal \is{demonstratives}demonstrative, /\ipa{ʈʂʰɯ˥}/, is \is{homophony}homophonous with the third person singular \is{pronouns}pronoun. There are inclusive and exclusive second"=person pronouns, and collective forms referring to ‘me and my kin’, ‘you and your kin’, etc. Pronouns in Na behave syntactically like nouns (as in languages such as Chinese, Japanese, Bambara and Zarma: \citealt[29]{creissels1995syntaxe}) and not as pronominal indices.

\newpage
The clause relativizer is /-\ipa{hĩ˥}/, illustrated in (\ref{ex:familyGIVE}).

\begin{exe}
	\ex
	\label{ex:familyGIVE}
	\ipaex{go˧mi˧ {\kern2pt}|{\kern2pt} tʰi˧-ki˧-hĩ˧ {\kern2pt}|{\kern2pt} ʈʂʰɯ˧-ʑi˥…}\\
	\gll go˧mi˧		tʰi˧-		ki˧\textsubscript{a}			-hĩ˥		ʈʂʰɯ˥			ʑi˩\textsubscript{b}\\
	younger\_sister		\textsc{dur}	to\_give			\textsc{rel}		\textsc{dem.prox}		\textsc{clf}.households\\
	\glt ‘the family to which the younger sister pledged herself… / the household which the woman entered upon marriage…’ (Sister3.18)
\end{exe}


\subsection{Verb and verb phrase}
\label{sec:vvphrase}

Directionality is expressed before the verb, by /\ipa{gɤ˩}-/ ‘upward’, /\ipa{mv̩˩}-/ ‘downward’, or disyllabic expressions (\sectref{sec:themarkingofspatialorientationonverbs}). 
Verb serialization is used for many constructions including resultatives, equivalents of Chinese ‘\textit{de} \zh{得}’ constructions, and the expression of movement \citep[397-405]{lidz2010}. Imperatives are simply conveyed by intonational means: the syntactic structure is the same as for statements, except for the verb ‘to go’, which has a~distinct {imperative} form, as explained at the outset of this section (\sectref{sec:sketch}).

\is{adjectives}Adjectives can be considered as \isi{stative verbs}: they “can take aspect marking, be negated, and can be modified by the \is{intensifiers}intensifier” \citep[362]{lidz2010}. The intensifiers are /\ipa{ɖwæ˧˥}/, which appears before the adjective (‘very \textsc{Adj}’) or verb (‘V a~lot’) and /\ipa{ʐwæ˩}/ (which means ‘extremely’, and only appears with adjectives). The {copula} is not used with adjectives. It can be added after any verb to convey certainty \citep[354]{lidz2010}.

In addition to the discussion of the lexical tones of nouns (Chapter~\ref{chap:thelexicaltonesofnouns}) and verbs (\sectref{sec:thelexicaltonesofverbs}), the following chapters return to various syntactic phrase types in which tone changes occur: {compound} nouns (Chapter~\ref{chap:compoundnouns}), phrases containing nominal classifiers (Chapter~\ref{chap:classifiers}), combinations of content words with grammatical elements (nouns in Chapter~\ref{chap:combinationsofnounswithgrammaticalwords} and verbs and adjectives in \sectref{sec:reduplication}-\ref{sec:combinationsofadjectiveswithgrammaticalmorphemes}), object and verb (\sectref{sec:objectandnonprefixedverb}-\ref{sec:objectandprefixedverb}), and subject and verb (\sectref{sec:subjectandverb}).