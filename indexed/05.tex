\chapter{Combinations of nouns with grammatical elements}
\label{chap:combinationsofnounswithgrammaticalwords}

This chapter brings together data on a~wide range of constructions containing nouns, from morphological derivation~-- mostly nouns containing gender suffixes~-- and {reduplication}, to combinations of nouns with particles in discourse. 


\section{Derivational affixes: Gender suffixes and kinship prefix}
\label{sec:thegendersuffixes}
\is{derivation!morphological|(}
\subsection{Introduction to gender suffixes}
\label{sec:thegendersuffixesintro}

The most common derivational affixes in Na are gender suffixes: /\ipa{-mi}/, /\ipa{-zo}/ and /\ipa{-pʰv}/,\footnote{At this stage, no tone is indicated for these three suffixes: establishing their tone category is a~key objective of \sectref{sec:thegendersuffixes}.} carrying
the meaning ‘female, mother’, ‘son, young’, and ‘male’, for instance in /\ipa{ɬi˧mi˧}/
‘female roebuck’, /\ipa{ɬi˩zo˩}/ ({variant}: /\ipa{ɬi˧zo\#˥}/) ‘young roebuck’, and /\ipa{ɬi˩pʰv̩˩}/
({variant}: /\ipa{ɬi˧pʰv̩\#˥}/) ‘male roebuck’ \citep[177-179]{lidz2010}.\footnote{Liberty Lidz's notations are /\ipa{-mi33}/, /\ipa{-zɔ33}/ and /\ipa{-pʰu33}/. Consultant F4's word for ‘grandmother's brother’, /\ipa{ə˧pʰv̩˧}/, is glossed by L.~Lidz as ‘grandfather (father of mother or father)’ in the Luoshui dialect, and transcribed as /\ipa{ɑ33-pʰv̩33}/ (p. 766), but it also appears as /\ipa{ɑ33-pʰu33}/ at places, e.g.~pp. 269, 503 and 763.}

The suffixes /\ipa{-mi}/ and /\ipa{-zo}/ also serve as augmentative and diminutive suffixes,
respectively. ‘Mother’ stands for ‘large’, and ‘son’ for ‘small’, as in numerous languages of the
area (see \citealt{mazaudon2003b} on \ili{Tamang}, and a~cross"=language discussion by \citealt{matisoff1992}). For
instance, /\ipa{kʰɤ˧˥}/ ‘basket (carried on back)’ yields /\ipa{kʰɤ˧mi˥\$}/ ‘large basket’ and
/\ipa{kʰɤ˧zo\#˥}/ ‘small basket’. The augmentative or diminutive
meaning has faded away from many suffixed forms, which have thereby become lexicalized. For instance, {monosyllabic} /\ipa{ljɤ˩˥}/ and suffixed /\ipa{ljɤ˩mi˥}/ are both used to refer to the same object, namely main
(supporting) beams. The augmentative and diminutive suffixes are not highly productive: there is no diminutive counterpart (\ipa{$\dagger$ljɤ˩-zo\#˥}) to /\ipa{ljɤ˩mi˥}/ ‘main beam’, for instance. The slanting beams upholding the roofing (planks or tiles), which rest on the supporting beams, are called /\ipa{ʐv̩˩ɭɯ˥}/. When referring to a~large"=sized slanting beam, it is not possible to use an~augmentative $\dagger$\ipa{ʐv̩˩ɭɯ˥-mi˩}, with the intended meaning of ‘a large"=sized
/\ipa{ʐv̩˩ɭɯ˥}/’. Instead, constructions with the adjectives /\ipa{tɕi˩\textsubscript{a}}/ ‘small’ or /\ipa{ɖɯ˩\textsubscript{a}}/ ‘large’ are used. (About the association of adjectives to nouns in Yongning Na, see \sectref{sec:productiveconstruction}.)

Etymologically, the second syllable of the words
/\ipa{ɲi˧mi\#˥}/ ‘sun’ and /\ipa{ɬi˧mi˧}/ ‘moon’, which currently lack \is{monosyllables}{monosyllabic}
counterparts, probably originates in the same morpheme /\ipa{mi}/. Since these two words also have the same structure in \ili{Naxi} (/\ipa{ɲi˧me˧}/ and
/\ipa{he˧me˧}/) and \ili{Laze} (/\ipa{ɲie˧mie˧}/ and /\ipa{ɬie˧mie˧}/), two languages closely related to Na, their disyllabic status is likely
to have some historical depth. The \is{suffixes}suffix /\ipa{zo}/ appearing in /\ipa{ɲi˧zo\#˥}/ ‘fish’, another
disyllable without a~{monosyllabic} counterpart, is also found in \ili{Laze} (/\ipa{ze˧}/ ‘son’,
/\ipa{ɲi˩ze˥}/ ‘fish’), whereas \ili{Naxi} has a~{monosyllable}: /\ipa{ɲi˧}/.

The discussion below covers names of animals and peoples with gender suffixes, as well as the
augmentative and diminutive suffixes. Needless to say, words
that contain a~/\ipa{mi}/ syllable of other origin were excluded, e.g.~the name of the Yongning monastery,
/\ipa{ɖæ˩mi˧}/, which is a~\is{loanwords}loanword from \ili{Tibetan} \textit{dgra med}. 

The suffixes for \textit{female}, \textit{young} and \textit{male} are related to the free morphemes /\ipa{mi˩˧}/, /\ipa{zo˥}/ and /\ipa{pʰv̩˧}/, which
appear in contexts such as (\ref{ex:isfemale})"=(\ref{ex:ismale}). In this case, there is no mystery in the evolution from the nouns to the derivational affixes. 

\begin{exe}
	\ex
	\label{ex:isfemale}
	\ipaex{ʈʂʰɯ˧ {\kern2pt}|{\kern2pt} mi˩ ɲi˥.}\\
	\gll ʈʂʰɯ˥	mi˩˧	ɲi˩\\
	\textsc{dem.prox}		female	\textsc{cop}\\
	\glt ‘This is a~female.’
\end{exe}

\begin{exe}
	\ex
	\label{ex:isyoung}
	\ipaex{ʈʂʰɯ˧ {\kern2pt}|{\kern2pt} zo˧ ɲi˥.}\\
	\gll ʈʂʰɯ˥	zo˥	ɲi˩\\
	\textsc{dem.prox}		son/young/male	\textsc{cop}\\
	\glt ‘This is a~young/male.’
\end{exe}

\begin{exe}
	\ex
	\label{ex:ismale}
	\ipaex{ʈʂʰɯ˧ {\kern2pt}|{\kern2pt} pʰv̩˧ ɲi˩.}\\
	\gll ʈʂʰɯ˥	pʰv̩˧	ɲi˩\\
	\textsc{dem.prox}		male	\textsc{cop}\\
	\glt ‘This is a~male.’
\end{exe}

The free forms /\ipa{zo˥}/ ‘young/male’ and /\ipa{pʰv̩˧}/ ‘male’ have different tones (H and M, respectively), whereas the suffixes /\ipa{-zo}/ and /\ipa{-pʰv}/ always have the same tonal patterns, even sharing the same tonal variants, as in the example of ‘roebuck’. However, \isi{neutralization} of tonal oppositions on nouns in the process of \isi{grammaticalization} as gender suffixes is not thoroughgoing: the tonal behaviour of /\ipa{-zo}/ and /\ipa{-pʰv}/ differs from that of the female \is{suffixes}suffix /\ipa{-mi}/. 

From a~tonal point of view, there is thus evidence that these three derivational elements have become distinct from the free nouns in which they originate. This is reminiscent of classifiers: the study of classifiers provided in Chapter \ref{chap:classifiers} reveals that the tone system of classifiers is not identical to that of free nouns, and that the tone of a~classifier is not necessarily the closest equivalent of that of the noun from which it derives. To repeat an example from \sectref{sec:howthetonalcategorieswerebroughtoutandlabelled}, there are two tonal correspondences among classifiers for H-tone nouns, the one illustrated by ‘beam’, /\ipa{ɖʐo˥}/, which has /\ipa{ɖʐo˥\textsubscript{a}}/ (category H\textsubscript{a}) as its self"=classifier, and the other illustrated by /\ipa{kɯ˥}/ ‘star’, which yields /\ipa{kɯ˧\textsubscript{b}}/ (M\textsubscript{b} tone category) as a~self"=classifier. Seen in this light, differences in tone between a~noun as a~full form and as a~derivational \is{suffixes}suffix are not a~particularly out-of-the-way finding in the context of Yongning Na morphotonology.
 
The difference in tone patterns between /\ipa{-mi}/ and the other two suffixes suffices to establish that suffixes are not toneless. But ascertaining the tones of these suffixes is not an~easy matter. A~simple test consists in combining them with a~M-tone noun: the M tone has properties that make it suitable for use in tonal tests. M can be followed by any tone (unlike H, which can only be followed by L), and it does not spread (unlike L), so it would seem to offer the best possible context for the lexical tone of the following morpheme to manifest itself in. This works out well for verbs: a~useful tonal test consists in observing a~verb's tone after a~M-tone morpheme such as the {negation} \is{prefixes}prefix, /\ipa{mɤ˧-}/. In this context, H-tone verbs surface with H tone, MH-tone verbs with MH tone, and so on (see \sectref{sec:overview}). But the three gender suffixes all yield the same result after a~M-tone {monosyllable}: for instance, /\ipa{lɑ˧}/ ‘tiger’ yields /\ipa{lɑ˧mi\#˥}/ ‘female tiger’, /\ipa{lɑ˧zo\#˥}/ ‘baby tiger’ and /\ipa{lɑ˧pʰv̩\#˥}/ ‘male tiger’. After disyllabic M-tone nouns, on the other hand, the results are different: thus, /\ipa{si˧gɯ˧}/ ‘lion’ yields /\ipa{si˧gɯ˧-mi˩}/ ‘female lion’, with L tone on the suffix, and /\ipa{si˧gɯ˧-zo\#˥}/ ‘baby lion’, with a~floating H tone. On this slender basis, the two sets of suffixes are provisionally transcribed as carrying lexical L and H tone, respectively. They are transcribed hereafter as //\ipa{-mi˩}//, //\ipa{-zo˥}// and //\ipa{-pʰv̩˥}//. It must be cautioned that this tentative identification does by no means encapsulate all the information about the tonal behaviour of these two types of suffixes, which is set out in tabular form below.


\subsection{The facts}
\label{sec:thegendersuffixesfacts}

Tables~\ref{tab:gendertwo}a--g and Tables~\ref{tab:genderthree}a--b
present the data concerning the suffixes //\ipa{-mi˩}//, //\ipa{-zo˥}// and //\ipa{-pʰv̩˥}//. The examples are arranged by tone of the suffixed form. Tables~\ref{tab:gendertwo}a--g
present disyllables, and Tables~\ref{tab:genderthree}a--b
trisyllables. Nouns in which the suffixes are augmentative or diminutive
and not gender suffixes are italicized (i.e.\ nouns that do not refer
to animals or ethnic groups). As elsewhere, a~slash separates
variants. A~dash ‘--’ in a~cell indicates that the
form does not exist: for instance, it is not possible to use a word suffixed with /\ipa{-zo}/ for ‘piglet’ (the attested form is
/\ipa{bæ˧bv̩˥}/). A~double dagger $\ddagger$ preceding a~noun indicates that it is
a~form that was proposed by the investigator and rejected by the
consultant. For instance, “\ipa{ʐæ˩mi\#˥} (\ipa{$\ddagger${\kern2pt}ʐæ˩mi˩})” for ‘female
leopard’ indicates that the tonal {variant} $\ddagger${\kern2pt}\ipa{ʐæ˩mi˩}, tested by
the investigator on the \isi{analogy} of the existence of a~L-tone {variant}
/\ipa{ʑi˩mi˩}/ for ‘female ape’, whose root belongs in the same
category as ‘leopard’, was rejected by the consultant.

Table~\ref{tab:gendertwosyllableslm} includes two roots, ‘thumb' and ‘bee', which are not synchronically attested as monosyllables, but whose internal \is{comparative method (historical linguistics)}reconstruction yields a~LM"=tone root.

\begin{subtables}
	\label{tab:gendertwo}
%	\begin{table}[t]
%		\caption{\label{tab:gendertwosyllableslm}Nouns with gender suffixes or \textsc{augmentative/{\allowbreak}diminutive} suffixes. Disyllabic
%			words. LM"=tone roots.}
%		\begin{tabularx}{\textwidth}{ P{21mm} P{16mm} P{20mm} Q Q }
%			\lsptoprule
%			\multirow{3}{21mm}{correspon\-dences} & root & \multicolumn{3}{l}{suffixed forms}\\ \cmidrule{3-5}
%			& & //\ipa{-mi˩}// & //\ipa{-zo˥}// & //\ipa{-pʰv̩˥}//\\
%			& & ‘female'/{\allowbreak}\textsc{aug} & ‘child'/{\allowbreak}\textsc{dim} & ‘male'\\ \midrule
%			\multirow{2}{21mm}{first type} & sow & \ipa{bo˩mi˧} & -- & \ipa{bo˩pʰv̩˧}\\
%			& hen & \ipa{æ˩mi˧} & -- & --\\
%			& \textit{thumb} & \ipa{lo˩mi˧} & -- & --\\
%			& bee & \ipa{dze˩mi˧} & -- & --\\ \addlinespace \hdashline \addlinespace
%			\multirow{4}{21mm}[1.5\baselineskip]{second type} & yak & \ipa{bv̩˧mi˧} & \ipa{bv̩˧zo\#˥ / bv̩˩zo˩} & \ipa{bv̩˧pʰv̩\#˥ / bv̩˩pʰv̩˩}\\
%			& sparrow & \ipa{ɖʐwæ˧mi˧} & \ipa{ɖʐwæ˧zo\#˥ / ɖʐwæ˩zo˩} & \ipa{ɖʐwæ˧pʰv̩\#˥ / ɖʐwæ˩pʰv̩˩}\\
%			& hawk & \ipa{kɤ˧mi˧} & \ipa{kɤ˧zo\#˥ / kɤ˩zo˩} & \ipa{kɤ˧pʰv̩\#˥ / kɤ˩pʰv̩˩}\\
%			& \textit{food steamer} & \ipa{bv̩˧mi˧ } & \ipa{bv̩˩zo˩ / bv̩˧zo\#˥} & --\\ \addlinespace \hdashline \addlinespace
%			\multirow{5}{21mm}[2\baselineskip]{third type} & weasel & \ipa{dv̩˩mi\#˥ / dv̩˩mi˧} & \ipa{dv̩˩zo\#˥ / dv̩˩zo˧} & \ipa{dv̩˩pʰv̩\#˥ / dv̩˩pʰv̩˧}\\
%			& goose & \ipa{ɑ˩mi\#˥ / ɑ˩mi˧} & \ipa{ɑ˩zo\#˥ / ɑ˩zo˧} & \ipa{ɑ˩pʰv̩\#˥ / ɑ˩pʰv̩˧}\\
%			& lizard & \ipa{dzo˩mi\#˥ / dzo˩mi˧} & \ipa{dzo˩zo\#˥ / dzo˩zo˧} & \ipa{dzo˩pʰv̩\#˥ / dzo˩pʰv̩˧}\\
%			& \textit{ladle} & \ipa{tɕʰo˩mi\#˥ / tɕʰo˩mi˧} & \ipa{tɕʰo˩zo\#˥ / tɕʰo˩zo˧} & --\\
%			& Na (people) & \ipa{nɑ˩mi\#˥ / nɑ˩mi˧} & \ipa{nɑ˩zo\#˥ / nɑ˩zo˧} & --\\ \addlinespace \hdashline \addlinespace
%			\multirow{2}{21mm}[.5\baselineskip]{fourth type}	& jackal & \ipa{pʰɤ˩mi˩} & \ipa{pʰɤ˧zo\#˥ / pʰɤ˩zo˩} & \ipa{pʰɤ˧pʰv̩\#˥ / pʰɤ˩pʰv̩˩}\\
%			& \textit{road, path} & \ipa{ʐɤ˩mi˩} & -- & --\\
%			\lspbottomrule
%		\end{tabularx}
%	\end{table}

	\begin{sidewaystable}[p]
		\caption{\label{tab:gendertwosyllableslm}Nouns with gender suffixes or {augmentative/{\allowbreak}diminutive} suffixes. Disyllabic
			words. LM"=tone roots.}
		\begin{tabularx}{\textwidth}{ P{21mm} P{19mm} Q Q P{42mm}}
			\lsptoprule
			\multirow{2}{21mm}{correspon\-dences} & root & \multicolumn{3}{l}{suffixed forms}\\ \cmidrule{3-5}
			& & //\ipa{-mi˩}// ‘female'/{\allowbreak}\textsc{aug} & //\ipa{-zo˥}// ‘child'/{\allowbreak}\textsc{dim} & //\ipa{-pʰv̩˥}// ‘male'\\ \midrule
%\multicolumn{3}{l}{suffixed forms}\\ \cmidrule{3-5}
%& & //\ipa{-mi˩}// & //\ipa{-zo˥}// & //\ipa{-pʰv̩˥}//\\
%& & ‘female'/{\allowbreak}\textsc{aug} & ‘child'/{\allowbreak}\textsc{dim} & ‘male'\\ \midrule
			\multirow{2}{21mm}{first type} & sow & \ipa{bo˩mi˧} & -- & \ipa{bo˩pʰv̩˧}\\
			& hen & \ipa{æ̃˩mi˧} & -- & --\\
			& \textit{thumb} & \ipa{lo˩mi˧} & -- & --\\
			& bee & \ipa{dze˩mi˧} & -- & --\\ \addlinespace \hdashline \addlinespace
			\multirow{4}{21mm}[1.5\baselineskip]{second type} & yak & \ipa{bv̩˧mi˧} & \ipa{bv̩˧zo\#˥ / bv̩˩zo˩} & \ipa{bv̩˧pʰv̩\#˥ / bv̩˩pʰv̩˩}\\
			& sparrow & \ipa{ɖʐwæ˧mi˧} & \ipa{ɖʐwæ˧zo\#˥ / ɖʐwæ˩zo˩} & \ipa{ɖʐwæ˧pʰv̩\#˥ / ɖʐwæ˩pʰv̩˩}\\
			& hawk & \ipa{kɤ˧mi˧} & \ipa{kɤ˧zo\#˥ / kɤ˩zo˩} & \ipa{kɤ˧pʰv̩\#˥ / kɤ˩pʰv̩˩}\\
			& \textit{steamer} & \ipa{bv̩˧mi˧} & \ipa{bv̩˩zo˩ / bv̩˧zo\#˥} & --\\ \addlinespace \hdashline \addlinespace
			\multirow{5}{21mm}[2\baselineskip]{third type} & weasel & \ipa{dv̩˩mi\#˥ / dv̩˩mi˧} & \ipa{dv̩˩zo\#˥ / dv̩˩zo˧} & \ipa{dv̩˩pʰv̩\#˥ / dv̩˩pʰv̩˧}\\
			& goose & \ipa{ɑ˩mi\#˥ / ɑ˩mi˧} & \ipa{ɑ˩zo\#˥ / ɑ˩zo˧} & \ipa{ɑ˩pʰv̩\#˥ / ɑ˩pʰv̩˧}\\
			& lizard & \ipa{dzo˩mi\#˥ / dzo˩mi˧} & \ipa{dzo˩zo\#˥ / dzo˩zo˧} & \ipa{dzo˩pʰv̩\#˥ / dzo˩pʰv̩˧}\\
			& \textit{ladle} & \ipa{tɕʰo˩mi\#˥ / tɕʰo˩mi˧} & \ipa{tɕʰo˩zo\#˥ / tɕʰo˩zo˧} & --\\
			& Na (people) & \ipa{nɑ˩mi\#˥ / nɑ˩mi˧} & \ipa{nɑ˩zo\#˥ / nɑ˩zo˧} & --\\ \addlinespace \hdashline \addlinespace
			\multirow{2}{21mm}[.5\baselineskip]{fourth type}	& jackal & \ipa{pʰɤ˩mi˩} & \ipa{pʰɤ˧zo\#˥ / pʰɤ˩zo˩} & \ipa{pʰɤ˧pʰv̩\#˥ / pʰɤ˩pʰv̩˩}\\
			& \textit{road, path} & \ipa{ʐɤ˩mi˩} & -- & --\\
			\lspbottomrule
		\end{tabularx}
%	\end{table}
	\end{sidewaystable}	
	
	\begin{table}[t!]
		\caption{\label{tab:gendertwosyllableslh}Nouns with gender suffixes or {augmentative/{\allowbreak}diminutive} suffixes. Disyllabic words. LH"=tone roots.}
		\begin{tabularx}{\textwidth}{ P{29mm} P{15mm} P{20mm} Q Q }
			\lsptoprule
			correspondences & root & \multicolumn{3}{l}{suffixed forms}\\ \cmidrule{3-5}
			& & //\ipa{-mi˩}// ‘female'/{\allowbreak}\textsc{aug} & //\ipa{-zo˥}// ‘child'/{\allowbreak}\textsc{dim} & //\ipa{-pʰv̩˥}// ‘male'\\ \midrule	
			\multirow{2}{32mm}{first type} &	leopard & \ipa{ʐæ˩mi\#˥ ($\ddagger${\kern2pt}ʐæ˩mi˩)} & \ipa{ʐæ˩zo\#˥ ($\ddagger${\kern2pt}ʐæ˩zo˩)} & \ipa{ʐæ˩pʰv̩\#˥ ($\ddagger${\kern2pt}ʐæ˩pʰv̩˩)}\\ 
			& monkey & \ipa{ʑi˩mi\#˥ / ʑi˩mi˩} & \ipa{ʑi˩zo\#˥ / ʑi˩zo˩} & \ipa{ʑi˩pʰv̩\#˥ / ʑi˩pʰv̩˩}\\ 
			& buffalo & \ipa{tʰɑ˩mi\#˥ ($\ddagger${\kern2pt}tʰɑ˩mi˩)} & \ipa{tʰɑ˩zo\#˥} & \ipa{tʰɑ˩pʰv̩\#˥}\\ 
			& muntjac & \ipa{tɕʰɯ˩mi\#˥ / tɕʰɯ˩mi˩} & \ipa{tɕʰɯ˩zo\#˥ / tɕʰɯ˩zo˩} & \ipa{tɕʰɯ˩pʰv̩\#˥ / tɕʰɯ˩pʰv̩˩}\\ 
			& \textit{plane (tool)} & \ipa{tʰi˩mi\#˥ ($\ddagger${\kern2pt}tʰi˩mi˩)} & \ipa{tʰi˩zo\#˥ ($\ddagger${\kern2pt}tʰi˩zo˩)} & --\\ \addlinespace \hdashline \addlinespace
			second type (isolated example) & \textit{slope} & \ipa{to˩mi˩} & \ipa{to˩zo˩} & --\\ \addlinespace \hdashline \addlinespace
			third type & \textit{plain} & \ipa{di˧mi˧} & -- & --\\ 
			 & woman & \ipa{mv̩˧mi˧} & \ipa{mv̩˩zo˩} & --\\ 
			\lspbottomrule
		\end{tabularx}
	\end{table}
	
%	%check on proofs: is it useful to introduce a
%	\clearpage
%	%here?
	\begin{table}[t!]
		\caption{\label{tab:gendertwosyllablesm}Nouns with gender suffixes or {augmentative/{\allowbreak}diminutive} suffixes. Disyllabic words. M-tone roots. Only one type of {correspondence}.}
		\begin{tabularx}{\textwidth}{ P{20mm} P{38mm} Q Q }
			\lsptoprule
			root & \multicolumn{3}{l}{suffixed forms}\\ \cmidrule{2-4}
			& //\ipa{-mi˩}// ‘female'/{\allowbreak}\textsc{aug} & //\ipa{-zo˥}// ‘child' & //\ipa{-pʰv̩˥}// ‘male'\\ \midrule
			tiger & \ipa{lɑ˧mi\#˥} & \ipa{lɑ˧zo\#˥} & \ipa{lɑ˧pʰv̩\#˥}\\
			goral & \ipa{se˧mi\#˥} & \ipa{se˧zo\#˥} & \ipa{se˧pʰv̩\#˥}\\
			\textit{message} & \ipa{qʰwæ˧mi\#˥} & -- & --\\
			\lspbottomrule
		\end{tabularx}
	\end{table}
	
	
	\begin{table}[t!]
		\caption{\label{tab:gendertwosyllablesl}Nouns with gender suffixes or {augmentative/{\allowbreak}diminutive} suffixes. Disyllabic
			words. L-tone roots. Only one type of {correspondence}.}
		{\setlength\tabcolsep{4pt}
		\begin{tabularx}{\textwidth}{ P{15mm} P{34mm} Q Q }
			\lsptoprule
			root & \multicolumn{3}{l}{suffixed forms}\\ \cmidrule{2-4}
			& //\ipa{-mi˩}// ‘female'/{\allowbreak}\textsc{aug} & //\ipa{-zo˥}// ‘child'/{\allowbreak}\textsc{dim} & //\ipa{-pʰv̩˥}// ‘male'\\ \midrule
			daughter & \ipa{mv̩˧mi˧} & -- & --\\
			sheep & \ipa{jo˧mi˧} & \ipa{jo˧zo\#˥ / jo˩zo˩} & \ipa{jo˧pʰv̩\#˥ / jo˩pʰv̩˩}\\
			roebuck & \ipa{ɬi˧mi˧} & \ipa{ɬi˧zo\#˥ / ɬi˩zo˩} & \ipa{ɬi˧pʰv̩\#˥ / ɬi˩pʰv̩˩}\\
			\textit{bottle} & \ipa{kɤ˧mi˧} & \ipa{kɤ˩zo˩ ($\ddagger${\kern2pt}kɤ˧zo\#˥)} & --\\
			\textit{river} & \ipa{dʑɯ˧mi˧} & -- & --\\
			\lspbottomrule
		\end{tabularx}}
	\end{table}
	
	\begin{table}[t!]
		\caption{\label{tab:gendertwosyllablesh}Nouns with gender suffixes or {augmentative/{\allowbreak}diminutive} suffixes. Disyllabic words. H-tone roots.}
		\begin{tabularx}{\textwidth}{ P{21mm} P{25mm} P{20mm} Q Q }
			\lsptoprule
			\multirow{2}{21mm}{correspon\-dences} & root & \multicolumn{3}{l}{suffixed forms}\\ \cmidrule{3-5}
			& & //\ipa{-mi˩}// ‘female'/{\allowbreak}\textsc{aug} & //\ipa{-zo˥}// ‘child'/{\allowbreak}\textsc{dim} & //\ipa{-pʰv̩˥}// ‘male'\\ \midrule
			first type & cow & \ipa{ʝi˩mi˩} & \ipa{ʝi˧zo\#˥} & \ipa{ʝi˧pʰv̩\#˥}\\
			& horse & \ipa{ʐwæ˩mi˩} & \ipa{ʐwæ˧zo\#˥} & \ipa{ʐwæ˧pʰv̩\#˥}\\
			& dog & \ipa{kʰv̩˩mi˩} & \ipa{kʰv̩˧zo\#˥} & \ipa{kʰv̩˧pʰv̩\#˥}\\ \addlinespace \hdashline \addlinespace
			second type & \ili{Pumi} (people) & \ipa{bɤ˧mi\#˥} & \ipa{bɤ˧zo\#˥} & --\\
			& pheasant & \ipa{ho˧mi\#˥} & \ipa{ho˧zo\#˥} & \ipa{ho˧pʰv̩\#˥}\\
			& \textit{cooking pan} & \ipa{v̩˧mi\#˥} & \ipa{v̩˧zo\#˥} & --\\ \addlinespace \hdashline \addlinespace
			third type & \textit{door} & \ipa{kʰi˧mi˧} & \ipa{kʰi˧zo\#˥} & --\\
			& \textit{canal} & \ipa{qʰæ˧mi˧} & \ipa{qʰæ˧zo\#˥} & --\\
			& \textit{tree trunk}  & \ipa{ɻ̍̃˧mi˧} & -- & --\\
			\lspbottomrule
		\end{tabularx}
	\end{table}
	
	
	\begin{table}[t!]
		\caption{\label{tab:gendertwosyllablesmh}Nouns with gender suffixes or {augmentative/{\allowbreak}diminutive} suffixes. Disyllabic words. MH"=tone roots.}
		\begin{tabularx}{\textwidth}{ P{29mm} P{14mm} Q Q Q }
			\lsptoprule
			correspondences & root & \multicolumn{3}{l}{suffixed forms}\\ \cmidrule{3-5}
			& & //\ipa{-mi˩}// ‘female'/{\allowbreak}\textsc{aug} & //\ipa{-zo˥}// ‘child'/{\allowbreak}\textsc{dim} & //\ipa{-pʰv̩˥}// ‘male'\\ \midrule
			first type & cat & \ipa{hwɤ˧mi˥\$} & \ipa{hwɤ˧zo\#˥ / hwɤ˧zo˥\$} & \ipa{hwɤ˧pʰv̩\#˥ / hwɤ˧pʰv̩˥\$}\\
			& doe & \ipa{ʈʂʰæ˧mi˥\$} & \ipa{ʈʂʰæ˧zo\#˥ / ʈʂʰæ˧zo˥\$} & \ipa{ʈʂʰæ˧pʰv̩\#˥ / ʈʂʰæ˧pʰv̩˥\$}\\
			& goat & \ipa{tsʰɯ˧mi˥\$} & \ipa{tsʰɯ˧zo\#˥ / tsʰɯ˧zo˥\$} & \ipa{tsʰɯ˧pʰv̩\#˥ / tsʰɯ˧pʰv̩˥\$}\\
			& crane & \ipa{ʁv̩˧mi˥\$} & \ipa{ʁv̩˧zo\#˥ / ʁv̩˧zo˥\$} & \ipa{ʁv̩˧pʰv̩\#˥ / ʁv̩˧pʰv̩˥\$}\\
			& wasp & \ipa{tɕɯ˧mi˥\$} & \ipa{tɕɯ˧zo\#˥ / tɕɯ˧zo˥\$} & \ipa{tɕɯ˧pʰv̩\#˥ / tɕɯ˧pʰv̩˥\$}\\
			& \textit{basket} & \ipa{kʰɤ˧mi˥\$} & \ipa{kʰɤ˧zo˥\$ ($\ddagger${\kern2pt}kʰɤ˧zo\#˥)} & --\\
			& \textit{needle} & \ipa{ʁo˧mi˥\$} & \ipa{ʁo˧zo\#˥ ($\ddagger${\kern2pt}ʁo˧zo˥\$)} & --\\
			& \textit{scales} & \ipa{tɕɯ˧mi˥\$} & \ipa{tɕɯ˧zo˥\$ ($\ddagger${\kern2pt}tɕɯ˧zo\#˥)} & --\\
			& \textit{bowl} & \ipa{qʰwɤ˧mi˥\$} & \ipa{qʰwɤ˧zo˥\$ ($\ddagger${\kern2pt}qʰwɤ˧zo\#˥)} & --\\
			& \textit{stomach} & \ipa{hu˧mi˥\$} & -- & --\\ \addlinespace \hdashline \addlinespace
			second type (isolated example) & \textit{building} & \ipa{ʑi˧mi˧} & -- & --\\
			\lspbottomrule
		\end{tabularx}
	\end{table}
	
	
	\begin{table}[p!]
		\caption{\label{tab:gendertwosyllablesrest}Nouns with gender suffixes or {augmentative/{\allowbreak}diminutive} suffixes. Disyllabic words without a~corresponding {monosyllable}.}
		\begin{tabularx}{\textwidth}{ P{22mm} P{26mm} P{20mm} Q Q }
			\lsptoprule
			\multirow{2}{22mm}{possible root tone} & meaning & \multicolumn{3}{l}{suffixed forms}\\ \cmidrule{3-5}
			& & //\ipa{-mi˩}// ‘female'/{\allowbreak}\textsc{aug} & //\ipa{-zo˥}// ‘child'/{\allowbreak}\textsc{dim} & //\ipa{-pʰv̩˥}// ‘male'\\ \midrule
			M or H & 	water buffalo & \ipa{dʑi˧mi\#˥} & \ipa{dʑi˧zo\#˥ ($\ddagger${\kern2pt}dʑi˩zo˩)} & \ipa{dʑi˧pʰv̩\#˥}\\
			& 	granddaughter & \ipa{ʐv̩˧mi\#˥} & -- & --\\
			& 	\textit{sun} & \ipa{ɲi˧mi\#˥} & -- & --\\ \addlinespace \hdashline \addlinespace
			L or LM & duck & \ipa{bæ˧mi˧} & \ipa{bæ˧zo\#˥} & \ipa{bæ˧pʰv̩\#˥}\\
			& 	\textit{large vat} & \ipa{dzo˧mi˧} & \ipa{dzo˧zo\#˥} & --\\
			& 	\textit{sword} & \ipa{ʁæ˧mi˧} & \ipa{ʁæ˧zo\#˥} & --\\ \addlinespace \hdashline \addlinespace
			\multirow{2}{22mm}{LH, L, H or MH} & 	\textit{tummy, belly} & \ipa{bi˧mi˧} & -- & --\\
			& 	fox & \ipa{ɖɤ˧mi˧} & -- & --\\
			& 	little sister & \ipa{go˧mi˧} & -- & --\\
			& 	\textit{moon} & \ipa{ɬi˧mi˧} & -- & --\\
			& 	\textit{king, lord} & \ipa{ʁo˧mi˧} & -- & --\\
			& 	louse & \ipa{ʂe˧mi˧} & -- & --\\
			& 	wife & \ipa{ʈʂʰv̩˧mi˧} & -- & --\\ \addlinespace \hdashline \addlinespace
			LM or LH & 	frog & \ipa{pɤ˩mi˩} & -- & \ipa{pɤ˩pʰv̩˩}\\ \addlinespace \hdashline \addlinespace
			LM, LH or H & 	\textit{tongue} & \ipa{hi˩mi˩} & -- & --\\
			& 	\textit{large comb} & \ipa{pv̩˩mi˩} & -- & --\\
			& 	\textit{axe} & \ipa{bi˩mi˩} & -- & --\\
			& 	\textit{heart} & \ipa{nv̩˩mi˩} & -- & --\\
			& 	niece & \ipa{ze˩mi˩} & -- & --\\
			& 	mule & \ipa{ɖɯ˩mi\#˥} & \ipa{ɖɯ˩zo\#˥} & \ipa{ɖɯ˩pʰv̩\#˥}\\ \addlinespace \hdashline \addlinespace
			LM or H & 	\textit{bow} & \ipa{ʐv̩˩mi˩} & \ipa{ʐv̩˧zo\#˥} & --\\ \addlinespace \hdashline \addlinespace
			any tone except LH & 	fish & -- & \ipa{ɲi˧zo\#˥} & --\\ \addlinespace \hdashline \addlinespace
			unclear & 	hwamei (bird) & \ipa{tɕɯ˩mi˥} & -- & --\\
			& 	\textit{cigarette lighter} & \ipa{tse˧mi˥} & -- & --\\
			\lspbottomrule
		\end{tabularx}
	\end{table}
\end{subtables}


\begin{subtables}
	\label{tab:genderthree}
	\begin{table}%[t]
		\caption{\label{tab:genderthreesyllablesdisyllablesm}Nouns with gender suffixes or augmentative/diminutive
			suffixes. Three"=syllable words derived from M-tone disyllables.}
		
		{\setlength\tabcolsep{4pt}
			\begin{tabularx}{\textwidth}{ P{18mm} Q l l l }
				\lsptoprule
				\multirow{2}{18mm}{correspon\-dences} & root & \multicolumn{3}{l}{suffixed forms}\\ \cmidrule{3-5}
				& & //\ipa{-mi˩}// ‘female' & //\ipa{-zo˥}// ‘child' & //\ipa{-pʰv̩˥}// ‘male'\\ \midrule
				\multirow{2}{18mm}{first\\ type} & rabbit & \ipa{tʰo˧li˧-mi˩} & \ipa{tʰo˧li˧-zo\#˥} & \ipa{tʰo˧li˧-pʰv̩\#˥}\\
				& snake & \ipa{ʐv̩˧bæ˧-mi˩} & \ipa{ʐv̩˧bæ˧-zo\#˥} & \ipa{ʐv̩˧bæ˧-pʰv̩\#˥}\\
				& lion & \ipa{si˧gɯ˧-mi˩} & \ipa{si˧gɯ˧-zo\#˥} & \ipa{si˧gɯ˧-pʰv̩\#˥}\\
				& earth\-worm  & \ipa{dʑɯ˧dv̩˧-mi˩} & \ipa{dʑɯ˧dv̩˧-zo\#˥} &
				\ipa{dʑɯ˧dv̩˧-pʰv̩\#˥}\\  \addlinespace \hdashline \addlinespace
				\multirow{2}{18mm}{second\\ type} & demon & \ipa{si˧bv̩˧-mi\#˥} & \ipa{si˧bv̩˧-zo\#˥} & --\\
				& ghost & \ipa{tsʰo˧qʰwɤ˧-mi\#˥} & \ipa{tsʰo˧qʰwɤ˧-zo\#˥} & --\\
				& Bai (people) & \ipa{ɬi˧bv̩˧-mi\#˥} & \ipa{ɬi˧bv̩˧-zo\#˥} & --\\
				\lspbottomrule
			\end{tabularx}}
		\end{table}
		
		%Table 4b.
		\begin{table}%[t]
			\caption{\label{tab:genderthreesyllables}Nouns with gender suffixes or augmentative/diminutive
				suffixes. Three"=syllable words without a~corresponding disyllable.}
			\begin{tabularx}{\textwidth}{ l Q l l l }
				\lsptoprule
				root tone & root & \multicolumn{3}{l}{suffixed forms}\\ \cmidrule{3-5}
				& & //\ipa{-mi˩}// ‘female' & //\ipa{-zo˥}// ‘child' & //\ipa{-pʰv̩˥}// ‘male'\\ \midrule
				L & bird & \ipa{v̩˩dze˩-mi˩} & \ipa{v̩˩dze˩-zo˩} & \ipa{v̩˩dze˩-pʰv̩˩}\\ \addlinespace \hdashline \addlinespace
				L\# & bat & \ipa{dze˧bɤ˩-mi˩} & \ipa{dze˧bɤ˩-zo˩} & \ipa{dze˧bɤ˩-pʰv̩˩}\\
				& owl & \ipa{mo˧jo˩-mi˩} & \ipa{mo˧jo˩mi˩-zo˩} & \ipa{mo˧jo˩mi˩-pʰv̩˩}\\ \addlinespace \hdashline \addlinespace
				LM+MH\# & wolf & \ipa{õ˩dv̩˧-mi˥} & \ipa{õ˩dv̩˧-zo\#˥} & \ipa{õ˩dv̩˧-pʰv̩\#˥}\\ \addlinespace \hdashline \addlinespace
				H\# & camel & \ipa{njɤ˧mv̩˥-mi˩} & \ipa{njɤ˧mv̩˥mi˩-zo˩} &
				\ipa{njɤ˧mv̩˥mi˩-pʰv̩˩}\\ \addlinespace \hdashline \addlinespace
				unclear & cicada & \ipa{dʑɯ˧dze˧mi\#˥} & -- & --\\
				& vulture & \ipa{se˩gwɤ˩-mi˧} & -- & --\\
				\lspbottomrule
			\end{tabularx}
		\end{table}
	\end{subtables}
	
	For two of the items that lack a~{monosyllabic} counterpart, the tone of the root can be arrived at
	through internal \is{comparative method (historical linguistics)}reconstruction. Since there is a~substantial number of examples (seven) of LM"=tone
	disyllables corresponding to LM"=tone monosyllables, and there is no other attested source for
	LM"=tone disyllables, /\ipa{dze˩mi˧}/ ‘bee’ and /\ipa{lo˩mi˧}/ ‘thumb’ can be hypothesized to be
	derived from LM"=tone roots: \ipa{*dze˩˧} and \ipa{*lo˩˧}. 

	But for most of the items that lack a~\is{monosyllables}monosyllabic counterpart,
	internal \is{comparative method (historical linguistics)}reconstruction does not lead to a~clear"=cut conclusion, because several tone categories of
	roots feed into the same tone categories of \is{disyllables}disyllables. For instance, /\ipa{ɲi˧zo\#˥}/ ‘fish’ may
	have originated in a~\is{monosyllables}monosyllabic root of any tone category except H, and M-tone words with the
	/\ipa{-mi˩}/ \is{suffixes}suffix, such as /\ipa{ɖɤ˧mi˧}/ ‘fox’, may be derived from any of the following four
	tone categories of monosyllables: LH, L, H or MH. The last two items in \tabref{tab:gendertwosyllablesrest}, ‘hwamei (a species of bird)' and ‘(cigarette) lighter', carry a~tone that does not correspond to any of the attested correspondences.
	
	The purpose of Tables~\ref{tab:gendertwo}a--g and \ref{tab:genderthree}a--b is to provide a~bird’s eye view of the tonal correspondences. It does not
	present information about \isi{etymology} and frequency of use: for instance, that /\ipa{ɻ̍̃˧mi˧}/ ‘treek trunk’ (in \tabref{tab:gendertwosyllablesh}) etymologically means ‘big bone’; that
	/\ipa{po˧lo˧}/ is a~more common form for ‘ram’ than the /\ipa{-pʰv̩˥}/ suffixed form /\ipa{jo˧pʰv̩\#˥}/{\kern2pt}\ipa{≈}{\kern2pt}/\ipa{jo˩pʰv̩˩}/ (literally ‘male sheep’) shown in \tabref{tab:gendertwosyllablesl}; or that
	/\ipa{-zo˥}/ suffixed forms are more common than /\ipa{-pʰv̩˥}/ suffixed forms to refer to male mules and
	water buffalos (\tabref{tab:gendertwosyllablesrest}). Some such facts are adduced in the discussion below; they can be looked up in the corresponding entries in the dictionary \citep{michauddict2015}.
	
	%check on proofs: is it useful to introduce a
	\clearpage
	%here?

	\subsection{Discussion}
	\label{sec:thegendersuffixesdisc}
	
	The tonal correspondences between \is{monosyllables}monosyllabic roots and \is{disyllables}disyllables are not one"=to"=one. The
	diversity of tonal correspondences suggests that disyllables with these suffixes have different degrees
	of \isi{lexicalization}, and different degrees of historical depth. It would thus be misleading to think
	of all the suffixed forms as the result of a~synchronic, currently productive morphological
	process. Semantically, there is a~continuum between disyllables with a~clearly female meaning, such
	as ‘sow’, and disyllables in which the semantic content of the \is{suffixes}suffix has become bleached,
	e.g.~/\ipa{kʰv̩˩mi˩}/ which simply means ‘dog’, not specifically ‘she"=dog’. After semantic bleaching, suffixes need to be added anew to specify gender. For instance, on the basis of /\ipa{njɤ˧mv̩˥mi˩}/ ‘camel’, the words for ‘camel calf’ and ‘male camel’ come out as /\ipa{njɤ˧mv̩˥mi˩-zo˩}/ and /\ipa{njɤ˧mv̩˥mi˩-pʰv̩˩}/, respectively. But these two forms, while readily understandable, are considered awkward by the main consultant. Does this mean that the /\ipa{-mi˩}/ component in /\ipa{njɤ˧mv̩˥mi˩}/ ‘camel’ is still perceived as carrying its {female} meaning? Not necessarily: the slight weirdness of /\ipa{njɤ˧mv̩˥mi˩-zo˩}/ ‘camel calf’ and /\ipa{njɤ˧mv̩˥mi˩-pʰv̩˩}/ ‘male camel’ seems due to the /{\dots}\ipa{mi.zo}/ and /{\dots}\ipa{mi.pʰv}/ sequences, where the \is{suffixes}suffix re"=activates, as it were, the gender denotation of /\ipa{-mi˩}/ in ‘camel’.
	
	The study of combinations for which two tone patterns are acceptable, e.g. /\ipa{hwɤ˧zo\#˥}/ and
	/\ipa{hwɤ˧zo˥\$}/ for ‘male kitten’, reveals that tonal variants are item"=specific
	(lexicalized). Out of the set of nine words suffixed in \mbox{/\ipa{-zo˥}/} or \mbox{/\ipa{-pʰv̩˥}/} corresponding
	with a~MH"=tone root, only four allow both variants, \#H and H\$. Among the nine words, four refer to
	names of objects: ‘basket’, ‘needle’, ‘scales’, and ‘bowl’. Interestingly, none of these four allows
	a~tonal {variant}: they all fall into one category or the other (\#H for ‘needle’, and H\$ for the
	other three). By contrast, four of the five animal names allow both variants. The difference between
	the gender suffixes and size suffixes in this respect is not clear"=cut: some nouns
	referring to objects allow two variants. For instance, ‘ladle’ conforms to the exceptionless
	existence of two tonal variants for LM"=tone roots: /\ipa{tɕʰo˩mi\#˥}/{\kern2pt}\ipa{≈}{\kern2pt}/\ipa{tɕʰo˩mi˧}/, and
	/\ipa{tɕʰo˩zo\#˥}/{\kern2pt}\ipa{≈}{\kern2pt}/\ipa{tɕʰo˩zo˧}/. Still, there appears to be a~statistical tendency for animal names
	to preserve a~tonal flexibility reflecting the stronger perception of their internal structure by
	the speakers.
	
	In some cases, it is possible to identify factors that have played a~role in contributing to the current tonal output: ‘little bottle’,
	/\ipa{kɤ˩zo˩}/, does not have the expected \#H-tone {variant} /\ipa{†kɤ˧zo\#˥}/; to the consultant,
	the latter form immediately summoned up the given name /\ipa{kɤ˧zo\#˥}/. Supposing that ‘little
	bottle’ once had two tonal variants, L and \#H, the pressure to avoid \isi{homophony} may have played
	a~role in the selection of the \is{variants}variant /\ipa{kɤ˩zo˩}/ to the exclusion of the other.
	
	The discussion below is arranged by tone of the root noun, in the same order as in Tables~\ref{tab:gendertwo}a--g and \ref{tab:genderthree}a--b. For each tone category of roots, forms suffixed with //\ipa{-mi˩}// are discussed first, followed by forms with //\ipa{-zo˥}// and //\ipa{-pʰv̩˥}//.
	
	
	\subsubsection{LM"=tone roots}
	\label{sec:lmtoneroots}
	
	
	Table~\ref{tab:gendertwo} brings out no fewer than four tone correspondences between monosyllables with LM tone and their suffixed
	forms: LM, as in /\ipa{bo˩mi˧}/ ‘sow’ and /\ipa{æ˩mi˧}/ ‘hen’; M, as in /\ipa{bv̩˧mi˧}/ ‘female yak’,
	/\ipa{ɖʐwæ˧mi˧}/ ‘female sparrow’ and /\ipa{kɤ˧mi˧}/ ‘female falcon’; LM+\#H, in /\ipa{dv̩˩mi\#˥}/
	‘female weasel’, /\ipa{ɑ˩mi\#˥}/ ‘goose’, /\ipa{dzo˧mi\#˥}/ ‘female lizard’, and /\ipa{nɑ˩mi\#˥}/ ‘Na
	woman’; and finally L, in /\ipa{pʰɤ˩mi˩}/ ‘female jackal’.
	
	Three LM"=tone nouns outside the semantic sphere of animal names can carry the /\ipa{-mi˩}/ \is{suffixes}suffix,
	with three of the above four tone patterns. One has M tone: /\ipa{bv̩˧mi˧}/ ‘big food steamer’ (from
	/\ipa{bv̩˩˧}/ ‘food steamer’). Another has LM+\#H tone: /\ipa{tɕʰo˩mi\#˥}/ ‘big ladle’ (from
	/\ipa{tɕʰo˩˧}/ ‘ladle’). The third has L tone: /\ipa{ʐɤ˩mi˩}/ ‘road, path’, which does not
	specifically mean ‘large road’ anymore: the \is{monosyllables}monosyllable /\ipa{ʐɤ˩˧}/, likewise meaning ‘road,
	path’, is falling into disuse.
	
	Given the diversity of these patterns, it is hard to establish the relative chronology of the tone
	rules that produced the four types (LM, M, LM+\#H, and L). A~few hints may be detected
	nonetheless. While LM"=tone monosyllables correspond to no fewer than four tone categories of
	suffixed forms, LM"=tone suffixed forms only correspond to LM"=tone monosyllables. In other words,
	LM"=tone monosyllables are the only source of LM"=tone disyllables. The examples, ‘sow’, ‘hen’, ‘bee’ and ‘thumb’, belong to basic vocabulary, suggesting that the LM::LM {correspondence} between \is{monosyllables}monosyllable and disyllable reflects an~older
	pattern.
	
	The tone patterns for the //\ipa{-zo˥}// and //\ipa{-pʰv̩˥}// suffixes are fully consistent with those for
	the \is{suffixes}suffix //\ipa{-mi˩}//: in the second type of correspondences in Table~\ref{tab:gendertwo}, forms suffixed with //\ipa{-zo˥}// or //\ipa{-pʰv̩˥}// always have \#H (with
	L as a~\is{variants}variant) when the form suffixed with //\ipa{-mi˩}// has M; in the third type, words with any of these three suffixes have LM+\#H, with LM as
	a~\is{variants}variant. 
	
	On the other hand, these patterns differ widely from those observed in other syntactic
	structures, such as \is{compounds}compound nouns and combinations of nouns with verbs or adjectives. Thus, different syntactic structures are reflected in different tone rules~-- with the added complexity of numerous lexicalized \is{irregularities}oddities.
	
	\subsubsection{LH"=tone roots}
	\label{sec:lhtoneroots}
	
	Monosyllables with LH tone correspond to disyllables carrying LM+\#H. But two items also
	have a~L-tone \is{variants}variant, namely /\ipa{ʑi˩mi\#˥}/{\kern2pt}\ipa{≈}{\kern2pt}/\ipa{ʑi˩mi˩}/ for ‘female monkey’ and
	/\ipa{tɕʰɯ˩mi\#˥}/{\kern2pt}\ipa{≈}{\kern2pt}/\ipa{tɕʰɯ˩mi˩}/ for ‘female muntjac’. However, /\ipa{ʐæ˩mi\#˥}/ ‘female
	leopard’ does not allow this \is{variants}variant: the form \ipa{$\ddagger${\kern2pt}ʐæ˩mi˩} is not acceptable. Outside the field
	of animal names, /\ipa{tʰi˩mi\#˥}/ ‘large plane’ (from /\ipa{tʰi˩˥}/ ‘plane [carpentry tool]’)
	cannot be realized as \ipa{$\ddagger${\kern2pt}tʰi˩mi˩}; conversely, the word for ‘large slope’ (from /\ipa{to˩˥}/
	‘slope’) is /\ipa{to˩mi˩}/, and \ipa{$\ddagger${\kern2pt}to˩mi\#˥} is not acceptable. No certainty can be reached
	at present on whether two different tone rules applied at different times (in which case the
	existence of two variants would be a~development due to \isi{analogy} or dialect contact) or the
	two variants used to coexist for all items, some of which lost one of the variants. %The fact that
%	variants are present for the majority of items within the same {correspondence} class is consistent with the
%	latter hypothesis.
	
	Two other patterns are illustrated by only one example each. These words may be relatively old: M tone
	in /\ipa{di˧mi˧}/ ‘plain’ (compare ‘earth, land’, /\ipa{di˩˥}/), and LH tone in /\ipa{ljɤ˩mi˥}/
	‘main beam’.
	
	\subsubsection{M-tone roots}
	\label{sec:mtoneroots}
	
	Monosyllables with M tone yield disyllables with \#H tone: /\ipa{lɑ˧mi\#˥}/ ‘female tiger’ and
	/\ipa{se˧mi\#˥}/ ‘female goral’. This pattern is also found, outside the semantic field of animal
	names, in /\ipa{qʰwæ˧mi\#˥}/ ‘message; letter’. That this rule is currently productive was verified
	by adding the augmentative \is{suffixes}suffix to the M-tone noun /\ipa{qwæ˧}/ ‘bed mat’, yielding /\ipa{qwæ˧mi\#˥}/ ‘large bed mat’.
	
	The situation is more complex for disyllables with //\ipa{-mi˩}// added as a~\is{suffixes}suffix. There are two attested patterns: --L, in /\ipa{si˧gɯ˧-mi˩}/
	‘lionness’, /\ipa{ʐv̩˧bæ˧-mi˩}/ ‘female snake’ and /\ipa{tʰo˧li˧-mi˩}/ ‘hare’, and \#H, in
	/\ipa{ɬi˧bv̩˧-mi\#˥}/ ‘woman of the Bai ethnic group’. The latter {correspondence} coincides with that
	found for monosyllables. To determine which of the two is currently productive, the \is{suffixes}suffix was added
	to a~word to which it is not normally attached: ‘earthworm’, /\ipa{dʑɯ˧dv̩˧}/, as the earthworm is a~hermaphrodite. For elicitation, the imagined context was the following: a~child wonders whether there are such things as male and female earthworms, and asks, ‘Are
	there such things as female earthworms?/Do female earthworms exist?’ The speaker had no hesitation in formulating
	the {question} as (\ref{ex:femaleearthworm}), with --L tone on the suffixed form
	/\ipa{dʑɯ˧dv̩˧-mi˩}/ ‘female earthworm’.
	
\Hack{\newpage}

	\begin{exe}
		\ex
		\label{ex:femaleearthworm}
		\ipaex{dʑɯ˧dv̩˧-mi˩ {\kern2pt}|{\kern2pt} ə˩-dʑo˩˥?}\\
		\gll dʑɯ˧dv̩˧-mi˩		ə˩-				dʑo˩\textsubscript{b}\\
		female\_earthworm		\textsc{interrog}	\textsc{exist}\\
		\glt ‘Are there such things as female earthworms?/Do female earthworms exist?’ (Source: field notes.)
	\end{exe}
	
	
%	In this second type, the tone patterns for the /\ipa{-zo˥}/ and /\ipa{-pʰv̩˥}/ suffixes are identical with those for /\ipa{-mi˩}/.
	
	
	\subsubsection{L-tone roots}
	\label{sec:ltoneroots}
	
	Monosyllabic words carrying L tone yield M-tone disyllables: /\ipa{jo˧mi˧}/ ‘ewe’,
	/\ipa{ɬi˧mi˧}/ ‘female roebuck’, and /\ipa{mv̩˧mi˧}/ ‘woman’. This last example is
	attested in a~proverb: see (\ref{ex:womanARCH}).
	
	\begin{exe}
		\ex
		\label{ex:womanARCH}
		\ipaex{mv̩˧mi˧ ʈʂʰwɤ˩ mɤ˩-ɖɯ˩,  {\kern2pt}|{\kern2pt}  kʰv̩˧nɑ˥ ʐo˩ mɤ˩-ɖɯ˩.}\\
		\gll mv̩˧mi˧	ʈʂʰwɤ˥						mɤ˧-			ɖɯ˧\textsubscript{b}		 kʰv̩˧nɑ˥			ʐo˩							mɤ˧-	ɖɯ˧\textsubscript{b}\\
		woman	dinner~(evening~meal)		\textsc{neg}	to\_get							black\_dog		lunch		\textsc{neg}	to\_get\\
		\glt ‘No dinner for the [married] woman, no lunch for the black dog.’ (Sister.130, 131, 139, 158,
		171 and Sister3.3, 113, 117) \textit{Explanation:} ‘No dinner for the married woman’:  if a~married woman goes back to her original home for a~visit during the day, she cannot stay for the evening meal, as she has obligations back at the household she has married into. ‘No lunch for the black dog’: ‘black dog’ here refers to any dog, in fact. Dogs only get two meals a~day, one in the morning and one in the evening. 
	\end{exe}
	
	
	The word /\ipa{mv̩˧mi˧}/ ‘woman’ is no longer intelligible to younger speakers, such as M23. It is
	likely that the {correspondence} between L-tone roots and M-tone disyllabic forms reflects
	a~tone rule that has a~relatively great time depth. The same tone (M) is found in
	/\ipa{kɤ˧mi˧}/ ‘large bottle’ and /\ipa{dʑɯ˧mi˧}/ ‘large river’.
	
	The tone patterns for the /\ipa{-zo˥}/ and /\ipa{-pʰv̩˥}/ suffixes are identical for roots with tones L
	and LM.
	
	
	\subsubsection{H-tone roots}
	\label{sec:htoneroots}
	
	
	For words derived from H-tone monosyllables, we find
    %Suggestion by S. Nordhoff (accepted): \rephrase{there are}{we find } 
    three patterns: \#H tone, in
	/\ipa{ho˧mi\#˥}/ ‘female pheasant’ and /\ipa{bɤ˧mi\#˥}/ ‘\ili{Pumi} woman’; L tone, in /\ipa{ʐwæ˩mi˩}/ ‘mare’, /\ipa{ʝi˩mi˩}/
	‘cow’ and /\ipa{kʰv̩˩mi˩}/ ‘dog’ (discussed further below); and M tone, in /\ipa{kʰi˧mi˧}/ ‘main door’, /\ipa{qʰæ˧mi˧}/ ‘canal; large ditch’,
	and /\ipa{ɻ̍̃˧mi˧}/ ‘tree trunk’ (etymologically ‘large bone’).
	
	The first pattern makes good synchronic sense: the tones of the \is{monosyllables}monosyllable and disyllable
	correspond neatly with each other. The semantic relationship is also clear: the {monosyllabic} term (‘pheasant’, ‘\ili{Pumi}’) does not refer to gender, and the suffixed term does (‘female pheasant’, ‘\ili{Pumi} woman’). The second and third patterns are not phonologically
	transparent. Concordantly, the semantic relationship is less clear in some cases. The words for
	‘mare’, ‘cow’ and ‘dog’ illustrate three stages in the gradual evolution of the \is{suffixes}suffix’s
	meaning. The \is{monosyllables}monosyllable for ‘horse’, /\ipa{ʐwæ˥}/, is in common use, and the word for ‘mare’,
	/\ipa{ʐwæ˩mi˩}/, simply specifies gender. On the other hand, the \is{monosyllables}monosyllable /\ipa{ʝi˥}/ for ‘cow’
	is not in frequent use; there are more than ten different disyllables pronounced /\ipa{ʝi}/, six of
	them with H tone, and the //\ipa{-mi˩}// \is{suffixes}suffix serves the purpose of disambiguation. In this role, while
	the \is{suffixes}suffix retains its female meaning, it can be said to function as an~animal \is{suffixes}suffix just as much
	as a~female \is{suffixes}suffix. The third example, /\ipa{kʰv̩˩mi˩}/, ‘dog’, is further down this
	evolutionary path: it refers to dogs both male and female, and the \is{monosyllables}monosyllable /\ipa{kʰv̩˥}/ for
	‘dog’ is seldom used (but firmly attested, witness Dog.1, 3, 45 and Dog2.68, 74-77, 79). These observations suggest that \#H may be the tone of more recently derived words, and
	L a~tone that used to obtain at an~earlier stage, and that remains lexically preserved in some old
	words.
	
	This conjecture is confirmed by examples from outside the semantic field of animal names:
	/\ipa{sɑ˩mi˩}/ ‘Cannabis indica’ (the psychotropic plant), which corresponds to \is{monosyllables}monosyllabic /\ipa{sɑ˥}/
	‘Cannabis sativa’ (used to produce thread), has L tone. The semantic content of the \is{suffixes}suffix is
	‘large’, referring to the size of the leaves (as an~aside: in {Mandarin} too, ‘cannabis’ \textit{dàmá} \zh{大麻} is derived from ‘hemp’ \textit{má} \zh{麻} by addition of the augmentative \textit{dà} \zh{大} ‘large’). Again, this suggests that L was the tone that used to obtain at an~earlier stage. By
	contrast, ‘large pot’, /\ipa{v̩˧mi\#˥}/ (compare /\ipa{v̩˥}/ ‘pot’), a~less clearly lexicalized and probably more recent
	disyllable, carries \#H tone, as does /\ipa{sɯ˧ɻ̍̃˧mi\#˥}/ ‘backbone, spine’ (compare /\ipa{sɯ˧ɻ̍̃\#˥}/
	‘tree trunk’).
	
	The tone patterns for the /\ipa{-zo˥}/ and /\ipa{-pʰv̩˥}/ suffixes are identical for roots with tones H
	and M.
	
	
	\subsubsection{MH"=tone roots}
	\label{sec:mhtoneroots}
	
	MH"=tone monosyllables all correspond to disyllables with H\$ tone. The following examples were
	observed: /\ipa{hwɤ˧mi˥\$}/ ‘she"=cat’; /\ipa{ʈʂʰæ˧mi˥\$}/ ‘hind’; /\ipa{tsʰɯ˧mi˥\$}/ ‘nanny goat’;
	/\ipa{ʁv̩˧mi˥\$}/ ‘female crane’; /\ipa{tɕɯ˧mi˥\$}/ ‘wasp’; and, from outside the semantic field of
	animal names, /\ipa{kʰɤ˧mi˥\$}/ ‘large basket’, /\ipa{ʁo˧mi˥\$}/ ‘big needle’, /\ipa{tɕɯ˧mi˥\$}/
	‘large basket’, and /\ipa{hu˧mi˥\$}/ ‘stomach, bowels’ (this last noun is now more common than \is{monosyllables}monosyllabic
	/\ipa{hu˧˥}/, and has no strong connotation of ‘big’).
	
	Words with the /\ipa{-zo˥}/ and /\ipa{-pʰv̩˥}/ suffixes can carry either \#H, a~pattern which is widely
	attested with these suffixes, or H\$, the same tone found in items suffixed with /\ipa{-mi˩}/.
	
	\subsubsection{Some observations about other lexical tones}
	\label{sec:someobservationsaboutotherlexicaltones}
	
	As predicted by Rule 5 (“All syllables following a~H.L or
	M.L sequence receive L tone”: see \sectref{sec:alistoftonerules}), L\# tone spreads over the {suffix}, yielding M.L.L:
	/\ipa{dze˧bɤ˩-mi˩}/ ‘female bat’, /\ipa{dze˧bɤ˩-zo˩}/ ‘little bat, pup’, and /\ipa{dze˧bɤ˩-pʰv̩˩}/
	‘male bat’. On this basis, a~disyllabic \ipa{*mo˧jo˩} can confidently be extracted from
	/\ipa{mo˧jo˩-mi˩}/ ‘owl’. (Remember that the asterisk indicates a~\is{comparative method (historical linguistics)}reconstruction, not an~ungrammatical form.)
	
	The tone patterns of ‘cicada’, ‘vulture’, ‘hwamei (bird)’ and ‘(cigarette) lighter’ have no equivalent elsewhere, making it impossible (in the present stage of our knowledge) to extract the tones of their roots.
	
	
	\subsubsection{Concluding observations}
	\label{sec:concludinggeneralobservations}
	
	
	The patterns in Tables~\ref{tab:gendertwo}a--g and \ref{tab:genderthree}a--b are summarized in Tables~\ref{tab:tcorrd}a--b. It must be emphasized that the classification
	into currently productive patterns and older patterns for the //\ipa{-mi˩}// {suffix} is speculative. As elsewhere, a~slash (/) separates variants.
	
	\begin{subtables}
		\label{tab:tcorrd}
		%Table 5.
		\begin{table}%[t]
			\caption{\label{tab:tcorrmono}Tonal correspondences between {monosyllabic} base forms and disyllables containing the suffixes /\ipa{-mi˩}/,
				/\ipa{-zo˥}/ and /\ipa{-pʰv̩˥}/, with tentative indications on whether the tone pattern is currently
				productive.}
						\begin{tabularx}{\textwidth}{ l l Q l }
				\lsptoprule
				& \multicolumn{2}{l}{/\ipa{-mi˩}/} & /\ipa{-zo˥}/, /\ipa{-pʰv̩˥}/\\\cmidrule(lr){2-3}\cmidrule(lr){4-4}
				& older? & productive? & no distinctions in productiveness\\ \midrule
				LM & LM; L & M; LM+\#H / LM & LM+\#H / LM; \#H / L\\
				LH & M; LH; L & LM+\#H / L & LM+\#H / L\\
				M &  & \#H & \#H\\
				L &  & M & \#H / L\\
				H & L; M & \#H & \#H\\
				MH &  & H\$ & \#H / H\$\\
				\lspbottomrule
			\end{tabularx}
		\end{table}
		
		\begin{table}%[t]
			\caption{\label{tab:tcorrdi}Tonal correspondences between disyllabic base forms and {trisyllabic} nouns containing the suffixes /\ipa{-mi˩}/, /\ipa{-zo˥}/ and /\ipa{-pʰv̩˥}/, with tentative indications on whether the tone pattern is currently productive.}
			\begin{tabularx}{\textwidth}{ l l Q l }
				\lsptoprule
				& \multicolumn{2}{l}{/\ipa{-mi˩}/} & /\ipa{-zo˥}/, /\ipa{-pʰv̩˥}/\\\cmidrule(lr){2-3}\cmidrule(lr){4-4}
				& older? & productive? & no distinctions in productiveness\\ \midrule
				M & \#H & --L (L\#) & \#H\\
				H &  & \#H &\\
				L &  & L & L\\
				L\# &  & L\#-- & L\#--\\
				LM+MH\# &  & LM+H\# & LM+\#H\\
				H\# &  & H\#-- & --\\
				\lspbottomrule
			\end{tabularx}
		\end{table}
	\end{subtables}

	Roots with the same lexical tones correspond to diverse tones on suffixed forms, with as many as
	four types of correspondences for LM tone. The total number of noun subsets in Tables~\ref{tab:gendertwo}a--g and \ref{tab:genderthree}a--b, excluding
	disyllabic roots, is 14. Since /\ipa{-mi˩}/ on the one hand, and /\ipa{-zo˥}/ and /\ipa{-pʰv̩˥}/ on the
	other, fall into different tone categories, there are 2×14=28 potentially distinct tonal types of
	suffixed nouns. Given that many types have variants, the number of different tones on suffixed nouns
	could be considerable.
	%; one could expect it to cover the entire range of eleven existing tone patterns for disyllables. 
	Yet the set of tones observed on suffixed nouns is limited to six: \{M, \#H, H\$, L,
	LM, LM+\#H\}, apart from two outliers: tones that are only found in one example each. Thus some tone categories
	contain large numbers of words produced through \is{suffixes}suffixation or compounding, whereas others are not
	fed by any currently productive combination processes. Such facts contribute to giving different lexical tone categories
	their own specific morphological flavours.
	
	There are only five tone patterns for forms suffixed in //\ipa{-zo˥}// and //\ipa{-pʰv̩˥}//: \{\#H, H\$, L, LM, LM+\#H\}. One additional pattern is
	attested for //\ipa{-mi˩}//, namely M tone. The relatively greater simplicity of tone patterns for
	//\ipa{-zo˥}// and //\ipa{-pʰv̩˥}// may be linked to their more restricted distribution in the lexicon:
	words with the ‘male’ and ‘child’ \is{suffixes}suffix, being fewer in number, may have undergone more \isi{simplification} of tone patterns by
	the analogical extension of productive patterns.	
	

	\subsection{Other suffixes for ‘male’}
		\label{sec:othermale}

{\largerpage}
	
	In addition to the currently productive \is{suffixes}suffix //\ipa{-pʰv̩˥}// for ‘male’, there also exist other, non-productive suffixes. These are mentioned for the sake of completeness, although the small number of examples greatly limits possibilities for analysis of their tone patterns.
		
		\subsubsection{The suffix /\ipa{-ʂwæ˧}/}
	\label{sec:thesuffixformale}
		
	The free form /\ipa{ʂwæ˧}/ currently has the meaning ‘castrated/neutered male’.
	%: see (\ref{ex:ismale2}). %% A proofreader noted that "the example immediately follows the colon, so there is no need to cross-reference the example here".
	
	\begin{exe}
		\ex
		\label{ex:ismale2}
		\ipaex{ʈʂʰɯ˧ {\kern2pt}|{\kern2pt} ʂwæ˧ ɲi˩.}\\
		\gll ʈʂʰɯ˥	ʂwæ˧	ɲi˩\\
		\textsc{dem.prox}		castrated\_male	\textsc{cop}\\
		\glt ‘This is a~castrated male.’
	\end{exe}
	
	
	The morpheme /\ipa{ʂwæ˧}/ may have had the
	meaning ‘male’ at an~earlier stage, however, witness the noun /\ipa{æ̃˧ʂwæ˥}/, meaning ‘rooster, cock’ (compare /\ipa{æ̃˩˧}/ ‘chicken’). This noun has no competitor with the \is{suffixes}suffix /\ipa{-pʰv̩˥}/ (it is not possible to say \ipa{$\ddagger${\kern2pt}æ̃˩pʰv̩\#˥}). Roselle Dobbs (p.c.\ 2016) reports that in Lataddi \zh{喇塔地} ‘grandfather’ and ‘rooster’ sound comically similar; it could be that an earlier form \ipa{$\dagger${\kern2pt}æ̃˩pʰv̩\#˥} ‘rooster’ fell into disuse in the Alawa dialect (studied here) because of phonetic closeness with ‘grandfather’. 
	
	The \is{suffixes}suffix /\ipa{-ʂwæ}/ also appears in three other items in which it carries the meaning ‘castrated male’: see \tabref{tab:namesofanimalswiththesuffix}. Interestingly, the
	tone pattern is different for ‘cock’ and ‘castrated yak’, two words whose root has the same tone
	(LM) but in which the \is{suffixes}suffix takes different meanings: ‘male’ in one case, ‘castrated male’ in
	the other. It is a~safe guess that /\ipa{æ̃˧ʂwæ˥}/ ‘cock’ has greater time depth.
	
	Concerning the tone of the \is{suffixes}suffix, two of the words in which the \is{suffixes}suffix appears (‘castrated yak’ and ‘castrated male goat’) carry tone patterns that are among possible variants for the suffixes \mbox{//\ipa{-zo˥}//} ‘baby, male’ and \mbox{//\ipa{-pʰv̩˥}//} ‘male’, tentatively analyzed as carrying H tone. But the third word, /\ipa{tsʰɯ˧ʂwæ˥}/ ‘wether, castrated male goat’, does not have the same tone pattern as words containing the ‘baby, male’ and ‘male’ suffixes: /\ipa{tsʰɯ˧zo\#˥}/{\kern2pt}\ipa{≈}{\kern2pt}/\ipa{tsʰɯ˧zo˥\$}/ and /\ipa{tsʰɯ˧pʰv̩\#˥}/{\kern2pt}\ipa{≈}{\kern2pt}/\ipa{tsʰɯ˧{\allowbreak}pʰv̩˥\$}/ (see \tabref{tab:gendertwosyllablesmh}). This constitutes evidence that the \is{suffixes}suffix /\ipa{-ʂwæ}/ does not carry the same tone as the \mbox{//\ipa{-zo˥}//} and \mbox{//\ipa{-pʰv̩˥}//} suffixes.  The \is{suffixes}suffix /\ipa{-ʂwæ}/ is provisionally labelled here as carrying M tone, and transcribed as \mbox{//\ipa{-ʂwæ˧}//}, but it must be emphasized that this is mostly a~way of distinguishing its tone from that of the ‘baby, male’ and ‘male’ suffixes, provisionally transcribed as //\ipa{-zo˥}// and //\ipa{-pʰv̩˥}//, respectively.
	
	\begin{table}%[t]
		\caption{\label{tab:namesofanimalswiththesuffix}Names of animals with the suffix /\ipa{ʂwæ˧}/.}
		\begin{tabularx}{\textwidth}{ Q l l l }
			\lsptoprule
			tone of root & meaning of root & suffixed form & meaning\\ \midrule
			LM & chicken & \ipa{æ̃˧ʂwæ˥} & rooster (not castrated)\\
			LM & yak & \ipa{bv̩˩ʂwæ˩} & castrated yak\\
			L & sheep & \ipa{jo˩ʂwæ˩} & wether, castrated male sheep\\
			MH & goat & \ipa{tsʰɯ˧ʂwæ˥} & wether, castrated male goat\\
			\lspbottomrule
		\end{tabularx}
	\end{table}

		\subsubsection{The suffix /\ipa{-v̩}/}
		\label{sec:thesuffixv}
	\largerpage[-2]
	\citet[179]{lidz2010} proposes that the \is{suffixes}suffix in /\ipa{zɛ³¹-wu³³}/ ‘nephew' (F4: /\ipa{ze˩v̩˩}/) and /\ipa{ʐu³¹-wu³³}/ ‘grandson' (F4: /\ipa{ʐv̩˧v̩\#˥}/) comes from the root for ‘uncle/senior male
	relative’, which appears in /\ipa{ə˧v̩˧˥}/ ‘maternal uncle', and that this root also constitutes the origin of the classifier for individuals (F4: /\ipa{v̩˧}/).
	
		\subsubsection{The suffix /\ipa{-ʁo}/}
		\label{sec:thesuffixro}
	\largerpage[-1]

	The word for ‘castrated horse’ is /\ipa{ʐwæ˧ʁo˩}/. Horses have been the object of great care and interest in this part of the Himalayas for at least two millenia \citep{wang1980}, so it is no wonder that words belonging to this semantic field are numerous, some of them probably very old. This isolated example is clearly insufficient for linguistic analysis.
	

	\subsection{The kinship prefix /\ipa{ə˧-}/}
	\label{sec:thekinshipprefix}
	
	Another non-productive but readily identifiable {affix} is the kinship \is{prefixes}prefix /\ipa{ə˧-}/. It is common to various languages of the area, such as Qiang
	\citep[158–159]{evansetal2007}, \ili{Yi}, and {Mandarin}. \tabref{tab:kinshiptermswiththeprefix} presents the examples that were observed in
	Yongning Na, where this \is{prefixes}prefix is “the only common noun \is{prefixes}prefix” \citep[167]{lidz2010}.
	

	\begin{table}%[t]
		\caption{\label{tab:kinshiptermswiththeprefix}Kinship terms with the prefix /\ipa{ə˧-}/.}
		\begin{tabularx}{\textwidth}{ Q l l }
			\lsptoprule
			kinship term & tone & meaning\\ \midrule
			\ipa{ə˧mɑ˧} & M & mother ({term of address})\\
			\ipa{ə˧mi˧} & M & mother; aunt\\
			\ipa{ə˧pʰv̩˧} & M & grandmother’s elder brother\\
			\ipa{ə˧si˧} & M & great"=grandmother; ancestor\\
			\ipa{ə˧ɖo˧} & M & boyfriend/girlfriend, lover\\
			\ipa{ə˧ʑi˧˥} & MH\# & grandmother mother’s mother\\
			\ipa{ə˧v̩˧˥} & MH\# & maternal uncle\\
			\ipa{ə˧bo˥\$} & H\$ & paternal uncle\\
			\ipa{ə˧dɑ˥\$} & H\$ & father\\
			\ipa{ə˧ɕjɤ˩} & L\# & boyfriend/girlfriend, lover\\
			\ipa{ə˧jɤ˩} & L\# & maternal aunt: mother’s elder sister\\
			\ipa{ə˧tɕi˩} & L\# & maternal aunt: mother’s younger sister\\
			\ipa{ə˧mv̩˩} & L\# & elder sibling (brother or sister)\\
			\ipa{ə˧zɯ˩ / ə˩zɯ˩} & L\# / L & dual: us two\\
			\ipa{ə˧-sɯ˩kv̩˩ / ə˩-sɯ˧kv̩˥} & --L / LMH & 1\textsuperscript{st} person plural, inclusive\\
			\lspbottomrule
		\end{tabularx}
	\end{table}
	
	Monosyllabic forms do not exist, and no convincing method to extract the tone of the root could be
	found. It is tempting to hypothesize that the \is{prefixes}prefix does not make a~tonal contribution, and that
	the tone of the disyllable reflects that of the root: disyllables with M, MH\# and H\$ tone would
	originate in roots with M, MH and H tone, respectively, and disyllables with L\# tone would originate
	in roots with L, LM or LH tone. But this reasoning is highly speculative, and no evidence could be found
	to explore this issue further. The root /\ipa{mi}/ in /\ipa{ə˧mi˧}/ ‘mother’ is no doubt linked with
	the free form /\ipa{mi˩˧}/ ‘female’, and the root /\ipa{pʰv}/ in /\ipa{ə˧pʰv̩˧}/ ‘grandmother's elder brother’ with the free form /\ipa{pʰv̩˧}/ ‘male’, but unlike these two free forms, the kinship terms /\ipa{ə˧mi˧}/ ‘mother’ and /\ipa{ə˧pʰv̩˧}/ ‘grandmother's elder brother’ have the same tone, so it would be problematic to extract the tones of the roots from those of the suffixed kinship terms.
	
	From a~static"=synchronic point of view, it is also difficult to reach hard"=and"=fast conclusions, due
	to the limited amount of data: one \is{prefixes}prefix and four suffixes. With this qualification, one may
	observe that disyllables with M, H\$ or L tone can be the result of \is{suffixes}suffixation as well as
	\is{prefixes}prefixation; that disyllables with \#H, H\#, LM, LH, or LM+\#H tone can result from \is{suffixes}suffixation but not from
	\is{prefixes}prefixation; and that disyllables with MH\# or L\# tone can result from \is{prefixes}prefixation but not from
	\is{suffixes}suffixation. The issue of possible origins for the various tone categories of disyllables (by \is{suffixes}suffixation, \is{prefixes}prefixation, and compounding) is taken up in \sectref{sec:possibleoriginsfordisyllablesonthebasisoftheirtoneabirdseyeview}.
	
	Concerning the tone of the kinship \is{prefixes}prefix, it seems reasonable to analyze it as M, since the \is{prefixes}prefix always appears with M tone except for two variants shown in the last two lines of \tabref{tab:kinshiptermswiththeprefix}. This is not really different from an~analysis under which this \is{prefixes}prefix is underlyingly toneless, since M behaves in some respects as a~default tone, as mentioned in \sectref{sec:analysisofmasadefaulttone}.
	
	Kinship terms in the Luoshui dialect \citep[167]{lidz2010} are similar to those in Alawa (the dialect studied here). For instance, Luoshui /\ipa{ɑ³³ʐɯ³³}/
	for ‘grandmother’ is cognate with Alawa /\ipa{ə˧ʑi˧˥}/. Only three terms from
	L.~Lidz’s list are not attested in Alawa. One of these is /\ipa{ɑ³³pɔ³¹}/, for ‘uncle: father’s
	elder or younger brother’. This could be a~\is{loanwords}borrowing from \ili{Mandarin} \textit{ābó} \zh{阿伯}
	‘brother"=in"=law; father’s older brother’.\footnote{In \textit{Pinyin} romanization, \textit{b} stands for a~voiceless bilabial stop: /\ipa{p}/, not /\ipa{b}/.} Borrowing is facilitated by the similar structure in both
	languages, with a~similar \is{prefixes}prefix (in \ili{Mandarin}: \textit{ā} \zh{阿}). A~different term is in use in
	Alawa: /\ipa{ə˧bo˥\$}/; its voiced initial suggests that it is not a~recent \is{loanwords}borrowing from
	{Mandarin}, which does not retain voiced stops. Ethnological data sheds light on the fact that the terms
	for uncles on the father’s side do not correspond neatly across dialects: the social relationship
	with one’s father (and his household) was traditionally loose (see Appendix B, \sectref{sec:anthropologicalresearchthefascinationofnafamilystructure}); accordingly, kinship terms on the father's side were not as specific as on the mother's side. 
	
	The peculiar structure of Na families invites linguistic speculation as to the origin and
	evolution of the terms currently used for relatives on the father’s
	side. Fu Maoji (\citeyear[23]{fu1980}; \citeyear[38–39]{fu1983}) hypothesizes that /\ipa{ə˧bo˥\$}/ ‘uncle on the father’s side’ used to refer to male
	relatives of the father’s generation, on the father’s side, i.e.\ the father and his brothers, and
	that the introduction of the term /\ipa{ə˧dɑ˥\$}/ ‘father’ led to the specialization of
	/\ipa{ə˧bo˥\$}/ to refer to uncles on the father’s side. This would imply that people had a~term to
	refer to their paternal uncles (pooled together with their father under the term /\ipa{ə˧bo˥\$}/)
	before they had a~term for ‘father’. This may seem paradoxical: since children did not live in the same household as their
	paternal uncles, their link to these uncles was through the father, and the existence of the notion of ‘father’ would appear as a~logical prerequisite for conceptualizing the broader notion of \textit{male relatives on the father's side, belonging to the father's generation}. Fu Maoji's reasoning nonetheless makes good sense in a~conceptual universe that is not based on nuclear families but on clans and extended families. If an individual is primarily identified in terms of belonging to a~household, and to a~generation inside the family, it would seem possible that fathers are not differentiated from their brothers in kinship terminology. Moreover, the lack of a~terminological distinction between the father and his brothers does not seem excessively surprising in view of the fact that ties with the father's family were loose and distant both economically and socially.\footnote{This reflection was proposed by Christine Mathieu (p.c.\ 2016). She quotes Lamu Gatusa \zh{拉木·嘎吐萨} as reporting a~word in the dialect of Labai \zh{拉柏} that is cognate with /\ipa{ə˧bo˥\$}/ and refers precisely to this concept: male relatives on the father's side, belonging to the father's generation, i.e.\ the father and his brothers. Distinctions can be made by adding the adjective ‘small’ to refer to the father's younger brothers, and ‘big’ to refer to his elder brothers.}
	
	Synchronically, extension of terms used within the traditional household (i.e.\ on the mother’s side) to the family of the father is occasionally observed, for instance publicly addressing one's father as /\ipa{ə˧v̩˧˥}/ ‘uncle
	on the mother’s side’ (field notes, consultant F4). The \is{stylistics}stylistic effect is to convey closeness~-- through inclusion in the household~-- and honour, as maternal uncles are characters to whom highest respect is due (as explained in Appendix B, \sectref{sec:themainsourceofinformationonnafamilystructuresurveysconductedinthe1960s}). In view of the plasticity of terms of address, one could venture
	an~alternative hypothesis about the origin of the term for ‘uncle on the father’s side’: that the words /\ipa{ə˧v̩˧˥}/ ‘uncle on the mother’s side’ and /\ipa{ə˧bo˥\$}/ ‘uncle on the father’s side’ in the Alawa dialect come from terms that used to refer to the mother’s
	older and younger brothers, respectively. This hypothesis would entail that the word
	/\ipa{ə˧bo˥\$}/, corresponding to a~socially less important and prestigious role than /\ipa{ə˧v̩˧˥}/,
	was later applied to uncles on the father’s side, while /\ipa{ə˧v̩˧˥}/ was extended to all of the
	mother’s brothers irrespective of age~-- preserving the hierarchy between /\ipa{ə˧v̩˧˥}/ as the more
	important social figure and /\ipa{ə˧bo˥\$}/ as the less important social figure, while transforming
	the age hierarchy into one between the mother’s side and the father’s side. There are two different terms for aunts: ‘mother's elder sister’, /\ipa{ə˧jɤ˩}/, and ‘mother's younger sister’, /\ipa{ə˧tɕi˩}/, so there is some plausibility in hypothesizing the existence of two different terms for uncles in an earlier state of the language. This hypothesis remains highly speculative, however. 
		
	The second term reported by Liberty Lidz that is not found in the present research data (Alawa dialect) is /\ipa{ɑ³³mɔ¹³}/ as another term for
	‘grandmother’. The third is /\ipa{ɑ³³lɑ³¹}/, referring to great"=great"=grandparents: in Alawa, the
	term /\ipa{ə˧si˧}/ is used for all ancestors of the great"=grandmother’s generation and above.
	
\is{derivation!morphological|)}


\section{Reduplication}
\label{sec:reduplicationofnounphraseswithadiscussionofsomemarginalcasesofreduplication}

In languages that do not have morphophonological templates specific to reduplication, it is not always easy to distinguish between reduplication and other types of repetition or copying. For instance, \citet[301]{moravcsik1978} considers \textit{very very} in the English example \textit{He is very very bright} as a~case of reduplication; this is debatable, because the intensifier can be repeated more than once (either an even number of times, 2×n, or an odd number of times: \textit{He is very, very, very bright}), with gradual rather than categorical semantic"=stylistic effects. In Yongning Na, reduplication is not too difficult to delimit on a~phonological basis, as it has specific tonal templates as well as clearly identifiable syntactic and semantic values. \is{reduplication}Reduplication of verbs is most common; it will be described in \ref{sec:reduplication}. Reduplication involving nouns is nowhere as frequent: the only well"=attested case is the \isi{reduplication} of \is{numerals}numeral"=plus"=classifier phrases. 


\subsection{Reduplication of numeral"=plus"=classifier phrases}
\label{sec:numclred}

Reduplication of a~phrase consisting of the \is{numerals}numeral ‘one’ (analyzed as carrying MH tone: //\ipa{ɖɯ˧˥}//) plus a~\is{classifiers}classifier indicates iteration. The entire reduplicated
phrase is integrated into a~single \isi{tone group} (about this crucial unit of Na morphotonology, see Chapter \ref{chap:toneassignmentrulesandthedivisionoftheutteranceintotonegroups}). The tone pattern of the first part of the phrase
conditions that of the second, by application of the phonological rules set out in \sectref{sec:alistoftonerules}:

\begin{itemize}
	\item{after a~H-tone classifier, the second half of the phrase receives L
		tone by application of Rules 4 and 5, e.g.~//\ipa{ɖɯ˧-ɲi˥\$}// → /\ipa{ɖɯ˧-ɲi˥{$\sim$}ɖɯ˩-ɲi˩}/ ‘day after day’
		(Reward.155, BuriedAlive2.85, Caravans.259)}
	\item{after a~M-tone classifier, the second part is
		unaffected, as in //\ipa{ɖɯ˧-ʁwɤ˧}// → /\ipa{ɖɯ˧-ʁwɤ˧{$\sim$}ɖɯ˧-ʁwɤ˧}/ ‘one heap after another’
		(Housebuilding.51)}
	\item{after L, the second part is lowered to L by application of Rule 5,
		e.g.~//\ipa{ɖɯ˧-ʑi˩}// → /\ipa{ɖɯ˧-ʑi˩{$\sim$}ɖɯ˩-ʑi˩}/ ‘one family after the other’ (Healing.94,
		Caravans.237)}
	\item{after MH, the H part of the \is{tonal contour}contour lands onto the first syllable of the second half: //\ipa{ɖɯ˧-kɤ˧˥}// →
		/\ipa{ɖɯ˧-kɤ˧{$\sim$}ɖɯ˥-kɤ˩}/ ‘one tree after the other’ (Housebuilding.28). The final L tone in this expression obtains by application of Rule~4.}
\end{itemize}


\subsection[A reduplicated nominal suffix]{Addition of the reduplicated suffix /\ipa{-ʂo˧{$\sim$}ʂo˩}/ to nouns, conveying abundance}
\label{sec:additionofreduplicatedsuffix}

Addition of the \is{suffixes}suffix /\ipa{-ʂo˧{$\sim$}ʂo˩}/ to nouns conveys abundance: examples include
/\ipa{mɤ˩-ʂo˩{$\sim$}ʂo˥}/ ‘smeared with grease, covered with grease’ (\ref{ex:Lake315}), /\ipa{ʂe˧-ʂo˧{$\sim$}ʂo˥}/ ‘rich in meat, with lots of meat’ (\ref{ex:Dog35}), and /\ipa{si˧-ʂo˧{$\sim$}ʂo˥}/ ‘packed with wood’ (\ref{ex:Housebuilding281}). 

\begin{exe}
	\ex
	\label{ex:Lake315}
	\ipaex{mv̩˩kʰv̩˧˥ {\kern2pt}|{\kern2pt} le˧-tsʰɯ˩-dʑo˩, {\kern2pt}|{\kern2pt} ɲi˧to˧ {\kern2pt}|{\kern2pt} ʈʂʰɯ˧-qo˧ {\kern2pt}|{\kern2pt} le˧-tɑ˧˥, {\kern2pt}|{\kern2pt} ʈʂʰɯ˧-qo˧ {\kern2pt}|{\kern2pt} le˧-tɑ˧˥, {\kern2pt}|{\kern2pt} \textbf{mɤ˩-ʂo˩$\textasciitilde$ʂo˥} tsɤ˩ tsɯ˩ {\kern2pt}|{\kern2pt} mv̩˩!}\\
	\gll mv̩˩kʰv̩˧˥		le˧-					tsʰɯ˩\textsubscript{a}		-dʑo˥		ɲi˧to˧		ʈʂʰɯ˧-qo˧		le˧-tɑ˧˥	mɤ˩		\textbf{-ʂo˧$\textasciitilde$ʂo˩}		tsɤ˧	tsɯ˧˥	mv̩˧\\
	evening		\textsc{accomp}			to\_come.\textsc{pst}	\textsc{top}	mouth	here	up\_to		animal\_fat		\textbf{\textsc{abundance}}	to\_become	\textsc{rep}		\textsc{affirm}\\
	\glt ‘The story goes that in the evening, when [the dumb man] came back, [his] mouth was smeared with grease up to here, up to here!’ (Lake3.15)
\end{exe}

\begin{exe}
	\ex
	\label{ex:Dog35}
	\ipaex{kʰv̩˩mi˩-ki˥, {\kern2pt}|{\kern2pt} ə{\dots} ʂe˧! {\kern2pt}|{\kern2pt} ɲi˧zo˧ {\kern2pt}|{\kern2pt} ɖɯ˧-kʰwɤ˥, {\kern2pt}|{\kern2pt} æ̃˩-ʂe˧ {\kern2pt}|{\kern2pt} ɖɯ˧-kʰwɤ˥, {\kern2pt}|{\kern2pt} bo˩-ʂe˧ {\kern2pt}|{\kern2pt} ɖɯ˧-kʰwɤ˥, {\kern2pt}|{\kern2pt} ʝi˧-ʂe˧ {\kern2pt}|{\kern2pt} ɖɯ˧-kʰwɤ˥, {\kern2pt}|{\kern2pt} kʰv̩˩mi˩-ki˥ {\kern2pt}|{\kern2pt} tʰv̩˧-hɑ̃˩-dʑo˩, {\kern2pt}|{\kern2pt} \textbf{ʂe˧-ʂo˧~ʂo˥} {\kern2pt}|{\kern2pt} tʰi˧-ki˧-kv̩˧ tsɯ˥ {\kern2pt}|{\kern2pt} mv̩˩!}\\
	\gll kʰv̩˩mi˩	-ki˧		ə{\dots}					ʂe˥		ɲi˧zo\#˥		ɖɯ˧-kʰwɤ˥\$			æ̃˩-ʂe\#˥		ɖɯ˧-kʰwɤ˥\$		bo˩-ʂe\#˥		ɖɯ˧-kʰwɤ˥\$		ʝi˧-ʂe\#˥		ɖɯ˧-kʰwɤ˥\$		kʰv̩˩mi˩	-ki˧		tʰv̩˧-hɑ̃˩		-dʑo˥		ʂe˥		\textbf{-ʂo˧$\textasciitilde$ʂo˩}	tʰi˧-		ki˧\textsubscript{a}		-kv̩˧˥		tsɯ˧˥		mv̩˧\\
	dog		\textsc{dat}		\textit{hesitation}		meat	fish	one-\textsc{clf}.piece		chicken\_meat		one-\textsc{clf}.piece		pork		one-\textsc{clf}.piece	beef		one-\textsc{clf}.piece	dog		\textsc{dat}	that\_day	\textsc{top}	meat	\textbf{\textsc{abundance}}		\textsc{dur}	to\_give	\textsc{abilitive}	\textsc{rep}	\textsc{affirm}\\
	\glt ‘To the dog, erm{\dots} [one would give] meat! A~piece of fish, a~piece of chicken, a~piece of pork, a~piece of beef{\dots} On that day [New Year's Eve], one would give the dog plenty of meat!' (Dog.35)
\end{exe}

\begin{exe}
	\ex
	\label{ex:Housebuilding281}
	\ipaex{nɑ˩-ʑi˧mi˧ ʈʂʰɯ˧ {\kern2pt}|{\kern2pt} ɖɯ˧-ʈʂɤ˥$\textasciitilde$ʈʂɤ˩-ki˩-ze˩-se˩ {\kern2pt}|{\kern2pt} \textbf{si˧-ʂo˧$\textasciitilde$ʂo˥}-ɲi˩!}\\
	\gll nɑ˩˧		ʑi˧mi˧		ʈʂʰɯ˧						ɖɯ˧-			ʈʂɤ˧\textsubscript{a}	$\textasciitilde$	ki˧\textsubscript{a}		-ze˧	-se˩	si˥		\textbf{-ʂo˧$\textasciitilde$ʂo˩}	-ɲi˩\\
	Na~(\isi{endonym})	house		\textsc{top}		\textit{delimitative}		to\_count	\textsc{activity}	to\_give		\textsc{pfv}	\textsc{completion}		wood		\textbf{\textsc{abundance}}	\textsc{certitude}\\
	\glt ‘The Na house, if one is going to count every part of it, [one will realize that] it is packed with wood!' (i.e.\ a~huge deal of wood goes into its construction) (Housebuilding.281)
\end{exe}

This suffix does not have a non"=reduplicated counterpart, and is not part of a~broader set of reduplicated nominal suffixes, so one may wonder whether it is not a~reduplicated form but a~simple suffix that happens to consist of two identical syllables. 
The reason for analyzing it as a~reduplicated form is the structural parallel with reduplicated suffixes that get added to adjectives (presented in \sectref{sec:thereduplicationofadjectives}). 

The reduplicated \is{suffixes}suffix /\ipa{-ʂo˧{$\sim$}ʂo˩}/ can be added to a~wide range of
nouns, including count nouns, such as persons: a~household with numerous young men may be described
as /\ipa{pʰæ˧tɕi˥-ʂo˩{$\sim$}ʂo˩}/, ‘teeming with youngsters’. The wide range of semantic application of the
\is{suffixes}suffix allows for the elicitation of an~entire set, shown in \tabref{tab:thetonalbehaviourofthereduplicatedsuffixdependingonthetoneoftheprecedingnoun}. As elsewhere, the ‘+’ sign in the transcription of surface tone
patterns indicates the tone of the \isi{copula} when placed after the expression as a~test to ascertain
the type of syllabic {anchoring} of a~final H tone.

\begin{table}%[t]
	\caption{\label{tab:thetonalbehaviourofthereduplicatedsuffixdependingonthetoneoftheprecedingnoun}The tonal behaviour of the reduplicated suffix /\ipa{-ʂo˧{$\sim$}ʂo˩}/ depending on the tone of the preceding noun.}
	\begin{tabularx}{\textwidth}{ Q l l l Q }
		\lsptoprule
		example & tone & example & surface pattern & analysis\\ \midrule
		dust & LM & \ipa{ɖæ˩-ʂo˧{$\sim$}ʂo˩} & L.M.L & LM+L\#\\
		pimple & LH & \ipa{ʝi˩-ʂo˥{$\sim$}ʂo˩} & L.H.L (=L.M.L) & LH--\\
		star & M & \ipa{kɯ˧-ʂo˧{$\sim$}ʂo˥} & M.M.H+L & H\#\\
		grease & L & \ipa{mɤ˩-ʂo˩{$\sim$}ʂo˥} & L.L.H+L & L+H\#\\
		meat & H & \ipa{ʂe˧-ʂo˧{$\sim$}ʂo˥} & M.M.H+L & H\#\\
		mushroom & MH & \ipa{mo˧-ʂo˧{$\sim$}ʂo˥} & M.M.H+L & H\#\\ \addlinespace \hdashline \addlinespace
		dew & M & \ipa{ɖʐv̩˧qʰɑ˧-ʂo˧{$\sim$}ʂo˩} & M.M.M.L & L\#\\
		fly & \#H & \ipa{bv̩˧ɻ̍˧-ʂo˧{$\sim$}ʂo˥} & M.M.M.H+L & H\#\\
		paste & MH\# & \ipa{ho˧dʑɯ˧-ʂo˧{$\sim$}ʂo˥} & M.M.M.H+L & H\#\\
		mud & H\$ & \ipa{ɖʐæ˧qʰæ˧-ʂo˧{$\sim$}ʂo˥} & M.M.M.H+L & H\#\\
		egg & L & \ipa{æ̃˩ʁv̩˩-ʂo˩{$\sim$}ʂo˥} & L.L.L.H+L & L+H\#\\
		cake/bread & L\# & \ipa{dze˧dv̩˩-ʂo˩{$\sim$}ʂo˩} & M.L.L.L & L\#--\\
		bean chaff & LM+MH\# & \ipa{nv̩˩tsɑ˧-ʂo˧{$\sim$}ʂo˥} & L.M.M.H+L & LM+H\#\\
		potato & LM+\#H & \ipa{jɤ˩jo˧-ʂo˧{$\sim$}ʂo˥} & L.M.M.H+L & LM+H\#\\
		sow & LM & \ipa{nv̩˩ɭɯ˧-ʂo˧{$\sim$}ʂo˩} & L.M.M.L & LM+L\#\\
		button & LH & \ipa{pv̩˩ɭɯ˥-ʂo˩{$\sim$}ʂo˩} & L.H.L.L & LH--\\
		youngster & H\# & \ipa{pʰæ˧tɕi˥-ʂo˩{$\sim$}ʂo˩} & M.H.L.L & H\#--\\
		\lspbottomrule
	\end{tabularx}
\end{table}

The tone of the \is{suffixes}suffix /\ipa{-ʂo{$\sim$}ʂo}/ can be hypothesized to be L\# (hence the notation
//\ipa{ʂo˧{$\sim$}ʂo˩}// adopted here) on the basis of its behaviour after M-tone disyllables and
after LM"=tone disyllables. In detail, the tone patterns in \tabref{tab:thetonalbehaviourofthereduplicatedsuffixdependingonthetoneoftheprecedingnoun} are not straightforward; they differ from
those of disyllabic postpositions, discussed in \sectref{sec:spatialpostpositions}.


\section{Possessive constructions containing pronouns}
\label{sec:possessiveconstructionscontainingpronouns}

Possessive constructions were discussed in Chapter~\ref{chap:thelexicaltonesofnouns}, where the behaviour of nouns in association with the \isi{possessive} /\ipa{=bv̩˧}/ served as one of the tests for
determining lexical tone categories. Possessive constructions containing pronouns do not have quite the same tonal patterns, however. This is one of several respects in which pronouns are special.


\subsection[The 1$^{st}$, 2$^{nd}$ and 3$^{rd}$ person pronouns]{The 1\textsuperscript{st}, 2\textsuperscript{nd} and 3\textsuperscript{rd} person pronouns}
\label{sec:the1st2ndand3rdpersonpronouns}

A~\is{pronouns}pronoun's tonal category is established by matching up its tone \is{form!in isolation}in isolation and its tone when a~\isi{copula} is added (a test which is also useful for nouns, as was explained in \sectref{sec:dynamicview}). On the basis of the tones in /\ipa{njɤ˩ ɲi˩˥}/ ‘it's me’, /\ipa{no˩ ɲi˩˥}/ ‘it's you’ and /\ipa{ʈʂʰɯ˧ ɲi˥}/ ‘it's her/him’, the \textsc{1sg}, \textsc{2sg} and \textsc{3sg} pronouns are analyzed as //\ipa{njɤ˩}//, //\ipa{no˩}// and //\ipa{ʈʂʰɯ˥}//, respectively.

To build a~\isi{possessive} construction with a~\is{pronouns}pronoun, the \isi{possessive} /\ipa{=bv̩˧}/ is generally
used, as in (\ref{ex:longagofiveofmyfamilyswerestolen}). The forms are /\ipa{njɤ˧=bv̩˩}/, /\ipa{no˧=bv̩˩}/ and /\ipa{ʈʂʰɯ˧=bv̩˧}/ for the 1\textsuperscript{st},
2\textsuperscript{nd} and 3\textsuperscript{rd} persons.
\begin{exe}
  \ex
  \label{ex:longagofiveofmyfamilyswerestolen}
  \ipaex{ə˧ʝi˧-ʂɯ˥ʝi˩, {\kern2pt}|{\kern2pt} njɤ˧=bv̩˩ {\kern2pt}|{\kern2pt} ʐwæ˧ ʈʂʰɯ˧, {\kern2pt}|{\kern2pt} ŋwɤ˩-kv̩˩ ʈʂæ˥ {\kern2pt}|{\kern2pt} po˧ hɯ˧-ɲi˥!}\\
  \gll ə˧ʝi˧-ʂɯ˥ʝi˩	njɤ˩	=bv̩˧	ʐwæ˥	ʈʂʰɯ˧	ŋwɤ˧	kv̩˧˥	ʈʂæ˧˥ po˧˥			hɯ˧\textsubscript{c}		-ɲi˩\\
  in\_the\_past	\textsc{1sg}	\textsc{poss}	horse	\textsc{top}	five	\textsc{clf}	to\_rob to\_take\_away	to\_go.\textsc{pst}	\textsc{certitude}\\
  \glt ‘Once, long ago, five of my family’s horses were stolen!’ \textit{Literally:} ‘Long ago, my horses (=my
  family’s horses), five were stolen and taken away!’ (Caravans.183)
\end{exe}

This is unlike the pattern for nouns: L-tone nouns with the \isi{possessive} yield L+M, and H-tone nouns yield M+H.

A~further complication is that there are seemingly two variants for the
3\textsuperscript{rd}-person \is{pronouns}pronoun: /\ipa{ʈʂʰɯ˧=bv̩˧}/ and /\ipa{ʈʂʰɯ˧=bv̩˩}/. These
are not tonal variants, however: they have different meanings. The latter, /\ipa{ʈʂʰɯ˧=bv̩˩}/, is
a~reduced form of /\ipa{ʈʂʰɯ˧=ɻ̍˩=bv̩˩}/, where /\ipa{=ɻ̍˩}/ is the associative plural. So
/\ipa{ʈʂʰɯ˧=bv̩˩}/ means ‘their’, whereas /\ipa{ʈʂʰɯ˧=bv̩˧}/ simply means ‘her/his’. Ellipsis of the
associative plural /\ipa{=ɻ̍˩}/ is complete: there is no segmental trace of it, only a~tonal
difference on the \isi{possessive} particle. This example provides an~insight into the expansion of morphotonology through segmental
simplifications (a~type of {diachronic} change that is especially well"=attested among {Bantu} languages). This
topic will be taken up in Chapter~\ref{chap:yongningnatonesinadynamicsynchronicperspective}, where the Yongning Na tone system is approached from a~dynamic"=synchronic perspective.

A~\is{pronouns}pronoun may also immediately precede the noun to build a~\isi{possessive} construction, as in (\ref{ex:letmumdie}).

\Hack{\newpage}

\begin{exe}
  \ex
  \label{ex:letmumdie}
  \ipaex{njɤ˧ mv̩˩{\dots} {\kern2pt}|{\kern2pt} ə˧zɯ˩ {\kern2pt}|{\kern2pt} ʂɯ˧-bi˧, {\kern2pt}|{\kern2pt} ə˧mi˧ {\kern2pt}|{\kern2pt} tʰi˧-ʂɯ˧-kʰɯ˧!}\\
  \gll njɤ˩		mv̩˩˥		ə˧zɯ˩		ʂɯ˧	-bi˧	ə˧mi˧	tʰi˧-	ʂɯ˧		-kʰɯ˧˥\\
  1\textsc{sg}		daughter	1\textsc{pl}.\textsc{incl}		to\_die	\textsc{imm\_fut}	mother	\textsc{dur}	to\_die	\textsc{caus}\\
  \glt ‘My dear daughter{\dots}  We are going to die [=we can’t avoid death, now that the tiger is
    at our door]; let Mum die [=let me sacrifice myself, so you can survive]!' (Tiger.16)
\end{exe}

Combinations of pronouns and nouns as in (\ref{ex:letmumdie}) were systematically elicited. The
results are identical for the 1\textsc{sg} and 2\textsc{sg} pronouns, //\ipa{njɤ˩}// and
//\ipa{no˩}//. They are shown in \tabref{tab:thetonesofpossessiveconstructionsconsistingofa1sgpronounandanoun}. The corresponding recording is: PossessPro.

%Table 1.
\begin{table}%[t]
\caption{\label{tab:thetonesofpossessiveconstructionsconsistingofa1sgpronounandanoun}The tones of {possessive} constructions consisting of a~1\textsc{sg} pronoun and a~noun.}
\begin{tabularx}{\textwidth}{ Q Q P{25mm} l l }
\lsptoprule
	tone & head & meaning & example & tone pattern\\ \midrule
	LM & \ipa{bo˩˧} & pig & \ipa{njɤ˧ bo˩} & L\#\\
	LH & \ipa{mv̩˩˥} & daughter & \ipa{njɤ˧ mv̩˩} & L\#\\
	M & \ipa{zɯ˧} & life, existence & \ipa{njɤ˧ zɯ\#˥} & \#H\\
	L & \ipa{dʑɯ˩} & water & \ipa{njɤ˧ dʑɯ\#˥} & \#H\\
	\#H & \ipa{hĩ˥} & human being & \ipa{njɤ˧ hĩ\#˥} & \#H\\
	MH\# & \ipa{tsʰɯ˧˥} & goat & \ipa{njɤ˧ tsʰɯ˧˥} & MH\#\\ \addlinespace \hdashline \addlinespace
	M & \ipa{po˧lo˧} & ram & \ipa{njɤ˧ po˧lo˧} & M\\
	\#H & \ipa{ʐwæ˧zo\#˥} & colt & \ipa{njɤ˧ ʐwæ˧zo\#˥} & \#H\\
	MH\# & \ipa{hwɤ˧li˧˥} & cat & \ipa{njɤ˧ hwɤ˧li˧˥} & MH\#\\
	H\$ & \ipa{kv̩˧ʂe˥\$} & flea & \ipa{njɤ˧ kv̩˧ʂe˥\$} & H\$\\
	L & \ipa{kʰv̩˩mi˩} & dog & \ipa{njɤ˧ kʰv̩˩mi˩} & --L\\
	L\# & \ipa{dɑ˧ʝi˩} & mule & \ipa{njɤ˧ dɑ˧ʝi˩} & L\#\\
	LM+MH\# & \ipa{ʝi˩ʈʂæ˧˥} & waist & \ipa{njɤ˧ ʝi˩ʈʂæ˩} & --L\\
	LM+\#H & \ipa{bi˩ʈʂʰɤ\#˥} & whiskers & \ipa{njɤ˧ bi˩ʈʂʰɤ˩} & --L\\
	LM & \ipa{bo˩mi˧} & sow & \ipa{njɤ˧ bo˩mi˩} & --L\\
	LH & \ipa{bo˩ɬɑ˥} & boar & \ipa{njɤ˧ bo˩ɬɑ˩} & --L\\
	H\# & \ipa{kʰv̩˧nɑ˥} & dog & \ipa{njɤ˧ kʰv̩˧nɑ˥} & H\#\\
\lspbottomrule
\end{tabularx}
\end{table}

\newpage 
In all these phrases, the \is{pronouns}pronoun //\ipa{njɤ˩}// carries the same tone as \is{form!in isolation}in isolation: a~M tone. (Neutralization of //L//, \mbox{//M//} and //H// to /M/ \is{form!in isolation}in isolation is the general rule for {monosyllabic} nouns and pronouns: see \sectref{sec:monosyllabicnouns}.) The patterns for
\is{monosyllables}monosyllabic nouns cannot be obtained through the application of a~simple set of general rules. On
the other hand, the patterns for disyllables are extremely simple. They consist of the succession of
the \is{pronouns}pronoun, as said \is{form!in isolation}in isolation: /\ipa{njɤ˧}/, followed by the noun, which carries the same tone
as when it appears on its own, except that some tone levels are deleted to comply with the phonological
requirements on a~well"=formed \isi{tone group}. This affects the tone categories in which a~L tone is
attached to the first syllable of the noun: L, LM+MH\#, LM+\#H, LM, and LH. For these categories, the sequence
found on the first two syllables (M tone on the \is{pronouns}pronoun, and L tone on the initial syllable of the
disyllabic noun) is incompatible with any tone other than L on following syllables, by Rule~5
(“All syllables following a~H.L or M.L sequence receive L tone”: see \sectref{sec:alistoftonerules}). For instance, \ipa{$\ddagger${\kern2pt}njɤ˧ bo˩mi˧}
(obtained through simple concatenation) would not be a~well"=formed \isi{tone group}; this is repaired to
/\ipa{njɤ˧ bo˩mi˩}/, lowering the final M to L. The representation in \figref{fig:concat} assumes that the M part of the LM tone is associated, then deleted. One could also consider that this M does not associate at all. This is an area where psycholinguistic experiments would be necessary to approach more closely the processes taking place in speakers' brains.

\begin{figure}[b]
	\caption{Tone"=to"=syllable association in /\ipa{njɤ˧ bo˩mi˩}/ ‘my sow’.}
	\begin{tikzpicture}
	\node (1) at (0.5,-0.5) {L};
	\node (4) at (3,-0.5) {LM};
	
	\node (2) at (0.5,-1.5) {σ};
%	\node (3) at (1,-1.5) {σ};
	\node (5) at (2.5,-1.5) {σ};
	\node (91) at (3.5,-1.5) {σ};
	
	\node [anchor=mid] (s1l) at (0.5,-2) {/\ipa{njɤ}/ \textsc{1sg}};
	%  \node (s1ll) at (0.5,-2.5) {lexical tone: MH\#};
	
	\node [anchor=mid] (s1lll) at (3,-2) {/\ipa{bo.mi}/ ‘sow’};
	%  \node (s1llll) at (4,-2.5) {lexical tone: L};
	
	\node[text width=48mm] (s1) at (-3,-0.75) {Stage 1:\\ input};
%	\node[text width=48mm] (s1) at (-3,-0.75) {Stage 1:\\ input};
	
	
	
	\node (12) at (0.5,-3.2) {M};
	\node (42) at (2.5,-3.2) {L};
	\node (99) at (3.5,-3.2) {M};
	
	\node (22) at (0.5,-4.7) {σ};
%	\node (32) at (1,-5.5) {σ};
	\node (52) at (2.5,-4.7) {σ};
	\node (90) at (3.5,-4.7) {σ};
	
	\node[text width=48mm] (s2) at (-3,-3.95) {Stage 2:\\ separate tonal association\\  for the two words;\\ //L// surfaces as /M/};
	
	\draw[decoration={markings,mark=at position 1 with
		{\arrow[scale=2,>=stealth]{>}}},postaction={decorate}] (12) -- (22);
	\draw[decoration={markings,mark=at position 1 with
			{\arrow[scale=2,>=stealth]{>}}},postaction={decorate}] (42) -- (52);
	\draw[decoration={markings,mark=at position 1 with
				{\arrow[scale=2,>=stealth]{>}}},postaction={decorate}] (99) -- (90);
			
	\node (13) at (1,-6) {M};
	\node (63) at (2,-6) {L};
	\node (43) at (3,-6) {M};
	\node (98) at (4,-6) {L};
	
	\node (23) at (3,-6.7) {=}; % 6.75 puis 6.6 puis 6.65
	\node (3) at (3,-7.2) {};
	\node (33) at (1,-7.5) {σ};
	\node (53) at (2,-7.5) {σ};
	\node (92) at (3,-7.5) {σ};
	
	\node[text width=48mm] (s3) at (-3,-6.75) {Stage 3:\\ joining into one \isi{tone group};\\ replacement of M tone\\ by L, following Rule 5};
	
	\draw (13) -- (33);
	\draw (63) -- (53);
%	\draw (43) -- (92); % barred association line
	\draw (43) -- (3); % barred association line
		
	\draw[decoration={markings,mark=at position 1 with
		{\arrow[scale=2,>=stealth]{>}}},postaction={decorate}] (98) -- (92);
	% Now to final (surface) stage:
	
	\node (14) at (1,-8.8) {M};
	\node (64) at (2,-8.8) {L};
	\node (44) at (3,-8.8) {L};
	
	\node (34) at (1,-10.3) {σ};
	\node (54) at (2,-10.3) {σ};
	\node (94) at (3,-10.3) {σ};
	
	\node[text width=48mm] (s4) at (-3,-9.55) {Stage 4:\\ resulting surface \\ phonological tone};
	
	\draw (14) -- (34);
	\draw (64) -- (54);
	\draw (44) -- (94); 

	\end{tikzpicture}
	\label{fig:concat}
\end{figure}

This construction offers an~example of minimal tonal integration of two elements. It can be
described as concatenation of the two parts of the expression, followed by adjustments required by phonological rules.

For disyllables, the tone patterns after the 3\textsuperscript{rd} person \is{pronouns}pronoun, //\ipa{ʈʂʰɯ˥}//, are
identical with those after the 1\textsuperscript{st} and 2\textsuperscript{nd} persons. For
monosyllables, on the other hand, the tone patterns are different in the case of L-tone nouns: //\ipa{njɤ˧ dʑɯ\#˥}// ‘my water’ vs.\ //\ipa{ʈʂʰɯ˧ dʑɯ˧}// ‘her/his water’. This asymmetry poses yet another
challenge to the language learner, who must learn (i)~to distinguish the tone patterns for these two
tonal sets of pronouns when they associate with a~{monosyllabic} noun, and (ii)~to overlook the
difference when the following noun is disyllabic. \tabref{tab:thetonesofpossessiveconstructionsconsistingofa3sgpronounandanoun} shows the entire set.

\begin{table}%[t]
\caption{\label{tab:thetonesofpossessiveconstructionsconsistingofa3sgpronounandanoun}The tones of {possessive} constructions consisting of a~3\textsc{sg} pronoun and a~noun.}
\begin{tabularx}{\textwidth}{ Q Q l l l }
\lsptoprule
	tone & example & meaning & with \textsc{3sg} & tone pattern\\ \midrule
	LM & \ipa{bo˩˧} & pig & \ipa{ʈʂʰɯ˧ bo˩} & L\#\\
	LH & \ipa{mv̩˩˥} & daughter & \ipa{ʈʂʰɯ˧ mv̩˩} & L\#\\
	M & \ipa{zɯ˧} & life, existence & \ipa{ʈʂʰɯ˧ zɯ\#˥} & \#H\\
	L & \ipa{dʑɯ˩} & water & \ipa{ʈʂʰɯ˧ dʑɯ˧} & M\\
	\#H & \ipa{hĩ˥} & human being & \ipa{ʈʂʰɯ˧ hĩ\#˥} & \#H\\
	MH\# & \ipa{tsʰɯ˧˥} & goat & \ipa{ʈʂʰɯ˧ tsʰɯ˧˥} & MH\#\\ \addlinespace \hdashline \addlinespace
	M & \ipa{po˧lo˧} & ram & \ipa{ʈʂʰɯ˧ po˧lo˧} & M\\
	\#H & \ipa{ʐwæ˧zo\#˥} & colt & \ipa{ʈʂʰɯ˧ ʐwæ˧zo\#˥} & \#H\\
	MH\# & \ipa{hwɤ˧li˧˥} & cat & \ipa{ʈʂʰɯ˧ hwɤ˧li˧˥} & MH\#\\
	H\$ & \ipa{ə˧dɑ˥\$} & father & \ipa{ʈʂʰɯ˧-ə˧dɑ\#˥} & \#H\\
	L & \ipa{kʰv̩˩mi˩} & dog & \ipa{ʈʂʰɯ˧ kʰv̩˩mi˩} & --L\\
	L\# & \ipa{dɑ˧ʝi˩} & mule & \ipa{ʈʂʰɯ˧ dɑ˧ʝi˩} & L\#\\
	LM+MH\# & \ipa{ʝi˩ʈʂæ˧˥} & waist & \ipa{ʈʂʰɯ˧ ʝi˩ʈʂæ˩} & --L\\
	LM+\#H & \ipa{bi˩ʈʂʰɤ\#˥} & whiskers & \ipa{ʈʂʰɯ˧ bi˩ʈʂʰɤ˩} & --L\\
	LM & \ipa{bo˩mi˧} & sow & \ipa{ʈʂʰɯ˧ bo˩mi˩} & --L\\
	LH & \ipa{bo˩ɬɑ˥} & boar & \ipa{ʈʂʰɯ˧ bo˩ɬɑ˩} & --L\\
	H\# & \ipa{kʰv̩˧nɑ˥} & dog & \ipa{ʈʂʰɯ˧ kʰv̩˧nɑ˥} & H\#\\
\lspbottomrule
\end{tabularx}
\end{table}

 
Finally, there also exists a~looser construction: a~simple juxtaposition of the \is{pronouns}pronoun and the
noun, each in its own tone groups, as in (\ref{ex:goneawaytowork}). 

 
% \Hack{\newpage}

\begin{exe}
  \ex
  \label{ex:goneawaytowork}
  \ipaex{njɤ˧ {\kern2pt}|{\kern2pt} ɻ̍˩ʈʂʰe˧-ɖɯ˩mɑ˩ ʈʂʰɯ˩-dʑo˩, {\kern2pt}|{\kern2pt} no˧sɯ˩kv̩˩ {\kern2pt}|{\kern2pt} tʰv̩˧-ɲi˧ {\kern2pt}|{\kern2pt} lo˧ ʝi˧ hɯ˧ tsɯ˩.}\\
  \gll njɤ˩		ɻ̍˩ʈʂʰe˧-ɖɯ˩mɑ˩		ʈʂʰɯ˧	-dʑo˥	no˧sɯ˩kv̩˩ tʰv̩˧-ɲi˧	lo˧ 	ʝi˧	hɯ˧\textsubscript{c}	tsɯ˧˥\\
  1\textsc{sg}		given\_name	\textsc{top}	\textsc{top}	\textsc{2pl}.\textsc{excl} that.day	work	to\_do	go.\textsc{pst}	\textsc{rep}\\
  \glt ‘As for my [daughter] Erchei Ddeema{\dots} that day, you (=the members of your
  family) had gone away to work.' (BuriedAlive2.132)
\end{exe}

 \largerpage[-1]
This is not a~\isi{possessive}
construction in the proper sense: rather, the \is{pronouns}pronoun serves as a~topic. The context to this example is as follows. A~young woman is unhappy with a~marriage arranged by her parents, and she commits a~small offence, which takes on huge proportions. Her mother has now come over to the mother"=in"=law’s house to make things right by talking the matter over. The young woman’s mother first recapitulates the whole story, clarifying what has been done by the two parties, the members
of the two families. In this situation, the first person \is{pronouns}pronoun in the construction /\ipa{njɤ˧ {\kern2pt}|{\kern2pt}
  ɻ̍˩ʈʂʰe˧-ɖɯ˩mɑ˩}/ emphasizes the fact that the mother is speaking on her daughter’s behalf, and
that, as head of the family, she assumes some responsibility for her daughter’s actions.


\subsection{The pronoun ‘oneself’}
\label{sec:thepronounoneself}
\largerpage

The \is{pronouns}pronoun ‘oneself’, /\ipa{õ˧˥}/, has a~different behaviour from the 1\textsuperscript{st},
2\textsuperscript{nd} and 3\textsuperscript{rd} person pronouns. 
It is not followed by the \isi{possessive}, /\ipa{=bv̩˧}/, except in the frequent construction /\ipa{õ˧=bv̩˥-õ˩}/ ‘one’s own’, ‘one’s proper’, ‘by
oneself’, which can in turn be followed by a~noun, as in (\ref{ex:weeatourownproduce}).
\begin{exe}
  \ex
  \label{ex:weeatourownproduce}
  \ipaex{õ˧=bv̩˥-õ˩ hɑ˩, {\kern2pt}|{\kern2pt} õ˧=bv̩˥-õ˩ dzɯ˩!}\\
  \gll õ˧˥		=bv̩˧	õ˧˥		hɑ˥		dzɯ˥\\
  oneself	\textsc{poss}	oneself		food		to\_eat\\
  \glt ‘One’s own food, one eats it oneself!~/ We eat our own produce!’ (Agriculture.68; this
  summarizes the traditional self"=sufficiency of the Na, who grew their own food crops)
\end{exe}


After /\ipa{õ˧=bv̩˥-õ˩}/, all tonal oppositions are neutralized, because the following noun can only
carry L tones, due to a~constraint formulated in the present description as Rule 5: “All syllables following a~H.L or M.L sequence receive L tone”.

The \is{pronouns}pronoun /\ipa{õ˧˥}/ usually combines with a~following noun, without an~intervening morpheme, as
in (\ref{ex:thisismyownbrother}) and (\ref{ex:onesmeat}).
\begin{exe}
  \ex
  \label{ex:thisismyownbrother}
  \ipaex{õ˧ ə˧mv̩˥ ɲi˩-ze˩!}\\
  \gll õ˧˥	ə˧mv̩˩		ɲi˩	-ze˧\\
  self	elder\_sibling	\textsc{cop}	\textsc{pfv}\\
  \glt ‘This is my own brother!’ (Sister.57; Sister3.58. Context: a~young woman recognizes a~ragged
  stranger attending her wedding as being her long"=lost brother.)
\end{exe}

\begin{exe}
	\ex
	\label{ex:onesmeat}
	\ipaex{õ˧-ʂe˥, õ˩ ʈʰæ˩!}\\
	\gll õ˧˥	ʂe˥		õ˧˥	ʈʰæ˧˥\\
	self	meat	self	to\_bite\\
	\glt ‘Each person eats her/his own slab of meat!’ (Field notes. Context: describing
	table manners. Each family member used to receive
	one slice of meat and eat it up. This is unlike
	Chinese (Han) custom, in which each guest picks food
	mouthful by mouthful, with chopsticks, from dishes placed on the table.)
\end{exe}

The construction /\ipa{õ˧˥}/ plus N, ‘[my/one’s] own N’, has a~tonal behaviour of its
own. On the {analogy} of
(\ref{ex:onesmeat}), new maxims can be coined, such as (\ref{ex:ones2}), (\ref{ex:ones3}) and (\ref{ex:ones4}). 

\begin{exe}
	\ex
	\label{ex:ones2}
	\ipaex{õ˧-dʑɯ˥, õ˩ ʈʰɯ˩!}\\
	\gll õ˧˥	dʑɯ˩	õ˧˥	ʈʰɯ˩\textsubscript{b}\\
	self	water	self	to\_drink\\
	\glt ‘Each drinks from her own bottle!’ (Field notes. Context: a~toddler has grabbed another’s bottle; parents
	prevent her from drinking from it.)
\end{exe}

\begin{exe}
	\ex
	\label{ex:ones3}
	\ipaex{õ˧-ɖæ˥, õ˩ bæ˩!}\\
	\gll õ˧˥	ɖæ˩˧	õ˧˥	bæ˩\textsubscript{a}\\
	self	dust	self	to\_sweep\\
	\glt ‘one must sweep one’s
	own garbage’ (elicited example)
\end{exe}

\begin{exe}
	\ex
	\label{ex:ones4}
	\ipaex{õ˧-lv̩˥, õ˩ li˩!}\\
	\gll õ˧˥	lv̩˧		õ˧˥	li˧\textsubscript{a}\\
	self	field	self	to\_look\_after\\
	\glt ‘one must look after one’s own fields’ (elicited)
\end{exe}

Combinations were systematically elicited. The full set of tonal combinations is shown in \tabref{tab:thetonesofpossessiveconstructionsconsistingofnoun}.

\begin{table}%[t]
\caption{\label{tab:thetonesofpossessiveconstructionsconsistingofnoun}The tones of {possessive} constructions consisting of /\ipa{õ˧˥}/+N.}
\begin{tabularx}{\textwidth}{ P{16mm}@{\hspace{2mm}} P{17mm}@{\hspace{2mm}} P{23mm}@{\hspace{2mm}} P{36mm}@{\hspace{2mm}} l@{\hspace{1mm}} }
\lsptoprule
	tone & example & meaning & with /\ipa{õ˧˥}/ & tone pattern\\ \midrule
	LM & \ipa{ɖæ˩˧} & dirt, dust & \ipa{õ˧-ɖæ˥} & H\#\\
	LH & \ipa{mv̩˩˥} & daughter & \ipa{õ˧-mv̩˥\$} & H\$\\
	M & \ipa{lv̩˧} & field & \ipa{õ˧-lv̩˥\$} & H\$\\
	L & \ipa{dʑɯ˩} & water & \ipa{õ˧-dʑɯ˥\$} & H\$\\
	\#H & \ipa{zo˥} & son & \ipa{õ˧-zo\#˥} & H\#\\
	MH\# & \ipa{tsʰɯ˧˥} & goat & \ipa{õ˧-tsʰɯ˥\$} & H\$\\ \addlinespace \hdashline \addlinespace
	M & \ipa{go˧mi˧} & younger sister & \ipa{õ˧-go˧mi˥} & H\#\\
	\#H & \ipa{ʐwæ˧zo\#˥} & colt & \ipa{õ˧-ʐwæ˧zo\#˥} & \#H\\
	MH\# & \ipa{ə˧mv̩˧˥} & elder sibling & \ipa{õ˧-ə˧mv̩˧˥ / õ˧-ə˥mv̩˩} & MH\# / MH\#--\\
	H\$ & \ipa{ə˧dɑ˥\$} & father & \ipa{õ˧-ə˧dɑ˥ / õ˧-ə˧dɑ\#˥} & H\# / \#H\\
	L & \ipa{kʰv̩˩mi˩} & dog & \ipa{õ˧-kʰv̩˥mi˩} & MH\#--\\
	L\# & \ipa{dɑ˧ʝi˩} & mule & \ipa{õ˧-dɑ˧ʝi˥} & H\#\\
	LM+MH\# & \ipa{ʝi˩ʈʂæ˧˥} & waist & \ipa{õ˧-ʝi˥ʈʂæ˩} & MH\#--\\
	LM+\#H & \ipa{bi˩ʈʂʰɤ\#˥} & whiskers & \ipa{õ˧-bi˥ʈʂʰɤ˩} & MH\#--\\
	LM & \ipa{ɑ˩ʁo˧} & home & \ipa{õ˧-ɑ˥ʁo˩} & MH\#--\\
	LH & \ipa{bo˩ɬɑ˥} & boar & \ipa{õ˧-bo˥ɬɑ˩} & MH\#--\\
	H\# & \ipa{kʰv̩˧nɑ˥} & dog & \ipa{õ˧-kʰv̩˧nɑ˥} & H\#\\
\lspbottomrule
\end{tabularx}
\end{table}


The behaviour of /\ipa{õ˧˥}/ in association with disyllables coincides with that of determinative
compounds containing a~MH"=tone determiner. With monosyllables, however, the tone patterns only
coincide with those of determinative compounds for nouns with LM or MH\# tone.




\section[Monosyllabic enclitics, suffixes, and postpositions]{Monosyllabic morphemes appearing after nouns: Enclitics, suffixes, and postpositions}
\label{sec:enclitics}

Analysis of morphemes as affixes, clitics, postpositions, serial verbs, “particles” and other parts of speech raises interesting issues that differ widely from one language to another, as illustrated by the diversity of proposals and viewpoints found in a~collective book about the notion of “word” \citep{dixonetal2002b}. \citet[43]{aikhenvald2002} proposed that there is “a multidimensional continuum, from a~fully bound to a~fully independent morpheme”. Rather than proceeding from morphosyntactic categories, the method used here is to start out from tone patterns, progressing towards an analysis in light of the morphemes' tonal behaviour. This approach yields independent evidence for syntactic analysis and part-of-speech labelling. For instance, dative
/\ipa{-ki˧}/ and \isi{possessive} /\ipa{=bv̩˧}/ turn out to have exactly the same tonal behaviour, which is distinct from that of agentive /\ipa{ɳɯ˧}/ (see \sectref{sec:encliticsthatcarrymtonewhenfollowingamtonenoun}), suggesting that the former two belong to the same morphosyntactic class, distinct from the latter. This can be taken as confirmation for the observation (based on syntactic behaviour) that the dative and \isi{possessive} are “almost suffixal” \citep[155]{lidz2010}, whereas the agentive is analyzed as a~case adposition.

Enclitics, suffixes, and postpositions are divided into four subsets on the basis of their behaviour after M-tone nouns. 
Those that surface\is{form!surface} with L tone in that context are considered to carry lexical L tone; likewise, those that surface with M, MH and H tones are considered to carry M, MH and H tones respectively. The paragraphs that follow reveal that these four broad tonal categories are not fully homogeneous; they serve as convenient headings for setting out the data.

\subsection{L-tone morphemes}
\label{sec:ltoneencliticspluralandassociativeplural}

This subsection discusses three morphemes: the \is{postpositions}postposition ‘on; at', and the plural and associative clitics. 

The \is{postpositions}postposition //\ipa{bi˩}// ‘on; at' surfaces with L tone after a~M-tone noun, e.g.~in /\ipa{gv̩˧mi˧ bi˩}/ ‘on the body'. Other examples from texts and field notes include disyllabic nouns with L tone, as in /\ipa{ʐæ˩sɯ˩ bi˥}/ ‘on the felt cape', and with LM tone, as in /\ipa{lo˩qʰwɤ˧ bi˩}/ ‘on the hand'. In order to obtain a~full set, systematic elicitation was conducted, yielding the data shown in \tabref{tab:postpositionon}. 


\begin{table}%[t]
	\caption{\label{tab:postpositionon}The behaviour of the L-tone postposition //\ipa{bi˩}// ‘on; at' with {monosyllabic} and disyllabic nouns. There is an additional ‘L \textsc{pro}’ row because L-tone pronouns have exceptional behaviour.}
	\begin{tabularx}{\textwidth}{ Q Q P{30mm} l }
		\lsptoprule
		example & tone & with /\ipa{bi˩}/ & surface tone pattern\\ \midrule
		pig & LM & \ipa{bo˩ bi˥} & L.H\\
		leopard & LH & \ipa{ʐæ˩ bi˥} & L.H\\
		tiger & M & \ipa{lɑ˧ bi˩} & M.L\\
		sheep & L & \ipa{jo˩ bi˩˥} & L.LH\\
		\textsc{2sg} & L \textsc{pro} & \ipa{no˧ bi˩} & M.L\\
		horse & H & \ipa{ʐwæ˧ bi˥} & M.H\\
		deer & MH & \ipa{ʈʂʰæ˧ bi˥} & M.H\\ \addlinespace \hdashline \addlinespace
		fox & M & \ipa{ɖɤ˧mi˧ bi˩} & M.M.L\\
		colt & \#H & \ipa{ʐwæ˧zo˧ bi˥} & M.M.H\\
		cat & MH\# & \ipa{hwɤ˧li˧ bi˥} & M.M.H\\
		she"=cat & H\$ & \ipa{hwɤ˧mi˧ bi˥} & M.M.H\\
		dog & L & \ipa{kʰv̩˩mi˩ bi˥} & L.L.H\\
		mule & L\# & \ipa{dɑ˧ʝi˩ bi˩} & M.L.L\\
		wolf & LM+MH\# & \ipa{õ˩dv̩˧ bi˥} & L.M.H\\
		Naxi & LM+\#H & \ipa{nɑ˩hĩ˧ bi˥} & L.M.H\\
		sow & LM & \ipa{bo˩mi˧ bi˩} & L.M.L\\
		boar & LH & \ipa{bo˩ɬɑ˥ bi˩} & L.H.L\\
		rat & H\# & \ipa{hwæ˧tsɯ˥ bi˩} & M.H.L\\
		\lspbottomrule
	\end{tabularx}
\end{table}

The tone patterns indicated in \tabref{tab:postpositionon} are those observed at the \is{form!surface}surface phonological level, not the underlying tones. In the case of /\ipa{bo˩ bi˥}/ ‘on \mbox{(a/the)} pig' and /\ipa{ʐæ˩ bi˥}/ ‘on \mbox{(a/the)} leopard', it is unclear whether the pattern is to be analyzed as //L.M// or //L.H//, since both are neutralized at the surface-phonological level, by Rule 6 (see \sectref{sec:alistoftonerules}). The data in \tabref{tab:postpositionon} does not reveal whether the tone patterns of ‘on \mbox{(a/the)} pig' and ‘on \mbox{(a/the)} leopard' are underlyingly identical or not.

Additional evidence comes from plural
/\ipa{=ɻæ˩}/ and associative plural /\ipa{=ɻ̍˩}/. As in \ili{Japhug}, where the plural \is{clitics}clitic /\ipa{=ra}/ can express plurality or collective meaning, these enclitics are not obligatory for non"=singular arguments, even in the case of human referents \citep{jacquesforth}. Like the \is{postpositions}postposition /\ipa{bi˩}/ ‘on; at', these two enclitics are analyzed as having L tone on the basis of their tonal behaviour after M-tone nouns. Their tone patterns are identical to those of the \is{postpositions}postposition /\ipa{bi˩}/ ‘on; at'. Moreover, in the case of the plural and associative plural morphemes, it is possible to add the {possessive} /\ipa{=bv̩˧}/ to the \textit{N+{plural}} expression as a~test of its \is{form!underlying}underlying tone category, following the procedure used in the study of the tones of nouns
(Chapter~\ref{chap:thelexicaltonesofnouns}). This test distinguishes //L.M//, which does not depress a~following {possessive}, from //L.H//, which does. \tabref{tab:thetonalbehaviourofplural} presents the
facts for N+{\allowbreak}{plural}+{\allowbreak}{possessive}. 

\begin{sidewaystable}[p]
\caption{\label{tab:thetonalbehaviourofplural}The tonal behaviour of nouns followed by the {plural} clitic /\ipa{=ɻæ˩}/ plus the {possessive} suffix /\ipa{=bv̩˧}/.}
\begin{tabularx}{\textwidth}{ l l l Q l }
\lsptoprule
	example & tone & +\textsc{plural} & +\textsc{plural}+\textsc{possessive} & tonal analysis\\ \midrule
	Na (ethnic group) & LM & \ipa{nɑ˩=ɻæ˥} & \ipa{nɑ˩=ɻæ˥=bv̩˩} & LH\\
	daughter & LH & \ipa{mv̩˩=ɻæ˥} & \ipa{mv̩˩=ɻæ˥=bv̩˩} & LH\\
	Han (ethnic group) & M & \ipa{hæ˧=ɻæ˩} & \ipa{hæ˧=ɻæ˩=bv̩˩} & L\#\\
	sheep & L & \ipa{jo˩=ɻæ˩˥} & \ipa{jo˩=ɻæ˩=bv̩˥} & L\\
	person, human being & H & \ipa{hĩ˧=ɻæ˥} & \ipa{hĩ˧=ɻæ˥=bv̩˩ / hĩ˧=ɻæ˧=bv̩˥} & H\# / H\$\\
	deer & MH & \ipa{ʈʂʰæ˧=ɻæ˥} & \ipa{ʈʂʰæ˧=ɻæ˥=bv̩˩} & H\#\\ \addlinespace \hdashline \addlinespace
	aunt & M & \ipa{ə˧mi˧=ɻæ˩} & \ipa{ə˧mi˧=ɻæ˩=bv̩˩} & --L\\
	younger brother & \#H & \ipa{gi˧zɯ˧=ɻæ˥} & \ipa{gi˧zɯ˧=ɻæ˧=bv̩˥ / gi˧zɯ˧=ɻæ˥=bv̩˩} & H\$ / H\#\\
	maternal uncle & MH\# & \ipa{ə˧v̩˧=ɻæ˥} & \ipa{ə˧v̩˧=ɻæ˧=bv̩˥ / ə˧v̩˧=ɻæ˥=bv̩˩} & H\$ / H\#\\
	she"=cat & H\$ & \ipa{hwɤ˧mi˧=ɻæ˥} & \ipa{hwɤ˧mi˧=ɻæ˧=bv̩˥ / hwɤ˧mi˧=ɻæ˥=bv̩˩} & H\$ / H\#\\
	woman & L & \ipa{mi˩zɯ˩=ɻæ˥} & \ipa{mi˩zɯ˩=ɻæ˩=bv̩˥} & L+H\#\\
	elder sibling & L\# & \ipa{ə˧mv̩˩=ɻæ˩} & \ipa{ə˧mv̩˩=ɻæ˩=bv̩˩} & L\#--\\
	wolf & LM+MH\# & \ipa{õ˩dv̩˧=ɻæ˥} & \ipa{õ˩dv̩˧=ɻæ˧=bv̩˥ / õ˩dv̩˧=ɻæ˥=bv̩˩} & LM+H\# / LM+H\$\\
	Naxi (ethnic group) & LM+\#H & \ipa{nɑ˩hĩ˧=ɻæ˥} & \ipa{nɑ˩hĩ˧=ɻæ˧=bv̩˥ / nɑ˩hĩ˧=ɻæ˥=bv̩˩} & LM+H\# / LM+H\$\\
	sow & LM & \ipa{bo˩mi˧=ɻæ˩} & \ipa{bo˩mi˧=ɻæ˩=bv̩˩} & LM--L\\
	boar & LH & \ipa{bo˩ɬɑ˥=ɻæ˩} & \ipa{bo˩ɬɑ˥=ɻæ˩=bv̩˩} & LH--\\
	young man & H\# & \ipa{pʰæ˧tɕi˥=ɻæ˩} & \ipa{pʰæ˧tɕi˥=ɻæ˩=bv̩˩} & H\#--\\
\lspbottomrule
\end{tabularx}
\end{sidewaystable}


As elsewhere, not all the expressions containing pronouns are tonally identical to those containing nouns. The proximal
\is{demonstratives}demonstrative //\ipa{ʈʂʰɯ˥}// and the distal \is{demonstratives}demonstrative
//\ipa{tʰv̩˥}// yield /\ipa{ʈʂʰɯ˧=ɻæ˥\$}/ and /\ipa{tʰv=ɻæ˥\$}/, i.e.\ the same pattern as for {monosyllabic} nouns carrying H tone. But the first and second persons, //\ipa{njɤ˩}//
and //\ipa{no˩}//, yield /\ipa{njɤ˧=ɻæ˩}/ and /\ipa{no˧=ɻæ˩}/ with the plural; this is different from
the pattern that obtains for L"=tone nouns, e.g.~/\ipa{jo˩=ɻæ˩˥}/ ‘sheep’. 

The data in \tabref{tab:thetonalbehaviourofplural} brings out tonal differences between nouns and expressions made up of a~noun plus a~\is{clitics}clitic. For instance, the noun /\ipa{ə˧v̩˧˥}/ ‘uncle’ yields /\ipa{ə˧v̩˧=ɻæ˥}/ ‘the uncles’ and /\ipa{ə˧v̩˧=ɻæ˧=bv̩˥}/ ‘of the uncles’, i.e.\ a~tonal alternation that has no counterpart among nouns: there is no tonal category of disyllables that has a~final H tone \is{form!in isolation}in isolation and that yields a~final H tone on a~following {possessive}. The H\$ tone category of disyllables followed by the {possessive} yields M.M.M, not M.M.H: e.g.~/\ipa{kv̩˧ʂe˥}/ ‘flea’ and /\ipa{kv̩˧ʂe˧=bv̩˧}/ ‘of \mbox{(a/the)} flea’ (see \tabref{tab:thelexicaltonesofdisyllabicnouns}). Thus, the tone carried by the expression /\ipa{ə˧v̩˧=ɻæ˥}/ ‘the uncles’ is not strictly identical to any of the tonal categories of disyllables. Likewise, the behaviour of the expression /\ipa{hĩ˧=ɻæ˥}/ ‘the people’, which yields /\ipa{hĩ˧=ɻæ˧=bv̩˥}/ ‘of the people’, has no counterpart among nouns. Stated in impressionistic terms, it is as~if the behaviour of H\$, the \textit{gliding} H tone, were simplified when the expression that carries it contains a~\is{clitics}clitic: the H level then tends to attach to a~following syllable, even if this syllable is a~\is{suffixes}suffix. This is unlike H\$"=tone nouns, whose combinations with suffixes of different tone categories yield different results. 

The plural marker /\ipa{=ɻæ˩}/ is frequently used. By contrast, the associative plural /\ipa{=ɻ̍}{\kern2pt}/
has a~highly specific meaning: it refers to the clan (the extended family). It is therefore
mostly restricted to pronouns and family (clan) names, which are few in number. It cannot be used
with kinship terms. However, it is not implausible that the morpheme /\ipa{=ɻ̍˩}/ that partakes in nominalization processes is in fact the associative plural. An example is shown in~(\ref{ex:themountainsmakeupacouple}), where /\ipa{pʰæ˧{$\sim$}pʰæ˧}/ ‘to attach’, in association
with /\ipa{=ɻ̍˩}/, comes to mean ‘a couple; a~pair; a~set (of things, persons{\dots}) tied
together’. 

\begin{exe}
  \ex
  \label{ex:themountainsmakeupacouple}
  \ipaex{pʰæ˧{$\sim$}pʰæ˧=ɻ̍˩ ɲi˩-kv̩˩ tsɯ˩ {\kern2pt}|{\kern2pt} mv̩˩.}\\
  \gll pʰæ˧\textsubscript{b}	   {$\sim$}	=ɻ̍˩	ɲi˩	-kv̩˧˥		tsɯ˧˥	mv̩˧\\
  to\_tie/fasten	   \textsc{red}	\textsc{associative}	\textsc{cop}	\textsc{abilitive}	\textsc{rep}
  \textsc{affirm}\\
  \glt ‘[The mountains \ipa{kɤ˧mv̩˧˥} and \ipa{æ˧ʂæ˧}] make up a~couple/a pair!’ (Mountains.99)
\end{exe}

Not unexpectedly, pronouns followed by the associative plural have a~tonal behaviour of their own. For the two L-tone pronouns (1\textsc{sg} and
2\textsc{sg}), the tone patterns with the associative plural are the same as with the plural: /\ipa{njɤ˧=ɻ̍˩}/
and /\ipa{no˧=ɻ̍˩}/. As for the proximal \is{demonstratives}demonstrative (also serving as 3\textsuperscript{rd} person)
/\ipa{ʈʂʰɯ˥}/ and the distal \is{demonstratives}demonstrative /\ipa{tʰv̩˥}/, they yield /\ipa{ʈʂʰɯ˧=ɻ̍˩}/ and
/\ipa{tʰv̩˧=ɻ̍˩}/, whereas the forms with the plural are /\ipa{ʈʂʰɯ˧=ɻæ˥\$}/ and /\ipa{tʰv=ɻæ˥\$}/. This difference is enough to establish that the plural and associative do not always share the same tone patterns, despite their identical behaviour in almost all cases. It must be kept in mind that, as pointed out at the outset of this section (\sectref{sec:enclitics}), the four broad tonal categories set up here (morphemes with L tone, M tone, MH tone and H tone) are based on one test only: their behaviour after M-tone nouns. These four categories are not homogeneous, and only serve as convenient headings for setting out the data.

The associative /\ipa{=ɻ̍˩}/ is therefore presented separately in \tabref{tab:thetonalbehaviourofassociativeplural}. Its low frequency explains the great number of gaps in the table.  

The Alawa dialect of Yongning Na also has a~dual morpheme /\ipa{=zɯ˩}/ appearing in four pronouns: first person dual exclusive /\ipa{njæ˧=zɯ˩}/, first person dual inclusive /\ipa{ə˧=zɯ˩}/, second person dual /\ipa{no˧=zɯ˩}/, and third person dual /\ipa{ʈʂʰɯ˧{\allowbreak}=zɯ˩}/. On the basis of these forms, the tone of the dual can be classified as belonging to the same broad tonal category as the plural and associative, namely that of L-tone morphemes~-- keeping in mind that this tonal category serves as a~first"=pass label.

The surface phonological patterns for the \is{postpositions}postposition /\ipa{bi˩}/ ‘on; at', shown in \tabref{tab:postpositionon}, are identical in every case to those for the plural \is{clitics}clitic /\ipa{=ɻæ˩}/. The underlying forms for the \is{postpositions}postposition are more difficult to arrive at, for want of a~handy test such as addition of the \isi{possessive} /\ipa{=bv̩˧}/, which works for noun phrases containing the plural \is{clitics}clitic /\ipa{=ɻæ˩}/ but not for locative phrases containing the \is{postpositions}postposition /\ipa{bi˩}/ ‘on; at'. In view of the full identity of surface phonological patterns for the \is{clitics}clitic and the \is{postpositions}postposition, it is tempting to propose an extrapolation from the surface phonological forms in \tabref{tab:postpositionon} to the (hypothetical) underlying forms proposed in \tabref{tab:postpositiononsupplemented}. These underlying forms are based on those obtained for the plural \is{clitics}clitic /\ipa{=ɻæ˩}/ (\tabref{tab:thetonalbehaviourofplural}). 

\begin{table}%[t]
\caption{\label{tab:thetonalbehaviourofassociativeplural}The tonal behaviour of associative plural /\ipa{=ɻ̍}{\kern2pt}/.}
\begin{tabularx}{\textwidth}{ l Q l Q l }
\lsptoprule
	example & meaning & tone & with /\ipa{=ɻ̍}{\kern2pt}/ & output\\ \midrule
	– & – & LM & – & –\\
	– & – & LH & – & –\\
	– & – & M & – & –\\
	\ipa{njɤ˩}, \ipa{no˩} & \textsc{1sg}, \textsc{2sg} & L & \ipa{njɤ˧=ɻ̍˩}, \ipa{no=ɻ̍˩}  & L\#\\
	\ipa{ʈʂʰɯ˥}, \ipa{tʰv̩˥} & \textsc{dem}.\textsc{prox}, \textsc{dist} & H & \ipa{ʈʂʰɯ˧=ɻ̍˩}, \ipa{tʰv̩˧=ɻ̍˩} & L\#\\
	– & – & MH & – & –\\ \addlinespace \hdashline \addlinespace
	\ipa{dze˧bo˧} & family name & M & \ipa{dze˧bo˧=ɻ̍˩} & L\#\\
	– & – & \#H & – & –\\
	– & – & MH\# & – & –\\
	\ipa{kv̩˧tsʰɑ˥\$} & family name & H\$ & \ipa{kv̩˧tsʰɑ˧=ɻ̍˥\$} & H\$\\
	\ipa{lɑ˩mɑ˩} & family name & L & \ipa{lɑ˩mɑ˩-ɻ̍˥\$} & L+H\$\\
	\ipa{ə˧ɕjo˩} & family name & L\# & \ipa{ə˧ɕjo˩=ɻ̍˩} & L\#--\\
	– & – & LM+MH\# & – & –\\
	– & – & LM+\#H & – & –\\
	– & – & LM & – & –\\
	– & – & LH & – & –\\
	– & – & H\# & – & –\\
	\ipa{ɖʐɤ˩kɤ˥\$} & family name & LM+H\$ & \ipa{ɖʐɤ˩kɤ˧-ɻ̍˥\$} & LM+H\$\\
\lspbottomrule
\end{tabularx}
\end{table}


\begin{table}%[t]
	\caption{\label{tab:postpositiononsupplemented}The behaviour of the L-tone postposition //\ipa{bi˩}// ‘on; at' as interpreted on the {analogy} of the {plural} clitic. There is an additional ‘L \textsc{pro}’ row because L-tone pronouns have exceptional behaviour.}
	\begin{tabularx}{\textwidth}{ Q Q P{30mm} l }
		\lsptoprule
		example & tone & with /\ipa{bi˩}/ & underlying tone\\ \midrule
		pig & LM & \ipa{bo˩ bi˧} & LM\\
		leopard & LH & \ipa{ʐæ˩ bi˥} & LH\\
		tiger & M & \ipa{lɑ˧ bi˩} & L\#\\
		sheep & L & \ipa{jo˩ bi˩˥} & L\\
		\textsc{2sg} & L \textsc{pro} & \ipa{no˧ bi˩} & L\#\\
		horse & H & \ipa{ʐwæ˧ bi˥} & H\$ / \#H\\
		deer & MH & \ipa{ʈʂʰæ˧ bi˥} & H\#\\ \addlinespace \hdashline \addlinespace
		fox & M & \ipa{ɖɤ˧mi˧ bi˩} & --L\\
		colt & \#H & \ipa{ʐwæ˧zo˧ bi˥} & H\$\\
		cat & MH\# & \ipa{hwɤ˧li˧ bi˥} & H\$\\
		she"=cat & H\$ & \ipa{hwɤ˧mi˧ bi˥} & H\$\\
		dog & L & \ipa{kʰv̩˩mi˩ bi˥} & L+H\$\\
		mule & L\# & \ipa{dɑ˧ʝi˩ bi˩} & L\#--\\
		wolf & LM+MH\# & \ipa{õ˩dv̩˧ bi˥} & LM+H\$ \\
		Naxi & LM+\#H & \ipa{nɑ˩hĩ˧ bi˥} & LM+H\$\\
		sow & LM & \ipa{bo˩mi˧ bi˩} & LM--L\\
		boar & LH & \ipa{bo˩ɬɑ˥ bi˩} & LH--\\
		rat & H\# & \ipa{hwæ˧tsɯ˥ bi˩} & H\#--\\
		\lspbottomrule
	\end{tabularx}
\end{table}

\subsection[M-tone morphemes]{M-tone morphemes: {Agentive}, {dative} and {topic}}
\label{sec:encliticsthatcarrymtonewhenfollowingamtonenoun}

This section presents three morphemes that carry M tone when following a~M-tone noun and are therefore analyzed as having a~M lexical tone. 

Tables~\ref{tab:topic}a--b present the tonal behaviour of agentive /\ipa{ɳɯ˧}/, dative
/\ipa{-ki˧}/ (whose tonal behaviour is identical with that of the \isi{possessive}, /\ipa{=bv̩˧}/) and topic
marker /\ipa{ʈʂʰɯ˧}/. This last morpheme can be hypothesized to be an~extension of the 3\textsuperscript{rd}
person singular \is{pronouns}pronoun /\ipa{ʈʂʰɯ˥}/, which also serves as a~proximal demonstrative, but in view of its tonal behaviour, the topic marker is considered to have M tone, as against H tone for the 3\textsuperscript{rd}
person singular \is{pronouns}pronoun /\ipa{ʈʂʰɯ˥}/. Table~\ref{tab:topicfull} presents examples, and 
Table~\ref{tab:topicabstract} the underlying tone patterns. Note that the tones of /\ipa{bo˩ ɳɯ˧}/ and
/\ipa{ʐæ˩ ɳɯ˥}/ are neutralized at the
surface phonological level due to Rule 6 (“In tone"=group"=final position, H and M are neutralized to H if they follow a~L tone”: see \sectref{sec:alistoftonerules}). As elsewhere, variants are separated by a~slash. The difference in tone patterns between these morphemes when associated to `horse' was carefully verified: the tones are M.L in /\ipa{ʐwæ˧ ɳɯ˩}/, as opposed to M.M in /\ipa{ʐwæ˧-ki˧}/ and /\ipa{ʐwæ˧ ʈʂʰɯ˧}/.	

\begin{subtables}
\label{tab:topic}

\begin{table}[t]
\caption{\label{tab:topicfull}Tone patterns of {agentive} /\ipa{ɳɯ˧}/, {dative} /\ipa{-ki˧}/, and {topic} marker
    /\ipa{ʈʂʰɯ˧}/ with {monosyllabic} and disyllabic nouns: examples in full.}
{\setlength\tabcolsep{5pt}%\renewcommand{\arraystretch}{1.25}
  \begin{tabularx}{\textwidth}{ l l Q P{24mm} P{30mm} }
\lsptoprule
	example & tone & /\ipa{ɳɯ˧}/ & /\ipa{-ki˧}/ & /\ipa{ʈʂʰɯ˧}/\\ \midrule
	pig & LM & \ipa{bo˩ ɳɯ˧} & \ipa{bo˩-ki˧} & \ipa{bo˩ ʈʂʰɯ˧}\\
	leopard & LH & \ipa{ʐæ˩ ɳɯ˥ } & \ipa{ʐæ˩-ki˥} & \ipa{ʐæ˩ ʈʂʰɯ˥}\\
	tiger & M & \ipa{lɑ˧ ɳɯ˧} & \ipa{lɑ˧-ki˧} & \ipa{lɑ˧ ʈʂʰɯ˧}\\
	sheep & L & \ipa{jo˧ ɳɯ˧ / jo˩~ɳɯ˥} & \ipa{jo˧"=ki˧ / \mbox{jo˩-ki˥}} & \ipa{jo˧ ʈʂʰɯ˧ / jo˩~ʈʂʰɯ˥}\\
	horse & H & \ipa{ʐwæ˧ ɳɯ˩} & \ipa{ʐwæ˧-ki˧} & \ipa{ʐwæ˧ ʈʂʰɯ˧}\\
	deer & MH & \ipa{ʈʂʰæ˧ ɳɯ˥} & \ipa{ʈʂʰæ˧-ki˥} & \ipa{ʈʂʰæ˧ ʈʂʰɯ˧ / ʈʂʰæ˧~ʈʂʰɯ˥}\\ \addlinespace \hdashline \addlinespace
	fox & M & \ipa{ɖɤ˧mi˧ ɳɯ˧} & \ipa{ɖɤ˧mi˧-ki˧} & \ipa{ɖɤ˧mi˧ ʈʂʰɯ˧}\\
	colt & \#H & \ipa{ʐwæ˧zo˧ ɳɯ˩} & \ipa{ʐwæ˧zo˧-ki˧} & \ipa{ʐwæ˧zo˧ ʈʂʰɯ˧}\\
	cat & MH\# & \ipa{hwɤ˧li˧ ɳɯ˥} & \ipa{hwɤ˧li˧-ki˥} & \ipa{hwɤ˧li˧ ʈʂʰɯ˥}\\
	she"=cat & H\$ & \ipa{hwɤ˧mi˥ ɳɯ˩ / hwɤ˧mi˧ ɳɯ˥} & \ipa{hwɤ˧mi˥-ki˩} & \ipa{hwɤ˧mi˧ ʈʂʰɯ˧ / hwɤ˧mi˧ ʈʂʰɯ˥}\\
	dog & L & \ipa{kʰv̩˩mi˩ ɳɯ˥} & \ipa{kʰv̩˩mi˩-ki˥} & \ipa{kʰv̩˩mi˩ ʈʂʰɯ˥}\\
	mule & L\# & \ipa{dɑ˧ʝi˩ ɳɯ˩} & \ipa{dɑ˧ʝi˩-ki˩} & \ipa{dɑ˧ʝi˩ ʈʂʰɯ˩}\\
	wolf & LM+MH\# & \ipa{õ˩dv̩˧ ɳɯ˥} & \ipa{õ˩dv̩˧-ki˥} & \ipa{õ˩dv̩˧ ʈʂʰɯ˧ / õ˩dv̩˧ ʈʂʰɯ˥}\\
	Naxi & LM+\#H & \ipa{nɑ˩hĩ˧ ɳɯ˩} & \ipa{nɑ˩hĩ˧-ki˧} & \ipa{nɑ˩hĩ˧ ʈʂʰɯ˧}\\
	sow & LM & \ipa{bo˩mi˧ ɳɯ˧} & \ipa{bo˩mi˧-ki˧} & \ipa{bo˩mi˧ ʈʂʰɯ˧}\\
	boar & LH & \ipa{bo˩ɬɑ˥ ɳɯ˩} & \ipa{bo˩ɬɑ˥-ki˩} & \ipa{bo˩ɬɑ˥ ʈʂʰɯ˩}\\
	rat & H\# & \ipa{hwæ˧tsɯ˥ ɳɯ˩} & \ipa{hwæ˧tsɯ˥-ki˩} & \ipa{hwæ˧tsɯ˥ ʈʂʰɯ˩}\\
	\lspbottomrule
\end{tabularx}}
\end{table}

\begin{table}%[t]
\caption{\label{tab:topicabstract}Tone patterns of {agentive} /\ipa{ɳɯ˧}/, {dative} /\ipa{-ki˧}/, and {topic} marker
	/\ipa{ʈʂʰɯ˧}/ with {monosyllabic} and disyllabic nouns.}
  \begin{tabularx}{\textwidth}{ Q Q Q Q }
\lsptoprule
	tone of noun & /\ipa{ɳɯ}/ & /\ipa{-ki}/ & /\ipa{ʈʂʰɯ}/\\ \midrule
	LM & L.M & L.M & L.M\\
	LH & L.H & L.H & L.H\\
	M & M.M & M.M & M.M\\
	L & M.M / L.H & M.M / L.H & M.M / L.H\\
	H & M.L & M.M & M.M\\
	MH & M.H & M.H & M.H\\ \addlinespace \hdashline \addlinespace
	M & M.M.M & M.M.M & M.M.M\\
	\#H & M.M.L & M.M.M & M.M.M\\
	MH\# & M.M.H & M.M.H & M.M.H\\
	H\$ & M.H.L / M.M.H & M.M.M & M.M.M\\
	L & L.L.H & L.L.H & L.L.H\\
	L\# & M.L.L & M.L.L & M.L.L\\
	LM+MH\# & L.M.H & L.M.H & L.M.M / L.M.H\\
	LM+\#H & L.M.L & L.M.M & L.M.M\\
	LM & L.M.M & L.M.M & L.M.M\\
	LH & L.H.L & L.H.L & L.H.L\\
	H\# & M.H.L & M.H.L & M.H.L\\
\lspbottomrule
\end{tabularx}
\end{table}
\end{subtables}

 \largerpage
An exceptional pattern is observed for /\ipa{di˩˥}/ ‘earth’: in addition to the expected
/\ipa{di˩ ɳɯ˥}/, attested in (\ref{ex:reward145}), the form /\ipa{di˧ ɳɯ˧}/ is also acceptable: see (\ref{ex:reward121}). This \is{variants}variant is not acceptable for other LH"=tone nouns, e.g.~it is
not possible to say \ipa{$\ddagger${\kern2pt}ʐæ˧ ɳɯ˧} for ‘by \mbox{(a/the)} leopard’.

\begin{exe}
	\ex
	\label{ex:reward145}
	\ipaex{di˩-ɳɯ˥ {\kern2pt}|{\kern2pt} ə˧-sɯ˩kv̩˩ li˩-dʑo˩-ɲi˩!}\\
	\gll di˩˥	ɳɯ˧		ə˧-sɯ˩kv̩˩	li˧\textsubscript{a}		-dʑo˧		-ɲi˩\\
	earth		\textsc{a}	\textsc{1pl.incl}	to\_watch		\textsc{prog}			\textsc{certitude}\\
	\glt ‘The Earth is watching us!’ (Reward.145)
\end{exe}

\begin{exe}
	\ex
	\label{ex:reward121}
	\ipaex{“mv̩˧-ɳɯ˩ {\kern2pt}|{\kern2pt} ki˧! {\kern2pt}|{\kern2pt} di˧ ɳɯ˧ {\kern2pt}|{\kern2pt} ki˧!” {\kern2pt}|{\kern2pt} pi˧-ɲi˥ tsɯ˩ {\kern2pt}|{\kern2pt} mv̩˩.}\\
	\gll mv̩˥	ɳɯ˧		ki˧\textsubscript{a}		di˩˥	ɳɯ˧		ki˧\textsubscript{a}	pi˥		-ɲi˩		tsɯ˧˥	 mv̩˧\\
	sky		\textsc{a}	to\_give					earth	\textsc{a}	to\_give			to\_say			\textsc{rep}	\textsc{prog}	\textsc{affirm}\\
	\glt ‘“It is a~gift of the Heavens! It is a~gift of the Earth!” he said.’ (Reward.121)
\end{exe}

Where two variants are possible, \is{stylistics}stylistic nuances can occasionally be pinpointed. For instance, ‘to the father’ allows the two variants /\ipa{ə˧dɑ˧-ki˧}/ and /\ipa{ə˧dɑ˥-ki˩}/. The former is considered better than the latter.\footnote{To test the
	consultant’s preference, the investigator says one of the two alternatives while raising his right
	hand, then the second while raising his left hand. The consultant indicates whether both are
	acceptable, and expresses a~preference, often as an~understatement: /\ipa{ʈʂʰɯ˧ bæ˧, {\kern2pt}|{\kern2pt} ɖɯ˧-pi˧˥ {\kern2pt}|{\kern2pt}
		ho˩˥}/, “This one is \textit{pretty correct}”, meaning “This one is better”.} The form /\ipa{ə˧dɑ˥-ki˩}/ appears in ComingOfAge2.18, in a~context where emphasis is laid on the word ‘father’. In this context, /\ipa{ə˧dɑ˧-ki˧}/ would not be syntactically wrong, but it would be less appropriate from a~\is{stylistics}stylistic point of view. There appear to be (at least) two reasons for the emphatic, expressive value of /\ipa{ə˧dɑ˥-ki˩}/. First, the form /\ipa{ə˧dɑ˥-ki˩}/, as compared with /\ipa{ə˧dɑ˧-ki˧}/, involves a~higher pitch on the syllable that means ‘father’ (H tone, as against M tone). This has iconic value: there is a~cross"=linguistic tendency for raised pitch to be associated with emphasis, in tonal and nontonal languages alike (see \sectref{sec:pragmaticintonation}). A~second reason is that, in /\ipa{ə˧dɑ˥-ki˩}/, the word ‘father’ surfaces with the same tone that it would have if said \is{form!in isolation}in isolation: /\ipa{ə˧dɑ˥}/ (lexical form: //\ipa{ə˧dɑ˥\$}//). This makes the word stand out, as if extracting it from the flow of speech. By contrast, in /\ipa{ə˧dɑ˧-ki˧}/ the word ‘father’ is tonally amalgamated with its \is{suffixes}suffix, and the constituent made up of the noun and its \is{suffixes}suffix is, in turn, woven into the mesh of its~wider environment, as a~neatly packaged unit whose tone pattern testifies to its syntactic elaboration. This greater degree of morphosyntactic integration befits carefully planned and constructed speech. It accounts for the perception of /\ipa{ə˧dɑ˧-ki˧}/ as a~“better” form outside context. 

\newpage 
In association with {agentive} /\ipa{ɳɯ˧}/, {dative} /\ipa{-ki˧}/ and {topic} marker
/\ipa{ʈʂʰɯ˧}/, the 1\textsuperscript{st} and 2\textsuperscript{nd} person pronouns behave like other L-tone items:
/\ipa{njɤ˧ ɳɯ˧}/, /\ipa{no˧ ɳɯ˧}/. On the other hand, the demonstratives /\ipa{ʈʂʰɯ˥}/ (proximal, also 3\textsuperscript{rd} person singular) and
/\ipa{tʰv̩˥}/ (distal) have a~different behaviour from other H-tone items: they yield M.M in
association with /\ipa{ɳɯ˧}/: /\ipa{ʈʂʰɯ˧ ɳɯ˧}/, /\ipa{tʰv̩˧ ɳɯ˧}/ (also /\ipa{ʈʂʰɯ˧ lɑ˧}/,
/\ipa{tʰv̩˧ lɑ˧}/ ‘this/that one too’). 

\subsection[MH"=tone morphemes]{The MH"=tone morphemes /\ipa{-qɑ˧˥}/ (dative/comitative) and /\ipa{gi˧˥}/ ‘behind’}
\label{sec:mhtoneenclitics}

The morpheme /\ipa{-qɑ˧˥}/ has dative and comitative functions. It is analyzed as carrying MH tone on the
basis of its behaviour after M-tone words both {monosyllabic} and disyllabic, and after LM"=tone
disyllables. \tabref{tab:thetonalbehaviourofthedativecomitativemarkerfollowingnouns} sets out the data. 
%(No recording was conducted.) 
The data for the \is{postpositions}postposition /\ipa{gi˧˥}/
‘behind’ is identical.\footnote{Remember that, as a~convention, clitics are preceded by an \textit{equal} sign, and a~hyphen is placed before suffixes and after prefixes, whereas postpositions are shown as free morphemes.}

\begin{table}%[t]
\caption{\label{tab:thetonalbehaviourofthedativecomitativemarkerfollowingnouns}The tonal behaviour of the dative/comitative marker /\ipa{-qɑ˧˥}/ following nouns.}
\begin{tabularx}{\textwidth}{ Q Q P{27mm} l }
\lsptoprule
	example & tone & example & tone pattern\\ \midrule
	pig & LM & \ipa{bo˩-qɑ˧} & L.M\\
	leopard & LH & \ipa{ʐæ˩-qɑ˥} & L.H (on surface: same as LM)\\
	tiger & M & \ipa{lɑ˧-qɑ˧˥} & M.MH\\
	sheep & L & \ipa{jo˩-qɑ˩˥} & L.LH\\
	horse & H & \ipa{ʐwæ˧-qɑ˩} & M.L\\
	deer & MH & \ipa{ʈʂʰæ˧-qɑ˥} & M.H\\ \addlinespace \hdashline \addlinespace
	fox & M & \ipa{ɖɤ˧mi˧-qɑ˧˥} & M.M.MH\\
	colt & \#H & \ipa{ʐwæ˧zo˧-qɑ˩} & M.M.L\\
	cat & MH\# & \ipa{hwɤ˧li˧-qɑ˥} & M.M.H\\
	she"=cat & H\$ & \ipa{hwɤ˧mi˥-qɑ˩} & M.H.L\\
	dog & L & \ipa{kʰv̩˩mi˩-qɑ˥} & L.L.H\\
	mule & L\# & \ipa{dɑ˧ʝi˩-qɑ˩} & M.L.L\\
	wolf & LM+MH\# & \ipa{õ˩dv̩˧-qɑ˥} & L.M.H\\
	Naxi & LM+\#H & \ipa{nɑ˩hĩ˧-qɑ˩} & L.M.L\\
	sow & LM & \ipa{bo˩mi˧-qɑ˧˥} & L.M.MH\\
	boar & LH & \ipa{bo˩ɬɑ˥-qɑ˩} & L.H.L\\
	rat & H\# & \ipa{hwæ˧tsɯ˥-qɑ˩} & M.H.L\\
\lspbottomrule
\end{tabularx}
\end{table}


Note that, as in all other morphosyntactic contexts, the //L.H.L// pattern is neutralized with //L.M.L// at
the surface phonological level. The tone pattern for ‘boar’ could therefore be transcribed as /L.M.L/ in surface phonological representation. Notation as //L.H.L// reflects the analysis proposed here: that the tone of the en\is{clitics}clitic is lowered to L because of the presence of a~preceding H tone~--
the H part of the LH tone pattern lexically attached to the noun ‘boar’. The choice of a~notation as
/L.M.L/ could seem advisable in order to stay closer to surface phonological form, limiting the degree of
abstraction of the notation; on the other hand, this could
clash with the notation of the lexical categories, wrongly suggesting that the \mbox{//LH//} lexical category became \mbox{//LM//}
when this en\is{clitics}clitic is added. This is why notation as L.H.L was chosen in \tabref{tab:thetonalbehaviourofthedativecomitativemarkerfollowingnouns}.


%%subsec:6-4-4
\subsection[A~H-tone morpheme]{The H-tone topic marker /\ipa{-dʑo˥}/, with observations about tonal contours in non-final position}
\label{sec:theambivalentbehaviouroftheltonetopicmarker}
\largerpage%longdistance
The topic marker /\ipa{-dʑo˥}/ can appear after nouns and verbs. The tone patterns that obtain when it is associated to a~noun are shown in \tabref{tab:thetonalbehaviourofthetopicmarkerfollowingnouns}. Interestingly, a~MH"=tone noun or verb preceding the topic marker is realized with a~MH \is{tonal contour}contour, e.g.~/\ipa{ʈʂʰæ˧˥-dʑo˩}/
‘as for the deer’ (from /\ipa{ʈʂʰæ˧˥}/ ‘deer’) and /\ipa{mɤ˧-lɑ˧˥-dʑo˩}/ ‘as [he/she] did not strike’
(from /\ipa{lɑ˧˥}/ ‘to strike’). This suggests that there is a~\isi{tone group} \is{boundary (between tone groups)}boundary after the noun or
verb, since contours are only realized tone"=group"=finally. Such a~behaviour would not be
unparalleled. For instance, the contrastive topic marker /\ipa{-no˧˥}/ and the word /\ipa{tʰi˩˥}/ ‘then’ always
mark the beginning of a~new \isi{tone group}. But the
tonal behaviour of the topic marker after nouns or verbs bearing a~tone other than MH suggests that it is integrated within the same tone
group. For instance, after a~M-tone noun or verb, the pattern is M.M.H, e.g.~/\ipa{lɑ˧-dʑo˥}/ ‘as for
the tiger’ and /\ipa{mɤ˧-hwæ˧-dʑo˥}/ ‘as [she/he] does not buy’: these expressions clearly constitute a~single \isi{tone group}. The full data set is presented in \tabref{tab:thetonalbehaviourofthetopicmarkerfollowingnouns}.\footnote{The data was verified by using additional nouns
  illustrating the tonal categories: /\ipa{zo˥}/ ‘son’ for H tone, /\ipa{õ˧˥}/ ‘oneself’ for MH tone,
  /\ipa{pʰɤ˧bɤ˧}/ ‘gift’ for M tone, and /\ipa{ə˧dɑ˥\$}/ ‘father’ and /\ipa{mv̩˧ʁo˥\$}/ ‘heavens’ for H\$ tone. The results were the same, including variant patterns: for ‘gift’, there are two variants,
  /\ipa{pʰɤ˧bɤ˧-dʑo˧}/ and /\ipa{pʰɤ˧bɤ˧-dʑo˥}/, in the same way as for the example used in \tabref{tab:thetonalbehaviourofthetopicmarkerfollowingnouns} (‘fox’). The only
  unexpected result was with ‘plain’, //\ipa{di˧qo˧}//. One would expect /\ipa{†di˧qo˧-dʑo˥}/, on the {analogy} of /\ipa{ɖɤ˧mi˧-dʑo˥}/, but the observed pattern is M.M.M : /\ipa{di˧qo˧-dʑo˧}/. This
  unexpected result, confirmed across elicitation sessions, may have to do with the internal
  structure of this disyllable, literally meaning ‘on earth’.} These observations are taken up
  in~\sectref{sec:casesofbreachoftonalgroupingandconsequencesforthesystem}, as part of the discussion of cases of breach of
tonal grouping: how non"=final syllables can come to carry a~\is{tonal contour}contour, and following syllables become
extrametrical. 
\clearpage


\begin{table}%[t]
\caption{\label{tab:thetonalbehaviourofthetopicmarkerfollowingnouns}The tonal behaviour of the topic marker /\ipa{-dʑo˥}/ following nouns.}
\begin{tabularx}{\textwidth}{ Q Q l l }
\lsptoprule
	example & tone & example & tone pattern\\ \midrule
	pig & LM & \ipa{bo˩˧-dʑo˩} & LM.L\\
	leopard & LH & \ipa{ʐæ˩˥-dʑo˩} & LH.L\\
	tiger & M & \ipa{lɑ˧-dʑo˥} & M.H\\
	daughter & L & \ipa{mv̩˩-dʑo˩˥} & LH.L\\
	horse & H & \ipa{ʐwæ˧-dʑo˩} & M.L\\
	deer & MH & \ipa{ʈʂʰæ˧˥-dʑo˩} & MH.L\\ \addlinespace \hdashline \addlinespace
	fox & M & \ipa{ɖɤ˧mi˧-dʑo˥ / ɖɤ˧mi˧-dʑo˧} & M.M.H / M.M.M\\
	colt & \#H & \ipa{ʐwæ˧zo˧-dʑo˩} & M.M.L\\
	cat & MH\# & \ipa{hwɤ˧li˧˥-dʑo˩} & M.MH.L\\
	she"=cat & H\$ & \ipa{hwɤ˧mi˥-dʑo˩} & M.H.L\\
	dog & L & \ipa{kʰv̩˩mi˩˥-dʑo˩} & L.LH.L\\
	mule & L\# & \ipa{dɑ˧ʝi˩-dʑo˩} & M.L.L\\
	wolf & LM+MH\# & \ipa{õ˩dv̩˧˥-dʑo˩} & L.MH.L\\
	Naxi & LM+\#H & \ipa{nɑ˩hĩ˧-dʑo˩} & L.M.L\\
	sow & LM & \ipa{bo˩mi˧-dʑo˥} & L.M.H\\
	boar & LH & \ipa{bo˩ɬɑ˧-dʑo˩} & L.M.L\\
	rat & H\# & \ipa{hwæ˧tsɯ˥-dʑo˩} & M.H.L\\
\lspbottomrule
\end{tabularx}
\end{table}

On the basis of its behaviour after M-tone nouns (and after M-tone verbs, which will be presented in the next chapter), the topic marker is provisionally analyzed as carrying a~lexical H tone. If this tonal identification is confirmed, the morpheme constitutes an extreme case of distance between underlying form and surface form: in texts, realizations with a~L tone outnumber those with a~H tone by a~ratio of about 10 to 1. The particles indicating reported speech, /\ipa{tsɯ˧˥}/, and affirmation, /\ipa{mv̩˧}/, constitute even more spectacular cases: their underlying form hardly ever surfaces as such (see \ref{sec:implicationsforthetonesofsentenceparticles}).

%{\largerpage}

\section{Disyllabic postpositions}
\label{sec:spatialpostpositions}

\is{disyllables}Disyllabic locative postpositions include /\ipa{ʁo˧tʰo˩}/ ‘behind’,
/\ipa{ʁo˧dɑ˧}/ ‘in front of’, /\ipa{ɬo˧tɑ˧}/ ‘beside, to the side of’, /\ipa{ʁwæ˧gi\#˥}/ ‘to the
left’, /\ipa{jo˩gi˩}/ ‘to the right’, and /\ipa{-qo˧lo˩}/ ‘inside’. (The {monosyllabic} form /\ipa{-qo˧}/ is also attested, with the same meaning.) \tabref{tab:thetonalbehaviourofthespatialpostpositionsbesidetothesideofandbehind} shows the data for nouns followed by the locative postpositions /\ipa{ɬo˧tɑ˧}/ ‘beside, to
the side of’ and /\ipa{ʁo˧tʰo˩}/ ‘behind, to the back of’. The data for /\ipa{ʁo˧dɑ˧}/ ‘in front of’ is identical with that for /\ipa{ɬo˧tɑ˧}/ ‘beside, to the side of’, and is therefore not shown in the table. Likewise, the behaviour of /\ipa{ʈʰæ˧qo˩}/ ‘under’ is identical with that of /\ipa{ʁo˧tʰo˩}/
‘behind’. \tabref{tab:thetonalbehaviourofthespatialpostpositionstotheleftandtotheright} shows the data for the locative postpositions /\ipa{ʁwæ˧gi\#˥}/ ‘to the left’ and
/\ipa{jo˩gi˩}/ ‘to the right’. The corresponding recording is: LocativePostp.


\begin{table}[t]
\caption{\label{tab:thetonalbehaviourofthespatialpostpositionsbesidetothesideofandbehind}The tonal behaviour of the locative postpositions /\ipa{ɬo˧tɑ˧}/ ‘beside’ and /\ipa{ʁo˧tʰo˩}/ ‘behind’. There is an additional ‘L \textsc{pro}’ row because L-tone pronouns have exceptional behaviour.}
  \begin{tabularx}{\textwidth}{ l l l Q Q }
\lsptoprule
	tone & example & meaning & beside & behind\\ \midrule
	LM & \ipa{bo˩˧} & pig & \ipa{bo˩ ɬo˧tɑ˧} & \ipa{bo˩ ʁo˥tʰo˩}\\
	LH & \ipa{ʐæ˩˥} & leopard & \ipa{ʐæ˩ ɬo˧tɑ˧} & \ipa{ʐæ˩ ʁo˥tʰo˩}\\
	M & \ipa{lɑ˧} & tiger & \ipa{lɑ˧ ɬo˧tɑ˧} & \ipa{lɑ˧ ʁo˧tʰo˩}\\
	L & \ipa{jo˩} & sheep & \ipa{jo˩ ɬo˩tɑ˩} & \ipa{jo˩ ʁo˩tʰo˥}\\
	L \textsc{pro} & \ipa{no˩} & \textsc{2sg} & \ipa{no˧ ɬo˧tɑ˧ } & \ipa{no˧ ʁo˧tʰo˩}\\
	H & \ipa{ʐwæ˥} & horse & \ipa{ʐwæ˧ ɬo˧tɑ˥} & \ipa{ʐwæ˧ ʁo˧tʰo˥}\\
	MH & \ipa{ʈʂʰæ˧˥} & deer & \ipa{ʈʂʰæ˧ ɬo˧tɑ˥} & \ipa{ʈʂʰæ˧ ʁo˧tʰo˥}\\ \addlinespace \hdashline \addlinespace
	M & \ipa{ɖɤ˧mi˧} & fox & \ipa{ɖɤ˧mi˧ ɬo˧tɑ˧} & \ipa{ɖɤ˧mi˧ ʁo˧tʰo˩}\\
	\#H & \ipa{ʐwæ˧zo\#˥} & colt & \ipa{ʐwæ˧zo˧ ɬo˧tɑ˥} & \ipa{ʐwæ˧zo˧ ʁo˧tʰo˥}\\
	MH\# & \ipa{hwɤ˧li˧˥} & cat & \ipa{hwɤ˧li˧ ɬo˧tɑ˥} & \ipa{hwɤ˧li˧ ʁo˧tʰo˥}\\
	H\$ & \ipa{hwɤ˧mi˥\$} & she"=cat & \ipa{hwɤ˧mi˧ ɬo˧tɑ˥} & \ipa{hwɤ˧mi˧ ʁo˧tʰo˥}\\
	L & \ipa{kʰv̩˩mi˩} & dog & \ipa{kʰv̩˩mi˩ ɬo˩tɑ˥} & \ipa{kʰv̩˩mi˩ ʁo˩tʰo˥}\\
	L\# & \ipa{dɑ˧ʝi˩} & mule & \ipa{dɑ˧ʝi˩ ɬo˩tɑ˩} & \ipa{dɑ˧ʝi˩ ʁo˩tʰo˩}\\
	LM+MH\# & \ipa{õ˩dv̩˧˥} & wolf & \ipa{õ˩dv̩˧ ɬo˧tɑ˥} & \ipa{õ˩dv̩˧ ʁo˧tʰo˥}\\
	LM+\#H & \ipa{nɑ˩hĩ\#˥} & Naxi & \ipa{nɑ˩hĩ˧ ɬo˧tɑ˥} & \ipa{nɑ˩hĩ˧ ʁo˧tʰo˥}\\
	LM & \ipa{bo˩mi˧} & sow & \ipa{bo˩mi˧ ɬo˧tɑ˧} & \ipa{bo˩mi˧ ʁo˧tʰo˩}\\
	LH & \ipa{bo˩ɬɑ˥} & boar & \ipa{bo˩ɬɑ˥ ɬo˩tɑ˩} & \ipa{bo˩ɬɑ˧ ʁo˩tʰo˩}\\
	H\# & \ipa{hwæ˧tsɯ˥} & rat & \ipa{hwæ˧tsɯ˥ ɬo˩tɑ˩} & \ipa{hwæ˧tsɯ˥ ʁo˩tʰo˩}\\
\lspbottomrule
\end{tabularx}
\end{table}


\begin{table}[t]
\caption{\label{tab:thetonalbehaviourofthespatialpostpositionstotheleftandtotheright}The tonal behaviour of the locative postpositions /\ipa{ʁwæ˧gi\#˥}/ ‘to the left’ and /\ipa{jo˩gi˩}/ ‘to the right’. There is an additional ‘L \textsc{pro}’ row because L-tone pronouns have exceptional behaviour.}
\begin{tabularx}{\textwidth}{ l l l l Q }
\lsptoprule
	tone & example & meaning & to the left of & to the right of\\ \midrule
	LM & \ipa{bo˩˧} & pig & \ipa{bo˩ ʁwæ˧gi\#˥} & \ipa{bo˩ jo˧gi\#˥}\\
	LH & \ipa{ʐæ˩˥} & leopard & \ipa{ʐæ˩ ʁwæ˧gi\#˥} & \ipa{ʐæ˩ jo˧gi\#˥}\\
	M & \ipa{lɑ˧} & tiger & \ipa{lɑ˧ ʁwæ˧gi\#˥} & \ipa{lɑ˧ jo˩gi˩}\\
	L & \ipa{jo˩} & sheep & \ipa{jo˩ ʁwæ˩gi˩} & \ipa{jo˧ jo˩gi˩}\\
	L \textsc{pro} & \ipa{no˩} & 2\textsc{sg} & \ipa{no˧ ʁwæ˧gi\#˥} & \ipa{no˧ jo˩gi˩}\\
	H & \ipa{ʐwæ˥} & horse & \ipa{ʐwæ˧ ʁwæ˧gi\#˥} & \ipa{ʐwæ˧ jo˥gi˩}\\
	MH & \ipa{ʈʂʰæ˧˥} & deer & \ipa{ʈʂʰæ˧ ʁwæ˧gi\#˥} & \ipa{ʈʂʰæ˧ jo˥gi˩}\\ \addlinespace \hdashline \addlinespace
	M & \ipa{ɖɤ˧mi˧} & fox & \ipa{ɖɤ˧mi˧ ʁwæ˧gi\#˥} & \ipa{ɖɤ˧mi˧ jo˩gi˩}\\
	\#H & \ipa{ʐwæ˧zo\#˥} & colt & \ipa{ʐwæ˧zo˧ ʁwæ˧gi\#˥} & \ipa{ʐwæ˧zo˧ jo˥gi˩}\\
	MH\# & \ipa{hwɤ˧li˧˥} & cat & \ipa{hwɤ˧li˧ ʁwæ˧gi\#˥} & \ipa{hwɤ˧li˧ jo˥gi˩}\\
	H\$ & \ipa{hwɤ˧mi˥\$} & she"=cat & \ipa{hwɤ˧mi˧ ʁwæ˧gi\#˥} & \ipa{hwɤ˧mi˧ jo˥gi˩}\\
	L & \ipa{kʰv̩˩mi˩} & dog & \ipa{kʰv̩˩mi˩ ʁwæ˩gi˩} & \ipa{kʰv̩˩mi˩ jo˥gi˩}\\
	L\# & \ipa{dɑ˧ʝi˩} & mule & \ipa{dɑ˧ʝi˩ ʁwæ˩gi˩} & \ipa{dɑ˧ʝi˩ jo˩gi˩}\\
	LM+MH\# & \ipa{õ˩dv̩˧˥} & wolf & \ipa{õ˩dv̩˧ ʁwæ˧gi\#˥} & \ipa{õ˩dv̩˧ jo˥gi˩}\\
	LM+\#H & \ipa{nɑ˩hĩ\#˥} & Naxi & \ipa{nɑ˩hĩ˧ ʁwæ˧gi\#˥} & \ipa{nɑ˩hĩ˧ jo˥gi˩}\\
	LM & \ipa{bo˩mi˧} & sow & \ipa{bo˩mi˧ ʁwæ˧gi\#˥} & \ipa{bo˩mi˧ jo˩gi˩}\\
	LH & \ipa{bo˩ɬɑ˥} & boar & \ipa{bo˩ɬɑ˧ ʁwæ˩gi˩} & \ipa{bo˩ɬɑ˧ jo˩gi˩}\\
	H\# & \ipa{hwæ˧tsɯ˥} & rat & \ipa{hwæ˧tsɯ˥ ʁwæ˩gi˩} & \ipa{hwæ˧tsɯ˥ jo˩gi˩}\\
\lspbottomrule
\end{tabularx}
\end{table}


There are two rows for the L tone in \tabref{tab:thetonalbehaviourofthespatialpostpositionsbesidetothesideofandbehind} and \tabref{tab:thetonalbehaviourofthespatialpostpositionstotheleftandtotheright} because L-tone pronouns, 1\textsc{sg} /\ipa{njɤ˩}/
and 2\textsc{sg} /\ipa{no˩}/, have exceptional behaviour. The difference in tonal output between pronouns and nouns is clear. For instance, with the
M-tone \is{postpositions}postposition ‘beside’, a~L-tone \is{pronouns}pronoun yields a~M-tone pattern: /\ipa{no˧ ɬo˧tɑ˧}/, and it
is not possible to say \ipa{$\ddagger${\kern2pt}no˩ ɬo˩tɑ˩}. Conversely, a~L-tone noun yields a~L-tone pattern:
/\ipa{jo˩ ɬo˩tɑ˩}/, and it is not possible to say \ipa{$\ddagger${\kern2pt}jo˧ ɬo˧tɑ˧}.

The system would look nice and economical if the tone patterns in these tables were identical with those for
other constructions, such as determinative compounds. Such is the case for /\ipa{ʁo˧tʰo˩}/ ‘at the
back’, which behaves tonally like a~L\#-tone head noun in determinative compounds, and for
/\ipa{ɬo˧tɑ˧}/ ‘beside, to the side of’, which behaves like a~M-tone noun in determinative compounds. Not all locative
postpositions share this behaviour, however: compare /\ipa{hwɤ˧li˧-hi˧kʰɯ˧˥}/ ‘cat’s gums (body
part)’ (input tones: MH\# and \#H; output tone: MH\#) and /\ipa{hwɤ˧li˧ ʁwæ˧gi\#˥}/ ‘to the left of
the cat’ (same input; output tone: \#H).

\largerpage
\section{Adverbs}
\label{sec:adverbs}

Contrary to what their name suggests, \textit{adverbs} (a loosely defined class of words) can appear not only with verbs, but also with nouns and some other linguistic units. 

\subsection{The homophonous adverbs /\ipa{lɑ˧}/ ‘only’ and ‘too, and’}
\label{sec:onlyand}

The data for the two \is{homophony}homophonous adverbs /\ipa{lɑ˧}/ ‘only’ and ‘too, and’ is shown in \tabref{tab:thebehaviourofonlyalsowithmonosyllabicanddisyllabicnouns}. Systematic elicitation of expressions made up of a~noun plus these adverbs was up against some slight initial difficulties, as these expressions do not constitute complete utterances. An illustration of the pitfalls of this type of elicitation is that the combination ‘only \mbox{(a/the)} she"=cat' was first recorded with a~M.M.L pattern, as /\ipa{hwɤ˧mi˧ lɑ˩}/, then with a~M.M.M pattern, as /\ipa{hwɤ˧mi˧ lɑ˧}/. Later the consultant pointed out that these were both wrong, and that the correct pattern was M.H.L: /\ipa{hwɤ˧mi˥ lɑ˩}/. Homophony between this expression and ‘to beat \mbox{(a/the)} she"=cat' (likewise /\ipa{hwɤ˧mi˥ lɑ˩}/) may have contributed to the difficulty encountered at elicitation.

\begin{table}%[t]
	\caption{\label{tab:thebehaviourofonlyalsowithmonosyllabicanddisyllabicnouns}The behaviour of /\ipa{lɑ˧}/ ‘only; also’ with {monosyllabic} and disyllabic nouns.}
	\begin{tabularx}{\textwidth}{ Q Q P{30mm} l }
		\lsptoprule
		example & tone & example & abstract tone pattern\\ \midrule
		pig & LM & \ipa{bo˩ lɑ˧} & L.M\\
		leopard & LH & \ipa{ʐæ˩ lɑ˥ } & L.H\\
		tiger & M & \ipa{lɑ˧ lɑ˧} & M.M\\
		sheep & L & \ipa{jo˩ lɑ˥} & L.H\\
		horse & H & \ipa{ʐwæ˧ lɑ˩} & M.L\\
		deer & MH & \ipa{ʈʂʰæ˧ lɑ˥} & M.H\\ \addlinespace \hdashline \addlinespace
		fox & M & \ipa{ɖɤ˧mi˧ lɑ˧} & M.M.M\\
		colt & \#H & \ipa{ʐwæ˧zo˧ lɑ˩} & M.M.L\\
		cat & MH\# & \ipa{hwɤ˧li˧ lɑ˥} & M.M.H\\
		she"=cat & H\$ & \ipa{hwɤ˧mi˥ lɑ˩} & M.H.L\\
		dog & L & \ipa{kʰv̩˩mi˩ lɑ˥} & L.L.H\\
		mule & L\# & \ipa{dɑ˧ʝi˩ lɑ˩} & M.L.L\\
		wolf & LM+MH\# & \ipa{õ˩dv̩˧ lɑ˥} & L.M.H\\
		Naxi & LM+\#H & \ipa{nɑ˩hĩ˧ lɑ˩} & L.M.L\\
		sow & LM & \ipa{bo˩mi˧ lɑ˧} & L.M.M\\
		boar & LH & \ipa{bo˩ɬɑ˥ lɑ˩} & L.H.L\\
		rat & H\# & \ipa{hwæ˧tsɯ˥ lɑ˩} & M.H.L\\
		\lspbottomrule
	\end{tabularx}
\end{table}

Comparison with the data for the three M-tone morphemes in Tables~\ref{tab:topic}a--b shows that the tonal behaviour of the four morphemes /\ipa{lɑ˧}/, /\ipa{ɳɯ˧}/, /\ipa{-ki˧}/, and /\ipa{ʈʂʰɯ˧}/ is tantalizingly similar: their tone patterns are identical for eleven of the seventeen
categories of nouns. The only difference between /\ipa{lɑ˧}/ and /\ipa{ɳɯ˧}/ is in combination with
the L category of monosyllables, where the pattern is L.H for /\ipa{lɑ˧}/, as in (\ref{ex:sheepandgoats}), and M.M for /\ipa{ɳɯ˧}/, as in (\ref{ex:bythesheep}). Needless to say, the data was carefully verified across work sessions. (A recording is available: OnlyAnd.)

 \begin{exe}
 	\ex
 	\label{ex:sheepandgoats}
 	\ipaex{jo˩ lɑ˥ {\kern2pt}|{\kern2pt} tsʰɯ˧˥}\\
 	\gll jo˩		lɑ˧		tsʰɯ˧˥\\
 	sheep		and		goat\\
 	\glt ‘sheep and goats’
 \end{exe}
 
 \begin{exe}
 	\ex
 	\label{ex:bythesheep}
 	\ipaex{jo˧ ɳɯ˧}\\
 	\gll jo˩		ɳɯ˧\\
 	sheep		\textsc{a}\\
 	\glt ‘by the sheep’
 \end{exe}

To venture a~hypothesis concerning these incomplete similarities, it seems plausible that \isi{grammaticalization} is accompanied by a~tonal evolution away from the tone of the free root. As mentioned in \sectref{sec:enclitics}, it may not be coincidental that dative
/\ipa{-ki˧}/ and \isi{possessive} /\ipa{=bv̩˧}/, which have exactly the same tonal behaviour, also share the morphosyntactic property of being “almost suffixal” \citep[155]{lidz2010}, and thereby distinct from agentive /\ipa{ɳɯ˧}/, analyzed as a~case adposition. Seen in this light, the difference in tone patterns would match that of parts of speech: one of the tone patterns is for suffixes, one for adpositions, one for adverbs/conjunctions, and one for discourse particles (in this case the topic marker). 

\subsection{The adverb /\ipa{pɤ˧to˩}/ ‘even’}
\label{sec:evenplusnoun}

The behaviour of the adverb /\ipa{pɤ˧to˩}/ ‘even’ is presented in \tabref{tab:thetonalbehaviourofeven}; the corresponding recording is NounsEven. This data differs from that of L\#-tone locative postpositions such as /\ipa{ʁo˧tʰo˩}/
‘behind’ and /\ipa{ʈʰæ˧qo˩}/ ‘under’, studied above (\ref{sec:spatialpostpositions}). The tonal behaviour of /\ipa{pɤ˧to˩}/ also differs from that of L\#-tone heads in
\is{compounds}compound nouns.


\begin{table}%[t]
\caption{\label{tab:thetonalbehaviourofeven}The tonal behaviour of /\ipa{pɤ˧to˩}/ ‘even’. There is an additional ‘L \textsc{pro}’ row because L-tone pronouns have exceptional behaviour.}
\begin{tabularx}{\textwidth}{ Q Q Q l }
\lsptoprule
	tone & example & meaning & N+/\ipa{pɤ˧to˩}/ ‘even’\\ \midrule
	LM & \ipa{bo˩˧} & pig & \ipa{bo˩ pɤ˥to˩}\\
	LH & \ipa{ʐæ˩˥} & leopard & \ipa{ʐæ˩ pɤ˥to˩}\\
	M & \ipa{lɑ˧} & tiger & \ipa{lɑ˧ pɤ˧to˩}\\
	L & \ipa{jo˩} & sheep & \ipa{jo˩ pɤ˩to˥}\\
	L \textsc{pro} & \ipa{no˩} & \textsc{2sg} & \ipa{no˧ pɤ˧to˩}\\
	\#H & \ipa{ʐwæ˥} & horse & \ipa{ʐwæ˧ pɤ˧to˩}\\
	MH\# & \ipa{ʈʂʰæ˧˥} & deer & \ipa{ʈʂʰæ˧ pɤ˥to˩}\\ \addlinespace \hdashline \addlinespace
	M & \ipa{ɖɤ˧mi˧} & fox & \ipa{ɖɤ˧mi˧ pɤ˧to˩}\\
	\#H & \ipa{ʐwæ˧zo\#˥} & colt & \ipa{ʐwæ˧zo˧ pɤ˧to˩}\\
	MH\# & \ipa{hwɤ˧li˧˥} & cat & \ipa{hwɤ˧li˧ pɤ˥to˩}\\
	H\$ & \ipa{hwɤ˧mi˥\$} & she"=cat & \ipa{hwɤ˧mi˧ pɤ˥to˩}\\
	L & \ipa{kʰv̩˩mi˩} & dog & \ipa{kʰv̩˩mi˩ pɤ˥to˩}\\
	L\# & \ipa{dɑ˧ʝi˩} & mule & \ipa{dɑ˧ʝi˩ pɤ˩to˩}\\
	LM+MH\# & \ipa{õ˩dv̩˧˥} & wolf & \ipa{õ˩dv̩˧ pɤ˥to˩}\\
	LM+\#H & \ipa{nɑ˩hĩ\#˥} & Naxi & \ipa{nɑ˩hĩ˧ pɤ˧to˩}\\
	LM & \ipa{bo˩mi˧} & sow & \ipa{bo˩mi˧ pɤ˧to˩}\\
	LH & \ipa{bo˩ɬɑ˥} & boar & \ipa{bo˩ɬɑ˧ pɤ˧to˩}\\
	H\# & \ipa{hwæ˧tsɯ˥} & rat & \ipa{hwæ˧tsɯ˥ pɤ˩to˩}\\
\lspbottomrule
\end{tabularx}
\end{table}


\section{Concluding note}
\label{sec:asummary}
\largerpage
% The phenomena do not obey general rules such as those found in polar-tone systems, where the tones of suffixes obtain through a~small set of rules.

The facts approached in this chapter are more diverse than in the three previous ones. There is a~great variety of tonal categories of grammatical elements in Yongning Na: the dative \is{suffixes}suffix /\ipa{-ki˧}/ and the \isi{possessive} clitic /\ipa{-bv̩˧}/ have a~different behaviour from the agentive adposition /\ipa{ɳɯ˧}/; a~third pattern is observed for the \is{conjunctions}conjunction /\ipa{lɑ˧}/ ‘too, and’; and a~fourth for the topic marker /\ipa{ʈʂʰɯ˧}/. The complexity of
Yongning Na morphotonology results from the great number of tonal paradigms.
%, none of which is especially complex in itself.
There remains much room for progress in the description and analysis of this morphotonological archipelago. In particular, systematic study of the lexicon holds potential for opening new windows onto processes of morphological derivation, revealing traces of former processes of \is{derivation!morphological}derivation such as an animal \is{suffixes}suffix /\ipa{-li}/ plausibly found in /\ipa{hwɤ˧li˧˥}/ ‘cat' and /\ipa{pʰi˧li˩}/ ‘butterfly' (\ili{Naxi} cognates: /\ipa{hwɑ˥le˧}/ and /\ipa{pʰe˧le˩}/). 