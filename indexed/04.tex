\chapter{Classifiers}
\label{chap:classifiers}

\is{classifiers|(}
\is{numerals|(}

A classifier \is{classifiers|textbf} is “a type of limited noun that occurs only after numerals ({\dots}), and whose
selection is determined by a~preceding (overt or implicit) noun”
\citep[88]{matisoff1973a}. The term ‘classifier’ is understood here in the
syntactic sense of any noun that may appear directly after
a~numeral. This includes measure words (‘inch', ‘armspan', ‘heap'{\dots}) and time words (‘day',
‘month', ‘year'{\dots}), which immediately follow the numeral. In view of their great importance in the language, and of the richness of their tone patterns, classifiers are dealt with in a~chapter of their own. The bulk of the chapter (\sectref{sec:numeralplusclassifierphrases}) deals with the tonal behaviour of classifiers in association with numerals. But classifiers
also associate with demonstratives (\sectref{sec:demonstrativeplusclassifierphrases}). In addition, numeral"=plus"=classifier or
demonstrative"=plus"=classifier phrases can interact tonally with a~preceding noun (\sectref{sec:interactionnoun}).

Devoting tens of pages to the topic of nominal classifiers, after long chapters devoted to the tones of nouns (Chapter~\ref{chap:thelexicaltonesofnouns}) and of compound nouns (Chapter~\ref{chap:compoundnouns}), may cause alarm to the best"=disposed of readers. Will the archipelago of tables never come to an end?
Is there no better way to describe the tonal grammar of Yongning Na than by recording lump after lump of morphotonological detail?

Indeed, the tone patterns of phrases containing classifiers cause a~disappointment to the linguist: they cannot be accounted for by phonological sandhi, nor do they obey a~well"=defined set of morphotonological rules.\footnote{Some clarifications about the terms ‘{tone sandhi}’, ‘morphotonology’ and ‘tonal morphology’ are set out in \sectref{sec:definitionofterms}.} The tonal paradigms of the various categories of classifiers need to be learnt one by one. The tables recording these patterns are admittedly the most heavy"=going in the present volume. But paradoxically, this chapter, which reports on the most irregular part of the Yongning Na tone system, carries a~comforting message: there \textit{is} a~limit to the seemingly endless complexity of the system, even in those areas that contain most irregularities. I~think (taking full responsibility for this immodest stance) that the present chapter gets close to an exhaustive description of its object, thereby showing that a~comprehensive account of tone in Yongning Na is not an unreachable goal.

%They have no more synchronic motivation than other irregular patterns such as \isi{suppletion} between the classifier /\ipa{v̩˧}/, in association with the numeral ‘one', and /\ipa{kv̩˧˥}/, with higher numerals, to count human beings. 


\section{Numeral"=plus"=classifier phrases}
\label{sec:numeralplusclassifierphrases}

Within the group of \ili{Naish} languages, tonal alternations in numeral"=plus"=classifier phrases are
vestigial in \ili{Naxi} (as spoken in the plain of Lijiang), limited in \ili{Laze}, and ubiquitous in Yongning
Na. In the Alawa dialect of Yongning Na, the tone of the
classifier is affected by the numeral to such an extent that arriving at the classifier's \is{form!underlying}underlying tone is not a~straightforward task. For instance, the classifier for days carries a~H tone in /\ipa{ɲi˧-ɲi˥}/ ‘two
days’, a~LH tone in /\ipa{so˩-ɲi˩˥}/ ‘three days’, and a~M tone in /\ipa{ʐv̩˧-ɲi˧}/ ‘four days’; and the
classifier for years has a~MH tone in /\ipa{ɲi˧-kʰv̩˧˥}/ ‘two years’, a~LH tone in /\ipa{so˩-kʰv̩˩˥}/
‘three years’, and a~L tone in /\ipa{ʐv̩˧-kʰv̩˩}/ ‘four years’. This section proposes a~synchronic description and analysis.\footnote{An earlier version of this section was published as
  \citet{michaud2013d}. Among other differences, subcategories of tones for classifiers are now distinguished through subscript letters, e.g.~M\textsubscript{a} and M\textsubscript{b} instead of the earlier
  notation as M1 and M2.}

For each noun in the vocabulary list collected in the course of fieldwork, a~note was made of the
classifiers typically associated with the noun, and this piece of information was included in the dictionary of Yongning Na \citep{michauddict2015}, following the example of Eugénie Henderson’s dictionary
of Bwe Karen \citep{henderson1997}. This obviously does not capture the full range of \is{stylistics}stylistic possibilities in
the choice of classifiers, which are best studied from their actual use: “a noun can be accompanied
by various classifiers depending on context, 
so it may not be adequate to describe one
of these as the primary classifier at a~lexical level” \citep[167]{francois2000a}.\footnote{\textit{Original text:} le même nom peut être accompagné de plusieurs classificateurs selon les contextes, sans qu'il soit légitime d'en privilégier un comme fondamental dès le lexique.} But this information came in handy to put together a~list of classifiers, yielding a~total of over one hundred monosyllabic classifiers. (Disyllabic classifiers will be dealt with
separately in
\sectref{sec:disyllabicclassifiersandimplicationsforthetonesofnumerals}.) 


\subsection{Elicitation procedures}
\label{sec:elicitationprocedures}
\label{sec:recordings}

The logical first step in the study of the tones of classifiers consisted of conducting systematic elicitation with numerals.\footnote{Classifiers are bound forms, which cannot be elicited in isolation (without an accompanying numeral or demonstrative). Numerals do not usually appear on their own either, but it is possible, at a~push, to elicit them in isolation. The forms from one to ten are: /\ipa{ɖɯ˧˥}/ ‘1’, /\ipa{ɲi˧˥}/ ‘2’, /\ipa{so˩˥}/ ‘3’, /\ipa{ʐv̩˧}/ ‘4’, /\ipa{ŋwɤ˧}/ ‘5’, /\ipa{qʰv̩˧˥}/ ‘6’, /\ipa{ʂɯ˧}/ ‘7’, /\ipa{hõ˧˥}/ ‘8’, /\ipa{gv̩˧}/ ‘9’, and /\ipa{tsʰe˧}/ ‘10’.} A~first data set was elicited in 2009, covering the range of numerals from 1 to 30. Elicitation yielded less than fully consistent results, due in part to the fact that the consultant was not accustomed to lengthy
counting tasks. One and the same noun"=plus"=classifier combination was realized with different tone patterns during different elicitation sessions. When such discrepancies were
pointed out to the consultant, she would identify one pattern as correct and reject the others as
mistaken. However, these \textit{a posteriori} judgments also wavered: a~\is{variants}variant that had been
brushed aside as \is{mistakes}mistaken would come up again in a~later session, and the consultant would then
insist that it was correct. Initially, I wrongly assumed that only one tone pattern could be
correct, but then it gradually became clear that there are two variants for some of the phrases. For
instance, the association of ‘9’ with the classifier for threads can be realized either as
/\ipa{gv̩˧-kʰɯ˥}/ or as /\ipa{gv̩˧-kʰɯ˩}/.
%(To preview the result of the analysis proposed in
%\sectref{sec:aboutvariantsoftonepatterns}: in the first \is{variants}variant, the phrase is parsed
%into two tone groups, whereas in the second, it is parsed as one \isi{tone group}.) 
Taking into
account these two factors~-- that occasional \isi{mistakes} are not uncommon in systematic elicitation,
and that two variants can both be correct~--, a~comprehensive description could finally be arrived
at, dissipating earlier perplexity and frustration.

The data collected in 2009, covering the range of numerals from 1 to 30, shows that some classifiers have a~tantalizingly similar but not identical behaviour: for instance, the classifier for days and the classifier for steps (of stairs) yield the same tone patterns for numerals up to thirteen, but differ for 14, 15, 16, 18, 19 and 22, as shown in \tabref{tab:stepsanddays}. Thus, ‘fourteen steps' is //\ipa{tsʰe˩ʐv̩˩-ɖwæ˥\$}// (tone: L+H\$; surface phonological form: /\ipa{tsʰe˩ʐv̩˩-ɖwæ˥}/), whereas ‘fourteen days' is //\ipa{tsʰe˩ʐv̩˩-ɲi˩}// (tone: L; surface phonological form: /\ipa{tsʰe˩ʐv̩˩-ɲi˩˥}/).

\begin{table}[p]
	\caption{Surface phonological form of the classifiers for days and for steps, in association with numerals from 1 to 30.}
	\begin{tabularx}{.85\textwidth}{ l Q Q l}
		\lsptoprule
		\textsc{num} & \textsc{clf}.days & \textsc{clf}.steps & comparison\\ \midrule
		1 & \ipa{ɖɯ˧-ɲi˥} & \ipa{ɖɯ˧-kʰwɤ˥} & same\\
		2 & \ipa{ɲi˧-ɲi˥} & \ipa{ɲi˧-kʰwɤ˥} & same\\
		3 & \ipa{so˩-ɲi˩˥} & \ipa{so˩-kʰwɤ˩˥} & same\\
		4 & \ipa{ʐv̩˧-ɲi˧} & \ipa{ʐv̩˧-kʰwɤ˧} & same\\
		5 & \ipa{ŋwɤ˧-ɲi˧} & \ipa{ŋwɤ˧-kʰwɤ˧} & same\\
		6 & \ipa{qʰv̩˧-ɲi˥} & \ipa{qʰv̩˧-kʰwɤ˥} & same\\
		7 & \ipa{ʂɯ˧-ɲi˧} & \ipa{ʂɯ˧-kʰwɤ˧} & same\\
		8 & \ipa{hõ˧-ɲi˥} & \ipa{hõ˧-kʰwɤ˥} & same\\
		9 & \ipa{gv̩˧-ɲi˧} & \ipa{gv̩˧-kʰwɤ˧} & same\\
		10 & \ipa{tsʰe˩-ɲi˩˥} & \ipa{tsʰe˩-kʰwɤ˩˥} & same\\
		11 & \ipa{tsʰe˩ɖɯ˩-ɲi˩˥} & \ipa{tsʰe˩ɖɯ˩-kʰwɤ˩˥} & same\\
		12 & \ipa{tsʰe˩ɲi˩-ɲi˩˥} & \ipa{tsʰe˩ɲi˩-kʰwɤ˩˥} & same\\
		13 & \ipa{tsʰe˩so˩-ɲi˩˥} & \ipa{tsʰe˩so˩-kʰwɤ˩˥} & same\\
		14 & \ipa{tsʰe˩ʐv̩˩-ɲi˩˥} & \ipa{tsʰe˩ʐv̩˩-kʰwɤ˥} & \textit{different}\\
		15 & \ipa{tsʰe˩ŋwɤ˩-ɲi˩˥} & \ipa{tsʰe˩ŋwɤ˩-kʰwɤ˥} & \textit{different}\\
		16 & \ipa{tsʰe˩qʰv̩˩-ɲi˩˥} & \ipa{tsʰe˩qʰv̩˩-kʰwɤ˥} & \textit{different}\\
		17 & \ipa{tsʰe˩ʂɯ˩-ɲi˩˥} & \ipa{tsʰe˩ʂɯ˩-kʰwɤ˩˥} & same\\
		18 & \ipa{tsʰe˩hõ˩-ɲi˩˥} & \ipa{tsʰe˩hõ˩-kʰwɤ˥} & \textit{different}\\
		19 & \ipa{tsʰe˩gv̩˩-ɲi˩˥} & \ipa{tsʰe˩gv̩˩-kʰwɤ˥} & \textit{different}\\
		20 & \ipa{ɲi˧tsi˧-ɲi˧} & \ipa{ɲi˧tsi˧-kʰwɤ˧} & same\\
		21 & \ipa{ɲi˧tsi˧ɖɯ˧-ɲi˥} & \ipa{ɲi˧tsi˧ɖɯ˧-kʰwɤ˥} & same\\
		22 & \ipa{ɲi˧tsi˧ɲi˧-ɲi˧} & \ipa{ɲi˧tsi˧ɲi˧-kʰwɤ˥} & \textit{different}\\
		23 & \ipa{ɲi˧tsi˧so˩-ɲi˩˥} & \ipa{ɲi˧tsi˧so˩-kʰwɤ˩˥} & same\\
		24 & \ipa{ɲi˧tsi˧ʐv̩˧-ɲi˧} & \ipa{ɲi˧tsi˧ʐv̩˧-kʰwɤ˧} & same\\
		25 & \ipa{ɲi˧tsi˧ŋwɤ˧-ɲi˧} & \ipa{ɲi˧tsi˧ŋwɤ˧-kʰwɤ˧} & same\\
		26 & \ipa{ɲi˧tsi˧qʰv̩˧-ɲi˥} & \ipa{ɲi˧tsi˧qʰv̩˧-kʰwɤ˥} & same\\
		27 & \ipa{ɲi˧tsi˧ʂɯ˧-ɲi˧} & \ipa{ɲi˧tsi˧ʂɯ˧-kʰwɤ˧} & same\\
		28 & \ipa{ɲi˧tsi˧hõ˧-ɲi˥} & \ipa{ɲi˧tsi˧hõ˧-kʰwɤ˥} & same\\
		29 & \ipa{ɲi˧tsi˧gv̩˧-ɲi˧} & \ipa{ɲi˧tsi˧gv̩˧-kʰwɤ˧} & same\\
		30 & \ipa{so˧tsʰi˧-ɲi˧} & \ipa{so˧tsʰi˧-kʰwɤ˧} & same\\
		\lspbottomrule
	\end{tabularx}
	\label{tab:stepsanddays}
\end{table}

This finding led to the decision to include the full range of numerals from 1 to 100 in a~second set of recordings. A total of 2,810 numeral"=plus"=classifier phrases were recorded. In the recordings that cover the range from 1 to 100, the intervals
[50..59] and [80..89] were not recorded, because previous elicitation had shown that their tone
patterns were identical with those of [40..49] and [60..69], respectively. Shortening the list
of numerals reduced consultant fatigue. 

The entire set of transcribed recordings is available online, with both a~surface phonological
transcription and an~indication of the underlying tonal pattern. 

No amount of continuous speech would be enough to obtain all the numeral"=plus"=classifier
combinations from 1 to 100, hence the choice to resort to systematic elicitation. However, some of
the combinations are also attested in the transcribed narratives.

\newpage 
\subsection{Results: Nine tonal categories for monosyllabic classifiers}
\label{sec:results}
%\subsubsection{How the tonal categories were brought out and labelled}
\label{sec:howthetonalcategorieswerebroughtoutandlabelled}

The nine tonal categories were brought out by distributional analysis, distinguishing sets of {monosyllabic} classifiers that have the same behaviour when
combined with numerals. These nine categories are shown in the nine
data columns of Tables~\ref{tab:1to100hmh}a--d and \ref{tab:1to100ml}a--d. 
%In turn, these categories were
%grouped into four sets on the basis of their similarities. For instance, the tone categories H\textsubscript{a} and H\textsubscript{b} (illustrated by the classifier for days and the classifier for steps of stairs, respectively) are identical
%except for fourteen of the numerals: {[14..16], [18..19], 22, 32, 42, 52, 62, 72, 82, 92,
%  99}. 

The choice of labels for the
nine categories brought out by distributional analysis was guided by structural hints. The tone patterns in association with the numerals ‘6’ and ‘8’ is not highly informative, since almost all tonal oppositions are
neutralized in this context (the only tones that are observed are H\# and H\$). Likewise, in phrases
involving ‘3’ and ‘10’, only two patterns are observed. After ‘4’ and ‘5’, four groups can be
distinguished, but if these patterns were indicative of the classifiers’ lexical tone, the system
would only contain two High tones (\#H and H\#) and two L tones (L\# and L). There would be no Mid
tones, and no contours. This would be completely unlike the lexical tones of the other {monosyllabic}
nouns found in Yongning Na, which consist of H, M, L and two types of rising contours, analyzed as
MH and LM.

On the other hand, the tone patterns in association with the numerals ‘1’ and ‘2’ make good sense as
labels for tonal categories. These four patterns are H, MH, M and L, all of which exist as lexical
tones for nouns. They are therefore adopted, adding a~letter to distinguish the
\is{subcategories of lexical tones}subcategories (two for H, MH and M, and three for L), by order of decreasing frequency (e.g.~there
are more classifiers in category M\textsubscript{a} than M\textsubscript{b}). Such labels for \is{subcategories of lexical tones}subcategories are also used for verbs,
among which it is necessary to distinguish two \is{subcategories of lexical tones}subcategories of L tones, L\textsubscript{a} and L\textsubscript{b}, and three
\is{subcategories of lexical tones}subcategories of M tones, M\textsubscript{a}, M\textsubscript{b} and M\textsubscript{c} (see Chapter~\ref{chap:verbsandtheircombinatoryproperties}). Note, however, that tonal \is{subcategories of lexical tones}subcategories are
established separately for nouns and verbs, and that no arguments can be proposed to identify these
\is{subcategories of lexical tones}subcategories across parts of speech. For instance, the label ‘L\textsubscript{a}’ as used for verbs does not refer
to the same category as ‘L\textsubscript{a}’ for classifiers. In order to preclude misunderstandings, it would be
possible to use different subscript letters, for instance using Greek letters for \is{subcategories of lexical tones}subcategories of
classifiers, and Latin letters for \is{subcategories of lexical tones}subcategories of verbs. But it seemed advisable to use Latin letters in all cases
to avoid an~extravagant profusion of symbols.

%The investigation brought out nine tonal categories for \is{monosyllables}monosyllabic classifiers. They were labelled by assuming that the tone that surfaces after the numerals ‘one' and ‘two' (which always have the same tonal behaviour) was indicative of the underlying category,\footnote{The argument for this choice is set out in \sectref{sec:howthetonalcategorieswerebroughtoutandlabelled}.} yielding four main categories: H, M, L, or MH.  These four main categories need to be split into two \is{subcategories of lexical tones}subcategories for H tone (H\textsubscript{a} and H\textsubscript{b}), two for MH tone (MH\textsubscript{a} and MH\textsubscript{b}), two for M tone (M\textsubscript{a} and M\textsubscript{b}), and three for L tone (L\textsubscript{a}, L\textsubscript{b} and L\textsubscript{c}). 

The nine categories (H\textsubscript{a}, H\textsubscript{b}, MH\textsubscript{a}, MH\textsubscript{b}, M\textsubscript{a}, M\textsubscript{b}, L\textsubscript{a}, L\textsubscript{b} and L\textsubscript{c}) are illustrated in \tabref{tab:oneexampleofeachofthetonalcategoriesofclassifiers}. 

\newpage 
\begin{table}[h]
	\caption{One example of each of the nine tonal categories of {monosyllabic} classifiers.}
	\begin{tabularx}{\textwidth}{ Q Q l }
		\lsptoprule
		classifier & tone & description: classifier for{\dots}\\ \midrule
		\ipa{ɖwæ˥} & H\textsubscript{a} & steps (of stairs)\\
		\ipa{ɲi˥} & H\textsubscript{b} & days\\
		\ipa{hɑ̃˧˥} & MH\textsubscript{a} & nights\\
		\ipa{kv̩˧˥} & MH\textsubscript{b} & people, persons\\
		\ipa{nɑ˧} & M\textsubscript{a} & tools\\
		\ipa{dzi˧} & M\textsubscript{b} & pairs of non"=separable objects, e.g.~shoes\\
		\ipa{dze˩} & L\textsubscript{a} & pairs of separable objects, e.g.~pots, bottles\\
		\ipa{dzi˩} & L\textsubscript{b} & trees, bamboos\\
		\ipa{ʐɤ˩} & L\textsubscript{c} & lines, patterns (in weaving or drawing)\\
		\lspbottomrule
	\end{tabularx}
	\label{tab:oneexampleofeachofthetonalcategoriesofclassifiers}
\end{table}


A summary of all the tone patterns which these nine categories of classifiers yield in association with numerals in the range from 1 to 100 is presented in
Tables~\ref{tab:1to100hmh}a--d and \ref{tab:1to100ml}a--d. These tables contain, in tightly packed form, the
information necessary to generate the surface phonological tone patterns of all
numeral"=plus"=classifier phrases in the Alawa dialect of Yongning Na. 

% To be checked on proofs: how to get all the figures exactly HERE.

\afterpage{
	\begin{subtables}
		\label{tab:1to100hmh}
		\begin{table}[p]
			\caption{\label{tab:1to25hmh}The underlying tone patterns of the nine categories of numeral"=plus"=classifier phrases. H and MH tones. Numerals from 1 to 25.}
			\begin{tabularx}{\textwidth}{ P{10mm} Q Q Q Q }
			\lsptoprule
				 & H\textsubscript{a} & H\textsubscript{b} & MH\textsubscript{a} & MH\textsubscript{b}\\\midrule
				1 & H\$ & H\$ & MH\# & MH\#\\
				2 & H\$ & H\$ & MH\# & MH\#\\
				3 & L & L & L & L\\
				4 & \#H & \#H & L\# & L\\
				5 & \#H & \#H & L\# & L\\
				6 & H\$ & H\$ & H\# & H\$\\
				7 & \#H & \#H & MH\# & MH\#\\
				8 & H\$ & H\$ & H\# & H\$\\
				9 & \#H & \#H & L\# & L\\
				10 & L & L & L & L\\
				11 & L & L & L & L\\
				12 & L & L & L & L\\
				13 & L & L & L & L\\
				14 & L+H\$ & L & L+H\# & L+H\#\\
				15 & L+H\$ & L & L+H\# & L+H\#\\
				16 & L+H\$ & L & L+H\# & L+H\#\\
				17 & L & L & L & L\\
				18 & L+H\$ & L & L+H\# & L+H\#\\
				19 & L+H\$ & L & L+H\# & L+H\#\\
				20 & \#H & \#H & MH\# & MH\#\\
				21 & H\$ & H\$ & MH\# & MH\#\\
				22 & H\$ & \#H & MH\# & MH\#\\
				23 & --L & --L & --L & --L\\
				24 & \#H & \#H & --L\# & --L\\
				25 & \#H & \#H & --L\# & --L\\
			\lspbottomrule
			\end{tabularx}
		\end{table}
		
		\begin{table}[p!]
			\caption{\label{tab:26to50hmh}The underlying tone patterns of the nine categories of numeral"=plus"=classifier phrases. H and MH tones. Numerals from 26 to 50.}
			\begin{tabularx}{\textwidth}{ P{10mm} Q Q Q Q }
			\lsptoprule
				 & H\textsubscript{a} & H\textsubscript{b} & MH\textsubscript{a} & MH\textsubscript{b}\\\midrule
				26 & H\$ & H\$ & H\# & H\$\\
				27 & \#H & \#H & MH\# & MH\#\\
				28 & H\$ & H\$ & H\# & H\$\\
				29 & \#H & \#H & --L\# & --L\\
				30 & \#H & \#H & MH\# & MH\#\\
				31 & H\$ & H\$ & MH\# & MH\#\\
				32 & H\$ & \#H & MH\# & MH\#\\
				33 & --L & --L & --L & --L\\
				34 & \#H & \#H & --L\# & --L\\
				35 & \#H & \#H & --L\# & --L\\
				36 & H\$ & H\$ & H\# & H\$\\
				37 & \#H & \#H & MH\# & MH\#\\
				38 & H\$ & H\$ & H\# & H\$\\
				39 & \#H/--L & \#H & --L\# & --L\\
				40 & L\#-- & L\#-- & L\#-- & L\#--\\
				41 & L\#--H\$ / L\#-- & L\#--H\$ / L\#-- & L\#--MH\# / L\#-- & L\#--MH\#\\
				42 & L\#--H\$ / L\#-- & L\#--\#H / L\#-- & L\#--MH\# / L\#-- & L\#--MH\#\\
				43 & L\#--L & L\#--L / L\#-- & L\#--L & L\#--L\\
				44 & L\#--\#H / L\#-- & L\#--\#H / L\#-- & L\#--L\# & L\#--L\\
				45 & L\#--\#H / L\#-- & L\#--\#H / L\#-- & L\#--L\# & L\#--L\\
				46 & L\#--H\$ & L\#--H\$ & L\#--H\# & L\#--H\$\\
				47 & L\#--\#H & L\#--\#H & L\#--MH\# & L\#--MH\#\\
				48 & L\#--H\$ & L\#--H\$ & L\#--H\# & L\#--H\$\\
				49 & L\#--\#H / L\#--L & L\#--\#H / L\#-- & L\#--L\# & L\#--L\\
				50 & L\#-- & L\#-- & L\#-- & L\#--\\
			\lspbottomrule
			\end{tabularx}
		\end{table}
		
		\begin{table}[p!]
		\caption{\label{tab:51to75hmh}The underlying tone patterns of the nine categories of numeral"=plus"=classifier phrases. H and MH tones. Numerals from 51 to 75.}
		\begin{tabularx}{\textwidth}{ P{10mm} l Q Q Q }
		\lsptoprule
			 & H\textsubscript{a} & H\textsubscript{b} & MH\textsubscript{a} & MH\textsubscript{b}\\\midrule
			51 & L\#--H\$ / L\#-- & L\#--H\$ / L\#-- & L\#--MH / L\#-- & L\#--MH\#\\
			52 & L\#--H\$ / L\#-- & L\#--\#H / L\#-- & L\#--MH / L\#-- & L\#--MH\#\\
			53 & L\#-- & L\#-- & L\#--L & L\#--L\\
			54 & L\#--\#H / L\#-- & L\#--\#H / L\#-- & L\#--L\# & L\#--L\\
			55 & L\#--\#H / L\#-- & L\#--\#H / L\#-- & L\#--L\# & L\#--L\\
			56 & L\#--H\$ & L\#--H\$ & L\#--H\# & L\#--H\$\\
			57 & L\#--\#H & L\#--\#H & L\#--MH\# & L\#--MH\#\\
			58 & L\#--H\$ & L\#--H\$ & L\#--H\# & L\#--H\$\\
			59 & L\#--\#H / L\#-- & L\#--\#H / L\#-- & L\#--L\# & L\#--L\\
			60 & LM--H\$ & LM--H\$ & LM--H\# & LM--\#H\\
			61 & LM--H\$ & LM--H\$ & LM--H\# & LM--\#H\\
			62 & LM--H\$ / LM--\#H & LM--\#H & LM--H\# & LM--H\#\\
			63 & LM--H\$ / LM--L & LM--H\$ / LM--L & LM--H\# & LM--H\#\\
			64 & LM--\#H & LM--\#H & LM--H\# & LM--H\#\\
			65 & LM--\#H & LM--\#H & LM--H\# & LM--H\#\\
			66 & LM--H\$ & LM--H\$ & LM--H\# & LM--H\#\\
			67 & LM--\#H & LM--\#H & LM--H\# & LM--H\#\\
			68 & LM--H\$ & LM--H\$ & LM--H\# & LM--H\#\\
			69 & LM--\#H / LM--L & LM--L & LM--H\# & LM--H\#\\
			70 & L\#-- & L\#-- & L\#-- & L\#--\\
			71 & L\#--H\$ / L\#-- & L\#--H\$ & L\#--MH\# / L\#-- & L\#--MH\#\\
			72 & L\#--H\$ / L\#-- & L\#--\#H & L\#--MH\# / L\#-- & L\#--MH\#\\
			73 & L\#--L & L\#--L & L\#--L & L\#--L\\
			74 & L\#--\#H & L\#--\#H & L\#--MH\# / L\#-- & L\#--L\\
			75 & L\#--\#H & L\#--\#H & L\#--MH\# / L\#-- & L\#--L\\
		\lspbottomrule
		\end{tabularx}
		\end{table}
		
		\begin{table}[p!]
		\caption{\label{tab:76to100hmh}The underlying tone patterns of the nine categories of numeral"=plus"=classifier phrases. H and MH tones. Numerals from 76 to 100.}
		\begin{tabularx}{\textwidth}{ P{10mm} Q Q Q Q }
		\lsptoprule
			 & H\textsubscript{a} & H\textsubscript{b} & MH\textsubscript{a} & MH\textsubscript{b}\\\midrule
			76 & L\#--H\$ & L\#--H\$ & L\#--H\# / L\#-- & L\#--H\#\\
			77 & L\#--\#H & L\#--\#H & L\#--MH\# / L\#-- & L\#--MH\#\\
			78 & L\#--H\$ & L\#--H\$ & L\#--H\# / L\#-- & L\#--H\#\\
			79 & L\#--\#H / L\#--L & L\#--\#H & L\#--L\# & L\#--L\\
			80 & LM--H\$ & LM--H\$ & LM--H\# & LM--H\#\\
			81 & LM--H\$ & LM--H\$ & LM--H\# & LM--H\#\\
			82 & LM--H\$ & LM--\#H & LM--H\# & LM--H\#\\
			83 & LM--H\$ & LM--H\$ / LM--L & LM--H\# & LM--H\#\\
			84 & LM--\#H & LM--H\$ & LM--H\# & LM--H\#\\
			85 & LM--\#H & LM--\#H & LM--H\# & LM--H\#\\
			86 & LM--H\$ & LM--H\$ & LM--H\# & LM--H\#\\
			87 & LM--\#H & LM--\#H & LM--H\# & LM--H\#\\
			88 & LM--H\$ & LM--H\$ & LM--H\# & LM--H\#\\
			89 & LM--\#H & LM--\#H & LM--H\# & LM--H\#\\
			90 & L\#-- & L\#-- & L\#-- & L\#--\\
			91 & L\#--H\$ & L\#--H\$ & L\#--MH\# / L\#-- & L\#--MH\# / L\#--\\
			92 & L\#--H\$ & L\#--\#H & L\#--MH\# / L\#-- & L\#--MH\# / L\#--\\
			93 & L\#--L & L\#--L & L\#--L & L\#--L\\
			94 & L\#--\#H & L\#--\#H & L\#--L\# / L\#-- & L\#--L\\
			95 & L\#--\#H & L\#--\#H & L\#--L\# / L\#-- & L\#--L\\
			96 & L\#--H\$ & L\#--H\$ & L\#--H\# / L\#-- & L\#--H\# / L\#--\\
			97 & L\#--\#H & L\#--\#H & L\#--MH\# / L\#-- & L\#--MH\# / L\#--\\
			98 & L\#--H\$ & L\#--H\$ & L\#--H\# / L\#-- & L\#--H\# / L\#--\\
			99 & L\#--L & L\#--\#H & L\#--L\# / L\#-- & L\#--L\\
			100 & \#H & \#H & MH\# & MH\#\\
		\lspbottomrule
		\end{tabularx}
		\end{table}
	\end{subtables}
	
	
	\begin{subtables}
		\label{tab:1to100ml}
		\begin{table}[h!!]
			\caption{\label{tab:1to25ml}The underlying tone patterns of the nine categories of numeral"=plus"=classifier phrases. M and L tones. Numerals from 1 to 25.}
			\begin{tabularx}{\textwidth}{ P{10mm} Q Q Q Q Q }
			\lsptoprule
				 & M\textsubscript{a} & M\textsubscript{b} & L\textsubscript{a} & L\textsubscript{b} & L\textsubscript{c}\\\midrule
				1 & M & M & L\# & L\# & L\#\\
				2 & M & M & L\# & L\# & L\#\\
				3 & M & M & L & M & M\\
				4 & L & L & H\# & H\# & H\#\\
				5 & L & L & H\# & H\# & H\#\\
				6 & H\# & H\$ & H\# & H\# & H\#\\
				7 & M & M & L\# & L\# & L\#\\
				8 & H\# & H\$ & H\# & H\# & H\#\\
				9 & L & L & H\# & H\# / L\# & H\#\\
				10 & M & M & L & M & L\\
				11 & M & M & L\# & L\# & L\#\\
				12 & M & M & L\# & L\# & L\#\\
				13 & M & M & L\# & L\# & L\#\\
				14 & L+H\# & L & L+H\# & L+H\# & L+H\#\\
				15 & L+H\# & L & L+H\# & L+H\# & L+H\#\\
				16 & L+H\# & L & L+H\# & L+H\# & L+H\#\\
				17 & M & M & L\# & L\# & L\#\\
				18 & L+H\# & L & L+H\# & L+H\# & L+H\#\\
				19 & L+H\# & L & L+H\# & L+H\# & L+H\#\\
				20 & M & M & --L & --L & --L\\
				21 & M & M & --L\# & --L\# & --L\#\\
				22 & M & M & --L\# & --L\# & --L\#\\
				23 & --L / M & M / --L & --L & --L & --L\\
				24 & --L & --L & H\# & H\# & H\#\\
				25 & --L & --L & H\# & H\# & H\#\\
			\lspbottomrule
			\end{tabularx}
		\end{table}
		
		\begin{table}[h!!]
			\caption{\label{tab:26to50ml}The underlying tone patterns of the nine categories of numeral"=plus"=classifier phrases. M and L tones. Numerals from 26 to 50.}
			{\fontsize{10}{11}\selectfont
				\begin{tabularx}{\textwidth}{ P{10mm} Q Q Q Q Q }
					\lsptoprule
						 & M\textsubscript{a} & M\textsubscript{b} & L\textsubscript{a} & L\textsubscript{b} & L\textsubscript{c}\\\midrule
						26 & H\# & H\$ & H\# & H\# & H\#\\
						27 & M & M & --L\# & --L\# & --L\#\\
						28 & H\# & H\$ & H\# & H\# & H\#\\
						29 & --L & --L & H\# & --L\# / H\# & H\#\\
						30 & M & M & --L\# & --L & --L\#\\
						31 & M & M & --L\# & --L\# & --L\#\\
						32 & M & M & --L\# & --L\# & --L\#\\
						33 & --L & --L & --L & --L & --L\\
						34 & --L & --L & H\# & H\# & H\#\\
						35 & --L & --L & H\# & H\# & H\#\\
						36 & H\# & H\$ & H\# & H\# & H\#\\
						37 & M & L & --L\# & --L\# & --L\#\\
						38 & H\# & H\$ & H\# & H\# & H\#\\
						39 & --L & --L & H\# / --L\# & H\# / --L\# & H\# / --L\#\\
						40 & L\#-- & L\#-- & L\#-- & L\#-- & L\#--\\
						41 & L\#-- & L\#--M / L\#-- & L\#--L\# / L\#-- & L\#--L\# / L\#-- & L\#--L\# / L\#--\\
						42 & L\#-- & L\#--M / L\#-- & L\#--L\# / L\#-- & L\#--L\# / L\#-- & L\#--L\# / L\#--\\
						43 & L\#-- & L\#--M / L\#-- & L\#--M / L\#--L & L\#--M / L\#-- & L\#--M / L\#--\\
						44 & L\#--H\# / L\#-- & L\#-- & L\#--H\# / L\#-- & L\#--H\# / L\#-- & L\#--H\# / L\#--\\
						45 & L\#--H\# / L\#-- & L\#-- & L\#--H\# / L\#-- & L\#--H\# / L\#-- & L\#--H\# / L\#--\\
						46 & L\#--H\# / L\#-- & L\#--H\$ / L\#-- & L\#--H\# / L\#-- & L\#--H\# / L\#-- & L\#--H\# / L\#--\\
						47 & L\#--M / L\#-- & L\#--M / L\#-- & L\#--L\# / L\#-- & L\#--L\# / L\#-- & L\#--L\# / L\#--\\
						48 & L\#--H\# / L\#-- & L\#--H\$ / L\#-- & L\#--H\# / L\#-- & L\#--H\# / L\#-- & L\#--H\# / L\#--\\
						49 & L\#--H\# / L\#-- & L\#--L & L\#--H\# / L\#-- & L\#--H\# / L\#-- & L\#--H\# / L\#--\\
						50 & L\#-- & L\#-- & L\#-- & L\#-- & L\#--\\
					\lspbottomrule
				\end{tabularx}
			}
		\end{table}
		
		\begin{table}[h!!]
			\caption{\label{tab:51to75ml}The underlying tone patterns of the nine categories of numeral"=plus"=classifier phrases. M and L tones. Numerals from 51 to 75.}
			{\fontsize{10}{11}\selectfont
				\begin{tabularx}{\textwidth}{ P{5mm} Q Q Q Q Q }
				\lsptoprule
					 & M\textsubscript{a} & M\textsubscript{b} & L\textsubscript{a} & L\textsubscript{b} & L\textsubscript{c}\\\midrule
					51 & L\#-- & L\#--M / L\#-- & L\#--L\# / L\#-- & L\#--L\# / L\#-- & L\#--L\# / L\#--\\
					52 & L\#-- & L\#--M / L\#-- & L\#--L\# / L\#-- & L\#--L\# / L\#-- & L\#--L\# / L\#--\\
					53 & L\#-- & L\#--M / L\#-- & L\#--M / L\#--L & L\#--M / L\#-- & L\#--M / L\#--\\
					54 & L\#--H\# / L\#-- & L\#-- & L\#--H\# / L\#-- & L\#--H\# / L\#-- & L\#--H\# / L\#--\\
					55 & L\#--H\# / L\#-- & L\#-- & L\#--H\# / L\#-- & L\#--H\# / L\#-- & L\#--H\# / L\#--\\
					56 & L\#--H\# / L\#-- & L\#--H\$ / L\#-- & L\#--H\# / L\#-- & L\#--H\# / L\#-- & L\#--H\# / L\#--\\
					57 & L\#--M / L\#-- & L\#--M / L\#-- & L\#--L\# / L\#-- & L\#--L\# / L\#-- & L\#--L\# / L\#--\\
					58 & L\#--H\# / L\#-- & L\#--H\$ / L\#-- & L\#--H\# / L\#-- & L\#--H\# / L\#-- & L\#--H\# / L\#--\\
					59 & L\#--H\# / L\#-- & L\#--L & L\#--H\# / L\#-- & L\#--H\# / L\#-- & L\#--H\# / L\#--\\
					60 & LM--H\# & LM--H\$ & LM--H\# & LM--H\# & LM--H\#\\
					61 & LM--H\# / L+MH\#--M & LM--H\$ & LM--H\# & LM--H\# & LM--H\#\\
					62 & LM--H\# / L+MH\#--M & LM--H\$ & LM--H\# & LM--H\# & LM--H\#\\
					63 & LM--H\# / L+MH\#--M & LM--H\$ & LM--H\# & LM--H\# & LM--H\#\\
					64 & LM--H\# / L+MH\#--L & LM--H\$ & LM--H\# & LM--H\# & LM--H\#\\
					65 & LM--H\# / L+MH\#--L & LM--H\$ & LM--H\# & LM--H\# & LM--H\#\\
					66 & LM--H\# & LM--H\$ & LM--H\# & LM--H\# & LM--H\#\\
					67 & LM--H\# / L+MH\#--M & LM--\#H & LM--H\# & LM--H\# & LM--H\#\\
					68 & LM--H\# & LM--H\$ & LM--H\# & LM--H\# & LM--H\#\\
					69 & LM--H\# / L+MH\#--L & LM--H\$ / LM--L & LM--H\# & LM--H\# & LM--H\#\\
					70 & L\#-- & L\#-- & L\#-- & L\#-- & L\#--\\
					71 & L\#-- / L\#--M & L\#--M / L\#-- & L\#--L\# / L\#-- & L\#--L\# / L\#-- & L\#--L\# / L\#--\\
					72 & L\#-- / L\#--M & L\#--M / L\#-- & L\#--L\# / L\#-- & L\#--L\# / L\#-- & L\#--L\# / L\#--\\
					73 & L\#-- / L\#--M & L\#--M / L\#-- & L\#--L & L\#--M / L\#-- & L\#--M / L\#--\\
					74 & L\#-- / L\#--H\# & L\#--L & L\#--H\# / L\#-- & L\#--H\# / L\#-- & L\#--H\# / L\#--\\
					75 & L\#-- / L\#--H\# & L\#--L & L\#--H\# / L\#-- & L\#--H\# / L\#-- & L\#--H\# / L\#--\\
				\lspbottomrule
				\end{tabularx}
			}
		\end{table}
		
		\begin{table}[h!!]
			\caption{\label{tab:75to100ml}The underlying tone patterns of the nine categories of numeral"=plus"=classifier phrases. M and L tones. Numerals from 75 to 100.}
			{\setlength\tabcolsep{4.5pt}
			{\fontsize{10}{11}\selectfont
				\begin{tabularx}{\textwidth}{ l P{22mm} Q Q Q Q }
				\lsptoprule
					 & M\textsubscript{a} & M\textsubscript{b} & L\textsubscript{a} & L\textsubscript{b} & L\textsubscript{c}\\\midrule
					76 & L\#--H\# / L\#-- & L\#--H\$ / L\#-- & L\#--H\# / L\#-- & L\#--H\# / L\#-- & L\#--H\# / L\#--\\
					77 & L\#--M / L\#-- & L\#--M / L\#-- & L\#--L\# / L\#-- & L\#--L\# / L\#-- & L\#--L\# / L\#--\\
					78 & L\#--H\# / L\#-- & L\#--H\$ / L\#-- & L\#--H\# / L\#-- & L\#--H\# / L\#-- & L\#--H\# / L\#--\\
					79 & L\#--L / L\#--H\# & L\#--L & L\#--H\# / L\#-- & L\#--H\# / L\#-- & L\#--H\# / L\#--\\
					80 & LM--H\# & LM--H\$ & LM--H\# & LM--H\# & LM--H\#\\
					81 & LM--H\# / L+MH\#--M & LM--H\$ & LM--H\# & LM--H\# & LM--H\#\\
					82 & LM--H\# / L+MH\#--M & LM--H\$ & LM--H\# & LM--H\# & LM--H\#\\
					83 & LM--H\# / L+MH\#--M & LM--H\$ & LM--H\# & LM--H\# & LM--H\#\\
					84 & LM--H\# / L+MH\#--L & LM--H\$ & LM--H\# & LM--H\# & LM--H\#\\
					85 & LM--H\# / L+MH\#--L & LM--H\$ & LM--H\# & LM--H\# & LM--H\#\\
					86 & LM--H\# & LM--H\$ & LM--H\# & LM--H\# & LM--H\#\\
					87 & LM--H\# / L+MH\#--M & LM--\#H & LM--H\# & LM--H\# & LM--H\#\\
					88 & LM--H\# & LM--H\$ & LM--H\# & LM--H\# & LM--H\#\\
					89 & LM--H\# / L+MH\#--L & LM--H\$ / LM--L & LM--H\# & LM--H\# & LM--H\#\\
					90 & L\#-- / L\#-- & L\#-- & L\#-- & L\#-- & L\#--\\
					91 & L\#--M / L\#-- & L\#--M / L\#-- & L\#--L\# / L\#-- & L\#--L\# / L\#-- & L\#--L\#\\
					92 & L\#--M / L\#-- & L\#--M / L\#-- & L\#--L\# / L\#-- & L\#--L\# / L\#-- & L\#--L\#\\
					93 & L\#--M / L\#-- & L\#--M / L\#-- & L\#--M / L\#-- & L\#--M / L\#-- & L\#--M / L\#--\\
					94 & L\#--L / L\#--H\# & L\#-- & L\#--H\# / L\#-- & L\#--H\# / L\#-- & L\#--H\#\\
					95 & L\#--L / L\#--H\# & L\#-- & L\#--H\# / L\#-- & L\#--H\# / L\#-- & L\#--H\#\\
					96 & L\#--H\# / L\#-- & L\#--H\$ / L\#-- & L\#--H\# & L\#--H\# / L\#-- & L\#--H\#\\
					97 & L\#--M / L\#-- & L\#--M / L\#-- & L\#--L\# / L\#-- & L\#--L\# / L\#-- & L\#--L\#\\
					98 & L\#--H\# / L\#-- & L\#--H\$ / L\#-- & L\#--H\# & L\#--H\# / L\#-- & L\#--H\#\\
					99 & L\#--L / L\#--H\# & L\#--L & L\#--H\# & L\#--H\# / L\#-- & L\#--H\# / L\#\\
					100 & M & M & L\# & L\# & L\#\\
				\lspbottomrule
				\end{tabularx}}
			}
		\end{table}
	\end{subtables}
	\clearpage
}

   
The mass of information presented in Tables~\ref{tab:1to100hmh}a--d and \ref{tab:1to100ml}a--d may
seem staggering. Were it not for the clear evidence from recorded data, one could suspect that this
multiplicity of tone patterns was an~artefact of elicitation procedures.

{\largerpage}

Variants are separated by a~slash (/). For typographical reasons, the table is divided into two
halves: the H and MH tones (H\textsubscript{a}, H\textsubscript{b}, MH\textsubscript{a} and MH\textsubscript{b}) in a~first table (\ref{tab:1to100hmh}), and the M and L tones (M\textsubscript{a}, M\textsubscript{b},
L\textsubscript{a}, L\textsubscript{b} and L\textsubscript{c}) in a~second table (\ref{tab:1to100ml}).

Numeral"=plus"=classifier phrases typically constitute one single \isi{tone group}\footnote{The \textit{tone group} is the unit within which tonal processes apply in Yongning Na. It could also be referred to as \textit{phonological phrase}. Chapter~\ref{chap:toneassignmentrulesandthedivisionoftheutteranceintotonegroups} is devoted to this morphotonological unit, which is fundamental to Na {prosody}.}~-- although
speakers can choose to split them into two groups for expressive (emphatic) purposes, as will be
discussed in Chapter~\ref{chap:toneassignmentrulesandthedivisionoftheutteranceintotonegroups}. The \is{juncture (inside a tone group)}juncture indicated by ‘--’ does
not separate the phrase into two tone groups, but into two parts, corresponding to tens on the one hand, units and the classifier on the other. This \is{juncture (inside a tone group)}juncture is found after the first two syllables, in phrases containing numerals above twenty. These phrases consist of two syllables referring to tens (‘two-ten’ for ‘twenty’, ‘three-ten’ for ‘thirty’, and so on) followed by the last digit followed by the classifier (except for the round figures ‘twenty’, ‘thirty’ and so on: no zero is indicated, and the classifier follows directly, as in /\ipa{ʐv̩˧tsʰi˩-kʰv̩˩}/ ‘forty years’). For instance, /\ipa{so˧tsʰi˧so˩-ɲi˩}/ ‘33 days’ can be represented as /\ipa{so˧tsʰi˧--so˩-ɲi˩}/ to show the \is{juncture (inside a tone group)}juncture between the two parts (glosses for the four syllables: ‘three~"=ten~--~three~"=~\textsc{clf}.days’). The items in Tables~\ref{tab:1to100hmh}a--d and \ref{tab:1to100ml}a--d that begin with ‘--’ do not have any specified tone on their first part; that
part receives a~Mid tone by default. Using again the phrase ‘33 days’ as an example, it has a~--L tone pattern: a~Low tone
that associates after the \is{juncture (inside a tone group)}juncture, yielding /{\dots}\ipa{so˩-ɲi˩}/. Since Low tones do not spread
regressively (‘right"=to"=left’), the first part of the phrase receives M tone by default (/\ipa{so˧tsʰi˧}{\dots}/),
resulting in the final output /\ipa{so˧tsʰi˧so˩-ɲi˩}/.

Likewise, the items whose tone pattern ends with ‘--’ do not have any specified tone on their second part. That
part receives its tones by the application of the phonological tone rules that govern tonal groups in
the Alawa dialect of Yongning Na. For instance, ‘forty years’ has the tone pattern L\#--: L\# tone (a final L tone)
associates to the first half of the phrase, yielding /\ipa{ʐv̩.tsʰi˩}{\dots}/. As already mentioned, tones do not spread leftward; the first syllable
receives a~Mid tone by default, hence /\ipa{ʐv̩˧tsʰi˩}{\dots}/. At this point a~phonological tonal rule applies (see \sectref{sec:alistoftonerules}):
Rule~5, “All syllables following a~H.L or M.L sequence receive L tone”. This amounts to saying that
a~tone cannot be surrounded by higher tones within a~\isi{tone group}: there are no |~MLM~|
sequences, nor are there |~HMH~|, |~HLM~|, |~MLMH~|, and so on. The only possible tone on the second
part of the phrase is therefore L. The final output is /\ipa{ʐv̩˧tsʰi˩-kʰv̩˩}/, as shown in \figref{fig:tone40years}.

\begin{figure}[h!!]
	\caption[{A detailed representation of tone"=to"=syllable association for ‘forty years'.}]{A detailed representation of tone"=to"=syllable association for the numeral"=plus"=classifier phrase /\ipa{ʐv̩˧tsʰi˩-kʰv̩˩}/ ‘forty years'.}
	\begin{tikzpicture}
	\node (1) at (0.5,-1.5) {L\#};
	\node (4) at (3.5,-1.5) {MH\textsubscript{a}};
%	\node (4) at (3.5,-0.5) {MH\#};
	
	\node (2) at (0,-2.5) {σ};
	\node (3) at (1,-2.5) {σ};
	\node (5) at (3.5,-2.5) {σ};
	
	\node [anchor=mid] (s1l) at (0.5,-3) {/\ipa{ʐv̩.tsʰi}/ ‘forty’};
	%  \node (s1ll) at (0.5,-2.5) {lexical tone: MH\#};
	
	\node [anchor=mid] (s1lll) at (3.5,-3) {/\ipa{kʰv̩}/ ‘year’};
	%	\node [anchor=mid] (s1lll) at (3,-2) {/\ipa{bv̩}/ \textsc{poss}};
	%  \node (s1llll) at (4,-2.5) {lexical tone: L};
	
%	\node[text width=40mm] (s1) at (-3,-0.75) {Stage 1:\\ input};
	\node[text width=50mm] (s1) at (-3,-1.75) {Stage 1:\\ input};

	\node (12) at (0.5,-4) {L\#};
%	\node (42) at (2,-4) {MH\#};
	
	\node (22) at (0,-5.5) {σ};
	\node (32) at (1,-5.5) {σ};
	\node (52) at (2,-5.5) {σ};
	
	\node[text width=50mm] (s2) at (-3,-4.75) {Stage 2:\\ \is{anchorage}anchoring of the phrase's\\ L\#-- tone (see \tabref{tab:26to50hmh})\\ to its
		phonologically\\ specified locus};
	
	\draw[decoration={markings,mark=at position 1 with
		{\arrow[scale=2,>=stealth]{>}}},postaction={decorate}] (12) -- (32);
	
	%	\draw[decoration={markings,mark=at position 1 with {\arrow[scale=2,>=stealth]{>}}},postaction={decorate}] (42) -- (52);
	
	
%	
%	\node (13) at (1,-7) {L};
%	%	\node (63) at (1.5,-7) {H};
%	\node (43) at (2,-7) {MH\#};
%	
%	\node (23) at (0,-8.5) {σ};
%	\node (33) at (1,-8.5) {σ};
%	\node (53) at (2,-8.5) {σ};
%	
%	\node[text width=40mm] (s3) at (-3,-7.75) {Stage 3:\\ one-to-one mapping\\ of levels to available syllables};
%	
%	\draw[decoration={markings,mark=at position 1 with
%		{\arrow[scale=2,>=stealth]{>}}},postaction={decorate}] (13) -- (33);
%	%	\draw[decoration={markings,mark=at position 1 with {\arrow[scale=2,>=stealth]{>}}},postaction={decorate}] (43) -- (53);
%	
	
	\node (14) at (0,-7) {M};
	\node (64) at (1,-7) {L};
%	\node (44) at (2,-7) {MH\#};
	
	\node (24) at (0,-8.5) {σ};
	\node (34) at (1,-8.5) {σ};
	\node (54) at (2,-8.5) {σ};
	
	\node[text width=50mm] (s4) at (-3,-7.5) {Stage 3:\\ addition of default\\ M tone};
	
	\draw[decoration={markings,mark=at position 1 with
		{\arrow[scale=2,>=stealth]{>}}},postaction={decorate}] (14) -- (24);
	\draw (64) -- (34);
	%	\draw (44) -- (54);
	
	
	\node (14) at (0,-10) {M};
	\node (64) at (1,-10) {L};
	\node (44) at (2,-10) {L};
	
	\node (24) at (0,-11.5) {σ};
	\node (34) at (1,-11.5) {σ};
	\node (54) at (2,-11.5) {σ};
	
	\node[text width=50mm] (s4) at (-3,-10.5) {Stage 4:\\ assignment of L tone\\ by {phonological rule}:\\ M.L can only be followed\\ by L};
	
	\draw (14) -- (24);
	\draw (64) -- (34);
%	\draw (44) -- (54);
	
		\draw[decoration={markings,mark=at position 1 with
			{\arrow[scale=2,>=stealth]{>}}},postaction={decorate}] (44) -- (54);
		
	\end{tikzpicture}
	\label{fig:tone40years}
\end{figure}

\subsection[The tones of classifiers and of corresponding free forms]{Why little evidence about the tones of classifiers can be gleaned from the free forms in which they originate}
\label{sec:nohelpfromFullForms}

In what precedes, the tones of classifiers were analyzed on the basis of synchronic distributional properties. Mention needs to be made of other methods, and why they have not provided decisive evidence so far on the phonological nature of the tonal categories of classifiers.

Relevant evidence could, in principle, come from those
classifiers that correspond transparently to a~free form: a~noun or a~verb. For example, the classifier for blows is /\ipa{dɑ˧˥}/, a~cognate object of the verb /\ipa{dɑ˧˥}/ ‘to hit, to strike’, as exemplified in (\ref{ex:strikeablow}). 

\Hack{\newpage}

\begin{exe}
	\ex
	\label{ex:strikeablow}
	\ipaex{ɖɯ˧-dɑ˧ tʰi˥-dɑ˩}\\
	\gll ɖɯ˧	dɑ˧˥	tʰi˧-	dɑ˧˥\\
	one	\textsc{clf}.blows			\textsc{dur}		to\_strike\\
	\glt ‘to strike blows’ (Sister3.135)
\end{exe}

The tonal {correspondence} with the verb looks transparent: both the verb and its grammaticalized form as a~classifier carry MH tone. Similarly, ‘mountain, hill’ is /\ipa{ʁwɤ˧}/, and as a~classifier it yields /\ipa{ʁwɤ˧}/
‘heap (of something)’, which has the same phonological form as the free noun, including its M tone. ‘Beam’, /\ipa{ɖʐo˥}/, is also identical in form to its
self"=classifier /\ipa{ɖʐo˥}/. Taking these three examples, it seems as if the tone category of a~classifier
were identical to that of the corresponding free form.

\newpage 
Other examples, however, do not exhibit such simple correspondences. A~different tonal
{correspondence} for H-tone nouns is exemplified by /\ipa{kɯ˥}/ ‘star’, which yields /\ipa{kɯ˧}/ (M\textsubscript{b}
category) as a~self"=classifier. ‘Bowl’ is /\ipa{qʰwɤ˩˧}/ (LM tone), whereas as a~classifier it
yields /\ipa{qʰwɤ˧˥}/ ‘bowlful’ (MH\textsubscript{a} category). To date, too few classifiers can be straightforwardly related to full nouns (or verbs) for the search of tonal correspondences between lexical word and classifier to be fruitful. 

%\subsubsection{Borrowed classifiers}
\label{sec:nohelpfromLoans}

Classifiers borrowed from \ili{Mandarin} constitute another potential source of evidence on the tones of Yongning Na classifiers. For instance, it can be assumed that the tone \is{subcategories of lexical tones}subcategories used to accommodate recent borrowings serve in synchrony as unmarked, default categories: H\textsubscript{a} or H\textsubscript{b}, MH\textsubscript{a} or MH\textsubscript{b}, and so on. However, only
one \is{loanwords}borrowing was observed: /\ipa{tɕi˧˥}/ for ‘pound (weight unit)’, from \ili{Mandarin} \textit{jīn} \zh{斤}. Its tonal category
is MH\textsubscript{a}, i.e.\ the majority category for classifiers with a~MH tone, but it would be unreasonable to
draw general conclusions from this isolated example. The way forward here would consist in a~more thorough study of Chinese loanwords than has been conducted so far. 


\subsection[Indirect confirmation for the H, MH, M and L categories]{Indirect confirmation for the H, MH, M and L categories of classifiers from frequency in surface forms}
\label{sec:indirectsupport}

The frequency of appearance of the various tonal levels provides support (albeit weak and indirect) for the four tonal ‘super"=categories’ of classifiers proposed here, namely H, MH, M and L. Under the (admittedly simplistic) assumption that
the contribution made by the tone of the classifier will be reflected statistically in the tone
patterns of numeral"=plus"=classifier phrases, the labels H, MH, M and L adopted here make good sense. Averaging over the entire range of tone patterns from
the number 1 to the number 100, the categories that have a~High tone after ‘1’ and ‘2’ (labelled H\textsubscript{a}
and H\textsubscript{b} in Tables~\ref{tab:1to100hmh}a--d and \ref{tab:1to100ml}a--d) are also those with the highest proportion of H tones (either on their own: H\#,
\#H, H\$, or as part of a~MH or MH\# \is{tonal contour}contour) and the lowest proportion of L tones. Conversely, the three tonal
categories of classifiers that have a~Low tone after ‘1’ and ‘2’ (analyzed phonologically as L\textsubscript{a}, L\textsubscript{b} and L\textsubscript{c}, respectively) have the
lowest proportion of H tones and the highest of L tones. 
The other two categories (with subgroups M\textsubscript{a} and M\textsubscript{b}, and MH\textsubscript{a}
and MH\textsubscript{b}) are between these two extremes. Again as expected, M\textsubscript{a} and M\textsubscript{b} have a~higher proportion of M
tones, and a~lower proportion of H tones, than MH\textsubscript{a} and MH\textsubscript{b}. These rule"=of"=thumb comparisons, which
do not carry {demonstrative} value, are simply mentioned to convey a~feel for the overall outlook of
the data. 

Another indirect way of approaching this data consists of examining the occasional
mistakes made by the consultant.


\subsection[Degree of tonal complexity and amount of mistakes]{The amount of mistakes in the recordings correlates with the degree of tonal complexity}
\label{sec:aboutmistakenrealizationsintherecordings}

\is{complexity}
\is{mistakes|textbf}

As was mentioned earlier (\sectref{sec:elicitationprocedures}), the task of realizing long series of numeral"=plus"=classifier phrases
was challenging for the consultant. Among the 2,810 tokens, 7\% have a~mistaken tone
pattern,\footnote{This figure includes some items that were deleted from the sound files at an~early
  stage of the study, before the principle of preserving the recordings unchanged was adopted.} i.e.\ a~tone pattern which the consultant (F4) consistently judged to be incorrect (a tonal slip of the
tongue) when we returned to the data after recording sessions.

These mistakes reflect in part the phonological complexity of the tone patterns at issue. The notion
that mistakes provide insights about language dates back at least to Henri Frei’s
\textit{Grammar of mistakes} (\citeyear{frei1929}); see also \citet{fromkin1973},
\citet{rossietal1998} and \citet{nooteboom2011}. The usefulness of speech errors and word games to gain
insights into tonal systems was shown by \citet[180–181]{hombert1986b}. Speakers of \ili{Mandarin},
Cantonese, Minnan and \ili{Thai} tend to move the tones with the syllables when changing
a~C\textsubscript{1}V\textsubscript{1}C\textsubscript{2}V\textsubscript{2} sequence into
C\textsubscript{1}V\textsubscript{2}C\textsubscript{2}V\textsubscript{1} or
C\textsubscript{2}V\textsubscript{2}C\textsubscript{1}V\textsubscript{1}, whereas speakers of
Bakwiri (also known as Bakweri; \ili{Bantu} branch of Niger"=Congo) tend to leave tone patterns unchanged. (See also
\citet{wanetal1998} on {Mandarin}.)

\figref{fig:numberofmistakesintherecordednumeralplusclassifierphrasesasafactoroftherangeoftens} shows the distribution of mistakes as a~factor of the range of tens: how many of the
observed \is{mistakes}mistakes concern numerals between 1 and 9 (leftmost bar), 10 and 19 (second bar),
etc. The ranges of numerals between 50 and 59, and between 80 and 89, are not represented: they were not included in the recordings (to reduce the consultant's fatigue), because the phonological tone patterns for 50--59 are always identical with those for 40--49, and those for 80--89 with 60--69.

\figref{fig:numberofmistakesintherecordednumeralplusclassifierphrasesasafunctionofthelastdigitunits} shows the distribution of \is{mistakes}mistakes as a~factor of the last digit (units): how many
mistakes concern numbers ending in ‘1’ (namely 1, 11, 21, 31, 41, and so on), in ‘2’, etc.

\begin{figure}%[t!]
  \caption{Number of mistakes in the recorded numeral"=plus"=classifier
    phrases as a~factor of the range of tens.}
\begin{tikzpicture}
  \begin{axis}[
      width=\textwidth,
      bar width=8mm,
      height=50mm,
      ymajorgrids,
      ylabel
      near ticks,
      xlabel near ticks,
      ylabel={Number of mistakes},
      xlabel={Range of numerals},
      tick pos=left,
      ymin=0,
      ymax=34,
      ytick={0,5,10,15,20,25,30},
      yticklabels={0,5,10,15,20,25,30},
      symbolic x coords={1, 10, 20, 30, 40, 60, 70, 90},
      %x tick label style  = {text width=1cm,align=center},
      xtick={1,10,20,30,40,60,70,90},
      xticklabels={1--9, 10--19, 20--29, 30--39, 40--49, 60--69, 70--79,
        90--99}]
    \pgfplotsset{ytick style={draw=none}}
    \pgfplotsset{major grid style={dashed}}
    \pgfplotsset{every x tick label/.append style={font=\scriptsize}}
    \pgfplotsset{every y tick label/.append style={font=\scriptsize}}
    \addplot[ybar,fill=lsRichGreen] coordinates { (1, 6) (10, 5) (20, 33) (30, 32) (40, 20) (60, 16)
      (70, 26) (90, 18) };
  \end{axis}
\end{tikzpicture}
\label{fig:numberofmistakesintherecordednumeralplusclassifierphrasesasafactoroftherangeoftens}
\end{figure}


\begin{figure}%[t!]
  \caption{Number of mistakes in the recorded numeral"=plus"=classifier phrases as a~function of the
    last digit (units).}
\begin{tikzpicture}
  \begin{axis}[
      ymin=0,
      width=\textwidth,
      bar width=6mm,
      height=50mm,
      ymajorgrids,
      ylabel near ticks,
      xlabel near ticks,
      ylabel={Number of mistakes},
      xlabel={Last digit of numeral},
      tick pos=left,
      ymin=0,
      ymax=34,
      ytick={0,5,10,15,20,25,30},
      yticklabels={0,5,10,15,20,25,30},
      symbolic x coords={0, 1, 2, 3, 4, 5, 6, 7, 8, 9},]
    \pgfplotsset{xtick style={draw=none}}
    \pgfplotsset{ytick style={draw=none}}
    \pgfplotsset{major grid style={dashed}}
    \pgfplotsset{every x tick label/.append style={font=\scriptsize}}
    \pgfplotsset{every y tick label/.append style={font=\scriptsize}}
    \addplot[ybar,fill=lsRichGreen] coordinates { (0, 5) (1, 14) (2, 25) (3, 10) (4, 25) (5,20) (6,
      1) (7, 23) (8, 3) (9, 30) };
  \end{axis}
\end{tikzpicture} 
\label{fig:numberofmistakesintherecordednumeralplusclassifierphrasesasafunctionofthelastdigitunits}
\end{figure}

  
The data is not symmetrical enough for a~full"=fledged statistical treatment. In particular,
(i)~there is more data for some tonal categories than others, (ii)~some combinations have several
repetitions, and (iii)~there is slightly more data in the range [1..10] than for higher numerals. Some observations can nonetheless be made. The numerals ending in ‘6’ and ‘8’ (e.g.~‘6’, ‘16’, ‘26’; ‘8’, ‘18’, ‘28’) are noticeably less subject to mistakes, and those beginning with these same numbers (e.g.~‘60’, ‘61’, ‘62’, ‘63’, ‘80’, ‘81’, ‘82’, ‘83’) are also slightly less often mispronounced than neighbouring “numeral-runs”. Importantly, numerals ending in ‘6’ and ‘8’ are also the least complex in tonal terms, due to the \isi{neutralization} of many tonal distinctions after these two numerals. By contrast, numeral"=plus"=classifier phrases containing numbers ending in ‘7’ and ‘9’, which display the greatest diversity in tone patterns, are among the most frequently mistaken, although ‘7’ does not stand out within the top half of the list of mistakes, which includes \{2, 4, 5, 7, 9\}.


\subsection{About variants of tone patterns}
\label{sec:aboutvariantsoftonepatterns}
\largerpage
As noted at the outset of this chapter, some phrases allow for \is{variants}variant tone patterns. This situation is more frequent for higher numerals than for numerals below twenty. For instance, the association of ‘47’ with the classifier for round objects can be realized either as
/\ipa{ʐv̩˧tsʰi˩--ʂɯ˧-ɭɯ˥}/ or as /\ipa{ʐv̩˧tsʰi˩--ʂɯ˩-ɭɯ˩}/. No observed numeral"=plus"=classifier phrase has more than two acceptable variants for its tone
pattern. 

Many of the variants can be explained in light of the phonological rules which hold within a~tone
group (see \sectref{sec:alistoftonerules}). If the numeral"=plus"=classifier phrase is treated as one \isi{tone group},
then these rules apply. For instance, the numeral ‘44’ in association with the classifier for tools, /\ipa{nɑ˧\textsubscript{a}}/, allows for the following two variants: /\ipa{ʐv̩˧tsʰi˩--ʐv̩˧-nɑ˥}/ and /\ipa{ʐv̩˧tsʰi˩--ʐv̩˩-nɑ˩}/. The first of these phrases, /\ipa{ʐv̩˧tsʰi˩--ʐv̩˧-nɑ˥}/, is not a~well"=formed \isi{tone group}, since it does not obey Rule~5: inside a \isi{tone group}, all syllables following a~M.L sequence receive L tone. The expression /\ipa{ʐv̩˧tsʰi˩--ʐv̩˧-nɑ˥}/ is therefore to be analyzed as consisting of two tone groups: /\ipa{ʐv̩˧tsʰi˩ {\kern2pt}|{\kern2pt} ʐv̩˧-nɑ˥}/ (tone
pattern: L\# for the first tone group, and H\# for the second). If this phrase were treated as one \isi{tone group}, the tones of its
last two syllables would be lowered to L. This is precisely what happens in the other \is{variants}variant that
is attested for this phrase: /\ipa{ʐv̩˧tsʰi˩ʐv̩˩-nɑ˩}/ (tone pattern: L\#--). The two variants can
therefore be described as (i)~a~form consisting of two tone groups, and (ii)~a~simplified
form, whose tonal pattern results straightforwardly from its treatment as a~single tone
group. (The same phenomenon is reported in the Lataddi dialect: see \citealt{dobbsetal2016}.)

The same applies to all tonal patterns in the range [40..59], [70..79] and [90..99], since the first
two syllables (corresponding to ‘40’, ‘50’, ‘70’ and ‘90’ respectively) have a~Mid"=plus"=Low
pattern. This pattern precludes any tone other than L on the following syllables within the same
\isi{tone group} (by Rule 5). One would therefore expect all
of these combinations to have two variants. This holds true as a~general rule: when the consultant
indicated a~complex form and I tried substituting a~simplified form, that form was never rejected by
the consultant. On the other hand, for some combinations only the simpler form is acceptable. For instance, for ‘44’ with a~classifier of category M\textsubscript{b}, such as /\ipa{ɭɯ˧\textsubscript{b}}/ (the classifier
for round objects), one has to say /\ipa{ʐv̩˧tsʰi˩ʐv̩˩-ɭɯ˩}/ (tone pattern: L\#--).

Supposing, on the {analogy} of category M\textsubscript{a}, that \ipa{$\dagger$ʐv̩˧tsʰi˩ {\kern2pt}|{\kern2pt} ʐv̩˧-ɭɯ˥} was acceptable at
an~earlier stage of the language, it must be supposed to have fallen out of use. For category M\textsubscript{a}, where there
are currently two variants in common use, the consultant’s intuition is that the complex \is{variants}variant,
/\ipa{ʐv̩˧tsʰi˩ {\kern2pt}|{\kern2pt} ʐv̩˧-nɑ˥}/, is somewhat “slow” and “clumsy”: it conveys special
emphasis and is only appropriate as part of an~expressive strategy to draw attention to the figure
at issue. To sum up, the integration of numeral"=plus"=classifier phrases into one tone group is the general rule.

Interestingly, when a~phrase ends in two Low"=tone syllables, it is possible to test whether these
Low tones result from the levelling down of originally non"=Low tones (as in the case of
/\ipa{ʐv̩˧tsʰi˩--ʐv̩˩-nɑ˩}/, mentioned above) or whether they reflect an~underlying Low tone. When the phrase is divided into two tone groups, if the second group has an underlying L tone, it receives a~post"=lexical final H tone, by the application of Rule 7 (“If a~tone
group only contains L tones, a~post"=lexical H tone is added to its last syllable”). For example, ‘23
years’ (category MH\textsubscript{a}) can be realized either as /\ipa{ɲi˧tsi˧--so˩-kʰv̩˩}/ or as /\ipa{ɲi˧tsi˧ {\kern2pt}|{\kern2pt}
  so˩-kʰv̩˩˥}/, revealing that its underlying tone pattern is M--L, whereas with the classifier for
tools it would be incorrect to say \ipa{$\ddagger${\kern2pt}ʐv̩˧tsʰi˩ {\kern2pt}|{\kern2pt} ʐv̩˩-nɑ˩˥}: the {variant} with a~division into
two groups is /\ipa{ʐv̩˧tsʰi˩ {\kern2pt}|{\kern2pt} ʐv̩˧-nɑ˥}/. This explains neatly why the \is{tonal contour}contour"=creating final H tone
is only allowed for some of the phrases. A~device for forcing the division of the phrase into two
tone groups consists of inserting the syllable /\ipa{lɑ˧}/ ‘and’ before the last digit:
e.g.~/\ipa{ʂɯ˧tsʰi˩ lɑ˩ {\kern2pt}|{\kern2pt} qʰv̩˧-ʁwɤ˥}/ ‘79 heaps’.

  
\subsection[Disyllabic classifiers]{Disyllabic classifiers, and what they reveal about the tones of numerals}
\label{sec:disyllabicclassifiersandimplicationsforthetonesofnumerals}

Only a~few \is{disyllables}disyllabic classifiers were observed. One is a~reduplicated {monosyllable}:
/\ipa{ʈʂʰe˧{$\sim$}ʈʂʰe˧}/ ‘the width of a~room’.\footnote{Roselle Dobbs (p.c.\ 2016) points out that the classifier /\ipa{ʈʂʰe˧{$\sim$}ʈʂʰe˧}/ ‘the width of a~room’ may be reduplicated from the verb /\ipa{ʈʂʰe˧\textsubscript{b}}/ ‘to stretch’.} Others are nouns, e.g.~/\ipa{ʝi˧qʰv̩\#˥}/ ‘bull’s horn’
can be used as a~classifier, since bull horns were used to drink or to pour liquids, for instance to
pour water into a~pot. The noun ‘bull’s horn’ is in competition, in its use as a~classifier, with
another classifier referring specifically to hornfuls of liquids, /\ipa{qʰv̩˧tʰv̩˧}/, which is more
commonly used than the noun. 
%The classifier for strides (large steps) is disyllabic:
%/\ipa{pɤ˧ʁɑ˧}/. 
‘Bottle’, /\ipa{to˩bi\#˥}/, can be used as a~noun, /\ipa{to˩bi\#˥ {\kern2pt}|{\kern2pt} ɖɯ˧-ɭɯ˧}/ ‘a
bottle’, or as a~classifier, /\ipa{ʐɯ˧ {\kern2pt}|{\kern2pt} ɖɯ˧-to˩bi˩}/ ‘a bottle of liquor/spirits’. The classifier for
ladlefuls is /\ipa{bv̩˩dze˩}/, and that for handfuls (using both hands) is /\ipa{lo˩dzi˩}/.

The disyllabic classifiers observed to date fall into one of four tonal categories: M, \#H, L, and LM+\#H. In terms of their behaviour in association with numerals, these four tonal categories fall into two sets: M and \#H on the one
hand, L and LM+\#H on the other. Tables~\ref{tab:hornfuls} and \ref{tab:bottles} present the data. Clearly, the relevant \is{juncture (inside a tone group)}juncture for tone assignment in these phrases is that which precedes the classifier, even for numerals under twenty: describing the tone pattern of /\ipa{ɖɯ˧-to˩bi˩}/ ‘one bottle’ requires the recognition of a~\is{juncture (inside a tone group)}juncture preceding the classifier, i.e.\ /\ipa{ɖɯ˧~-- to˩bi˩}/.

\begin{subtables}
\begin{table}[t!]
\caption{\label{tab:hornfuls}The tonal behaviour of disyllabic classifiers with lexical M or \#H tone.}
\begin{tabularx}{\textwidth}{ P{20mm} Q Q Q }
  \lsptoprule
  	numeral & example form & output tone & meaning\\ \midrule
	1 & \ipa{ɖɯ˧-qʰv̩˧tʰv̩\#˥} & \#H & one hornful\\
	2 & \ipa{ɲi˧-qʰv̩˧tʰv̩\#˥} & \#H & two hornfuls\\ 
	3 & \ipa{so˩-qʰv̩˩tʰv̩˩˥} & L & three hornfuls\\
	4 & \ipa{ʐv̩˧-qʰv̩˧tʰv̩\#˥} & \#H & four hornfuls\\
	5 & \ipa{ŋwɤ˧-qʰv̩˧tʰv̩\#˥} & \#H & five hornfuls\\
	6 & \ipa{qʰv̩˧-qʰv̩˧tʰv̩\#˥} & \#H & six hornfuls\\
	7 & \ipa{ʂɯ˧-qʰv̩˧tʰv̩\#˥} & \#H & seven hornfuls\\
	8 & \ipa{hõ˧-qʰv̩˧tʰv̩\#˥} & \#H & eight hornfuls\\
	9 & \ipa{gv̩˧-qʰv̩˧tʰv̩\#˥} & \#H & nine hornfuls\\
	10 & \ipa{tsʰe˩-qʰv̩˩tʰv̩˩˥} & L & ten hornfuls\\
\lspbottomrule
\end{tabularx}
\end{table}

\begin{table}[t!]
\caption{\label{tab:bottles}The tonal behaviour of disyllabic classifiers with lexical L or LM+\#H tone.}
\begin{tabularx}{\textwidth}{ P{20mm} Q Q Q }
  \lsptoprule
  	numeral & example form & output tone & meaning\\ \midrule
	1 & \ipa{ɖɯ˧-to˩bi˩} & --L & one bottle\\
	2 & \ipa{ɲi˧-to˩bi˩} & --L & two bottles\\ 
	3 & \ipa{so˩-to˩bi˩˥} & L & three bottles\\
	4 & \ipa{ʐv̩˧-to˥bi˩} & \#H-- & four bottles\\
	5 & \ipa{ŋwɤ˧-to˥bi˩} & \#H-- & five bottles\\
	6 & \ipa{qʰv̩˧-to˥bi˩} & \#H-- & six bottles\\
	7 & \ipa{ʂɯ˧-to˩bi˩} & --L & seven bottles\\
	8 & \ipa{hõ˧-to˥bi˩} & \#H-- & eight bottles\\
	9 & \ipa{gv̩˧-to˥bi˩} & \#H-- & nine bottles\\
	10 & \ipa{tsʰe˩-to˩bi˩˥} & L & ten bottles\\
\lspbottomrule
\end{tabularx}
\end{table}
\end{subtables}

%Overall, there is much less diversity of tones for disyllabic classifiers than for \is{monosyllables}monosyllabic
%ones.

\subsection[Two numerals plus a~classifier]{Two numerals plus a~classifier, conveying approximation}
\label{sec:twonumerals}

Two numerals can accompany a~classifier, conveying an~approximative number. This can be likened to the coordinative construction ‘\textsc{num} or \textsc{num}’ in {English}: ‘one or two’, ‘two or three’{\dots} This construction is less common than coordinative compounds consisting of two numeral"=plus"=classifier phrases, such as /\ipa{ɖɯ˧-ɲi˧ - ɲi˧-ɲi˧}/ ‘one day"=two days’ and /\ipa{ɲi˧-ɲi˧ - so˧-ɲi˥}/ ‘two days"=three days’, presented in \sectref{sec:coordinativecompounds}. Table~\ref{tab:twonum} sets out the results of elicitation with the noun ‘day’. A~dash ‘--’ indicates that the combination is not in use. This is not to say that there is a~hard"=and"=fast rule against the combinations marked with a~dash in Table~\ref{tab:twonum}, only that those combinations seemed less felicitous to the consultant, for various reasons that may involve considerations of \isi{homophony} along with frequency of use and semantics. For instance, the expression /\ipa{ɖɯ˧-ɲi˧ ɲi˧}/, literally ‘one-two days’, actually means ‘a few days’, somewhat like ‘a~couple of days’ in {English}. Since this expression easily covers the range from one to four, it makes the combinations ‘2 and 3’ and ‘3 and 4’ unnecessary. Among higher numbers, a~reason why the combination ‘7 and 8’ is deemed acceptable may be that it corresponds to ‘a week or so’, a~span of time that has some relevance in the consultant's current conceptualization of time, in which weeks are a~relevant unit because the grandchildren's schooling is punctuated by the succession of weeks. Seen in this light, the combination ‘6 and 7’ should be about as relevant as ‘7 and 8’, and could be acceptable at a~push: $\ddagger${\kern2pt}\ipa{qʰv̩˧-ʂɯ˧ ɲi˧} ‘six or seven days’.

\begin{table}%[t]
	\caption{Expressions conveying an~approximative number: two numerals plus a~classifier.}
	\begin{tabularx}{\textwidth}{ l l l }
		\lsptoprule
		numerals & association of two numerals & meaning\\\midrule
		1 and 2 & \ipa{ɖɯ˧-ɲi˧ ɲi˧}  &  ‘a few days’\\
		2 and 3 & -- &\\
		3 and 4 & -- &\\
		4 and 5 & \ipa{ʐv̩˧-ŋwɤ˧ ɲi˧} & ‘four or five days’\\
		5 and 6 & \ipa{ŋwɤ˧-qʰv̩˧ ɲi˧} & ‘five or six days’\\
		6 and 7 & -- &\\
		7 and 8 & \ipa{ʂɯ˧-hõ˧ ɲi˧} & ‘seven or eight days’\\
		8 and 9 & -- &\\
		9 and 10 & \ipa{gv̩˧-tsʰe˧ ɲi˥} & ‘nine or ten days’\\
		\lspbottomrule
	\end{tabularx}
	\label{tab:twonum}
\end{table}

The fact that the expressions in \tabref{tab:twonum} do not belong to a~full, productive paradigm in the consultant's speech led to the conclusion that it would not be appropriate to attempt systematic elicitation of phrases consisting of two numerals and a~classifier. There was a~concern that elicitation of unusual forms (many of them new coinages) would not yield consistent results (as mentioned in \sectref{sec:examinationoftranscribedtextsanddirectelicitation}). This is one of many topics that remain for future research. 

\largerpage
\subsection[Conclusions]{Conclusions about numeral"=plus"=classifier phrases}
\label{sec:Conclusions}

From the mass of information set out above, it is clear that the tone patterns of
numeral"=plus"=classifier phrases encapsulate information not derived from phonological rules. The
system as presented in Tables~\ref{tab:1to100hmh}a--d and \ref{tab:1to100ml}a--d is regular and productive, in that all the classifiers of a~given
tone category have the same tone patterns. As this system lends itself straightforwardly to computer
implementation, a~simple Perl script was written.\footnote{I plan to make computer tools available from an institutional repository in future. Until this plan comes to fruition, interested readers are invited to get in touch with me.} It takes as its input
the classifier’s tone category and segmental composition and a~numeral (or range of numerals) from 1
to 100. The data in Tables~\ref{tab:1to100hmh}a--d and \ref{tab:1to100ml}a--d is stored inside the script, allowing for the tone pattern to be
recovered through table lookup. The surface phonological tone pattern is then assigned to the phrase
on the basis of the general rules governing tone assignment in this dialect (rules which are encoded into the
script), such as that simple L and M tones attach to all the syllables within their domain, tone
sequences attach to syllables “left"=to"=right”, and so on. For instance, providing as input the numeral ‘44’ and the
classifier for tools, /\ipa{nɑ˧\textsubscript{a}}/ (tonal category: M\textsubscript{a}), the script yields the following
two variants for ‘44 tools’: /\ipa{ʐv̩˧tsʰi˩--ʐv̩˧-nɑ˥}/ (tone pattern: L\#--H\#) and /\ipa{ʐv̩˧tsʰi˩--ʐv̩˩-nɑ˩}/ (tone
pattern: L\#--).

In the current version of the script, all of the information set out in Tables~\ref{tab:1to100hmh}a--d and \ref{tab:1to100ml}a--d is encoded in full,
specifying the tone patterns of 900 combinations (9 tone categories of classifiers times 100
numerals). This allows for straightforward table"=lookup, but is uneconomical from the point of view
of linguistic modelling. The addition of some rules could significantly reduce the number of
combinations that need to be indicated. In particular, the tone patterns of [40..49] are identical
with those of [50..59]; likewise for [60..69] and [80..89]. Numerals ending in ‘1’ also have
identical patterns with those ending in ‘2’, with a~few exceptions in category H\textsubscript{b}. The information
provided for each \is{subcategories of lexical tones}subcategory (H\textsubscript{a} and H\textsubscript{b}, M\textsubscript{a} and M\textsubscript{b}, and so on) could also be simplified by
considering one of the \is{subcategories of lexical tones}subcategories as the norm, and only supplying the forms for the other
\is{subcategories of lexical tones}subcategories where they differ from that norm.
However, it is clear that even after this {simplification} task had been conducted, large numbers of
tonal patterns would still need to be specified individually. For instance, neither H\textsubscript{a} nor H\textsubscript{b} can be
considered as a~simplified version of the other. The systematic presence of a~L tone in all the phrases from ‘10’ to ‘19’ could suggest that the patterns for H\textsubscript{b} constitute a~simplified version of those for H\textsubscript{a}, but there also exists a~complication for H\textsubscript{b} and
not for H\textsubscript{a}, namely the presence of different tones after numerals ending in ‘1’ and ‘2’. This observation is a~striking \isi{counterexample} to the pan"=\ili{Naish} generalization that
the numerals ‘1’ and ‘2’ always have the same tone patterns~-- a~generalization which holds for \ili{Naxi}
and \ili{Laze}, and for all the rest of the Na data. Finally, an idiosyncratic tone pattern is observed for /\ipa{to˥}/ ‘armful’: this classifier belongs to the H\textsubscript{a} category, but the combination ‘11 armfuls’ is realized as
/\ipa{tsʰe˧ɖɯ˧-to˧}/ instead of the expected /\ipa{†tsʰe˩ɖɯ˩-to˩˥}/.


\section{Demonstrative"=plus"=classifier phrases}
\label{sec:demonstrativeplusclassifierphrases}

A \is{demonstratives}demonstrative and a~following classifier are always integrated into one \isi{tone group}, as in (\ref{ex:dress}), where the proximal \is{demonstratives}demonstrative //\ipa{ʈʂʰɯ˥}// and the classifier for pairs or sets //\ipa{dzi˧\textsubscript{b}}// combine as /\ipa{ʈʂʰɯ˧-dzi˧˥}/ ‘this set'.

\begin{exe}
	\ex
	\label{ex:dress}
	\ipaex{no˩bv̩˧ {\kern2pt}|{\kern2pt} ʈʂʰɯ˧-dzi˧˥ {\kern2pt}|{\kern2pt} le˧-ʑi˩, {\kern2pt}|{\kern2pt} tʰi˧-mv̩˧-kʰɯ˧˥.}\\
	\gll no˩bv̩˧		ʈʂʰɯ˥					dzi˧\textsubscript{b}			le˧-		ʑi˩\textsubscript{b}	tʰi˧-	mv̩˧\textsubscript{a}	-kʰɯ˧˥\\
	given\_name		\textsc{dem.prox}	\textsc{clf}.sets/pairs				\textsc{accomp}	to\_bring\_along		\textsc{dur}		to\_put\_on		\textsc{caus}\\
	\glt ‘Nobbu brought that set [of clothes], and made [her] wear [the clothes].’ (BuriedAlive3.73)
\end{exe}

The
elicitation procedure for demonstrative"=plus"=classifier phrases was similar to that used for numeral"=plus"=classifier phrases, and similar
puzzles were encountered.

As explained in the previous section, the tonal categories of classifiers were established on the
basis of their behaviour when combined with numerals. The nine tonal categories of monosyllabic classifiers are: H\textsubscript{a} and H\textsubscript{b}; MH\textsubscript{a} and
MH\textsubscript{b}; M\textsubscript{a} and M\textsubscript{b}; and L\textsubscript{a}, L\textsubscript{b} and L\textsubscript{c}. As for the proximal \is{demonstratives}demonstrative, /\ipa{ʈʂʰɯ\#˥}/, and the
distal \is{demonstratives}demonstrative, /\ipa{tʰv̩\#˥}/, they both carry lexical \#H tone. The expectations were that,
in \is{demonstratives}demonstrative"=plus"=classifier phrases, (i)~there would be no difference between proximal and distal
demonstratives, since they have the same lexical tone, and (ii)~all classifiers within each of the
nine tonal categories would have the same tonal behaviour.

The first prediction was verified: phrases containing /\ipa{ʈʂʰɯ\#˥}/ ‘this’ and /\ipa{tʰv̩\#˥}/
‘that’ always have the same tonal behaviour. The second prediction, on the other hand, was incorrect: some of the tonal categories for classifiers proved to be less than fully homogeneous. The
account provided below starts out from the simplest cases, and progresses towards the category for
which the greatest degree of divergence was found.

When combined with demonstratives, H\textsubscript{a} and H\textsubscript{b} behave in the same way, as do MH\textsubscript{a} and MH\textsubscript{b}. This is the
simplest part of the picture: the distinction between H\textsubscript{a} and H\textsubscript{b} is neutralized in this context, and
the distinction between MH\textsubscript{a} and MH\textsubscript{b} is likewise neutralized. Categories M\textsubscript{a} and M\textsubscript{b} behave differently
from each other, but in a~consistent and simple way, with only one possible pattern:
demonstrative plus M\textsubscript{a}-tone classifier yields L\#, e.g.~/\ipa{tʰv̩˧-nɑ˩}/ (classifier for tools);
demonstrative plus M\textsubscript{b}-tone classifier yields \#H, e.g.~/\ipa{tʰv̩˧-ɭɯ\#˥}/ (generic classifier).

Among Low tone categories of classifiers, L\textsubscript{a} and L\textsubscript{c} are relatively straightforward. All L\textsubscript{a}-tone
classifiers have the same behaviour, allowing two variants: H\# and H\$. Both variants are firmly
attested. The speaker expresses a~preference for the former, but this slight imbalance appears to be
the same for all examples, suggesting that there does not exist any clear (lexicalized) preference
for the one or the other in association with a~specific classifier. L\textsubscript{c}-tone
classifiers allow no less than three variants: MH\#, H\# and H\$.

Category L\textsubscript{b} is the most problematic. In the production data, there are three variants, H\#, H\$ and
MH\#. But for some classifiers of this category (e.g.~/\ipa{mi˩\textsubscript{b}}/, the classifier for animals, and
/\ipa{kʰɯ˩\textsubscript{b}}/, the classifier for long objects), MH\# is by far the most frequent pattern, with H\$
as an~occasional {variant}, and H\# rarely attested. For other classifiers, H\# and H\$ appear
with comparable frequency, whereas MH\# is seldom found.

When several \is{variants}variants were proposed by the investigator and the consultant was
asked which ones were correct, a~similar picture emerged: MH\# is strongly preferred for some classifiers, with H\$ as
an~acceptable {variant}, whereas H\# is dispreferred (either refused as incorrect, or judged as marginally acceptable only). For other classifiers, the preferred form is with H\# tone, H\$ tone is
an~acceptable {variant}, and MH\# tone is not. \tabref{tab:thetwosubcategoriesoflbtoneclassifiersbasedontheirbehaviourinassociationwithdemonstratives} presents these two sets, provisionally labelled
as ‘type I’ and ‘type II’.

\begin{table}%[t]
  \caption{The two subcategories of L\textsubscript{b}-tone classifiers, based on their behaviour in association with demonstratives.}
\begin{tabularx}{\textwidth}{ P{20mm} Q P{15mm} P{25mm} }
\lsptoprule
	type & tone pattern in association with {demonstrative} & form & classifier for\\ \midrule
	L\textsubscript{b}, type I & MH\# most common; & \ipa{bo˩\textsubscript{b}} & headdresses\\
	 & H\$ attested;  & \ipa{dv̩˩\textsubscript{b}} & small groups\\
	 & H\# dispreferred or refused & \ipa{dzi˩\textsubscript{b}} & trees\\
	 &  & \ipa{jo˩\textsubscript{b}} & ounces\\
	 &  & \ipa{kʰɯ˩\textsubscript{b}} & long objects\\
	 &  & \ipa{lo˩\textsubscript{b}} & valleys\\
	 &  & \ipa{ɬi˩\textsubscript{b}} & armspans\\
	 &  & \ipa{mi˩\textsubscript{b}} & animals\\
	 &  & \ipa{pʰv̩˩\textsubscript{b}} & fields\\
	 &  & \ipa{tʰv̩˩\textsubscript{b}} & sets of ten\\
	 &  & \ipa{tɕʰi˩\textsubscript{b}} & meals\\
	 &  & \ipa{wɤ˩\textsubscript{b}} & loads\\
	 &  & \ipa{wo˩\textsubscript{b}} & teams of oxen\\ \midrule
	L\textsubscript{b}, type II & H\# most common; & \ipa{pɤ˩\textsubscript{b}} & ladders, doors\\
	 & H\$ attested; & \ipa{po˩\textsubscript{b}} & packs\\
	 & MH\# dispreferred or refused & \ipa{ʁo˩\textsubscript{b}} & types, sorts\\
	 &  & \ipa{ʂɯ˩\textsubscript{b}} & times\\
	 &  & \ipa{tsʰe˩\textsubscript{b}} & leaves\\
	 &  & \ipa{ʈv̩˩\textsubscript{b}} & large chunks\\
\lspbottomrule
\end{tabularx}
\label{tab:thetwosubcategoriesoflbtoneclassifiersbasedontheirbehaviourinassociationwithdemonstratives}
\end{table}

The consultant’s acceptance of variants fluctuated from session to session, but the distinction
between types I and II was confirmed in the course of a~long series of elicitation sessions. Examples from the recorded texts provide evidence to the same effect.

The tone patterns for all categories of classifiers are set out in \tabref{tab:thetonepatternsofdemonstrativeplusclassifierphrases}, using the distal
demonstrative, /\ipa{tʰv̩\#˥}/, as an~example. (As mentioned above, the tone patterns for the
proximal \is{demonstratives}demonstrative, /\ipa{ʈʂʰɯ\#˥}/, are the same.) The corresponding online recordings are DemClf, DemClf2 and DemClf3.

\begin{table}%[t]
\caption{The tone patterns of demonstrative"=plus"=classifier phrases.}
\begin{tabularx}{\textwidth}{ l P{30mm} Q l }
\lsptoprule
	\multicolumn{2}{l}{tone pattern} & \multicolumn{2}{l}{example}\\\cmidrule(lr){1-2}\cmidrule(lr){3-4}
	tone of \textsc{clf} &  tone of~\textsc{dem}+\textsc{clf} phrase & classifier for &
   \textsc{dem}+\textsc{clf} phrase\\ \midrule
	H\textsubscript{a}, H\textsubscript{b} & H\$ / \#H & chunks & \ipa{tʰv̩˧-kʰwɤ˥\$} / \ipa{tʰv̩˧-kʰwɤ\#˥}\\
	MH\textsubscript{a}, MH\textsubscript{b} & L\# & cattle & \ipa{tʰv̩˧-pʰo˩}\\
	M\textsubscript{a} & L\# & tools & \ipa{tʰv̩˧-nɑ˩}\\
	M\textsubscript{b} & \#H & \textit{generic} & \ipa{tʰv̩˧-ɭɯ\#˥}\\
	L\textsubscript{a} & H\# / H\$ & quantities & \ipa{tʰv̩˧-mɤ˥ / tʰv̩˧-mɤ˥\$}\\
	L\textsubscript{b}, type I & MH\# / H\$ & animals & \ipa{tʰv̩˧-mi˧˥ / tʰv̩˧-mi˥\$}\\
	L\textsubscript{b}, type II & H\# / H\$ & doors & \ipa{tʰv̩˧-pɤ˥ / tʰv̩˧-pɤ˥\$}\\
	L\textsubscript{c} & MH\# / H\# / H\$  & plains & \ipa{tʰv̩˧-di˧˥ / tʰv̩˧-di˥ / tʰv̩˧-di˥\$}\\
\lspbottomrule
\end{tabularx}
\label{tab:thetonepatternsofdemonstrativeplusclassifierphrases}
\end{table}

The ‘type I’ and ‘type II’ distinction among L\textsubscript{b}-tone classifiers constitutes an~impressive extra intricacy within the classifiers' system of nine tonal categories. Category L\textsubscript{b} is a~subdivision within the L category of tones; its
further division into types I~and II constitutes a~subdivision within a~subdivision. From the point
of view of phonological output, type I within the L\textsubscript{b} category yields an~output (MH\# / H\$)
which does not coincide with any other within the system.



\section{Tonal interactions with a~preceding noun}
\label{sec:interactionnoun}
\label{sec:introductioninteractionnoun}

Occasionally, there is tonal interaction between a~numeral"=plus"=classifier or
{\linebreak}demonstrative"=plus"=classifier phrase and the preceding noun, as in (\ref{ex:wowsheisgivingyouthreesilvercoins}). 
\begin{exe}
  \ex
  \ipaex{ə˧mi˧! {\kern2pt}|{\kern2pt} pæ˧kʰwɤ˧ so˧-ɭɯ˥ ki˩ mæ˩!}\\
  \label{ex:wowsheisgivingyouthreesilvercoins}
  \gll ə˧mi˧	pæ˧kʰwɤ\#˥	so˩	ɭɯ˧	ki˧\textsubscript{a}	mæ˧\\
  \textsc{intj}	silver\_coin	three	\textsc{clf}	to\_give	\textsc{disc}.\textsc{ptcl}\\
  \glt ‘Wow! [(S)he] is giving you three silver coins!’
\end{exe}

\newpage 
According to the main consultant's memories, (\ref{ex:wowsheisgivingyouthreesilvercoins}) is the
type of comment that uncles and aunts would make when a~child received significant amounts of money
on the occasion of their coming"=of"=age ceremony. To offer only one coin would be inappropriate, as
gifts should come in pairs. Two coins is a~beautiful gift. Three coins is beyond expectations (the equivalent at the time of about half a~month's salary).

The tonal \is{derivation!tonal}derivation for the phrase ‘three silver coins’ is as follows: 
\begin{enumerate}[label=(\roman*), itemsep=0pt]
\item the input tones are: \#H for ‘silver coin’, //\ipa{pæ˧kʰwɤ\#˥}//; L for ‘three’; and M
  for the classifier, //\ipa{ɭɯ˧}//
\item the numeral"=plus"=classifier phrase ‘three-\textsc{clf}’ carries M tone: //\ipa{so˧-ɭɯ˧}//
\item the phrase that results from the combination of (i) and (ii) carries a~H\# tone: //\ipa{pʰæ˧kʰwɤ˧
  so˧-ɭɯ˥\#}//
\end{enumerate}

Other examples of such tonal interaction include (\ref{ex:thatgirl}) and (\ref{ex:thatuncle}).

\begin{exe}
	\ex
	\label{ex:thatgirl}
	\ipaex{mv̩˩zo˩ tʰv̩˩-ɭɯ˩˥}\\
	\gll mv̩˩zo˩	tʰv̩˥	ɭɯ˧\textsubscript{b}\\
	 girl	\textsc{dem.dist}	\textsc{clf}.generic\\
	\glt ‘that girl’ (Coming\-Of\-Age2.60; Tiger2.139, 141, 147)
\end{exe}

\begin{exe}
	\ex
	\label{ex:thatuncle}
	\ipaex{ə˧v̩˧ tʰv̩˧-v̩˧}\\
	\gll ə˧v̩˧˥	tʰv̩˥	v̩˧\\
	uncle	\textsc{dem.dist}	\textsc{clf}.individual\\
	\glt ‘that uncle’ (Tiger2.109)
\end{exe}

The \is{form!underlying}underlying tonal category of such
expressions can be established in the same way as that of nouns, by tests such as adding the
\isi{copula}. The expression ‘three silver coins’ /\ipa{pʰæ˧kʰwɤ˧ so˧-ɭɯ˥}/ in (\ref{ex:wowsheisgivingyouthreesilvercoins}) yields /\ipa{pʰæ˧kʰwɤ˧ so˧-ɭɯ˥ ɲi˩}/,
revealing that its tone is H\#: //\ipa{pʰæ˧kʰwɤ˧ so˧-ɭɯ˥\#}//. By contrast, ‘that uncle’, /\ipa{ə˧v̩˧ tʰv̩˧-v̩˧}/ (\ref{ex:thatuncle}),
yields /\ipa{ə˧v̩˧ tʰv̩˧-v̩˧ ɲi˥}/, revealing that its tone is \#H: //\ipa{ə˧v̩˧ tʰv̩˧-v̩\#˥}//. (As for (\ref{ex:thatgirl}), its surface form suffices to determine its underlying tone: it can only be //L//.)

Why is tonal change only occasional? Here are some speculations. 

First, if tone change were systematic, it would entail the \isi{neutralization} of some of the tonal distinctions
among classifiers. This is because the number of possible tone patterns in tone groups of three syllables or
more is not as high as the combinatorial diversity in sequences of short tone
groups. For instance, after disyllabic nouns carrying a~L\# tone, determiners would have their tones
lowered to L (by application of Rule 5: “All syllables following a~H.L or M.L sequence receive L tone”), which would entirely \is{neutralization}neutralize tonal oppositions among classifiers. Even if the majority of
contrasts among different categories of classifiers were retained, tonal change would increase
opacity, as there would be more cases where \is{form!surface}surface phonological tone on determiners differed from
\is{form!underlying}underlying tone. Given the richness of the tone system of classifiers, an~increase
in opacity would be likely to create a~pressure towards the \isi{simplification} of the system. The fact
that tonal interaction between a~noun and a~following determiner is only occasional may thus favour
the preservation of the current system. Functionally, having a~choice between two options widens the range of \is{stylistics}stylistic possibilities.

% Here: a subsection header was deleted.
\label{sec:anasymmetricalsystemsometonesaremorepronetochangethanothers}

While discourse factors are paramount in determining whether there is interaction between the noun and the phrase that determines it, the tonal category may also play a~role. All other things being equal, some
tones appear to resist interaction because the tonal output would be non"=trivial, requiring the speaker to remember the morphologically conditioned tone combination rule that applies for this combination. Other tones are more prone to interaction because adjusting the noun with the following determiner phrase is simply a~matter of applying phonological rules. For instance, if the noun's tone is L\#, it feels easiest to integrate this noun with the following phrase, because the phonological consequence is a simple levelling down of the tones of the following syllables within the \isi{tone group}: the noun's two syllables get M and L tone, respectively, and all syllables following this M.L sequence receive L tone by Rule~5. In this case, division into two tone groups would require the extra effort of preserving the tones of the determiner phrase from lowering. This is possible, but it
requires a~\is{stylistics}stylistic motivation to emphasize the two constituents. 

The implication for the analysis of the tone system would be that some tones are more combination"=prone than others, because the tone combination rules in which they are involved follow from phonological rules, and hence tend to be exceptionless and simple to implement. The \is{tone spreading}spreading of L\# onto following syllables, mentioned in the previous paragraph, appears as the best example. Other tones are less combination"=prone because tone changes in which they are involved are less straightforward,
and tend to allow several variants. The H\$ tone category is a~case in point. 

Such asymmetries inside the tone system, with some tones more change"=prone than others, would
be reflected in the frequency of tonal interactions between a~noun and a~following determiner
(N+\textsc{dem}+\textsc{clf}, N+\textsc{num}+\textsc{clf}). Systematic verification of these hypotheses must be deferred until later.

\largerpage[-2]
\section[Concluding note]{Concluding note}
\label{sec:generalconclusion}

While the tone patterns of classifiers may appear staggeringly complex, phrases containing classifiers are frequent in discourse, a~factor which is known to favour the preservation of irregular morphology. As to the system's {diachronic} origin, some reflections are set out in \sectref{sec:applicationtoyongningna}. 

\is{classifiers|)}
\is{numerals|)}

If anyone read this whole chapter with uninterrupted excitement and engagement, they are invited to imagine how much more elaborate the picture would have been if Yongning Na had had the fortune to be used as a~medium for the expression of concepts of number theory such as ordinals and fractions. \ili{Tibetan}, for instance, has ordinals, and means to express fractions and other mathematical concepts \citep{liu2010tib}, which tend to seep into other languages in the area of cultural \ili{Tibetan} influence (e.g.~Nar-Phu, spoken in the Manang district of Nepal: \citealt[342]{noonan2003}). Dzongkha (a~Bodic language spoken in Bhutan) has fascinating complexities such as the use of fractions in its counting system \citep{mazaudon1985a}. But in Yongning Na there are no ordinal numbers or fractions. Examples that contain ordinal numbers in other languages (such as {English} and Chinese) correspond to turns of phrase with cardinal numbers in Yongning Na. For instance, in (\ref{ex:seventh}), ‘on the eighth day’ is expressed as ‘one day [after] seven days’, i.e.\ ‘the day after seven days had elapsed’.\footnote{The context to example (\ref{ex:seventh}) is as follows. A~young woman choked because she gulped down boiled eggs that got stuck in her throat; family members assumed that she was dead, and buried her. She came to herself when robbers looting her grave put her body upright to rip off the valuable garments in which she had been interred. She returned to see her mother; the mother was terrified, thinking that her daughter was become a~ghost. Ghost lore has it that the dangerous period within which ghosts can return is limited to seven days, so the mother asked the daughter to stay in hiding until seven days had elapsed, and only return home after that time.} In (\ref{ex:bridge}), the equivalent of ‘the first person who will come' is ‘whoever comes', understood by inference from context as ‘the \textit{first} person who comes'.\footnote{This example requires an explanation about context. The parents of a~sick infant are advised to build a~bridge, and to request the first person who crosses this bridge to give a~new name to their child. This is a~form of symbolic adoption, intended to give the child a~new start in life. The stranger gives the infant a~new name, and a~small token, such as a~button from their dress. Example (\ref{ex:bridge}) is part of the instructions that the parents receive about what they should do for their child.} Other examples are found in Renaming.29, 38 and Funeral.2. 

 \begin{exe}
 	\ex
 	\ipaex{ʂɯ˧-hɑ̃˧ gv̩˧ {\kern2pt}|{\kern2pt} ɖɯ˧-ɲi˥, {\kern2pt}|{\kern2pt} ə˧mi˧ lɑ˧ {\kern2pt}|{\kern2pt} mv̩˩˥ {\kern2pt}|{\kern2pt} tʰi˧-to˧{$\sim$}to˧, {\kern2pt}|{\kern2pt} ŋv̩˩ ɲi˥ tsɯ˩ {\kern2pt}|{\kern2pt} mv̩˩!}\\
 	\label{ex:seventh}
 	\gll ʂɯ˧-hɑ̃˧˥					gv̩˧\textsubscript{c}			ɖɯ˧-ɲi\$˥					ə˧mi˧		lɑ˧		mv̩˩˥		tʰi˧-		to˧{$\sim$}to˧		ŋv̩˩\textsubscript{a}		-ɲi˩		tsɯ˧˥		mv̩˧\\
 	seven-\textsc{clf}.nights	to\_elapse		one-\textsc{clf}.days	mother		and		daughter		\textsc{dur}	to\_hug		to\_cry			\textsc{certitude}		\textsc{rep}		\textsc{affirm}\\
 	\glt ‘On the eighth day (\textit{literally}: the day after seven days had gone by), mother and daughter fell into each other's arms and wept!' (BuriedAlive3.129)
 \end{exe}


 \begin{exe}
 	\ex
 	\ipaex{hĩ˧ {\kern2pt}|{\kern2pt} ɲi˩ le˧-tsʰɯ˩-ɻ̍˩-dʑo˩~{\dots}}\\
 	\label{ex:bridge}
 	\gll hĩ˥		ɲi˩							le˧-					tsʰɯ˩\textsubscript{a}		ɻ̍˩								-dʑo˥\\
 	person		\textsc{interrog}.who	\textsc{accomp}		to\_come					to\_turn\_toward	\textsc{top}\\
 	\glt ‘The first person who comes along{\dots} (\textit{literally:} whoever comes along{\dots})' (Renaming.14)
 \end{exe}

Contact with \il{Sinitic}Chinese, including training in mathematics at school, currently creates a~pressure towards the use of ordinals and fractions, but the result is the wholesale adoption of Chinese forms. 

% Indexing stuff
%\is{in isolation (realization of word in isolation)|see{form!in isolation}}