\chapter{Conclusion}
\label{chap:conclusion}

\epigraph{When one aims to please others, one may fail, whereas things that we do to please ourselves always have a~chance to interest someone or other.}{Marcel Proust\footnotemark}{}\footnotetext{This sentence is followed by reflections that are close to linguists' concerns: ``No one is unique: our individualities are made out of a~universal fabric; this is what allows for sympathy and understanding, which are such great pleasures in life. If we could analyze the soul as we analyze matter, it would become apparent that below the surface diversity of minds, as under that of material objects, there are but a~few simple substances and irreducible elements; and that what we think of as our personality is made up from elements which are quite common, and which are met again pretty much everywhere in the universe.'' \textit{Original text:} Quand on travaille pour plaire aux autres on peut ne pas réussir, mais les choses qu'on a~faites pour se contenter soi-même ont toujours chance d'intéresser quelqu'un. Il est impossible qu'il n'existe pas de gens qui prennent quelque plaisir à ce qui m'en a~tant donné. Car personne n'est original et fort heureusement pour la sympathie et la compréhension qui sont de si grands plaisirs dans la vie, c'est dans une trame universelle que nos individualités sont taillées. Si l'on savait analyser l'âme comme la matière, on verrait que, sous l'apparente diversité des esprits aussi bien que sous celle des choses, il n'y a~que peu de corps simples et d'éléments irréductibles et qu'il entre dans la composition de ce que nous croyons être notre personnalité, des substances fort communes et qui se retrouvent un peu partout dans l'Univers. (\textit{Pastiches et mélanges}, Paris: Gallimard, 1919, pp. 108--109)}

%Command \noindent added to avoid having a first indent in cases where a paragraph starts after an epigraph without an intervening title.
{\noindent}The Yongning Na tone system comprises (i)~a~set of phonological rules governing tone"=to"=syllables association, set out in \sectref{sec:asummaryoftonetosyllableassociationrules}, and (ii)~a~host of rules that are specific to certain morphosyntactic
contexts, set out in Chapters~\ref{chap:compoundnouns}-\ref{chap:verbsandtheircombinatoryproperties}. Different rules apply in the association of a~verb with a~subject or
an~object, the association of two nouns into a~{compound}, that of a~{numeral} and classifier, or that
of a~word and its affixes, for instance. These tonal paradigms constitute the core of the
morphotonology of Yongning Na, and represent the bulk of what language learners must acquire to master this tone system. 

As a~conclusion, let us return to the initial puzzle: the first example presented in the
introduction. It is now possible to set out the mechanisms whereby the surface phonological tone sequences of examples (\ref{ex:ihavetogoandtakemyluggagenowREP}--\ref{ex:ihavetogoimafraidihavetoleaveREP}) obtain. 

\begin{exe}
	\ex %\label{1}
	\begin{xlist}
		\ex
		\label{ex:ihavetogoandtakemyluggagenowREP}
		\gll njɤ˧	ʑi˩	bi˩	-zo˩	-ho˥.\\
		\textsc{1sg}	to\_take	to\_go	\textsc{obligative}	\textsc{desiderative}\\
		\glt ‘I have to go and take [my luggage] now.'
		
		\ex
		\label{ex:ihavetogoimafraidihavetoleaveREP}
		\gll	njɤ˧	bi˧	-zo˧	-ho˩.\\
		\textsc{1sg}	to\_go	\textsc{obligative}	\textsc{desiderative}\\
		\glt ‘I have to go. / I’m afraid I have to leave.' 
	\end{xlist}
\end{exe}

With morpheme"=level transcriptions indicating lexical tone in terms of the lexical tone categories of Yongning Na, these sentences can be represented as (\ref{ex:ihavetogoandtakemyluggagenow2REP}--\ref{ex:ihavetogoimafraidihavetoleave2REP}).

\begin{exe}
	\ex
	\begin{xlist}
		\ex
		\label{ex:ihavetogoandtakemyluggagenow2REP}
		\ipaex{njɤ˧ {\kern2pt}|{\kern2pt} ʑi˩ bi˩-zo˩-ho˥.}\\
		\gll njɤ˩ 	ʑi˩\textsubscript{a}		bi˧\textsubscript{c}	-zo˧\textsubscript{a}		-ho˩\\
		\textsc{1sg}	to\_take		to\_go	\textsc{obligative}	\textsc{desiderative}\\
		\glt ‘I have to go and take [my luggage] now.'
		
		\ex
		\label{ex:ihavetogoimafraidihavetoleave2REP}
		\ipaex{njɤ˧ {\kern2pt}|{\kern2pt} bi˧-zo˧-ho˩.}\\
		\gll njɤ˩ 	bi˧\textsubscript{c}	-zo˧\textsubscript{a}		-ho˩\\
		\textsc{1sg}	to\_go	\textsc{obligative}	\textsc{desiderative}\\
		\glt ‘I have to go. / I’m afraid I have to leave.'
	\end{xlist}
\end{exe}

A crucial piece of information is the tone"=group {boundary} after the \textsc{1sg} subject: these utterances contain two tone groups, and tonal processes apply independently for the two groups, as set out in Chapter~\ref{chap:toneassignmentrulesandthedivisionoftheutteranceintotonegroups}.

The first {tone group} only contains one syllable. Its lexical tone is L. As reported in Chapter~\ref{chap:thelexicaltonesofnouns}, the realization of this tone in isolation is as a~level, non"=low tone: M, hence the surface phonological form /\ipa{njɤ˧}/ in (\ref{ex:ihavetogoandtakemyluggagenow2REP}) and (\ref{ex:ihavetogoimafraidihavetoleave2REP}). 

In (\ref{ex:ihavetogoandtakemyluggagenow2REP}), the second {tone group} consists of two serialized verbs and two
suffixes. 
%The second verb, `to go', behaves tonally like its
%grammaticalized counterpart, the {immediate future}
%{suffix}. 
Following the morphotonological rules brought out in
Chapter~\ref{chap:verbsandtheircombinatoryproperties}, //\ipa{ʑi˩\textsubscript{a}}// in association
with //\ipa{bi˧\textsubscript{c}}// `to go' yields //\ipa{ʑi˩-bi˩}//. Addition of the {obligative} //\ipa{-zo˧\textsubscript{a}}// yields //\ipa{ʑi˩-bi˩-zo˩}//. The last {suffix}, {desiderative}
//\ipa{ho˩}//, carries L tone. The tonal behaviour of a~L-tone {suffix}
depends on the number of syllables of the expression to which it is
attached: if suffixed directly to a~L-tone verb, it carries L tone; if
suffixed to a~L-tone expression of two syllables or more, the {suffix}
carries H tone, hence //\ipa{ʑi˩-ho˩}// `will take', with a~L.L
pattern, but //\ipa{ʑi˩-zo˩-ho˥}// `will need to take', with a~L.L.H
pattern. Association of this {suffix} to the three"=syllable
expression //\ipa{ʑi˩-bi˩-zo˩}// thus yields a~final H tone, hence
//\ipa{ʑi˩-bi˩-zo˩-ho˥}//.

Both tone groups, //\ipa{njɤ˧}// and //\ipa{ʑi˩-bi˩-zo˩-ho˥}//,
contain at least one tone other than L, so that the repair rule for
all-L tone groups (referred to as Rule 7 in \sectref{sec:alistoftonerules}) does not apply. These tone
groups therefore proceed unmodified to the surface phonological level,
as /\ipa{njɤ˧ {\kern2pt}|{\kern2pt} ʑi˩-bi˩-zo˩-ho˥}/.

In (\ref{ex:ihavetogoimafraidihavetoleave2REP}), the second {tone group} consists of a main verb, `to go', and the same two suffixes as in (\ref{ex:ihavetogoandtakemyluggagenow2REP}). Following the morphotonological rules brought out in
Chapter~\ref{chap:verbsandtheircombinatoryproperties}, the M tone on the main verb does not modify the tones of the suffixes. The M tone on the {obligative} suffix //\ipa{-zo˧\textsubscript{a}}// likewise leaves the following morpheme unaffected, so that all three syllables in the tone group simply surface with their lexical tone.

While it is satisfying to verify that the morphotonological patterns described in the present volume
shed light on these and other examples, it must be acknowledged that
this book is only one~step towards the goal of
advanced linguistic modelling of tone in Yongning Na. A~mid- to
long"=term perspective is the computer"=aided analysis of individual
utterances on the basis of a~computer model of the grammar
(finite"=state modelling), following the methodological suggestion of
\citet{karttunen2006}. This will require (i)~implementing the
entire tonal grammar of Yongning Na by computer scripts, (ii)~glossing Yongning Na texts at the morpheme level, providing a~unique link to the lexicon, and (iii)~encoding the
morphosyntactic structure of each utterance. The aim will be to
generate surface phonological tone patterns for an~utterance, spelling
out the full set of possible variants in the division of utterances
into tone groups. 

The model will make it possible to verify
quantitatively, over the full set of available data, the
generalizations which were proposed in this volume.
More ambitiously, the model
will allow the investigator to set the patterns observed in a~textual
utterance (a real sentence from a~text) against the backdrop of a~set of alternatives. This opens new perspectives for
appraising the speaker’s stylistic choices in a~narrative, such as the
observed division of the sentence into tone groups, and the choice of
one {variant} rather than another in cases where two or more tone
patterns are acceptable. This holds promise for uncovering factors at
play in so"=called nonconditioned {variation}. Modelling of the various
components of the language's {prosody} and morphosyntax could allow for an~exploration of the range of stylistic possibilities allowed by the linguistic
system, e.g.~through the choice of congruence vs.\ dissonance between
message and form, and between syntax and prosodic {phrasing} (a source
of inspiration here is Delattre’s approach to {French} {intonation}:
\citeyear{delattre1966a,delattre1970}).

A second perspective will consist in examining the phonetic
implementation of surface phonological tone sequences. This study has
not begun in earnest yet, because the approved order of business
consists of first understanding the system (the morphotonology, and the
{intonation}) before launching into experimental investigation into
acoustic correlates and fine phonetic details
\citep{rice2014,mazaudon2014}. From the beginning of fieldwork on
Yongning Na, this objective has always been kept in view, however. It
motivates constant efforts to collect data that will be exploitable
for this purpose: high"=fidelity audio, and, for some recordings, an~electroglottographic signal. The first steps will consist of modelling
the tonal targets and studying {coarticulation} patterns. Since tone"=group boundaries are systematically
indicated in the Yongning Na annotations, it should be possible to
obtain quantified evidence on issues such as: To what extent are
tone"=group boundaries accompanied by pauses? Do fine phonetic details
in the realization of segments cue the presence of tone"=group
boundaries, i.e.\ to what extent are tone"=group boundaries signalled by
“segmental {intonation}” in the sense of \citet{niebuhr2009}?

The ultimate aim consists of assessing the contribution of various factors to the final phonetic realization of each syllable, teasing apart and spelling out the various components of the speech signal, and their linguistic interpretation. Computer implementation may be used as a~tool to bring out, by contrast, intonational phenomena, as components that are not predictable on the basis of the utterance’s contrastive units: the sequence of phonemes, and the tonal string parsed into tone groups. 

%\begin{figure}[t]
%	\caption{Working out tone in Yongning Na: field notes from 2007, with comments added in 2008.}
%	\begin{minipage}{.5\textwidth}
%		\centering
%		\includegraphics[width=.95\linewidth]{figures/ms/1027.jpg}
%%		\captionof{figure}{A figure}
%		\label{fig:test1}
%	\end{minipage}%
%	\begin{minipage}{.5\textwidth}
%		\centering
%		\includegraphics[width=.95\linewidth]{figures/ms/1029.jpg}
%%		\captionof{figure}{Another figure}
%		\label{fig:test2}
%	\end{minipage}
%\end{figure}
%

%\clearpage
\largerpage
\begin{figure}[h!!]
	\includegraphics[width=.93\textwidth]{figures/ms/1029.jpg}
%	\caption{Working out tone in Yongning Na: field notes from 2007, with comments added in 2008.}
	\caption{Working out tone in Yongning Na: field notes, 2007.}
	\label{fig:ms2}
\end{figure}

% \begin{figure}[h!!]
	% \includegraphics[width=.9\textwidth]{figures/ms/1107Pred.jpg}
	% \caption{Working out tone in Yongning Na: field notes, 2007.}
	% \label{fig:ms3}
% \end{figure}

%\footnote{Marc Brunelle’s work on {Vietnamese} is an~inspirational example of this strand of research \citep{brunelle2015}.}