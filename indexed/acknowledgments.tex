\addchap{Acknowledgments}
% \begin{refsection}

\epigraph{[Writing a~grammar] takes an individual who loves language in general, the target language in particular, and is trained and happy to spend time doing this work.}{\citep[xxiv]{nurse2011}}

%Command \noindent added to avoid having a first indent in cases where a paragraph starts after an epigraph without an intervening title.
{\noindent}I am grateful to my teachers for long years of patient and inspiring training. Laurent Danon-Boileau, my first linguistics tutor, advised students to study lesser"=documented languages. 
% – a~field about which I knew nothing. 
I realized at once that this was what I wanted to do: to apply myself to the documentation of a~lesser"=known language. In 1994, reading Michel Launey’s newly published description of Classical Nahuatl as “an omnipredicative grammar” \citep{launey1994}, my imagination was fired by the mention (p. 16) of the professional and personal coincidences that had led him to study a~language remote in time and space, which yielded groundbreaking insights into human language. I dreamt of making a~scientific contribution by exploring a~distant language and culture. I finally experienced this wonderful blend of discovery, excitement and fulfilment in my fieldwork in Southwest China. 

Many thanks to Jacqueline Vaissière for undertaking to raise a~novice steeped in the cloudiest romanticism to the status of Doctor in phonetics, and for her continued guidance over the years. 

My work on Yongning Na began in 2006, at the same time as I joined the \textit{Langues et Civilisations à Tradition Orale} (LACITO) research centre within \textit{Centre National de la Recherche Scientifique} (CNRS). With its tradition of immersion fieldwork and in"=depth research on endangered languages, LACITO proved a stimulating and congenial work environment. I am grateful for the opportunity allowed me by CNRS of staying in China in 2011--2012 for fieldwork, through a~temporary
affiliation with the \textit{Centre d’Etudes Français sur la Chine contemporaine} (CEFC). I would also like to thank the Institute of Linguistics, Academia Sinica for hosting me for three months (January to March 2011). From November 2012 to June 2016, I was based at the International Research Institute MICA (Hanoi, Vietnam), in a stimulating environment allowing for close collaboration with colleagues from Asia and elsewhere. Special thanks to the heads of the institute, Phạm Thị Ngọc Yến (succeeded in 2015 by Nguyễn Việt Sơn) and Eric Castelli, for their support and encouragement.

I am grateful to the Dongba Culture Research Institute (\zh{丽江市东巴文化研究院}) in Lijiang and the Horse-Tea Road Culture Research Centre (\zh{云南大学茶马古道文化研究所}) in Kunming for facilitating administrative and practical matters; special thanks to Li Dejing \zh{李德静} and Mu Jihong \zh{木霁弘}. At Yunnan University, many thanks are due to Duan Bingchang \zh{段炳昌}, Wang Weidong \zh{王卫东}, Zhao Yanzhen \zh{赵燕珍} and Yang Liquan \zh{杨立权} for their sensitive management of fieldwork"=related administrative matters. 

%, and for inviting me to become an Adjunct member (\zh{外聘研究员})

Many thanks to Picus Ding for putting me in touch with the Mosuo scholar Latami Dashi \zh{拉他咪·达石} (\ipa{lɑ˧tʰɑ˧mi˥ ʈæ˧ʂɯ˧}), and to Latami Dashi for supporting and encouraging my work with his mother Mrs. Latami Dashilame \zh{拉他咪·达石拉么} (\ipa{lɑ˧tʰɑ˧mi˥ ʈæ˧ʂɯ˧"=lɑ˩mv̩˩}) over the years. I am grateful to the Yongning Na language consultants (in particular my main consultant, Mrs.~Latami Dashilame) for their patience and support. 

Many thanks to connoisseurs of the Na culture and language for our exchanges, and for their useful comments on draft versions: Lamu Gatusa \zh{拉木·嘎吐萨} (Chinese pen-name: Shi Gaofeng \zh{石高峰}), Liberty Lidz, Christine Mathieu, Pascale"=Marie Milan and Ho Sana \zh{何撒娜}. Special thanks to Roselle Dobbs for extensive discussions and vigorous text editing over the years. Many thanks to the three anonymous reviewers for wonderfully thorough and helpful reviews, and to Chen Yen-ling \zh{陳彥伶}, Katia Chirkova, Denis Creissels, Stéphane Gros, Nathan Hill, Guillaume Jacques, Martine Mazaudon, Boyd Michailovsky, Frédéric Pain, Phạm Thị Thu Hà, Annie Rialland, Martine Toda and Meng Yang for their useful comments on draft chapters. 

Many thanks are also due to Séverine Guillaume, the engineer in charge of the Pangloss Collection (an online archive of recordings of rare languages), for her help in the archiving of the annotated audio recordings that constitute the empirical foundations of the present volume. Many thanks to Luise Dorenbusch for conversion of the original draft to \LaTeX{}.
 %and for working out \textit{haute couture} technical solutions for various typesetting issues; 
 Many thanks to Guillaume Jacques, Thomas Pellard and Sebastian Nordhoff for their help with \LaTeX{}. Many thanks to Jérôme Picard for drawing the map. Many thanks to the series editor, Martin Haspelmath, to the Language Science Press coordinator, Sebastian Nordhoff, and to the wonderful team of volunteers who proofread the volume. Remaining errors and shortcomings are mine alone.

Many thanks to my wife and my daughter for their patience and support. 

So many people have supported this project that I must apologize for those names that should be here but were inadvertently left off the list. 

% Right overhead missing; added with special command
\rohead{Acknowledgments}

Fieldwork on Yongning Na was funded through two grants from the \textit{Agence Nationale de la Recherche} (ANR, France): PASQi (ANR-07-JCJC-0063) and Himalco (ANR-12-CORP-0006). The present book is a~contribution to %the work package “Evolutionary approaches to phonology: New goals and new methods (in diachrony and panchrony)” of 
the Labex project “Empirical Foundations of Linguistics” (ANR-10-LABX-0083). 

% \printbibliography[heading=subbibliography]
% \end{refsection}

