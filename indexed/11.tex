\chapter{Yongning Na in its areal context}
\label{chap:arealperspectives}
\label{sec:acomparisonwithothersinotibetanlanguages}

This chapter presents some observations about Yongning Na in its areal
context, pointing out similarities and differences with a~few other \il{Sino-Tibetan}Sino"=Tibetan languages of the area with which Na may have been in contact in the past. The chapter is much too short to provide adequate coverage of its topic: in Wikipedia parlance, it would be called a “stub”. It nonetheless appeared useful to gather areal observations (no matter how tentative) in a~separate chapter, rather than blend them into the typological discussion in Chapter~\ref{chap:arealandtypologicaldiscussion}, where they do not really belong.


\section{Naxi and Laze: Close relatives, but not part of a~convergence area}
\label{sec:compwithnaxi}
%\section{Comparison within the Naish group of languages}
\label{sec:comparisonwithinnaish}

It has been observed that “the same phonological processes such as \isi{tone sandhi} and \isi{lengthening}
obviously make reference to different prosodic domains within the same language family ({Bantu})”
\citep[132]{zerbian2006a}. High diversity is also found within Naish, even though this lower"=level grouping is incomparably {\linebreak}smaller than Bantu.

Comparison of Na with \ili{Naxi} and \il{Laze|textbf}Laze is fundamental for {diachronic} investigation, since Naxi and Laze are the closest known relatives of Yongning Na (see \sectref{sec:thepositionofnaandnaxiwithinsinotibetan}). But from an areal point of view, I have not been able to find evidence of diffusional convergence between Na and the other two. It seems as if these languages had not been part of a~convergence area. On the contrary, there has been strong social, political and cultural \textit{divergence} since the 14\textsuperscript{th} century: the Naxi chiefdom of Lijiang was increasingly Sinicized (and finally came under direct Chinese administration in the 18\textsuperscript{th} century), whereas the
feudal chieftain system was continued in Yongning and Muli until the mid"=20\textsuperscript{th} century due to the failure of attempts at military conquest of the Liangshan Yi area, which constitutes the gateway to these peripheral regions (see Appendix B, \sectref{sec:historicaloutline}). As for the Laze, they are a~small group of some four hundred people who migrated from Yanbian to their
current location in Muli towards the end of the 19\textsuperscript{th} century. They are reported to be among the speakers of \ili{Naish} languages who left Yanbian as it became a~dominantly \ili{Yi} area, but no information is available about sociolinguistic configurations before the influx of Yi clans, so I~am not in a~position to tell how areal convergence may have contributed to shaping the prosody of Laze in past centuries. The present"=day areal situation is not well"=understood either, as the other Naish dialects spoken in the vicinity of the Laze villages remain undocumented, as far as I know.

%The Laze who settled in the valley of Xiangjiao \zh{项脚} (in Muli \zh{木里} county) found themselves in an environment where their neighbours in were speakers of other Naish dialects; they developed no bilingualism in languages that are spoken in other areas of Muli, such as Pumi, Tibetan, Shixing (Xumi), Namuyi, or Lizu. This goes a~long way towards explaining why Laze did not undergo convergence towards prosodic culminativity.

From a~prosodic point of view, \ili{Naxi} and \ili{Laze} currently exemplify the \textit{nonrestricted tone} type in the sense of \citet{
voorhoeve1973}: tones are assigned to individual syllables without regard to the tone pattern of the entire word or tone phrase. In particular, in these 
two languages there is no limitation on the number of H tones that appear in succession, unlike in Yongning Na, where H tone is culminative (there is 
at most one H tone in each tone group). Interestingly, culminativity has been proposed as an areal characteristic of the languages that have long 
coexisted in Muli, a~county that neighbours on Yongning \citep[160]{chirkova2012}.\footnote{Muli was a~semi"=independent kingdom ruled by 
Pumi hereditary lama kings until the mid-20\textsuperscript{th} century. A caveat about contact phenomena is in order here: the idea is not that Muli 
county \textit{as a whole} constituted one convergence area, but that convergence took place in some parts of Muli that were multilingual over long 
stretches of time. The high mountains and deep valleys of Muli create formidable obstacles to communication, and different parts of Muli constitute 
strikingly different sociolinguistic environments, with different languages in contact and different relationships of prestige between ethnic groups and 
their languages.} In terms of this important property of the prosodic system, Yongning Na does not pattern together with Naxi or Laze, but with 
languages that have long been spoken in Muli: \ili{Pumi}, \ili{Namuyi}, \ili{Shixing}~/ Xumi, \ili{Lizu},\footnote{\textit{Lizu} is not to be confused with \textit{Lisu} \zh{傈僳语}. The former is an Ersuish language, spoken by approximately 7,000 
people who reside along the banks of the Yalong \zh{雅砻} River (Tibetan: Nyag chu) \citep{chirkovaetal2012}; the latter is a~Yi (Loloish) language spoken by about 900,000 people in a~wide area that straddles boundaries between China (Yunnan and Sichuan), Thailand, Burma and India.} and the local dialect 
of \ili
{Tibetan}. Among these, Pumi deserves particular attention, because it is also spoken in Yongning.


\section{Comparison with Pumi}
\label{sec:compwithpumi}

The prosodic system of Na is remarkably close to that of \ili{Pumi} (also known as Prinmi), a~neighbouring \il{Sino-Tibetan}Sino"=Tibetan language. 

\subsection[The tone group and its ties with information structure]{The tone group and its role in conveying information structure}
\label{sec:thetonegroupasbuildingblockofutterancesanditsroleinconveyinginformationstructure}

In Wadu \zh{瓦都} \ili{Pumi} as in Yongning Na, there are \textit{tone groups}, similarly defined by
a~tonal criterion:

\begin{quotation}
	Within a~\isi{tone group}, the underlying tone of one lexical element (usually the left"=most element)
	spreads (usually rightwards) to the adjacent morphemes in the same \isi{tone group} ({\dots}). The
	remaining elements in a~\isi{tone group} are assigned default low surface tone. Tone does not
	spread across \isi{tone group} boundaries. \citep[66]{daudey2014}
\end{quotation}

%Command \noindent added to avoid having an indent. Proofreader suggestion: since this sentence continues the argument, it is better not to indent. 
{\noindent}As in Yongning Na, the \isi{tone group} plays a~key role in conveying \isi{information structure}.

\begin{quotation}
	[S]ome elements always combine with others into a~single \isi{tone group}, some elements always form
	a~\isi{tone group} by themselves, and for some elements, speakers can decide to combine or not combine
	them into tone groups. The latter elements are the most interesting, in that they allow the speaker
	to express pragmatic differences through the choice of combining them or not. \citep[68]{daudey2014}
\end{quotation}

%Command \noindent added to avoid having an indent. Proofreader suggestion: since this sentence continues the argument, it is better not to indent. 
{\noindent}The parallel with the observations about Na set out in Chapter~\ref{chap:toneassignmentrulesandthedivisionoftheutteranceintotonegroups} is striking. Such similarities raise the issue of whether \isi{language contact} is involved. The variety of \ili{Pumi} studied by H. Daudey (Wadu \ili{Pumi}) is spoken in the plain of Yongning, where the Na and the \ili{Pumi} have lived together on good terms for at least eight centuries, so the similarities could be due to \isi{language contact}. The two groups “frequently intermarry and so a~fair amount of \ili{Pumi} speak or understand Yongning Na to some degree. The reverse is not necessarily true” \citep[5]{daudey2014}. But similar characteristics are also observed in another dialect of \ili{Pumi}, that of Niuwozi \zh{牛窝子}, which is not in contact with Yongning Na. This dialect is spoken close to the Ninglang county seat; it is in contact with another variety of the Na language (\ipa{lo˧gv̩˩}; in Chinese: Běiqúbà \zh{北渠坝}), not mutually intelligible with that of Yongning. (As a~piece of anecdotal evidence about the degree of mutual comprehension: one of the daughters of consultant F4 married a~Na from that area; the differences in dialect led the couple to adopt {Mandarin} to communicate with each other.) While the vocabulary adopted in the linguistic description is slightly different, the observations appears to match closely those made about Wadu \ili{Pumi} and Yongning Na. 

\begin{quotation}
	Under the influence of {intonation}, the underlying H tone of a~phonological word is readily removable when it is situated in the final unit of the clause ({\dots}). When this happens, the phonological word is merged with the other prosodic domain (removal of the original {boundary} of phonological word due to a~loss of H tone in a~following word [{\dots}]). Sometimes, a~series of low tones may appear in the ending syllables of an~utterance after the {boundary} of phonological word is eliminated ({\dots}). \citep[69]{ding2014}
\end{quotation}

%Command \noindent added to avoid having an indent. Proofreader suggestion: since this sentence continues the argument, it is better not to indent. 
{\noindent}Thus, the similarities in prosodic organization between Pumi and Na are not necessarily the outcome of areal convergence. As pointed out in \sectref{sec:concludingremark}, the division of an utterance into intonation phrases plays a central role in conveying phrasing and prominence in the most diverse languages, including thoroughly unrelated (and extensively studied) languages such as English.\footnote{Within the Sino"=Tibetan family, a~distant echo to the Na facts is found in Zhuokeji Rgyalrong, where “clauses that are
	juxtaposed without any overt linkage marker that denotes coordination or
	consecutivization, or any morphosyntactic marking that signals dependency of
	one clause on the other” can be integrated into one group (called “intonation unit”) to highlight “strong rhetorical links” between these clauses \citep[208]{lin2009}.}

\subsection{Other similarities}
\label{sec:othersimilarities}

There are also similarities between Na and \ili{Pumi} phonological rules, such as that each prosodic domain requires at least one non"=L tone, and when none is present, a~H tone is added to the final syllable, yielding a~rising tone, LH \citep[60]{ding2014}. Concerning numeral"=plus"=classifier\is{classifiers} phrases, \citet[69]{ding2014} notes that their tone patterns “are not utterly predictable from the tones of the two formatives, as other factors beyond phonology are at work”. This is parallel to the situation found in Na (studied in Chapter~\ref{chap:classifiers} of the present volume), although judging from Picus Ding's book, the degree of complexity found in that particular variety of \ili{Pumi} would seem to be smaller than in the variety of Na studied here. 
%(The issue of assessing the degree of complexity of the tone system is addressed in \sectref{sec:morphophonologicalcomplexity}.)

The high number of points of similarity suggests that further comparison of Na and \ili{Pumi} could be highly revealing. The aim would be to attain the degree of depth and precision reached by \citet{wagneretal2010} in their comparison of {English} and French. 



%\subsection{Role of the number of syllables of the words involved a~given tone rule}
%\label{sec:roleofthenumberofsyllablesofthewordsinvolvedagiventonerule}
%
%No tone change takes place in compounding when one of the two input nouns has more than two
%syllables (see~\ref{sec:themainfactscoordinativecompounds}). The influence of the number of syllables on the type of
%phonological processes that take place in compounds is a~point of similarity with other languages of
%the area, such as \ili{Shixing}.

\section{Comparison with Yi}
\label{sec:compwithyi}

\ili{Naish} languages have striking typological similarities with languages of the \ili{Yi} (Loloish) branch of {Burmese}-{Yi} ({Burmese}-Lolo). From a~tonal point of view, there are similarities of the kind that one would expect given the similarities in syntax: \isi{tone sandhi} occurs in contexts such as compounds and \is{numerals}numeral"=plus"=classifier phrases. In \ili{Nosu}, for instance, there exists an~alternation whereby tone ³³ changes to the sandhi tone ⁴⁴ (transcribed in the orthography as a~final \textit{x}) when followed by another ³³ tone. This dissimilatory process is morphosyntactically conditioned, witness the existence of a~tonal distinction between \textit{nga gu} ‘I called (someone)’ and \textit{ngax gu} ‘(Someone) called me’: in this case, the tonal difference reflects one between {agent} and {patient} \citep[28]{gerner2013}. The morphotonology has limited extent, however: in total, there are eight contexts where \isi{tone sandhi} occurs. Their description takes up no more than three pages in a~grammar of half a~thousand pages \citep[28--30]{gerner2013}.

\section[Contact between two"=level and three"=level tone systems]{A hypothesis about contact between two"=level and three"=level tone systems}
\label{sec:twolevelsthreelevels}

Contact between two"=level and three"=\is{level tones}level tone systems appears as an especially interesting topic for areal studies of tone. \ili{Pumi}, with
which Na has been in at least occasional contact for centuries, has two levels, L and H. Among Na dialects, two"=level systems are found in Wuzhiluo \zh{五指落}, on the north edge of the swamp area known as the Grass Sea which forms the
eastern end of Lake Lugu \citep{dobbsetal2016} and in Shuiluo \zh{水落}, in the county of Muli \zh{木里} (source: unpublished field notes, 2009). The dialect of Luoshui \zh{落水}, geographically close to the Yongning plain, has three levels, but among these, the highest level has a~relatively marginal status \citep{lidz2010}. Alawa (the dialect studied in this volume) has three levels, but with a~strongly restricted distribution. By contrast, dialects spoken further to the West and the Northwest, such as Labai \zh{拉柏}, clearly have three levels (L, M and H). Past contact between two"=level and three"=\is{level tones}level tone systems could shed light on synchronic phenomena found in Alawa, such as the exceptionless {phonological rule} prohibiting tone"=group"=initial H
tone, effectively limiting the number of tonal contrasts to two in group"=initial position. This synchronic rule has far"=reaching consequences: at the surface phonological level, it is
impossible to have H tone (as distinct from M) on a~monosyllable said \is{form!in isolation}in isolation, or tone patterns H.M, H.MH, H.L or
H.H on disyllables said \is{form!in isolation}in isolation, because an isolated form (often referred to as a~\textit{citation form}) constitutes a~\isi{tone group} on its own. As pointed out in \sectref{sec:thecreationoffloatinghtonesaconsequenceofphonotacticconstraints}, correspondences between overt H~tones in the Labai dialect and \is{floating tone}floating H~tones in Alawa suggest the possibility that word"=initial H~tones in Alawa were shifted from their position on the
first syllable of the word as a~response to the enforcement of the rule prohibiting initial H. In this scenario, the next {question} is why word"=initial H tones ceased to be phonotactically licit. At this point, one can entertain the possibility of contact with a~two"=\is{level tones}level tone system, whose speakers had special difficulty handling a~three"=term tonal opposition on an initial syllable.  Synchronic case studies of contact between two"=level and three"=level tonal systems would be useful to gain insights into the types of processes to be expected, and the possible consequences for the linguistic systems in contact. The findings of such studies, combined with additional data on present-day dialectal diversity, could shed light on the historical role played by contact in shaping the tonal systems that can be observed today.
