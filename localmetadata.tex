\title{Tone in Yongning~Na}  
\subtitle{Lexical tones and morphotonology}
\BackTitle{Tone in Yongning~Na}
\BackBody{Yongning Na, also known as Mosuo, is a~Sino-Tibetan language spoken in Southwest China. This book provides a~description and analysis of its tone system, progressing from lexical tones towards morphotonology. Tonal changes permeate numerous aspects of the morphosyntax of~Yongning Na; they are not the product of a~small set of phonological rules, but of a~host of rules that are restricted to specific morphosyntactic contexts. Rich morphotonological systems have been reported in this area of Sino-Tibetan, but book-length descriptions remain few. This study of an endangered language contributes to a~better understanding of the diversity of prosodic systems in East Asia.
	
	The analysis is based on original fieldwork data (made available online), collected over the course of ten years, commencing in 2006.
%	\\
%	
%	\zh{2006年秋,我第一次听到摩梭人的对话,对于以“捕捉声音”为职业的语音学家来说,职业的敏感令我预感到这里有一座声调的语音迷宫等待着被破解,让我如获至宝。自此,我一头扎进了摩梭话声调研究当中。整整十年间,我一直在这座声调迷宫里游弋。
%	
%	语音转瞬即逝,在摩梭话语流中抓取规律并非易事,然而这就是语音学的魅力之所在。不间断语言采风,深入第一线语音记录,大量实验室语音数据分析是支撑我最终完成此书的重要基础。
%	
%	本书系统地描述了摩梭话词汇层的调类与声调变化规律:不同组合的声调形态变化以及句法变调。最后,指出了近年来摩梭话声调系统演变方向。书中也涉及到了语音类型学比较。
%		
%	摩梭话声调,这个只有当语言流动起来才会扇动翅膀的隐身精灵,是我十年来一直追逐不懈的主题。籍《摩梭话声调研究——从词汇层次的调类到声调形态》一书,希望揭开摩梭话声调声态语法作用的序幕。也寄望各界读者不吝赐教,共同探讨。世界语言的声音宝库众多,让我们一起努力探秘。}
}
\dedication{À mon père}
\typesetter{Luise Dorenbusch, Benjamin Galliot, Guillaume Jacques, Alexis Michaud, Sebastian Nordhoff, Thomas Pellard}
\proofreader{Eran Asoulin, Rosey Billington, Mykel Brinkerhoff, Henriëtte Daudey, Aude Gao (Gao Yang \zh{高扬}), Andreas Hölzl, Lachlan Mackenzie, Maria Isabel Maldonado, Claudia Marzi, Jean Nitzke, Sebastian Nordhoff, Ikmi Nur Oktavianti, Ahmet Bilal Özdemir, Ludger Paschen, Brett Reynolds, Alec Shaw}
\author{Alexis Michaud} % anonymized for reviewing
% \keywords{tone; prosody; morphotonology; morphophonology; Yongning Na}%add 5 keywords
\renewcommand{\lsISBNdigital}{978-3-946234-86-9}
\renewcommand{\lsISBNhardcover}{978-3-946234-87-6}
\renewcommand{\lsISBNsoftcover}{978-3-946234-68-5}
\renewcommand{\lsSeries}{sidl} % use lowercase acronym, e.g. sidl, eotms, tgdi
\renewcommand{\lsSeriesNumber}{13} %will be assigned when the book enters the proofreading stage
\renewcommand{\lsURL}{http://langsci-press.org/catalog/book/109} % contact the coordinator for the right number
\BookDOI{10.5281/zenodo.439004}

\renewcommand{\lsAdditionalFontsImprint}{, Charis SIL, AR PL UMing}

